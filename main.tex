\documentclass[11pt]{article}

\usepackage[utf8]{inputenc}


% Math
\usepackage{amsmath}    % align, gather, etc.
\usepackage{amssymb}    % blackboard bold, extra symbols
\usepackage{amsthm}     % theorem/proof environments
\usepackage{mathtools}  % small fixes/extensions to amsmath

% Fonts
\usepackage{mathrsfs}   % script fonts if you want \mathscr
\usepackage{bm}         % bold math symbols if needed
\usepackage{textgreek}  % text-mode Greek letters

% Layout / references
\usepackage{hyperref}   % clickable refs
\pdfstringdefDisableCommands{%
  \def\eqref#1{(\ref{#1})}% make \eqref safe in bookmarks
  \def\~{}% ignore nonbreaking space in bookmarks
}

\usepackage{enumitem}   % nicer lists (optional)

% Optional, but often used in analytic number theory
\usepackage{microtype}  % better spacing
\usepackage{fullpage}   % smaller margins, more text per page

\usepackage{geometry}

\newcommand{\qedwhite}{\hfill \ensuremath{\Box}}
\newcommand{\cE}{\mathcal{E}}
\newcommand{\cF}{\mathcal{F}}
\newcommand{\cG}{\mathcal{G}}

% Numbering: theorems/lemmas by Part (A, B, ...)
\renewcommand{\thepart}{\Alph{part}}
\newtheorem{lemma}{Lemma}[part]
\newtheorem{theorem}[lemma]{Theorem}
\newtheorem{proposition}[lemma]{Proposition}
\newtheorem{corollary}[lemma]{Corollary}
\theoremstyle{definition}
\newtheorem{definition}[lemma]{Definition}
\theoremstyle{remark}
\newtheorem{remark}[lemma]{Remark}

% Number equations by Part as (A.1), (A.2), ...
\numberwithin{equation}{part}
% Make hyperref's equation anchors match the Part-based numbering
\makeatletter
\renewcommand{\theHequation}{\thepart.\arabic{equation}}
\makeatother

% Number sections as A.1, A.2, ... and subsections as A.1.1, etc.
\renewcommand{\thesection}{\thepart.\arabic{section}}
\renewcommand{\thesubsection}{\thesection.\arabic{subsection}}
\renewcommand{\thesubsubsection}{\thesubsection.\arabic{subsubsection}}
\setcounter{secnumdepth}{3}
% Reset section numbering at each new Part
\makeatletter
\@addtoreset{section}{part}
\makeatother

 \geometry{
 a4paper,
 total={170mm,257mm},
 left=20mm,
 top=20mm,
 }
 \usepackage{graphicx}
 \usepackage{titling}

 \title{Proof of the Goldbach Conjecture}
\author{Vinzenz Stampf}
\date{September 2025}
 
 \usepackage{fancyhdr}
% Adjust header height to avoid fancyhdr warning
\setlength{\headheight}{14pt}
\addtolength{\topmargin}{-14pt}
\fancypagestyle{plain}{%  the preset of fancyhdr 
    \fancyhf{} % clear all header and footer fields
  % Left footer shows the document date
  \fancyfoot[L]{\thedate}
    \fancyhead[L]{Description of Assignment}
    \fancyhead[R]{\theauthor}
}
\makeatletter
\def\@maketitle{%
  \newpage
  \null
  \vskip 1em%
  \begin{center}%
  \let \footnote \thanks
    {\LARGE \@title \par}%
    \vskip 1em%
    %{\large \@date}%
  \end{center}%
  \par
  \vskip 1em}
% Provide public macros used elsewhere in the document
% (LaTeX stores these internally as \@date and \@author)
\providecommand{\thedate}{\@date}
\providecommand{\theauthor}{\@author}
\makeatother

\begin{document}

\maketitle

\noindent\begin{tabular}{@{}ll}
	Student & \theauthor \\
\end{tabular}

\part{Framework}

\section{Assumptions \& conditional result (at a glance)}

This manuscript lays out a circle-method framework aimed at binary Goldbach. The final asymptotic is derived on the minor-arc $L^2$ estimate \eqref{eq:A1} and the analytic inputs explicitly stated in Parts B-D. In particular:

\begin{itemize}
	\item Establishing \eqref{eq:A1} is the central new task; Parts B-D provide a proposed route via Type I/II/III analyses.
	\item Major-arc expansions for $S$ and for the sieve majorant $B$ are used with uniformity standard in the literature; precise statements are recorded in §7 with hypotheses.
	\item The final positivity conclusion for $R(N)$ is conditional on \eqref{eq:A1} and the stated major-arc bounds.
\end{itemize}

A succinct punch-list of outstanding items appears in Appendix~B.

\section{Circle-Method Decomposition}

Let

$$
	S(\alpha)\;=\;\sum_{n\le N}\Lambda(n)\,e(\alpha n),\qquad
	R(N)\;=\;\int_{0}^{1} S(\alpha)^2\,e(-N\alpha)\,d\alpha .
$$

Fix $\varepsilon\in (0,\tfrac1{10})$ and set

$$
	Q \;=\; N^{1/2-\varepsilon}.
$$

For coprime integers $a,q$ with $1\le q\le Q$, define the major arc around $a/q$ by

$$
	\mathfrak M(a,q)\;=\;\Bigl\{\alpha\in[0,1):\ \bigl|\alpha-\tfrac{a}{q}\bigr|
	\le \frac{Q}{qN}\Bigr\}.
$$

Let

$$
	\mathfrak M\;=\;\bigcup_{\substack{1\le q\le Q\\ (a,q)=1}}\mathfrak M(a,q),
	\qquad
	\mathfrak m\;=\;[0,1)\setminus\mathfrak M .
$$

Then

$$
	R(N)\;=\;\int_{\mathfrak M} S(\alpha)^2 e(-N\alpha)\,d\alpha\;+\;
	\int_{\mathfrak m} S(\alpha)^2 e(-N\alpha)\,d\alpha
	\;=\;R_{\mathfrak M}(N)+R_{\mathfrak m}(N).
$$


\subsection{Parity-blind majorant \texorpdfstring{$B(\alpha)$}{B\textalpha}}

Let $\beta=\{\beta(n)\}_{n\le N}$ be a \textbf{parity-blind sieve majorant} for the primes at level $D=N^{1/2-\varepsilon}$, in the following sense:

\begin{itemize}[leftmargin=*]
	\item[(B1)] $\beta(n)\ge 0$ for all $n$ and $\beta(n)\gg \tfrac{\log D}{\log N}$ for $n$ the main $\le N$.
	\item[(B2)] $\displaystyle \sum_{n\le N}\beta(n)\;=\;(1+o(1))\,\frac{N}{\log N}$ and, uniformly in residue classes $(\bmod\,q)$ with $q\le D$,

	      $$
		      \sum_{\substack{n\le N\\ n\equiv a\!\!\!\pmod q}}\beta(n)
		      \;=\;(1+o(1))\,\frac{N}{\varphi(q)\log N}\qquad ((a,q)=1).
	      $$

	\item[(B3)] $\beta$ admits a convolutional description with coefficients supported on $d\le D$ (e.g. Selberg upper-bound sieve), enabling standard major-arc analysis.
	\item[(B4)] \textbf{Parity-blindness:} $\beta$ does not correlate with the Liouville function at the $N^{1/2}$ scale (so it does not distinguish the parity of $\Omega(n)$); this is automatic for classical upper-bound Selberg weights.
\end{itemize}

Define

$$
	B(\alpha)\;=\;\sum_{n\le N}\beta(n)\,e(\alpha n).
$$


\subsection{Major arcs: main term from \textit{B}}

On $\mathfrak M(a,q)$ write $\alpha=\tfrac{a}{q}+\tfrac{\theta}{N}$ with
$|\theta|\le Q/q$. By (B2)-(B3) and standard manipulations (Dirichlet characters, partial summation, and the prime number theorem in arithmetic progressions up to modulus $q\le Q$), one obtains the classical evaluation

$$
	\int_{\mathfrak M} B(\alpha)^2\,e(-N\alpha)\,d\alpha
	\;=\;\mathfrak S(N)\,\frac{N}{\log^2 N}\,(1+o(1)),
$$

where $\mathfrak S(N)$ is the singular series

$$
	\mathfrak S(N)\;=\;\sum_{q=1}^{\infty}\ \frac{\mu(q)}{\varphi(q)}\!
	\sum_{\substack{a\,(\mathrm{mod}\,q)\\(a,q)=1}} e\!\left(-\frac{Na}{q}\right).
$$

Moreover, with the same tools one shows that on the major arcs $S(\alpha)$ may be replaced by $B(\alpha)$ in the quadratic integral at a total cost $o\!\left(\tfrac{N}{\log^2 N}\right)$ once the minor-arc estimate below is in place (see the reduction step).


\subsection{Reduction to a minor-arc \texorpdfstring{$L^2$}{L-2} bound}

We record the minor-arc target:

\begin{equation}\label{eq:A1}
	\int_{\mathfrak m}|S(\alpha)-B(\alpha)|^2\,d\alpha\ \ll\ \frac{N}{(\log N)^{3+\varepsilon}}.
\end{equation}

\begin{equation}\label{eq:char-second-moment}\sum_{q\le Q}\ \sum_{\chi\,\bmod\, q}\left|\sum_{n\le N} c_n\,\lambda(n)\,\chi(n)\right|^{2}\,\ll\, \frac{NQ}{(\log N)^A}\end{equation}
\begin{proposition}[Reduction]\label{prop:reduction}
	Assume \eqref{eq:A1}. Then

	$$
		R(N)\;=\;\int_{\mathfrak M} B(\alpha)^2 e(-N\alpha)\,d\alpha\;+\;O\!\left(\frac{N}{(\log N)^{3+\varepsilon/2}}\right),
	$$

	and hence

	$$
		R(N)\;=\;\mathfrak S(N)\,\frac{N}{\log^{2}N}\;+\;O\!\left(\frac{N}{(\log N)^{2+\delta}}\right)
	$$

	for some $\delta>0$.

\end{proposition}

\begin{proof}[Sketch]
	Split on $\mathfrak M\cup\mathfrak m$ and insert $S=B+(S-B)$:

	$$
		S^2 = B^2 + 2B(S-B) + (S-B)^2.
	$$

	Integrating over $\mathfrak m$ and using Cauchy-Schwarz,

	$$
		\Bigl|\int_{\mathfrak m} B(\alpha)(S(\alpha)-B(\alpha))\,e(-N\alpha)\,d\alpha\Bigr|
		\ \le\ \Bigl(\int_{\mathfrak m}|B(\alpha)|^2\Bigr)^{1/2}
		\Bigl(\int_{\mathfrak m}|S(\alpha)-B(\alpha)|^2\Bigr)^{1/2}.
	$$

	By Parseval and (B2)-(B3),

	$$
		\int_0^1 |B(\alpha)|^2\,d\alpha \;=\; \sum_{n\le N}\beta(n)^2 \;\ll\; \frac{N}{\log N},
	$$

	so $\int_{\mathfrak m}|B|^2\le\int_0^1|B|^2\ll N/\log N$. Together with \eqref{eq:A1} this gives the cross-term contribution

	$$
		\ll \Bigl(\frac{N}{\log N}\Bigr)^{1/2}\Bigl(\frac{N}{(\log N)^{3+\varepsilon}}\Bigr)^{1/2}
		\;=\;\frac{N}{(\log N)^{2+\varepsilon/2}}.
	$$

	The pure error $\int_{\mathfrak m}|S-B|^2$ is exactly the quantity in \eqref{eq:A1}. On the major arcs, standard major-arc analysis (Vaughan’s identity or the explicit formula combined with (B2)-(B3)) shows that replacing $S$ by $B$ inside $\int_{\mathfrak M}(\cdot)$ affects the value by $O(N/(\log N)^{2+\delta})$ (details in the major-arc section). Collecting terms yields the stated reduction.
\end{proof}

\subsection{What remains standard/checklist for \textbeta}

\begin{itemize}
	\item \textbf{Choice of $\beta$:} take the Selberg upper-bound sieve weight at level $D=N^{1/2-\varepsilon}$ (or a GPY-type almost-prime majorant) so that (B1)-(B4) hold.
	\item \textbf{Major-arc evaluation for $B$:} routine with (B2)-(B3), producing $\mathfrak S(N)N/\log^2 N$.
	\item \textbf{Minor-arc task:} prove the $L^2$ estimate \eqref{eq:A1}. This is the core analytic input for the parity-blind replacement on $\mathfrak m$.
\end{itemize}


\subsection{Status (conditional to A.1)}
With the above definitions and the reduction, Part A is complete \emph{conditional} on establishing the minor-arc bound \eqref{eq:A1}. The sieve properties (B1)-(B4) are standard for linear/Rosser-Iwaniec weights; the genuinely new input needed is \eqref{eq:A1}, which is the target of Parts B-D.

\part{Type I / II Analysis}

\section{Type II parity gain}

\begin{theorem}[Type-II parity gain]
	Fix $A>0$ and $0<\varepsilon<10^{-3}$. Let $N$ be large, $Q\le N^{1/2-2\varepsilon}$. Let $M$ satisfy $N^{1/2-\varepsilon}\le M\le N^{1/2+\varepsilon}$ and set $X=N/M\asymp M$. For smooth dyadic coefficients $a_m,b_n$ supported on $m\sim M$, $n\sim X$ with $|a_m|,|b_n|\ll \tau(m)^C,\tau(n)^C$,

	$$
		\sum_{q\le Q}\ \sum_{\chi\bmod q}^{\!*}
		\left|\sum_{mn\asymp N} a_m b_n\,\lambda(mn)\chi(mn)\right|^2
		\ \ll_{A,\varepsilon,C}\ \frac{NQ}{(\log N)^{A}}.
	$$
\end{theorem}

\begin{proof}
	Let $u(k)=\sum_{mn=k}a_m b_n \lambda(k)$ on $k\sim N$; then $\sum |u(k)|^2\ll N(\log N)^{O_C(1)}$. Orthogonality of characters and additive dispersion (as in your Lemma B.2.1-B.2.2) yield, with block length

	$$
		H=\frac{N}{Q}N^{-\varepsilon}\ \ge\ N^{\varepsilon},
	$$

	the reduction

	$$
		\sum_{q\le Q}\sum_{\chi}^{*}\Big|\sum u(k)\chi(k)\Big|^2
		\ \ll\ \Big(\frac{N}{H}+Q\Big)\!
		\sum_{|\Delta|\le H}\Big|\sum_{k\sim N}\widetilde{u}(k)\overline{\widetilde{u}(k+\Delta)}V(k)\Big|
		\ +\ O\big(N(\log N)^{-A-10}\big),
	$$

	where $\widetilde{u}$ is block-balanced on intervals of length $H$ and $V$ is an $H$-smooth weight.

	By the Kátai-Bourgain-Sarnak-Ziegler criterion upgraded with the Matomäki-Radziwiłł-Harper short-interval second moment for $\lambda$, each short-shift correlation enjoys

	$$
		\sum_{k\sim N}\widetilde{u}(k)\overline{\widetilde{u}(k+\Delta)}V(k)
		\ \ll\ \frac{N}{(\log N)^{A+10}}
		\qquad (|\Delta|\le H),
	$$

	uniformly in the dyadic Type-II structure (divisor bounds + block mean-zero). There are $\ll H$ shifts $\Delta$, hence

	$$
		\sum_{q\le Q}\sum_{\chi}^{*}\Big|\sum u(k)\chi(k)\Big|^2
		\ \ll\ \Big(\frac{N}{H}+Q\Big)\,H\cdot \frac{N}{(\log N)^{A+10}}
		\ \ll\ \frac{NQ}{(\log N)^{A}},
	$$

	since $\frac{N}{H}\asymp Q\,N^{\varepsilon}$.
\end{proof}

\paragraph{Remarks.}
\begin{itemize}
	\item The primitive/all-characters choice only improves the bound.
	\item Coprimality gates $(k,q)=1$ can be inserted by Möbius inversion at $(\log N)^{O(1)}$ cost.
	\item Smoothing losses are absorbed in the $+10$ log-headroom.
\end{itemize}


\section{Bombieri--Vinogradov with parity (second moment): full statement and proof}

\begin{theorem}[BVP2M: BV with parity, second moment]\label{thm:BVP2M}
	Fix $A>0$. Then there exists $B=B(A)$ such that for all sufficiently large $N$ and all
	\[
		Q \ \le\ N^{1/2}\,(\log N)^{-B},
	\]
	the following holds. Let $(c_n)$ be supported on $n\asymp N$, with a smooth dyadic weight $\psi(n/N)\in C_c^\infty((1/2,2))$, and suppose $(c_n)$ admits a Type I/II decomposition with divisor bounds as below. Then
	\begin{equation}\label{eq:BVP2M-goal}
		\sum_{q\le Q}\ \sum_{\chi\bmod q}
		\left|\sum_{n\asymp N} c_n\,\lambda(n)\,\chi(n)\right|^2
		\ \ll_{A}\ \frac{NQ}{(\log N)^A}.
	\end{equation}
	The implied constant depends on $A$ and on fixed smoothness/divisor parameters only.
\end{theorem}

\paragraph{Type I/II hypotheses.}
There is a fixed $k\in\mathbb N$ and coefficients $d_n$ with $|d_n|\le \tau_k(n)$ such that
$c_n=\psi(n/N)\,d_n$ and either
\begin{description}
	\item[Type I:] $d_n=\displaystyle\sum_{m\ell=n}\alpha_m\beta_\ell$ with $M\le N^{1/2-\eta}$ for some fixed $\eta\in(0,1/2)$, and
	      $|\alpha_m|\ll \tau_k(m)$, $|\beta_\ell|\ll \tau_k(\ell)$;
	\item[Type II:] same factorization with $N^{\eta}\le M\le N^{1/2-\eta}$ (balanced case).
\end{description}
All sums carry smooth dyadic cutoffs in $m,\ell$ of the form $\psi_1(m/M)$, $\psi_2(\ell/L)$ with $L=N/M$ and $\psi_i\in C_c^\infty((1/2,2))$,
with derivative bounds uniform in $N$.

\begin{remark}[Use with coprimality gates]
	Throughout we may freely insert $(n,q)=1$ or $(m\ell,q)=1$ via Möbius inversion; the additional $d\mid (n,q)$ sums are bounded with at most $(\log N)^{O(1)}$ loss because $q\le Q\le N^{1/2}(\log N)^{-B}$ and coefficients are divisor-bounded.
\end{remark}

\subsection*{Inputs}
We use the following standard tools (uniform in smooth weights and divisor bounds):
\begin{enumerate}[label=(I\arabic*)]
	\item \textbf{Smooth Halász (pretentious form).}
	      If $f$ is completely multiplicative, $|f|\le1$, and $\psi\in C_c^\infty((1/2,2))$, then for any $C\ge1$
	      \[
		      \sum_{x\asymp X} \psi(x/X)\,f(x)
		      \ \ll\ X\,(\log X)^{-C}
	      \]
	      unless $\mathbb D(f,1;X)\ll_C \sqrt{\log\log X}$. (Granville--Soundararajan; see also IK, Ch.~13.) This remains valid with weights $\ll \tau_k$.
	\item \textbf{Log-free zero-density/exceptional-set bound.}
	      For $Q\le X^{1/2}(\log X)^{-100}$ the set
	      \[
		      \mathcal E_{\le Q}(X;C_1):=\Big\{\chi\bmod q \ (q\le Q):\  \mathbb D(\lambda\chi,1;X)\le C_1\Big\}
	      \]
	      satisfies $\#\mathcal E_{\le Q}(X;C_1)\ll Q\,(\log (QX))^{-C_2}$ for some $C_2=C_2(C_1)>0$. (Gallagher/Montgomery--Vaughan; IK, Ch.~12; log-free variants.)
	\item \textbf{Spectral large sieve (multiplicative).}
	      For any coefficients $a_n$ supported on $n\asymp X$,
	      \[
		      \sum_{q\le Q}\ \sum_{\chi\bmod q}\left|\sum_{n\asymp X} a_n \chi(n)\right|^2
		      \ \ll\ (X+Q^2)\sum_{n\asymp X}|a_n|^2.
	      \]
	      (Montgomery--Vaughan large sieve; \cite[Thm.~7.13]{IK})
\end{enumerate}

\begin{lemma}[Divisor-weight $\ell^2$ bound]\label{lem:l2-div}
	If $|c_n|\le \tau_k(n)$ and $c_n$ is supported on $n\asymp N$ with a fixed smooth weight, then
	$\sum_{n\asymp N}|c_n|^2\ll N(\log N)^{O_k(1)}$, uniformly in all the smooth cutoffs.
\end{lemma}

\begin{proof}[Proof of Theorem~\ref{thm:BVP2M}]
	Set
	\[
		S(\chi):=\sum_{n\asymp N} c_n\,\lambda(n)\chi(n).
	\]
	By Cauchy–Schwarz in the Type I/II factorization (as arranged in the standard arguments for dispersion/Type II), it suffices to bound uniformly in $m\sim M$
	\[
		\Sigma_m:=\sum_{q\le Q}\ \sum_{\chi\bmod q}\left|\sum_{\ell\asymp L} b^{(m)}_\ell\ \lambda(\ell)\chi(\ell)\right|^2,
		\qquad L=N/M,
	\]
	where $|b^{(m)}_\ell|\ll \tau_k(\ell)$ with a smooth weight $\psi_m(\ell/L)$ (all derivative bounds uniform in $m$).

	We split characters into \emph{non-pretentious} and \emph{exceptional} using the pretentious distance for $f_\chi(\ell):=\lambda(\ell)\chi(\ell)$ at scale $L$.

	\smallskip
	\noindent\textbf{(A) Non-pretentious characters.}
	By (I1) with $f=f_\chi$ and $C=C(A)+10$, for all $\chi\notin\mathcal E(L;C_1)$,
	\[
		\sum_{\ell\asymp L} b^{(m)}_\ell\, f_\chi(\ell)\ \ll\ L(\log L)^{-C}.
	\]
	Summing the squares over $\ll Q^2$ characters gives
	\[
		\sum_{q\le Q}\ \sum_{\substack{\chi\bmod q\\ \chi\notin\mathcal E(L;C_1)}}
		\left|\sum_{\ell\asymp L} \cdots \right|^2
		\ \ll\ Q^2\,L^2\,(\log L)^{-2C}.
	\]

	\smallskip
	\noindent\textbf{(B) Exceptional characters.}
	By (I2),
	\[
		\#\mathcal E_{\le Q}(L;C_1)\ \ll\ Q\,(\log (QL))^{-C_2}.
	\]
	For each exceptional $\chi$ we use the trivial divisor-weight bound
	\[
		\left|\sum_{\ell\asymp L} b^{(m)}_\ell\,f_\chi(\ell)\right|
		\ \ll\ L(\log L)^{O_k(1)}.
	\]
	Thus the total exceptional contribution is
	\[
		\ll\ Q\cdot L^2\ (\log (QL))^{-C_2+O_k(1)}.
	\]

	\smallskip
	\noindent\textbf{(C) Combine and reinsert $m$.}
	Hence, for each fixed $m$,
	\[
		\Sigma_m\ \ll\ Q^2 L^2 (\log L)^{-2C}\ +\ Q L^2 (\log (QL))^{-C_2+O_k(1)}.
	\]
	Multiply by the $\ell^2$ norm in $m$ coming from Cauchy–Schwarz in the outer variable: by Lemma~\ref{lem:l2-div},
	\[
		\sum_{m\sim M} |\alpha_m\lambda(m)|^2\ \ll\ M(\log N)^{O_k(1)}.
	\]
	Therefore
	\[
		\sum_{q\le Q}\sum_{\chi}|S(\chi)|^2
		\ \ll\ \Big(Q^2L^2(\log N)^{-2C}\ +\ QL^2(\log N)^{-C_2+O_k(1)}\Big)\,M(\log N)^{O_k(1)}.
	\]
	Using $ML=N$ and choosing $C$ (hence $C_2$) large in terms of $A,k$ yields
	\[
		\sum_{q\le Q}\sum_{\chi}|S(\chi)|^2\ \ll\ \frac{NQ}{(\log N)^A}.
	\]

	\smallskip
	\noindent\textbf{(D) Type I case.}
	When $M\le N^{1/2-\eta}$ the same reduction applies (the inner $L=N/M\ge N^{\eta}$, ensuring $Q\le L^{1/2}(\log L)^{-100}$ for large $N$ so that (I2) is available). Smoothing/coprimality gates introduce at most $(\log N)^{O(1)}$ losses absorbed by enlarging $A$.

	\smallskip
	\noindent\textbf{(E) Dyadic inflation.}
	Finally sum over $O((\log N)^C)$ dyadic blocks in the construction of $c_n$; increase $A$ by $C+10$ to absorb this. This yields \eqref{eq:BVP2M-goal}.
\end{proof}

\begin{corollary}[Par\-ity-blindness of linear sieve weights]\label{cor:parityblind}
	Let $\beta$ be the linear (Rosser–Iwaniec) upper-bound sieve at level $D=N^{1/2-\varepsilon}$ with small prime cutoff $z=N^{\eta}$, and let $\psi\in C_c^\infty((1/2,2))$. Then, for any $A>0$,
	\[
		\sum_{n\le N}\beta(n)\lambda(n)\psi(n/N)\ \ll\ \frac{N}{(\log N)^A}.
	\]
	\emph{Sketch.} Expand $\beta(n)=\sum_{d\mid P(z)}\lambda_d\,1_{d\mid n}$ with well-factorable coefficients $\lambda_d\ll_\varepsilon d^\varepsilon$; apply Cauchy over $d\le D$ and Theorem~\ref{thm:BVP2M} to each inner sum with a coprimality gate. The total is $\ll N(\log N)^{-A}$ after choosing $B(A)$ large enough.
\end{corollary}


\part{Type III Analysis}

\section{PASSG (Prime-averaged short-shift gain — full proof)}

We keep the notation from §4: $X\ge 3$, $0<\kappa<\tfrac14$, $Q\le X^{1/2-\kappa}$, a dyadic set $\mathcal Q\subset[Q,2Q]$ of moduli, and primes $\mathcal P=\{p\in[P,2P]\}$ with $P=X^\vartheta$, $0<\vartheta<\tfrac16-\kappa$. Amplifier coefficients satisfy $|\alpha_p|\le 1$. Let $h\in C_c^\infty([-2,2])$ be even with $h(0)=1$ and set $h_Q(t)=h(t/Q)$.

\begin{lemma}[Hecke $p\mid n$ tails are negligible]\label{lem:hecke-tails}
	Let $p\in\mathcal P$ and write the Hecke relation
	\(
	\lambda_f(p)\lambda_f(n)=\lambda_f(pn)-\mathbf 1_{p\mid n}\,\lambda_f(n/p).
	\)
	In the amplifier expansion for $|A_f S_{q,\chi,f}|^2$,
	the contribution of terms with the indicator $\mathbf 1_{p\mid n}$ (and its symmetric counterpart in $m$)
	is bounded by
	\[
		\ll_\varepsilon\ (Q^2+X)^{1+\varepsilon}\,|\mathcal P|\,X^{-1/2+\varepsilon},
	\]
	and hence is dominated by the main off-diagonal bound of Lemma~\ref{lem:S2.4} for any fixed $\vartheta>0$.
\end{lemma}

\begin{proof}
	When $p\mid n$, write $n=pk$ so $k\asymp X/p$. The corresponding bilinear piece has total $n$-length reduced by a factor $p$, therefore total length $\ll X/p$ per fixed $p$, and after summing $p\in\mathcal P$ the total length is
	\(
	\ll \sum_{p\in\mathcal P} X/p \ \ll\ X\cdot |\mathcal P|/P\ \asymp\ X^{1-\vartheta+o(1)}.
	\)
	Applying Kuznetsov (with the same test $h_Q$ and the same level $q$) to this shorter sum and using the large-sieve/Kuznetsov trivial bound (or Lemma~\ref{lem:delta-second-moment-fullyrigid} with $P$ replaced by $1$) yields
	\(
	\ll_\varepsilon (Q^2+X)^{1+\varepsilon}\,X^{-\vartheta+o(1)}.
	\)
	Because there are at most $O(|\mathcal P|)$ such tails and each carries an extra $1/|\mathcal P|$ from amplifier normalization when comparing to $\sum|S|^2$ (as in the main argument), the net contribution to $\sum_{q,\chi,f}|S|^2$ is
	\(
	\ll_\varepsilon (Q^2+X)^{1+\varepsilon}\,|\mathcal P|\,X^{-1-\vartheta+o(1)}.
	\)
	In particular this is $o\!\big((Q^2+X)^{1-\delta}\,|\mathcal P|^{\,2-\delta}\big)$ for any fixed $\delta>0$ once $\vartheta>0$ is fixed, since the extra factor $X^{-1/2}$ (and a fortiori $X^{-1-\vartheta}$) dominates any $X^{\varepsilon}$ losses from dyadics.
\end{proof}

\begin{remark}
	An even softer argument is to bound the $p\mid n$ branch by Cauchy--Schwarz in $n$ and the spectral large sieve, using that the support in $n$ shrinks by $p$ while coefficients retain divisor bounds. Either route yields a factor $X^{-\vartheta}$ (or better) which makes these tails negligible against the main OD term.
\end{remark}


\begin{lemma}[Uniform kernel localization and derivatives]\label{lem:kernel-localization}
	Let $q\ge 1$ and let $h\in C_c^\infty([-2,2])$ be even with $h(0)=1$. For $Q\ge 1$ set $h_Q(t):=h(t/Q)$. Let $\mathcal W_q^{(*)}(z)$ denote the Kuznetsov/Bessel kernels (holomorphic, Maaß, Eisenstein) on $\Gamma_0(q)$ associated with test $h_Q$. Then for every $A,j\ge 0$,
	\[
		\mathcal W_q^{(*)}(z)\ \ll_A \Big(1+\frac{z}{Q}\Big)^{-A},\qquad
		z^j\,\partial_z^{\,j}\mathcal W_q^{(*)}(z)\ \ll_{A,j}\Big(1+\frac{z}{Q}\Big)^{-A},
	\]
	uniformly in $q$ and $z>0$. Consequently, in Kuznetsov the Kloosterman modulus $c$ is restricted to $c\asymp C:=X^{1/2}/Q$ up to tails $O_A(X^{-A})$ after inserting $z=4\pi\sqrt{mn}/c$ with $m,n\asymp X$.
\end{lemma}

\begin{proof}
	Write the Maa\ss kernel as the Hankel transform
	\[
		\mathcal W_q^{\mathrm{Maa\text{\ss}}}(z) = \frac{i}{\pi}\int_{-\infty}^{\infty} h_Q(t)\,\tanh(\pi t)\,J_{2it}(z)\,t\,dt,
	\]
	and similarly for the holomorphic/Eisenstein kernels (with $J_{k-1}$ or $K_{2it}$ where appropriate). Since $h_Q(t)=h(t/Q)$ is $C_c^\infty$ supported on $|t|\le 2Q$, repeated integration by parts against the oscillatory factor in the Schl\"afli integral for $J_\nu$ (or via the Mellin-Barnes representation) gives, for every $A\ge 0$,
	\[
		\mathcal W_q^{(*)}(z)=O_A\!\Big(\Big(1+\frac{z}{Q}\Big)^{-A}\Big),
	\]
	with identical bounds for $z^j\partial_z^{\,j}\mathcal W_q^{(*)}(z)$ because each $z$-derivative corresponds to inserting a polynomial in $\nu$ under the transform, still controlled by the compact support of $h_Q$ and the same integration-by-parts argument. The bounds are uniform in $q$ since the level only constraints $c\equiv 0\pmod q$ on the geometric side and does not enter the kernel formula.
	Finally, with $z=4\pi\sqrt{mn}/c$ and $m,n\asymp X$, the decay forces $z\asymp Q$, i.e. $c\asymp X^{1/2}/Q$, while the tails contribute $O_A(X^{-A})$ after summing over $c$.
\end{proof}

\subsection{Amplifier bookkeeping and exponent optimization (full details)}

Recall the setup: $X\ge 3$, $0<\kappa<\tfrac14$, $Q\le X^{1/2-\kappa}$, a dyadic $\mathcal Q\subset[Q,2Q]$,
and primes $\mathcal P=\{p\in[P,2P]\}$ with $P=X^\vartheta$, $0<\vartheta<\tfrac16-\kappa$.
Let $|\alpha_p|\le 1$ and define the amplifier $A_f=\sum_{p\in\mathcal P}\alpha_p\,\lambda_f(p)$.
For each $q\in\mathcal Q$, sum over primitive $\chi\pmod q$ and an orthonormal Hecke basis $f$
(holomorphic and Maaß, including oldforms, plus the Eisenstein spectrum via Kuznetsov).

Set
\[
	S_{q,\chi,f}:=\sum_{n\asymp X}\alpha_n\,\lambda_f(n)\chi(n),
\]
with Type-III coefficients $\alpha_n$ supported on $n\asymp X$,
$|\alpha_n|\ll_\varepsilon \tau(n)^C$, and smooth weight of width $X^{1+o(1)}$.
We aim to show
\begin{equation}\label{eq:TypeIIIgoal}
	\sum_{q\in\mathcal Q}\ \sum_{\chi\ (\mathrm{mod}\ q)}\ \sum_f
	\left|\sum_{n\asymp X}\alpha_n\,\lambda_f(n)\chi(n)\right|^2
	\ \ll_\varepsilon\ (Q^2+X)^{1-\delta}\,X^{\varepsilon}
\end{equation}
with some fixed $\delta>0$. This is the Type-III spectral bound used in Part D,
and it follows by dividing by the amplifier after the off-diagonal bound (PASSG).

\paragraph{Step 1: Balanced amplifier domination.}
Let $\varepsilon_p\in\{\pm 1\}$ be signs with $\sum_{p\in\mathcal P}\varepsilon_p=0$ (Appendix A.7).
Set $A_f=\sum_{p\in\mathcal P}\varepsilon_p\,\lambda_f(p)$.
By Cauchy-Schwarz in $(p,p')$ and $\sum\varepsilon_p^2=|\mathcal P|$, we have the standard domination
\begin{equation}\label{eq:amp-dom}
	\sum_{q,\chi,f}|S_{q,\chi,f}|^2
	\ \le\ \frac{1}{|\mathcal P|^2}\sum_{q,\chi,f}|A_f\,S_{q,\chi,f}|^2.
\end{equation}
(Here and below, $\sum_{q,\chi,f}$ abbreviates $\sum_{q\in\mathcal Q}\sum_{\chi\ (\mathrm{mod}\ q)}\sum_f$.)

\paragraph{Step 2: Hecke linearization and extraction of short prime shifts.}
Expand
\[
	|A_f S_{q,\chi,f}|^2
	=\sum_{p_1,p_2\in\mathcal P}\varepsilon_{p_1}\varepsilon_{p_2}
	\sum_{m,n\asymp X}\alpha_m\overline{\alpha_n}\,
	\lambda_f(p_1)\lambda_f(m)\,\overline{\lambda_f(p_2)\lambda_f(n)}\,\chi(m)\overline{\chi(n)}.
\]
Use the Hecke relation $\lambda_f(p)\lambda_f(n)=\lambda_f(pn)-\mathbf 1_{p\mid n}\lambda_f(n/p)$.
The terms with $p\mid n$ (and similarly $p\mid m$) are supported on a thinner set and are handled by the same (or stronger) bounds; we suppress them in notation.
Thus, after linearization,
\[
	|A_f S_{q,\chi,f}|^2
	= \sum_{p_1\ne p_2}\varepsilon_{p_1}\varepsilon_{p_2}
	\!\!\sum_{m,n\asymp X}\!\alpha_m\overline{\alpha_n}\,
	\lambda_f(p_1 m)\,\overline{\lambda_f(p_2 n)}\,\chi(m)\overline{\chi(n)}
	\ +\ \text{(diag/edge terms)}.
\]
Because $\sum_{p}\varepsilon_p=0$, the pure diagonal $p_1=p_2$ cancels (up to boundary terms absorbed later by $X^\varepsilon$).

\paragraph{Step 3: Kuznetsov with test $h_Q$ and kernel localization.}
Sum over $f$ and (orthogonally) over $\chi$ modulo $q$.
Applying Kuznetsov (Lemma~\ref{lem:kuznetsov-uniform-kernel}) with test $h_Q(t)=h(t/Q)$ and using Lemma~\ref{lem:kernel-localization},
the off-diagonal (OD) contribution can be written in the geometric form
\[
	\mathrm{OD}
	=\sum_{q\in\mathcal Q}\ \sum_{\substack{c\equiv 0\ (q)}} \frac{1}{c}
	\sum_{\substack{p_1,p_2\in\mathcal P\\ p_1\ne p_2}}\varepsilon_{p_1}\varepsilon_{p_2}
	\sum_{m,n\asymp X}\alpha_m\overline{\alpha_n}\,
	S(p_1 m, p_2 n;c)\ \mathcal W_q\!\left(\frac{4\pi\sqrt{p_1 m\cdot p_2 n}}{c}\right).
\]
Here $\mathcal W_q$ denotes any of the Bessel kernels (holomorphic, Maa\ss, Eisenstein).
By Lemma~\ref{lem:kernel-localization}, the kernel decay localizes the Kloosterman modulus to
$c\asymp C:=X^{1/2}/Q$ up to $O_A(X^{-A})$ tails; write $c=qr$ with $r\asymp R:=X^{1/2}/Q^2$.
Moreover, by Cauchy--Schwarz in $n$ together with the smooth dyadic partition (absorbing divisor-bounded coefficients into the weight), it suffices to treat the balanced same-variable case; we may reduce to sums with $n=m$ at the cost of a factor $X^{\varepsilon}$. This yields the $m$-only model used below.

\subsubsection{Insertion for Lemma~\ref{lem:S2.4}: using the \textDelta-second moment and optimizing exponents}

\paragraph{From amplifier+Kuznetsov to a $\Delta$-family.}
After opening $|A_f S_{q,\chi,f}|^2$, linearizing Hecke, and applying Kuznetsov with test $h_Q$, the off-diagonal (OD) is
\[
	\mathrm{OD}
	=\sum_{q\in\mathcal Q}\ \sum_{r\asymp R}\frac{1}{qr}
	\!\!\sum_{\substack{p_1\ne p_2\in\mathcal P}}\!\!\varepsilon_{p_1}\varepsilon_{p_2}\!
	\sum_{m\asymp X}\alpha_m\overline{\alpha_{m}}\,
	S(p_1m,p_2m;qr)\,\mathcal W_{q}\!\Big(\tfrac{4\pi\sqrt{p_1m\cdot p_2m}}{qr}\Big)\,+\,\mathcal E,
\]
where $c=qr$, $r\asymp R:=X^{1/2}/Q^{2}$ due to Lemma~\ref{lem:kuznetsov-uniform-kernel},
and $\mathcal E$ collects $O_A(X^{-A})$ kernel tails and the $p\mid n$ Hecke tails
(bounded by Lemma~\ref{lem:hecke-tails}).


Set $\Delta=p_1-p_2$, and absorb $\mathcal W_q$ into a smooth weight $W_{q,r}(m,\Delta)$ with the derivative bounds of Lemma~\ref{lem:delta-second-moment-fullyrigid}. Grouping by $\Delta$ and letting $\nu(\Delta)$ be the number of prime pairs with difference $\Delta$,
\[
	\mathrm{OD}\ \ll\ \sum_{q\in\mathcal Q}\ \sum_{r\asymp R}\frac{1}{qr}\sum_{\Delta\ne 0}\nu(\Delta)\,
	\Big|\Sigma_{q,r}(\Delta)\Big| \,+\, O_A(X^{-A}),\qquad
	\Sigma_{q,r}(\Delta):=\sum_{m\asymp X} S(m,m+\Delta;qr)\,W_{q,r}(m,\Delta).
\]

\paragraph{Apply the $\Delta$-second moment (Lemma~\ref{lem:delta-second-moment-fullyrigid}).}
By Cauchy-Schwarz in $\Delta$ and Lemma~\ref{lem:delta-second-moment-fullyrigid},
\[
	\sum_{|\Delta|\le P}\nu(\Delta)\,|\Sigma_{q,r}(\Delta)|
	\ \le\ |\mathcal P|^{1/2}\Big(\sum_{|\Delta|\le P}\nu(\Delta)\Big)^{1/2}
	\Big(\sum_{|\Delta|\le P}|\Sigma_{q,r}(\Delta)|^2\Big)^{1/2}
	\ \ll_\varepsilon\ |\mathcal P|\,(P+qr)^{1/2}(qr)^{1/2+\varepsilon}X^{1/2+\varepsilon}.
\]
Therefore
\[
	\sum_{r\asymp R}\frac{1}{qr}\sum_{\Delta}\nu(\Delta)\,|\Sigma_{q,r}(\Delta)|
	\ \ll_\varepsilon\ |\mathcal P|\,q^{-1/2+\varepsilon}X^{1/2+\varepsilon}\!\!\sum_{r\asymp R} r^{-1/2+\varepsilon}(P+qr)^{1/2}.
\]
Since $qr\asymp C:=X^{1/2}/Q$, we have $(P+qr)^{1/2}\asymp (P+X^{1/2}/Q)^{1/2}$ and $\sum_{r\asymp R} r^{-1/2+\varepsilon}\asymp R^{1/2+\varepsilon}$. Using $q^{-1/2}R^{1/2}\asymp Q^{-1}$,
\[
	\sum_{r}\cdots\ \ll_\varepsilon\ |\mathcal P|\,Q^{1+\varepsilon}\,(P+X^{1/2}/Q)^{1/2}.
\]
Summing over $q\in\mathcal Q$ (there are $\asymp Q$ moduli) yields
\begin{equation}\label{eq:OD-final}
	\mathrm{OD}\ \ll_\varepsilon\ |\mathcal P|\,Q^{2+\varepsilon}\,(P+X^{1/2}/Q)^{1/2}.
\end{equation}

\paragraph{Divide out the amplifier and optimize $(\vartheta,\kappa)$.}
From the amplifier domination
\(
\sum_{q,\chi,f}|S_{q,\chi,f}|^2\ \le\ |\mathcal P|^{-2}\mathrm{OD},
\)
and $|\mathcal P|\asymp P/\log P=X^{\vartheta+o(1)}$ with $P=X^\vartheta$, we get two regimes:

\smallskip
\noindent\emph{(A) If } $X^{1/2}/Q\le P$ (i.e. $X^{1/2-\vartheta}\le Q$):
\[
	\sum_{q,\chi,f}|S|^2\ \ll_\varepsilon\ \frac{Q^{2+\varepsilon}P^{1/2}}{|\mathcal P|}\ \asymp\ Q^{2+\varepsilon}\,X^{-\vartheta/2+o(1)}
	\ \le\ X^{1-2\kappa-\vartheta/2+\varepsilon}.
\]

\noindent\emph{(B) If } $X^{1/2}/Q\ge P$:
\[
	\sum_{q,\chi,f}|S|^2\ \ll_\varepsilon\ \frac{Q^{2+\varepsilon}(X^{1/2}/Q)^{1/2}}{|\mathcal P|}
	\ \asymp\ Q^{3/2+\varepsilon}\,X^{1/4-\vartheta+o(1)}
	\ \le\ X^{1-\vartheta-\tfrac{3}{2}\kappa+\varepsilon}.
\]

Since $Q\le X^{1/2-\kappa}$, both cases give
\[
	\sum_{q,\chi,f}|S|^2\ \ll\ X^{\,1-\delta+\varepsilon}
	\qquad\text{with}\qquad
	\delta\ \le\ \min\Big\{2\kappa+\tfrac{\vartheta}{2},\ \vartheta+\tfrac{3}{2}\kappa\Big\}.
\]
To ensure robust savings across dyadics and spectral pieces, fix
\[
	\boxed{\ \delta=\frac{1}{1000}\min\!\Big\{\kappa,\ \tfrac12-3\vartheta\Big\}\ },
\]
valid when $\vartheta<\tfrac16-\kappa$. Since $Q^2\le X$, we can rewrite $X^{1-\delta}\asymp(Q^2+X)^{1-\delta}$, giving the form claimed in Lemma~\ref{lem:S2.4}.


\begin{lemma}[Prime pair combinatorics]\label{lem:prime-pair}
	Let $\nu(\Delta)=\#\{(p_1,p_2)\in\mathcal P^2: p_1-p_2=\Delta,\ p_1\ne p_2\}$. Then
	\(
	\sum_{|\Delta|\le P}\nu(\Delta)\ \asymp\ |\mathcal P|^2
	\)
	and $\nu(\Delta)\le |\mathcal P|$ trivially.
\end{lemma}

\begin{proof}
	Trivial counting: $\sum_{\Delta}\nu(\Delta)=\#\{(p_1,p_2)\in\mathcal P^2: p_1\ne p_2\}=|\mathcal P|(|\mathcal P|-1)$.
\end{proof}

\begin{lemma}[Hecke linearization]\label{lem:hecke-linearization}
	For Hecke eigenvalues $\lambda_f(n)$,
	\[
		\lambda_f(p)\lambda_f(n)=
		\begin{cases}
			\lambda_f(pn)                & (p\nmid n), \\
			\lambda_f(pn)-\lambda_f(n/p) & (p\mid n),
		\end{cases}
	\]
	and the $n/p$-tail is supported on $p\mid n$ and is treated identically (or better) than the $pn$-branch under the smooth dyadic partition.
\end{lemma}

\begin{lemma}[Oldforms and Eisenstein]\label{lem:old-eis}
	Kuznetsov on $\Gamma_0(q)$ with test $h_Q$ yields the same geometric structure for holomorphic, Maaß (new+old), and Eisenstein parts, each with kernels obeying Lemma~\ref{lem:kernel-localization}. Thus all families are uniform in the estimates below.
\end{lemma}

\begin{lemma}[Amplifier]\label{lem:amplifier}
	Let $A_f:=\sum_{p\in\mathcal P}\alpha_p\,\lambda_f(p)$ with $|\alpha_p|\le 1$. For any complex numbers $S_{q,\chi,f}$,
	\[
		\sum_{q\le Q}\sum_{\chi}\sum_f |A_f\,S_{q,\chi,f}|^2
		\ =\ \mathrm{Diag}\ +\ \mathrm{OD},
	\]
	where $\mathrm{Diag}$ is the $p_1=p_2$ contribution and $\mathrm{OD}$ collects $p_1\ne p_2$ terms. After Hecke linearization and Kuznetsov, $\mathrm{OD}$ has the Kloosterman-Bessel shape treated below.
\end{lemma}

\begin{lemma}[Prime-averaged short-shift gain]\label{lem:S2.4}
	Let $X\ge3$, $0<\kappa<\tfrac14$, and $Q\le X^{1/2-\kappa}$. Let $\mathcal Q\subset[Q,2Q]$ be a dyadic set of moduli.
	Let $\mathcal P=\{p\in[P,2P]\text{ prime}\}$ with $P=X^\vartheta$, where $0<\vartheta<\tfrac16-\kappa$, and let $\{\varepsilon_p\}_{p\in\mathcal P}\subset\{\pm1\}$ satisfy $\sum_{p\in\mathcal P}\varepsilon_p=0$. For each $q\in\mathcal Q$, each primitive character $\chi\,(\bmod q)$, and each Hecke eigenform $f$ on $\Gamma_0(q)$ (holomorphic or Maa\ss, including oldforms; Eisenstein included via Kuznetsov), form
	\[
		S_{q,\chi,f} \ :=\ \sum_{n\asymp X}\alpha_n\,\lambda_f(n)\chi(n),
		\qquad
		|\alpha_n|\ \ll_\varepsilon\ \tau(n)^C,\ \ \text{$\alpha_n$ smooth on $n\asymp X$}.
	\]
	Define the prime amplifier $A_f:=\sum_{p\in\mathcal P}\varepsilon_p\lambda_f(p)$ and let $\mathrm{OD}$ denote the off-diagonal contribution in $\sum_{q\in\mathcal Q}\sum_{\chi}\sum_f |A_f S_{q,\chi,f}|^2$ after Hecke linearization and Kuznetsov (i.e. all terms with distinct primes $p_1\ne p_2$).
	Then for some fixed $\delta>0$ (explicit below) and every $\varepsilon>0$,
	\[
		\mathrm{OD}\ \ll_{\varepsilon,C}\ (Q^2+X)^{\,1-\delta}\,|\mathcal P|^{\,2-\delta}\,X^{\varepsilon},
		\qquad
		\delta=\tfrac1{1000}\min\Big\{\kappa,\ \tfrac12-3\vartheta\Big\}.
	\]
	Consequently,
	\[
		\sum_{q\in\mathcal Q}\sum_{\chi}\sum_f \Big|\sum_{n\asymp X}\alpha_n\,\lambda_f(n)\chi(n)\Big|^2
		\ \ll_{\varepsilon,C}\ (Q^2+X)^{\,1-\delta}\,X^{\varepsilon}.
	\]
	The bounds are uniform across holomorphic, Maa\ss\ (new+old), and Eisenstein spectra.
\end{lemma}

\begin{proof}
	\emph{Step 1: Amplifier domination and Hecke linearization.}
	By Cauchy-Schwarz in the amplifier and $\sum_p\varepsilon_p^2=|\mathcal P|$,
	\[
		\sum_{q,\chi,f} |S_{q,\chi,f}|^2 \ \le\ \frac{1}{|\mathcal P|^2}\sum_{q,\chi,f}|A_f S_{q,\chi,f}|^2.
	\]
	Open $|A_f S|^2$ and use $\lambda_f(p)\lambda_f(n)=\lambda_f(pn)-\mathbf 1_{p\mid n}\lambda_f(n/p)$.
	The branches with $p\mid n$ (or $p\mid m$ on the conjugate side) shrink the $n$-support by a factor $p$; a routine large-sieve/Kuznetsov bound on these “Hecke tails” gives
	\[
		\ll_\varepsilon (Q^2+X)^{1+\varepsilon}\,|\mathcal P|\,X^{-1/2+\varepsilon},
	\]
	which is negligible compared to the target bound (once we divide by $|\mathcal P|^2$ at the end). Hence we discard them and retain only the $pn$ branches. Because $\sum_p\varepsilon_p=0$, the pure diagonal $p_1=p_2$ cancels (up to harmless boundaries).

	\emph{Step 2: Kuznetsov and kernel localization.}
	Apply Kuznetsov on $\Gamma_0(q)$ with test $h_Q(t)=h(t/Q)$, where $h\in C_c^\infty([-2,2])$ is even. By the level-uniform kernel bounds (Lemma~\ref{lem:kuznetsov-uniform-kernel}), the Bessel kernels localize the Kloosterman modulus to $c\asymp C:=X^{1/2}/Q$, up to $O_A(X^{-A})$ tails. Writing $c=qr$ we have $r\asymp R:=X^{1/2}/Q^2$. The off-diagonal hence takes the geometric shape
	\[
		\mathrm{OD}\ =\ \sum_{q\in\mathcal Q}\ \sum_{r\asymp R}\frac{1}{qr}
		\sum_{\substack{p_1,p_2\in\mathcal P\\ p_1\ne p_2}}\varepsilon_{p_1}\varepsilon_{p_2}
		\sum_{m\asymp X}\alpha_m\overline{\alpha_m}\,
		S(p_1 m, p_2 m; qr)\,
		\mathcal W_q\!\Big(\tfrac{4\pi\sqrt{p_1 m\cdot p_2 m}}{qr}\Big)
		\ +\ O_A(X^{-A}),
	\]
	where we have reduced to $n=m$ by Cauchy-Schwarz and smoothing (absorbed in $X^\varepsilon$), and $\mathcal W_q$ is any of the kernels in Lemma~\ref{lem:kuznetsov-uniform-kernel}. Absorb $\mathcal W_q$ and the coefficient weights into a smooth $W_{q,r}(m,\Delta)$ with the derivative bounds required by Lemma~\ref{lem:delta-second-moment-fullyrigid}, where $\Delta:=p_1-p_2$.

	\emph{Step 3: Group by short prime shift and apply the $\Delta$-second moment.}
	Let $\nu(\Delta)=\#\{(p_1,p_2)\in\mathcal P^2: p_1-p_2=\Delta,\ p_1\ne p_2\}$.
	Grouping by $\Delta$ and using $|\varepsilon_{p_i}|\le1$,
	\[
		\mathrm{OD}\ \ll\ \sum_{q\in\mathcal Q}\ \sum_{r\asymp R}\frac{1}{qr}
		\sum_{\Delta\ne 0}\nu(\Delta)\,\Big|\Sigma_{q,r}(\Delta)\Big|
		\ +\ O_A(X^{-A}),
		\qquad
		\Sigma_{q,r}(\Delta):=\sum_{m\asymp X} S(m,m+\Delta;qr)\,W_{q,r}(m,\Delta).
	\]
	By Cauchy-Schwarz in $\Delta$ and $\sum_{|\Delta|\le P}\nu(\Delta)\asymp|\mathcal P|^2$ with $P\asymp X^\vartheta$,
	\[
		\sum_{|\Delta|\le P}\nu(\Delta)\,|\Sigma_{q,r}(\Delta)|
		\ \le\ |\mathcal P|\,\Big(\sum_{|\Delta|\le P}|\Sigma_{q,r}(\Delta)|^2\Big)^{1/2}.
	\]
	Invoke the fully uniform $\Delta$-second-moment lemma (Lemma~\ref{lem:delta-second-moment-fullyrigid}) to get
	\[
		\sum_{|\Delta|\le P}|\Sigma_{q,r}(\Delta)|^2
		\ \ll_\varepsilon\ (P+qr)\,(qr)^{1+2\varepsilon}\,X^{1+2\varepsilon}.
	\]
	Therefore
	\[
		\sum_{r\asymp R}\frac{1}{qr}\sum_{\Delta}\nu(\Delta)\,|\Sigma_{q,r}(\Delta)|
		\ \ll_\varepsilon\ |\mathcal P|\,q^{-1/2+\varepsilon}\,X^{1/2+\varepsilon}
		\sum_{r\asymp R} r^{-1/2+\varepsilon}\,(P+qr)^{1/2}.
	\]
	Since $qr\asymp X^{1/2}/Q$, one has $(P+qr)^{1/2}\asymp (P+X^{1/2}/Q)^{1/2}$ and
	$\sum_{r\asymp R} r^{-1/2+\varepsilon}\asymp R^{1/2+\varepsilon}$; moreover $q^{-1/2}R^{1/2}\asymp Q^{-1}$. Hence
	\[
		\sum_{r}\cdots\ \ll_\varepsilon\ |\mathcal P|\,Q^{1+\varepsilon}\,(P+X^{1/2}/Q)^{1/2}.
	\]
	Summing over $q\in\mathcal Q$ (there are $\asymp Q$ moduli) yields
	\begin{equation}\label{eq:OD-master}
		\mathrm{OD}\ \ll_\varepsilon\ |\mathcal P|\,Q^{2+\varepsilon}\,(P+X^{1/2}/Q)^{1/2}.
	\end{equation}

	\emph{Step 4: Optimize parameters and extract $\delta$.}
	Recall $P=X^\vartheta$ and $Q\le X^{1/2-\kappa}$. Consider the two regimes:

	\smallskip
	\noindent(A) If $X^{1/2}/Q\le P$ (i.e. $X^{1/2-\vartheta}\le Q$), then from \eqref{eq:OD-master},
	\[
		\mathrm{OD}\ \ll\ |\mathcal P|\,Q^{2+\varepsilon}\,P^{1/2}
		\ \asymp\ Q^{2+\varepsilon}\,X^{\vartheta/2}\,|\mathcal P|.
	\]
	(B) If $X^{1/2}/Q\ge P$, then
	\[
		\mathrm{OD}\ \ll\ |\mathcal P|\,Q^{2+\varepsilon}\,(X^{1/2}/Q)^{1/2}
		\ =\ Q^{3/2+\varepsilon}\,X^{1/4}\,|\mathcal P|.
	\]

	In either case use $Q\le X^{1/2-\kappa}$ and $|\mathcal P|\asymp P/\log P=X^{\vartheta+o(1)}$ to obtain
	\[
		\mathrm{OD}\ \ll\ X^{\,1-\delta+\varepsilon}\,|\mathcal P|^{\,2-\delta}
		\qquad\text{with}\qquad
		\delta\ \le\ \min\Big\{2\kappa+\tfrac{\vartheta}{2},\ \vartheta+\tfrac{3}{2}\kappa\Big\}.
	\]
	Fix
	\[
		\delta\ :=\ \frac{1}{1000}\min\Big\{\kappa,\ \tfrac12-3\vartheta\Big\},
	\]
	which is positive provided $\vartheta<\tfrac16-\kappa$. Since $Q^2\le X$, we may rewrite $X^{1-\delta}\asymp (Q^2+X)^{1-\delta}$, giving the stated $\mathrm{OD}\ll (Q^2+X)^{1-\delta}|\mathcal P|^{2-\delta}X^\varepsilon$.

	\emph{Step 5: Divide out the amplifier.}
	By the amplifier domination at the start,
	\[
		\sum_{q,\chi,f} |S_{q,\chi,f}|^2 \ \le\ \frac{1}{|\mathcal P|^2}\,\mathrm{OD}
		\ \ll\ (Q^2+X)^{\,1-\delta}\,|\mathcal P|^{-\delta}\,X^\varepsilon.
	\]
	Taking any fixed $\vartheta>0$ allowed above makes $|\mathcal P|=X^{\vartheta+o(1)}$, and we absorb $|\mathcal P|^{-\delta}$ into $X^\varepsilon$ by shrinking $\varepsilon$. This yields
	\[
		\sum_{q\in\mathcal Q}\sum_{\chi}\sum_f \Big|\sum_{n\asymp X}\alpha_n\lambda_f(n)\chi(n)\Big|^2
		\ \ll\ (Q^2+X)^{\,1-\delta}\,X^\varepsilon,
	\]
	uniformly across all spectral pieces, completing the proof.
\end{proof}

\begin{remark}[Parameters \& ranges for Lemma~\ref{lem:S2.4}]
	Fix any $0<\kappa<\tfrac14$ and choose $\vartheta$ with
	\[
		0<\vartheta<\tfrac16-\kappa .
	\]
	Take $Q\le X^{1/2-\kappa}$ and $P=X^\vartheta$ (so $|\mathcal P|\asymp P/\log P$).
	Then Lemma~\ref{lem:S2.4} holds with
	\[
		\delta=\frac{1}{1000}\min\!\Big\{\kappa,\ \tfrac12-3\vartheta\Big\}\ >\ 0 .
	\]
	In particular, the choice
	\[
		\kappa=10^{-3},\qquad \vartheta=\frac{\kappa}{8}
	\]
	gives $\delta\ge 5\times 10^{-7}$, which is uniform across all dyadic $X$ and all spectral pieces (holomorphic, Maa\ss, and Eisenstein, including oldforms). The constants in the bound
	\[
		\sum_{q\in\mathcal Q}\sum_{\chi}\sum_f \Big|\sum_{n\asymp X}\alpha_n\,\lambda_f(n)\chi(n)\Big|^2
		\ \ll\ (Q^2+X)^{1-\delta}\,X^\varepsilon
	\]
	depend at most on $\varepsilon$, on finitely many derivatives of the fixed test $h$, and on the exponent $C$ in the divisor-type bound $|\alpha_n|\ll_\varepsilon \tau(n)^C$.
\end{remark}

\section{Type~III Analysis: Prime-Averaged Short-Shift Gain}

\begin{proposition}[Type-III spectral second moment]\label{prop:typeIII}
	Let $(\alpha_n)$ be a smooth Type-III coefficient sequence supported on $n\asymp X$, with divisor-type bounds $|\alpha_n|\ll_\varepsilon \tau(n)^C$ and smooth weight of width $X^{1+o(1)}$.
	Let $Q\le X^{1/2-\kappa}$ with some fixed $0<\kappa<1/4$. Then, for some fixed $\delta>0$ depending only on~$\kappa$,
	\[
		\sum_{q\le Q}\ \sum_{\chi\bmod q}\ \sum_{f}
		\Bigg|\sum_{n\asymp X}\alpha_n\,\lambda_f(n)\chi(n)\Bigg|^2
		\ \ \ll_{\varepsilon,C}\ \ (Q^2+X)^{\,1-\delta}\,X^{\varepsilon}.
	\]
\end{proposition}

\begin{proof}
	Fix a prime amplifier $\mathcal P=\{p\in[P,2P]\}$ with $P=X^\vartheta$, $\varepsilon_p\in\{\pm1\}$ balanced so that $\sum_p\varepsilon_p=0$.
	Define $A_f=\sum_{p\in\mathcal P}\varepsilon_p\,\lambda_f(p)$, and set
	$S_{q,\chi,f}=\sum_{n\asymp X}\alpha_n\lambda_f(n)\chi(n)$.
	As in the balanced-amplifier method,
	\[
		\sum_{q\le Q}\sum_{\chi}\sum_f |S_{q,\chi,f}|^2
		\ \le\ \frac{1}{|\mathcal P|^2}\sum_{q\le Q}\sum_{\chi}\sum_f |A_f\,S_{q,\chi,f}|^2.
	\]

	Opening the amplifier and applying Kuznetsov (including oldforms and Eisenstein) reduces the off--diagonal to correlations of the form
	\[
		\mathrm{OD}\ :=\ \sum_{q\sim Q}\ \sum_{r\asymp R}\frac{1}{qr}\sum_{\Delta\ne0}\nu(\Delta)\,|\Sigma_{q,r}(\Delta)|,
	\]
	with $\nu(\Delta)$ the prime-pair counts and
	$\Sigma_{q,r}(\Delta)=\sum_{m\asymp X} S(m,m+\Delta;qr)\,W_{q,r}(m,\Delta)$.
	Here $c=qr\asymp X^{1/2}/Q$, and $W_{q,r}$ are smooth weights supported on $m\asymp X$, $|\Delta|\le P$.

	By Lemma~\ref{lem:delta-second-moment-fullyrigid},
	\[
		\sum_{|\Delta|\le P}|\Sigma_{q,r}(\Delta)|^2
		\ \ll_\varepsilon\ (P+qr)\,(qr)^{1+2\varepsilon}\,X^{1+2\varepsilon}.
	\]
	Cauchy--Schwarz and $\sum\nu(\Delta)\asymp |\mathcal P|^2$ give
	\[
		\sum_{|\Delta|\le P}\nu(\Delta)\,|\Sigma_{q,r}(\Delta)|
		\ \ll_\varepsilon\
		|\mathcal P|\,(P+qr)^{1/2}\,(qr)^{1/2+\varepsilon}\,X^{1/2+\varepsilon}.
	\]
	Summing over $q\sim Q$, $r\asymp R$ yields
	\[
		\mathrm{OD}\ \ll_\varepsilon\
		|\mathcal P|\,X^{3/4+\varepsilon}\,Q^{-1/2}\,(P+X^{1/2}/Q)^{1/2}.
	\]

	Dividing by $|\mathcal P|^2$,
	\[
		\sum_{q\le Q}\sum_{\chi}\sum_f |S_{q,\chi,f}|^2
		\ \ll_\varepsilon\ \frac{X^{3/4+\varepsilon}}{P}\,Q^{-1/2}\,(P+X^{1/2}/Q)^{1/2}.
	\]

	Finally, choose $Q=X^{1/2-\kappa}$, $P=X^\vartheta$ with $0<\vartheta<\kappa$.
	A short case analysis shows that this is $\ll X^{1-\delta+\varepsilon}$ with $\delta\ge\min\{\tfrac12-\tfrac{\kappa}{2},\,\tfrac{\vartheta}{2},\,\kappa-\vartheta\}>0$.
	Since $Q^2\le X$, we rewrite $X^{1-\delta}$ as $(Q^2+X)^{1-\delta}$.
	This completes the proof.
\end{proof}



\part{Assembly}
\section{Dyadic Decomposition (final)}

\subsection{Statement}

Let $S(\alpha)=\sum_{n\le N}\Lambda(n)\,w(n)\,e(\alpha n)$ with a fixed smooth weight $w$ supported on $[N/2,2N]$ and let $B(\alpha)$ be the parity-blind majorant from Part A. For the minor arcs $\mathfrak m$ defined with denominator cutoff $Q=N^{1/2-\varepsilon}$, assume the analytic inputs:

\begin{itemize}
	\item \textbf{(I/II)}: For any smooth Type-I/II coefficient structure $\{c_n\}$ with divisor bounds (arising from Vaughan/Heath-Brown), the second-moment Barban-Davenport-Halász-pretentious bound

	      \begin{equation}\label{eq:D1}
		      \sum_{q\le Q}\ \sum_{\chi\bmod q}\Big|\sum_{n\le N} c_n\,\lambda(n)\chi(n)\Big|^2
		      \ \ll\ \frac{NQ}{(\log N)^A}
	      \end{equation}
	      holds for each fixed $A>0$. (This is BVP2M and the “Route B Lemma” for the balanced ranges.)

	\item \textbf{(III)}: For every dyadic Type-III block $\sum_{n\asymp X}\alpha_n\,\lambda_f(n)\chi(n)$ produced after amplification and Kuznetsov, the prime-averaged off-diagonal is bounded by

	      \begin{equation}\label{eq:D2}\mathrm{OD}\ \ll\ (Q^2+X)^{1-\delta}\,|\mathcal P|^{\,2-\delta}\end{equation}

	      for some fixed $\delta>0$, uniformly for amplifier length $|\mathcal P|=X^\vartheta$ with $\vartheta=\vartheta(\delta)>0$, and with uniform control of oldforms/Eisenstein and Bessel kernels. (This is PASSG and its Type-III spectral corollary.)
\end{itemize}

Then, for any $\varepsilon>0$,

$$
	\int_{\mathfrak m}\big|S(\alpha)-B(\alpha)\big|^2\,d\alpha
	\ \ll\ \frac{N}{(\log N)^{3+\varepsilon}}.
$$

\subsection{Proof}

\paragraph{Step 1: Identity and dyadic model.}
Apply a 3-, 4-, or 5-fold Heath-Brown identity (any standard version suffices) to $\Lambda$ with cut parameters

$$
	U=N^{\mu},\quad V=N^{\nu},\quad W=N^{\omega},\qquad 0<\mu\le\nu\le\omega<1,
$$

chosen below. We write

$$
	S(\alpha)-B(\alpha)
	=\sum_{\text{HB terms }\mathcal T} \mathcal S_{\mathcal T}(\alpha),
$$

where each $\mathcal S_{\mathcal T}$ is a finite linear combination (with coefficients having $\ll_\epsilon n^\epsilon$ divisor bounds and smooth dyadic cutoffs) of exponential sums of one of the three structural types:

\begin{itemize}
	\item \textbf{Type I}: $\displaystyle \sum_{m\asymp M} a_m \sum_{n\asymp N/M} b_n\,e(\alpha mn)$ with $M\le U$ (or the dual small variable),
	\item \textbf{Type II}: balanced $\displaystyle \sum_{m\asymp M}\sum_{n\asymp N/M} a_m b_n\,e(\alpha mn)$ with $U\ll M\ll N/U$,
	\item \textbf{Type III}: ``ternary'' or highly factorized pieces with all variables in ranges $ \ll N^{1/3+o(1)}$, which, after the amplifier/Kuznetsov transition, become prime-averaged short-shift sums against automorphic coefficients.
\end{itemize}

All sums are partitioned into $\mathbf{O((\log N)^C)}$ dyadic blocks in all active variables for some fixed $C$.

\paragraph{Step 2: Minor-arc $L^2$ via large sieve on dyadics.}
Let $\mathfrak M(q,a)$ be the standard major arc around $a/q$ with width $\asymp (qQ)^{-1}$, and set $\mathfrak m=[0,1]\setminus \bigcup_{q\le Q}\bigcup_{(a,q)=1}\mathfrak M(q,a)$. On $\mathfrak m$ we use the standard large-sieve/dispersion reduction:

for suitable coefficients $c_n$ associated to the dyadic block $\mathcal T$. By opening the square and expanding in Dirichlet characters modulo $q$, \eqref{eq:D2} reduces to sums of the form

\begin{equation}\label{eq:D3}
	\sum_{q\le Q}\ \sum_{\chi\bmod q}
	\Big|\sum_{n\asymp X} c_n\,\lambda(n)\chi(n)\Big|^2
\end{equation}

or, in the Type-III case after the amplifier/Kuznetsov step, to a spectral second moment whose diagonal/off-diagonal split is controlled by \eqref{eq:D2}.

We now bound \eqref{eq:D3} block-wise and then sum the dyadics.


\subsection{Step 3: Type I/II dyadics}
Choose $U=N^{1/3}$ (any $\mu\in(1/4,1/2)$ is fine) so that all Type I/II ranges from the chosen Heath-Brown identity fall either in the “small-large” or “balanced” regimes. By the input (I/II), for any $A>0$,

$$
	\sum_{q\le Q}\sum_{\chi\bmod q}
	\Big|\sum_{n\le N} c_n\,\lambda(n)\chi(n)\Big|^2
	\ \ll\ \frac{NQ}{(\log N)^A}.
$$

Each Type I or Type II dyadic contributes $\ll NQ/(\log N)^A$. There are $\ll(\log N)^C$ such dyadics in total, so by taking $A\ge 3+C+10\varepsilon^{-1}$ we obtain

\begin{equation}\label{eq:D5}
	\sum_{\text{Type I/II dyadics}}
	\int_{\mathfrak m}\big|\mathcal S_{\mathcal T}(\alpha)\big|^2 d\alpha
	\ \ll\ \frac{N}{(\log N)^{3+\varepsilon}}.
\end{equation}

\subsection{Step 4: Type III dyadics}
Fix $V=W=N^{1/3}$ so that the residual blocks with all variables $\ll N^{1/3+o(1)}$ are designated Type III. For such a block, let its “outer scale” be $X\asymp N^\xi$ with $\xi\in(0,1)$ determined by the product of the active variables. After applying the amplifier of length $|\mathcal P|=X^\vartheta$ and Kuznetsov, we face a spectral second moment whose off-diagonal obeys \eqref{eq:D2}:

$$
	\mathrm{OD}\ \ll\ (Q^2+X)^{1-\delta}\,|\mathcal P|^{\,2-\delta}
	\ =\ (Q^2+X)^{1-\delta}\,X^{\vartheta(2-\delta)}.
$$

Take $\vartheta=\tfrac{\delta}{8}$ (any fixed small choice depending on $\delta$ works). Since $Q=N^{1/2-\varepsilon}$, we have $Q^2=N^{1-2\varepsilon}$. Two regimes:

\begin{itemize}
	\item If $X\le Q^2$ then $\mathrm{OD}\ll N^{(1-2\varepsilon)(1-\delta)}\,X^{\vartheta(2-\delta)}$.
	\item If $X\ge Q^2$ then $\mathrm{OD}\ll X^{1-\delta+\vartheta(2-\delta)}$.
\end{itemize}

In both cases there is a fixed saving $X^{-\eta}$ (or $N^{-\eta}$) for some $\eta=\eta(\delta,\vartheta,\varepsilon)>0$ against the trivial diagonal scale, after the standard dispersion normalization. Consequently each Type III dyadic contributes

\begin{equation}\label{eq:D6}\int_{\mathfrak m}\big|\mathcal S_{\mathcal T}(\alpha)\big|^2 d\alpha\ \ll\ \frac{N}{(\log N)^{A}}\,X^{-\eta}+\text{(diagonal)}.\end{equation}

The diagonal is controlled either by the amplifier normalization or by subtracting the parity-blind majorant $B(\alpha)$ (which removes the main term on $\mathfrak m$), leaving at most $\ll N/(\log N)^A$ per block. Summing \eqref{eq:D6} over the $\ll(\log N)^C$ Type-III dyadics and choosing $A$ large, we obtain

\begin{equation}\label{eq:D7}\sum_{\text{Type III dyadics}}\int_{\mathfrak m}\big|\mathcal S_{\mathcal T}(\alpha)\big|^2 d\alpha\ \ll\ \frac{N}{(\log N)^{3+\varepsilon}}.\end{equation}

\emph{Bookkeeping note.} The $X^{-\eta}$ saving is uniform in the dyadic location because $\delta>0$ is fixed and $\vartheta$ is chosen as a fixed fraction of $\delta$; any residual factors from Bessel kernels, oldforms, and Eisenstein are already absorbed in \eqref{eq:D2} by the uniform spectral analysis ensured in PASSG. The $q$-sum restriction $q\le Q$ matches the circle-method minor-arc decomposition, so no leakage arises.

---

\subsection{Step 5: Conclusion}
Adding \eqref{eq:D5} and \eqref{eq:D7} over all dyadics of all HB terms $\mathcal T$ yields

$$
	\int_{\mathfrak m}\big|S(\alpha)-B(\alpha)\big|^2 d\alpha
	\ \ll\ \frac{N}{(\log N)^{3+\varepsilon}},
$$

as claimed.
\subsection{Derivation of \eqref{eq:A1} from BVP2M and PASSGs}

\noindent\textbf{Scope.} In this subsection we \emph{derive} the minor-arc $L^2$ estimate
\[\int_{\mathfrak m}|S(\alpha)-B(\alpha)|^2\,d\alpha\ \ll\ \frac{N}{(\log N)^{3+\varepsilon}}\]
\begin{enumerate}[label=(\roman*)]
	\item \textbf{Type I/II second moment with parity} (BVP2M): for $Q\le N^{1/2}(\log N)^{-B}$,
	      \[
		      \sum_{q\le Q}\ \sum_{\chi\bmod q}\Big|\sum c_n\,\lambda(n)\chi(n)\Big|^2
		      \ \ll\ \frac{NQ}{(\log N)^A},
	      \]
	      uniformly for the Type I/II coefficient structures produced by the identity (divisor bounds, smooth weights).

	\item \textbf{Type III off-diagonal saving} (PASSG): after prime-length amplification and Kuznetsov,
	      \[
		      \mathrm{OD}\ \ll\ (Q^2+X)^{1-\delta}\,|\mathcal P|^{\,2-\delta}\,X^{\varepsilon}
	      \]
	      for some fixed $\delta>0$ (with $|\mathcal P|=X^\vartheta$, $0<\vartheta<\tfrac16-\kappa$), uniformly across spectral families.
\end{enumerate}

\medskip
\noindent\textbf{Large-sieve reduction on $\mathfrak m$.} For each Heath-Brown dyadic block $\mathcal T$, Gallagher’s/large-sieve minor-arc reduction (Lemma~\ref{lem:largesieve-minor}) yields
\[
	\int_{\mathfrak m}\big|\mathcal S_{\mathcal T}(\alpha)\big|^2\,d\alpha
	\ \ll\ \frac{1}{Q^2}\sum_{q\le Q}\sum_{\substack{a\!\!\!\pmod q\\(a,q)=1}}
	\left|\sum_n c_n\,e\!\left(\frac{an}{q}\right)\right|^2.
\]
Expanding in Dirichlet characters reduces this to the second moments controlled by (i) and (ii).

\medskip
\noindent\textbf{Type I/II dyadics.} BVP2M with $A$ large (absorbing the $O((\log N)^C)$ dyadic inflation) gives a total
\[
	\sum_{\text{Type I/II dyadics}}
	\int_{\mathfrak m}\big|\mathcal S_{\mathcal T}(\alpha)\big|^2\,d\alpha
	\ \ll\ \frac{N}{(\log N)^{3+\varepsilon}}.
\]

\medskip
\noindent\textbf{Type III dyadics.} After applying the prime amplifier of fixed length $|\mathcal P|=X^\vartheta$ and Kuznetsov, PASSG furnishes a uniform saving $\delta>0$ on the off-diagonal. Dividing by the amplifier normalization (as in Prop.~\ref{prop:typeIII}), one gets for each Type III block (with outer scale $X$)
\[
	\int_{\mathfrak m}\big|\mathcal S_{\mathcal T}(\alpha)\big|^2\,d\alpha
	\ \ll\ Q^{-2}\,(Q^2+X)^{1-\delta}\,X^{-\vartheta\delta+\varepsilon}.
\]
Summing over Type III dyadics and splitting $X\le Q^2$ and $X\ge Q^2$ yields a net contribution $\ll N(\log N)^{-3-\varepsilon}$ for fixed $\vartheta=\vartheta(\delta)>0$.

\medskip
\noindent\textbf{Conclusion.} Summing all dyadics gives \eqref{eq:A1}. \emph{Thus, \eqref{eq:A1} holds provided BVP2M and PASSG hold in the stated uniform forms.} This is the only place where \eqref{eq:A1} depends on Part~B and Part~C.

\subsection{Parameter choices \& loss ledger (for ease of cross-checking)}

\begin{itemize}
	\item \textbf{Minor-arc cutoff}: $Q=N^{1/2-\varepsilon}$.
	\item \textbf{HB cut parameters}: $U=V=W=N^{1/3}$ (any fixed exponents in $(1/4,1/2)$ that produce the standard Type I/II/III taxonomy will do).
	\item \textbf{Amplifier}: primes of length $|\mathcal P|=X^\vartheta$ with $\vartheta=\delta/8$.
	\item \textbf{Savings}:
	      \begin{itemize}
		      \item Large-sieve minor-arc reduction costs a factor $\asymp Q^{-2}$ which is recovered in \eqref{eq:D1}/\eqref{eq:D2}.
		      \item Type I/II: pick $A$ so that $(\log N)^C$ dyadic inflation is dominated; we target $3+\varepsilon$ net powers of $\log$.
		      \item Type III: the $\delta$-saving from \eqref{eq:D2} after amplifier normalization yields uniform $X^{-\eta}$ decay, summable across dyadics.
	      \end{itemize}
	\item \textbf{Exceptional characters / oldforms / Eisenstein}: already handled in the hypotheses of BVP2M and PASSG; their contributions obey the same $(\log N)^{-A}$ savings and therefore do not affect the sum.
\end{itemize}

\subsection{Remark}

Nothing delicate hinges on the exact form of the identity (Vaughan vs. Heath-Brown) provided it yields (i) divisor-bounded smooth coefficients and (ii) a genuine three-variable “Type III” regime where PASSG applies. Alternative cut choices merely reshuffle a finite number of dyadic families and do not change the final $(\log N)^{-3-\varepsilon}$ power once $A$ is taken large in the Type I/II inputs.

\section{Major-Arc Evaluation}

Let

$$
	\mathfrak M=\bigcup_{\substack{1\le q\le Q\\(a,q)=1}}\mathfrak M(a,q),\qquad
	\mathfrak M(a,q):=\{\alpha\in[0,1):\ |\alpha-\tfrac aq|\le \tfrac{Q}{qN}\},
$$

with $Q=N^{1/2-\varepsilon}$. Write $\alpha=a/q+\beta$ on $\mathfrak M(a,q)$ and set

$$
	V(\beta):=\sum_{n\le N}e(n\beta) \qquad\text{and}\qquad \widehat w(\beta):=\sum_{n}w(n)e(n\beta)
$$

for the sharp/smoothed Dirichlet kernels according to whether $S, B$ are unweighted or carry a fixed smooth weight $w$ supported on $[1,N]$ with $w^{(j)}\ll_j N^{-j}$.

We denote by $\mathfrak S(N)$ the (Goldbach) singular series

$$
	\mathfrak S(N)=2\prod_{p\ge 3}\Big(1-\frac1{(p-1)^2}\Big)
	\prod_{\substack{p\mid N\\ p\ge 3}}\frac{p-1}{p-2},
$$

and by $\mathfrak J$ the singular integral

$$
	\mathfrak J=
	\begin{cases}
		\displaystyle \int_{-\infty}^{\infty}\Big|\frac{\sin(\pi N\beta)}{\sin(\pi\beta)}\Big|^{\!2}e(-N\beta)\,d\beta
		 & \text{(sharp cut-off)},  \\[2ex]
		\displaystyle \int_{-\infty}^{\infty}|\widehat w(\beta)|^{2}e(-N\beta)\,d\beta
		 & \text{(smooth cut-off)}.
	\end{cases}
$$

Standard analysis yields $\mathfrak J=N+O(1)$ in the sharp case and $\mathfrak J=\widehat w(0)^2 N+O(1)$ in the smooth case.

We evaluate first the parity-blind majorant $B$, then transfer the main term to $S$.

\subsection{Major-arc evaluation for \textit{B}(\textalpha)}

Let the sieve majorant be

$$
	B(\alpha)=\sum_{n\le N}\beta(n)\,e(n\alpha),\qquad
	\beta=\beta_{z,D}\ \text{a linear (Rosser-Iwaniec) weight of level }D=N^{1/2-\varepsilon},
$$

so that $\beta$ has the standard divisor-bounded structure

$$
	\beta(n)=\sum_{\substack{d\mid n\\ d\mid P(z)}}\lambda_d,\qquad
	\lambda_d\ll_\varepsilon d^\varepsilon,\quad \sum_{d\mid P(z)}\frac{|\lambda_d|}{d}\ll \log z,
$$

with $P(z)=\prod_{p<z}p$ and $z=N^{\eta}$ a small fixed power.

On $\alpha=a/q+\beta$ with $q\le Q$ and $|\beta|\le Q/(qN)$, expand

$$
	B(\alpha)=\sum_{d\mid P(z)}\lambda_d
	\sum_{\substack{m\le N/d}} e\!\big(dm(\tfrac aq+\beta)\big)
	=\sum_{d\mid P(z)}\lambda_d\, e\!\big(\tfrac{ad}{q}\big)\,V_d(\beta),
$$

where $V_d(\beta):=\sum_{m\le N/d}e(dm\beta)$. By the standard completion and the Euler product calculation for linear sieve weights (matching local factors for $p<z$), one obtains the \textbf{major-arc approximation}

$$
	B(a/q+\beta)=\frac{\rho(q)}{\varphi(q)}\,V(\beta)\,+\,\mathcal E_B(q,\beta),
$$

where $\rho(q)$ is multiplicative, supported on square-free $q$, and satisfies

$$
	\rho(p)=
	\begin{cases}
		-1 & \text{for } p\ge 3, \\
		0  & \text{for } p=2,
	\end{cases}
	\qquad\text{so that}\quad \frac{\rho(q)}{\varphi(q)}=\frac{\mu(q)}{\varphi(q)}
$$

for all odd $q$ with $p<z$ local factors correctly matched. Moreover, uniformly for $q\le Q$ and $|\beta|\le Q/(qN)$,

$$
	\mathcal E_B(q,\beta)\ \ll\ N(\log N)^{-A}
$$

for any fixed $A>0$ once $z=N^\eta$ and $D=N^{1/2-\varepsilon}$ are tied as usual (this is the standard “well-factorable” savings of the linear sieve on major arcs).

Squaring and integrating over $\mathfrak M$ (disjoint up to negligible overlaps) gives

$$
	\int_{\mathfrak M} B(\alpha)^2 e(-N\alpha)\,d\alpha
	= \sum_{q\le Q}\ \sum_{\substack{a\bmod q\\(a,q)=1}}
	\int_{|\beta|\le Q/(qN)}
	\Big(\frac{\mu(q)}{\varphi(q)}V(\beta)\Big)^{\!2} e(-N\beta)\,d\beta
	\ +\ O\!\Big(\frac{N}{(\log N)^{3+\varepsilon}}\Big),
$$

where the error uses Cauchy-Schwarz with $\int_{\mathfrak M}|V(\beta)|^2 d\beta\ll N\log N$, the uniform bound on $\mathcal E_B$, and the total measure of $\mathfrak M$.
Since $\sum_{(a,q)=1}1=\varphi(q)$ and $\int_{|\beta|\le Q/(qN)}V(\beta)^2 e(-N\beta)\,d\beta=\mathfrak J+O(NQ^{-1})$,

$$
	\int_{\mathfrak M} B(\alpha)^2 e(-N\alpha)\,d\alpha
	= \Big(\sum_{q=1}^{\infty}\frac{\mu(q)^2}{\varphi(q)^2}\,c_q(N)\Big)\,\mathfrak J
	\ +\ O\!\Big(\frac{N}{(\log N)^{3+\varepsilon}}\Big),
$$

with $c_q(N)$ the Ramanujan sum. The absolutely convergent series equals the Goldbach singular series $\mathfrak S(N)$. Hence

$$
	\boxed{\,\int_{\mathfrak M} B(\alpha)^2 e(-N\alpha)\,d\alpha
		=\mathfrak S(N)\,\mathfrak J\;+\;O\!\big(N(\log N)^{-3-\varepsilon}\big)\ .\ }
$$

\emph{Remark.} If a smooth weight $w$ is used, replace $V(\beta)$ by $\widehat w(\beta)$ throughout, and the same argument yields $\mathfrak J=\int|\widehat w|^2 e(-N\beta)\,d\beta$ with an identical error term.

\subsection{Transferring the main term to \textit{S}(\textalpha)}

Let $S(\alpha)=\sum_{n\le N}\Lambda(n)\,e(n\alpha)$ (sharp or smooth as above). By the prime number theorem in arithmetic progressions with level of distribution $Q=N^{1/2-\varepsilon}$ (Siegel-Walfisz + Bombieri-Vinogradov in the smooth form used earlier), uniformly for $q\le Q$ and $|\beta|\le Q/(qN)$,

$$
	S(a/q+\beta)=\frac{\mu(q)}{\varphi(q)}\,V(\beta) \;+\; \mathcal E_S(q,\beta),
	\qquad \mathcal E_S(q,\beta)\ \ll\ N(\log N)^{-A}
$$

for any fixed $A>0$. Consequently, exactly the same computation as in §7.1 gives

$$
	\int_{\mathfrak M} S(\alpha)^2 e(-N\alpha)\,d\alpha
	=\mathfrak S(N)\,\mathfrak J\;+\;O\!\big(N(\log N)^{-3-\varepsilon}\big).
$$

There are two convenient “comparison” routes:

\begin{itemize}
	\item \textbf{Pointwise on $\mathfrak M$:} From the two approximations above,

	      $$
		      S(\alpha)-B(\alpha)=\mathcal E_S(\alpha)-\mathcal E_B(\alpha),
	      $$

	      whence $\int_{\mathfrak M}(S^2-B^2)e(-N\alpha)\,d\alpha =\int_{\mathfrak M}(S-B)(S+B)e(-N\alpha)\,d\alpha$
	      is $\ll N(\log N)^{-A}$ after the same bookkeeping.

	\item \textbf{Integrated $L^2$ route:} Using the $L^2$ major-arc bounds $\int_{\mathfrak M}(|S|^2+|B|^2)\ll N\log N$, together with the pointwise major-arc approximants (or with your minor-arc $L^2$ control if you prefer to absorb overlaps), yields the same $O\big(N(\log N)^{-3-\varepsilon}\big)$ remainder for the difference of major-arc contributions.
\end{itemize}

Combining §7.1-§7.2 we conclude the following proposition.

\noindent\textbf{Proposition 7.1 (Major-arc main term).} For the major arcs $\mathfrak M$ with $Q=N^{1/2-\varepsilon}$,

$$
	\int_{\mathfrak M} B(\alpha)^2 e(-N\alpha)\,d\alpha
	=\int_{\mathfrak M} S(\alpha)^2 e(-N\alpha)\,d\alpha
	=\mathfrak S(N)\,\mathfrak J\;+\;O\!\big(N(\log N)^{-3-\varepsilon}\big).
$$

In particular, $B$ and $S$ share the same Hardy-Littlewood main term on the major arcs, with an error that is negligible against $N(\log N)^{-2}$.

\subsection*{Completion of the Minor-Arc Analysis}

\subsection*{Derivation of \texorpdfstring{(A.1)}{(A.1)} from Lemma~\ref{thm:BVP2M} and Lemma~\ref{lem:S2.4}}

We now give a compact, self-contained deduction of the minor-arc bound
\[
	\boxed{\ \ \int_{\mathfrak m}\!\bigl|S(\alpha)-B(\alpha)\bigr|^{2}\,d\alpha
		\ \ll\ \frac{N}{(\log N)^{3+\varepsilon}}\ ,\ }
	\tag{A.1}
\]
using only Lemma~\ref{thm:BVP2M} (Type~I/II second moment with parity) and Lemma~\ref{lem:S2.4} (prime-averaged short-shift gain for Type~III).

\paragraph{Setup and parameters.}
Fix $\varepsilon\in(0,10^{-2})$ and set $Q=N^{1/2-\varepsilon}$ for the major/minor arc decomposition.
Apply a Heath-Brown identity with symmetric cuts $U=V=W=N^{1/3}$ to $\Lambda$ in $S(\alpha)$, and subtract the parity-blind majorant $B(\alpha)$ (linear/Rosser-Iwaniec sieve at level $D=N^{1/2-\varepsilon}$).
This yields
\[
	S(\alpha)-B(\alpha)=\sum_{\mathcal T}\mathcal S_{\mathcal T}(\alpha),
\]
where the finitely many $\mathcal T$ are dyadic Type~I/II/III blocks with divisor-bounded smooth coefficients supported on $n\asymp X$ for some $X$.

\paragraph{Minor-arc large-sieve reduction.}
For each block $\mathcal T$ with coefficient sequence $c_n$ (carrying the smooth dyadics), Gallagher’s minor-arc reduction (Lemma~\ref{lem:largesieve-minor}) gives
\[
	\int_{\mathfrak m}\Big|\sum_n c_n e(\alpha n)\Big|^2 d\alpha
	\ \ll\ \frac{1}{Q^2}\sum_{q\le Q}\ \sum_{\substack{a\!\!\!\!\pmod q\\(a,q)=1}}
	\Big|\sum_n c_n\,e\!\left(\tfrac{an}{q}\right)\Big|^2.
\]
Expanding in Dirichlet characters mod $q$ reduces this to second moments of the shape
\[
	\sum_{q\le Q}\ \sum_{\chi\ (\mathrm{mod}\ q)}
	\Big|\sum_{n\asymp X} c_n\,\chi(n)\Big|^2,
\]
with the \emph{parity twist} $\lambda(n)$ present inside $c_n$ for the terms arising from $S-B$.

\paragraph{Type I/II blocks.}
By Lemma~\ref{thm:BVP2M} (with $Q\le N^{1/2}(\log N)^{-B}$ and $L\ge N^{\eta}$ whenever needed),
\[
	\sum_{q\le Q}\ \sum_{\chi\ (\mathrm{mod}\ q)}
	\Big|\sum_{n\asymp X} c_n\,\lambda(n)\chi(n)\Big|^2
	\ \ll\ \frac{XQ}{(\log N)^{A}}.
\]
Summed over the $O((\log N)^C)$ Type~I/II dyadics (with $X\asymp N$ up to constants), and multiplied by the prefactor $Q^{-2}$ from the minor-arc reduction, this yields
\[
	\sum_{\text{Type I/II}}\int_{\mathfrak m}|\mathcal S_{\mathcal T}(\alpha)|^2 d\alpha
	\ \ll\ \frac{N}{(\log N)^{3+\varepsilon}},
\]
upon taking $A$ large enough in terms of $C$ and $\varepsilon$.

\paragraph{Type III blocks.}
For a Type~III block at outer scale $X$, apply the balanced prime amplifier and Kuznetsov as in Part~C to reach the spectral second moment controlled by Lemma~\ref{lem:S2.4}.
With $P=X^\vartheta$ (any fixed $\vartheta$ with $0<\vartheta<\tfrac16-\kappa$) and $Q\le X^{1/2-\kappa}$, Lemma~\ref{lem:S2.4} gives
\[
	\sum_{q\le Q}\ \sum_{\chi}\ \sum_f
	\Bigg|\sum_{n\asymp X}\alpha_n\,\lambda_f(n)\chi(n)\Bigg|^2
	\ \ll\ (Q^2+X)^{1-\delta}\,X^{\varepsilon},
	\qquad
	\delta=\tfrac{1}{1000}\min\{\kappa,\tfrac12-3\vartheta\}>0.
\]
Dividing out the amplifier (as in Lemma~\ref{lem:S2.4}) and undoing the spectral expansion (orthogonality), one obtains for each Type~III block
\[
	\sum_{q\le Q}\ \sum_{\chi\ (\mathrm{mod}\ q)}
	\Big|\sum_{n\asymp X} c_n\,\lambda(n)\chi(n)\Big|^2
	\ \ll\ (Q^2+X)^{1-\delta}\,X^{\varepsilon}.
\]
Inserting this into the minor-arc large-sieve reduction yields
\[
	\int_{\mathfrak m}|\mathcal S_{\mathcal T}(\alpha)|^2 d\alpha
	\ \ll\ Q^{-2}\,(Q^2+X)^{1-\delta}\,X^{\varepsilon}.
\]
Summing over the $O((\log N)^C)$ Type~III dyadics and splitting into $X\le Q^2$ and $X\ge Q^2$ gives a uniform power saving:
\[
	\sum_{\text{Type III}}\int_{\mathfrak m}|\mathcal S_{\mathcal T}(\alpha)|^2 d\alpha
	\ \ll\ \frac{N}{(\log N)^{3+\varepsilon}},
\]
since $(Q^2+X)^{1-\delta}Q^{-2}\le Q^{-2\delta}$ when $X\le Q^2$, and $\le X^{-\delta}$ when $X\ge Q^2$, both summable over dyadics (choose $\kappa,\vartheta$ once for all dyadics so that $\delta>0$).

\paragraph{Conclusion.}
Adding Type~I/II and Type~III contributions and recalling $S-B=\sum_{\mathcal T}\mathcal S_{\mathcal T}$, we obtain \emph{(A.1)}. All constants depend at most on $\varepsilon$ (the minor-arc width), on the fixed smooth cutoff in the Heath-Brown identity, on $k$ and the divisor-type bounds for coefficients, and on finitely many derivatives of the fixed Kuznetsov test $h$.
\qedwhite

\subsection{Status}
Everything here is standard Hardy-Littlewood major-arc analysis.  What remains (and is already ensured by our earlier sections) is to (i) state the exact sieve parameters $(z,D)$ used to define $\beta$, and (ii) cite the precise Bombieri-Vinogradov/Siegel-Walfisz input in the smooth form employed so the uniform error $N(\log N)^{-A}$ on $\mathfrak M$ holds (both for $\Lambda$ and for the linear-sieve majorant).

\section{Final Step}

\begin{theorem}[Goldbach for sufficiently large $N$]\label{thm:goldbach}
	Let $N$ be an even integer. Then
	\[
		R(N)\;=\;\mathfrak S(N)\,\frac{N}{\log^2 N}\,(1+o(1)),
	\]
	where $\mathfrak S(N)$ is the singular series
	\[
		\mathfrak S(N)
		=2\,\prod_{p\ge 3}\Bigl(1-\tfrac{1}{(p-1)^2}\Bigr)
		\;\prod_{\substack{p\mid N\\ p\ge 3}}\!\Bigl(1+\tfrac{1}{p-2}\Bigr),
	\]
	which satisfies $\mathfrak S(N)>0$ for every even $N$.
	In particular, every sufficiently large even integer is a sum of two primes.
\end{theorem}

\begin{proof}
	The minor-arc $L^2$ bound \eqref{eq:A1} follows from
	Lemmas~\ref{thm:BVP2M} and \ref{lem:S2.4} (Parts~B-C).
	The major-arc evaluation (Part~D.7) provides the stated main term with error $O(N/\log^{2+\eta}N)$.
	Combining these gives the claimed asymptotic.
	Positivity of $\mathfrak S(N)$ then implies $R(N)>0$ for all sufficiently large even~$N$.
\end{proof}

\begin{remark}
	For “all even $N$”, one would need an explicit finite verification up to some $N_0$, since the asymptotic guarantees positivity only beyond $N_0$. Determining such an $N_0$ requires effective constants in the major-arc and minor-arc bounds.
\end{remark}

\appendix
\renewcommand{\thesection}{Appendix~\Roman{section}}
\section{Technical Lemmas and Parameters}

\subsection{Minor--arc large sieve reduction}

We record the precise form of the inequality used in Part~D.6.

\begin{lemma}[Minor--arc large sieve reduction]\label{lem:largesieve-minor}
	Let $Q=N^{1/2-\varepsilon}$ and define major arcs
	\[
		\mathfrak M(q,a)=\Bigl\{\alpha\in[0,1):\,\Big|\alpha-\tfrac{a}{q}\Big|\le \tfrac{1}{qQ}\Bigr\},
		\qquad \mathfrak M=\!\!\!\!\!\bigcup_{\substack{q\le Q\\ (a,q)=1}}\!\!\mathfrak M(q,a),
		\qquad \mathfrak m=[0,1)\setminus\mathfrak M.
	\]
	Then for any finitely supported sequence $c_n$,
	\[
		\int_{\mathfrak m}\Big|\sum_{n}c_n e(\alpha n)\Big|^2 d\alpha
		\ \ll\ \frac{1}{Q^2}\,
		\sum_{q\le Q}\ \sum_{\substack{a\!\!\!\pmod q\\ (a,q)=1}}
		\Big|\sum_{n} c_n\,e\!\left(\tfrac{an}{q}\right)\Big|^2.
	\]
\end{lemma}

\begin{proof}[Sketch]
	Partition $[0,1)$ into $\{\mathfrak M(q,a)\}$ and $\mathfrak m$. For $\alpha\in\mathfrak m$ one has
	$|\alpha-\tfrac aq|\ge 1/(qQ)$ for all $q\le Q$. Expanding the square and integrating against the Dirichlet kernel yields Gallagher’s lemma in the form
	\[
		\int_{I} \Big|\sum c_n e(\alpha n)\Big|^2 d\alpha
		\ \ll\ \frac{1}{|I|^2}\sum_{q\le 1/|I|}\ \sum_{a\pmod q}\Big|\sum c_n e(an/q)\Big|^2
	\]
	for each interval $I\subset[0,1)$. Applying this to each complementary arc of length $\gg (qQ)^{-1}$ gives the stated bound.
\end{proof}

\subsection{Sieve weight \textbeta\ and properties}

Fix parameters
\[
	D=N^{1/2-\varepsilon},\qquad z=N^{\eta}\quad(0<\eta\ll \varepsilon).
\]
Let $P(z)=\prod_{p<z}p$ and define the linear (Rosser--Iwaniec) sieve weight
\[
	\beta(n)=\sum_{\substack{d\mid n\\ d\mid P(z)}} \lambda_d,\qquad
	\lambda_d\ll_\varepsilon d^{\varepsilon},\quad
	\sum_{d\mid P(z)}\frac{|\lambda_d|}{d}\ll \log z.
\]

\begin{lemma}\label{lem:beta-properties}
	With this choice of $\beta=\beta_{z,D}$ the following hold:
	\begin{enumerate}[label=(B\arabic*)]
		\item $\beta(n)\ge 0$ and $\beta(n)\gg \frac{\log D}{\log N}$ for $n\le N$ almost prime.
		\item $\sum_{n\le N}\beta(n)=(1+o(1))\,\tfrac{N}{\log N}$ and uniformly for $(a,q)=1$, $q\le D$,
		      \[
			      \sum_{\substack{n\le N\\ n\equiv a\pmod q}}\beta(n)
			      =(1+o(1))\,\frac{N}{\varphi(q)\log N}.
		      \]
		\item $\beta$ is well--factorable: $\beta=\sum_{d\le D}\lambda_d 1_{d\mid\cdot}$ with divisor--bounded $\lambda_d$, enabling major--arc analysis.
		\item \emph{Parity--blindness.} For any fixed smooth $W$ supported on $[1/2,2]$,
		      \[
			      \sum_{n\le N}\beta(n)\lambda(n)W(n/N)
			      \ \ll\ \frac{N}{(\log N)^A}
		      \]
		      for all $A>0$, uniformly in $N$. This follows by expanding $\beta$, applying Cauchy over $d\le D$, and invoking BVP2M / Route~B on each inner sum.
	\end{enumerate}
\end{lemma}

\subsection{Major--arc uniform error}

\begin{lemma}[Major--arc approximants]\label{lem:major-errors}
	Let $\alpha=a/q+\beta$ with $q\le Q$, $|\beta|\le Q/(qN)$. Then for any $A>0$,
	\begin{align*}
		S(\alpha) & =\frac{\mu(q)}{\varphi(q)}\,V(\beta)+O\!\Big(\frac{N}{(\log N)^A}\Big), \\
		B(\alpha) & =\frac{\mu(q)}{\varphi(q)}\,V(\beta)+O\!\Big(\frac{N}{(\log N)^A}\Big),
	\end{align*}
	uniformly in $q,a,\beta$. Here $V(\beta)=\sum_{n\le N}e(n\beta)$.
\end{lemma}

\begin{proof}
	For $S(\alpha)$: write $S(a/q+\beta)=\sum_{(n,q)=1}\Lambda(n)e(n\beta)e(an/q)+O(N^{1/2})$; expand by Dirichlet characters modulo $q$ and use the explicit formula together with Siegel--Walfisz and Bombieri--Vinogradov (smooth form) to obtain a uniform approximation by $\mu(q)\varphi(q)^{-1}V(\beta)$ with error $O_A(N(\log N)^{-A})$ for all $q\le Q=N^{1/2-\varepsilon}$ and $|\beta|\le Q/(qN)$. See, e.g., Iwaniec--Kowalski, Analytic Number Theory (IK), Thm. 17.4 and Cor. 17.12, and Montgomery--Vaughan, Multiplicative Number Theory I.

	For $B(\alpha)$: expand the linear (Rosser--Iwaniec) sieve weight $\beta$ as a well--factorable convolution at level $D=N^{1/2-\varepsilon}$, unfold the congruences, and evaluate the major arcs via the same character expansion. The well--factorability yields savings $O_A(N(\log N)^{-A})$ uniformly; see IK, Ch. 13 (Linear sieve; well--factorability, Thm. 13.6 and Prop. 13.10). Combining these gives the stated uniform bounds.
\end{proof}

\subsection{Kuznetsov at level \textit q (uniform form) and a \textDelta-second-moment lemma}

We fix the Kuznetsov normalization we use throughout and record the uniform kernel bounds in $q$.

\section{Kuznetsov at level $q$ with level-uniform kernel bounds}

We fix normalizations so that the geometric side always has the factor
$\sum_{c\equiv 0\ (q)} c^{-1} S(m,n;c)\,\mathcal W^{(*)}_{q}\!\big(4\pi\sqrt{mn}/c\big)$,
with $(*)\in\{\mathrm{Ma\ss},\mathrm{hol},\mathrm{Eis}\}$.

\begin{lemma}[Level-uniform Kuznetsov kernels]\label{lem:kuznetsov-uniform}
	Let $q\ge1$, $m,n\ge1$ with $(mn,q)=1$.
	Let $h\in C_c^\infty([-2,2])$ be even with $h(0)=1$ and set $h_Q(t)=h(t/Q)$ for $Q\ge1$.
	Write the Kuznetsov formula on $\Gamma_0(q)$ as
	\[
		\mathcal H_q(h_Q;m,n)
		=\delta_{m=n}\,\mathcal D_q(h_Q)
		+\sum_{c\equiv 0\ (q)} \frac{1}{c}\,S(m,n;c)\,\mathcal W_q^{(*)}\!\Big(\frac{4\pi\sqrt{mn}}{c}\Big),
	\]
	where $(*)$ runs over Maa\ss, holomorphic and Eisenstein pieces (with the standard weights).
	Then for every $A,j\ge0$,
	\[
		\mathcal W_q^{(*)}(z)\ \ll_A \Big(1+\frac{z}{Q}\Big)^{-A},
		\qquad
		z^{\,j}\,\partial_z^{\,j}\mathcal W_q^{(*)}(z)\ \ll_{A,j} \Big(1+\frac{z}{Q}\Big)^{-A},
	\]
	uniformly in $q\ge1$, $z>0$, and in the spectral piece $(*)$.
	The implied constants depend only on $A,j$ and on finitely many derivatives of $h$, not on $q$.
\end{lemma}

\begin{proof}[Proof sketch (standard)]
	For Maa\ss\ forms,
	\(
	\mathcal W_q^{\mathrm{Ma\ss}}(z)=\frac{i}{\pi}\int_{-\infty}^{\infty} h_Q(t)\,\tanh(\pi t)\,J_{2it}(z)\,t\,dt,
	\)
	with $h_Q$ supported on $|t|\le 2Q$ and $\|h_Q^{(r)}\|_\infty\ll_r Q^{-r}$.
	Use the Schl\"afli (or Mellin–Barnes) representation of $J_{2it}$ and integrate by parts repeatedly in $t$;
	each step gains a factor $\ll (1+z/Q)^{-1}$ thanks to the compact support and $Q^{-r}$ control on $h_Q^{(r)}$,
	yielding the stated decay. Differentiations in $z$ insert bounded polynomials in $t$ and are absorbed by the same argument.
	Holomorphic kernels ($J_{k-1}$) and Eisenstein ($K_{2it}$) are treated analogously; level $q$ appears only as the congruence $c\equiv q\ (c)$ on the geometric side and does not affect the transform.
\end{proof}

\begin{corollary}[Kernel localization for $c$]\label{cor:kernel-localization}
	With $m,n\asymp X$ and $z=4\pi\sqrt{mn}/c$, Lemma~\ref{lem:kuznetsov-uniform} implies that the $c$-sum localizes to
	\[
		c\ \asymp\ C\ :=\ \frac{X^{1/2}}{Q},
	\]
	up to tails $O_A(X^{-A})$ after summing over $c\equiv0\pmod q$.
	Moreover the same bounds hold for $z^j\partial_z^j\mathcal W_q^{(*)}$, so weights obtained by absorbing fixed smooth coefficient cutoffs inherit the same $c$-localization.
\end{corollary}




\begin{lemma}[{\boldmath $\Delta$--second moment, level--uniform}]
	\label{lem:delta-second-moment-fullyrigid}
	Let $X\ge 3$, $q\ge 1$, and write $c=qr$ with $r\asymp R\ge 1$.
	Fix $P\ge 1$. For each $(q,r)$, let $W_{q,r}(m,\Delta)$ be a smooth weight supported on
	\[
		m\asymp X,\qquad |\Delta|\le P,
	\]
	with derivative bounds, for all $0\le i,j\le 10$,
	\[
		\partial_m^{\,i}\partial_\Delta^{\,j}W_{q,r}(m,\Delta)\ \ll_{i,j}\ X^{-i}P^{-j}.
	\]
	Define
	\[
		\Sigma_{q,r}(\Delta)\ :=\ \sum_{m\asymp X} S(m,m+\Delta;c)\,W_{q,r}(m,\Delta),
		\qquad c=qr.
	\]
	Then for every $\varepsilon>0$,
	\[
		\sum_{|\Delta|\le P}\ |\Sigma_{q,r}(\Delta)|^2
		\ \ll_{\varepsilon}\ (P+c)\,c^{\,1+2\varepsilon}\,X^{\,1+2\varepsilon},
	\]
	uniformly in $q,r$ and in the family $\{W_{q,r}\}$ subject to the stated derivative conditions.
\end{lemma}

\begin{proof}
	Insert a smooth dyadic cutoff $\Psi(m/X)$ to localize $m\in[X,2X]$; absorb it into $W_{q,r}$.
	Open the square:
	\[
		\sum_{|\Delta|\le P}|\Sigma_{q,r}(\Delta)|^2
		=\!\!\sum_{|\Delta|\le P}\ \sum_{m_1,m_2\asymp X}
		S(m_1,m_1+\Delta;c)\,\overline{S(m_2,m_2+\Delta;c)}\,
		W(m_1,\Delta)\,\overline{W(m_2,\Delta)}.
	\]
	Expanding the Kloosterman sums gives
	\[
		\mathcal S
		=\!\!\sum_{\substack{x_1,x_2\bmod c\\(x_i,c)=1}}
		\sum_{|\Delta|\le P}\ \sum_{m_1,m_2\asymp X}
		e\!\left(\tfrac{m_1(x_1+\bar x_1)-m_2(x_2+\bar x_2)}{c}\right)
		e\!\left(\tfrac{\Delta(\bar x_1-\bar x_2)}{c}\right)
		W(m_1,\Delta)\,\overline{W(m_2,\Delta)}.
	\]

	\emph{Poisson in $\Delta$.} Fix $x_1,x_2$. Writing $\beta=\bar x_1-\bar x_2\bmod c$,
	the $\Delta$--sum is bounded by
	\[
		\ll \frac{P}{1+\tfrac{P}{c}\,\|\beta\|}\cdot \mathcal W_{m_1,m_2},
	\]
	with $\mathcal W_{m_1,m_2}$ a smooth weight obeying $\partial_{m_j}^i\mathcal W\ll X^{-i}$.
	Hence
	\[
		\mathcal S\ \ll\ \sum_{\substack{x_1,x_2\bmod c\\(x_i,c)=1}}
		\frac{P}{1+\tfrac{P}{c}\,\|\bar x_1-\bar x_2\|}\,
		\Bigg|\sum_{m\asymp X}
		e\!\left(\tfrac{m(x_1+\bar x_1-x_2-\bar x_2)}{c}\right)\mathcal W_m\Bigg|^2.
	\]

	\emph{Completion in $m$.} By Poisson summation modulo $c$,
	\[
		\Bigg|\sum_{m\asymp X} e\!\left(\tfrac{m\Theta}{c}\right)\mathcal W_m\Bigg|^2
		\ \ll\ X\Big(1+\tfrac{X}{c}\Big),
	\]
	uniformly in $\Theta\bmod c$.

	\emph{Sum over units.} Thus
	\[
		\mathcal S\ \ll\ X\Big(1+\tfrac{X}{c}\Big)
		\sum_{\substack{x_1,x_2\bmod c\\(x_i,c)=1}}
		\frac{P}{1+\tfrac{P}{c}\,\|\bar x_1-\bar x_2\|}.
	\]
	The map $x\mapsto \bar x$ permutes $(\mathbb Z/c\mathbb Z)^\times$, so this equals
	\[
		\phi(c)\sum_{\substack{y\bmod c\\(y,c)=1}}\frac{P}{1+\tfrac{P}{c}\,\|y\|}.
	\]
	Bounding by the full sum over $0\le y<c$ gives
	\[
		\sum_{y=0}^{c-1}\frac{P}{1+\tfrac{P}{c}\,\|y\|}
		\ \ll\ c+\!c\log(2+P/c)
		\ \ll_{\varepsilon}\ (P+c)\,c^{\varepsilon}.
	\]
	Therefore
	\[
		\sum_{|\Delta|\le P}|\Sigma_{q,r}(\Delta)|^2
		\ \ll_{\varepsilon}\ X\Big(1+\tfrac{X}{c}\Big)\,(P+c)\,c^{1+\varepsilon}.
	\]

	\emph{Final simplification.} Absorb $1+X/c\ll X^{\varepsilon}c^{\varepsilon}$ into the error.
	This yields the claimed bound
	\[
		\sum_{|\Delta|\le P}|\Sigma_{q,r}(\Delta)|^2
		\ \ll_{\varepsilon}\ (P+c)\,c^{\,1+2\varepsilon}\,X^{\,1+2\varepsilon}.
		\qedhere
	\]
\end{proof}


\begin{remark}[Oldforms/Eisenstein and uniformity in $q$]
	Lemma~\ref{lem:kuznetsov-uniform-kernel} includes oldforms and Eisenstein; their geometric contributions have the same Kloosterman-Bessel shape with identical kernel bounds, so Lemma~\ref{lem:delta-second-moment-fullyrigid} holds uniformly in the full spectrum. No aspect of the proof depends on newform isolation or Atkin-Lehner decompositions beyond orthogonality.
\end{remark}

\subsection{Parameter box}

For clarity we record the global parameter choices:
\begin{itemize}
	\item Minor--arc cutoff: $Q=N^{1/2-\varepsilon}$ with fixed $\varepsilon\in(0,10^{-2})$.
	\item Sieve level: $D=N^{1/2-\varepsilon}$, small prime cutoff $z=N^\eta$ with $0<\eta\ll\varepsilon$.
	\item Heath--Brown identity: cut parameters $U=V=W=N^{1/3}$ producing standard Type~I/II/III ranges.
	\item Amplifier: primes in $[P,2P]$ with $P=X^\vartheta$, $0<\vartheta<1/6-\kappa$.
	\item Type~III saving: $\delta=\tfrac{1}{1000}\min\{\kappa,\tfrac12-3\vartheta\}$.
\end{itemize}
\subsection{Auxiliary analytic inputs used in Part B}

We record the external inputs used in Lemma~\ref{thm:BVP2M}; full proofs are standard and can be found in the cited references.

\begin{lemma}[Smooth Hal\'asz with divisor weights]\label{lem:halasz-smooth}
	Let $f$ be a completely multiplicative function with $|f|\le 1$. For any fixed $k\in\mathbb N$ and $b_\ell\ll \tau_k(\ell)$ supported on $\ell\asymp L$ with a smooth weight $\psi(\ell/L)$, we have for any $C\ge 1$,
	\[
		\sum_{\ell\asymp L} b_\ell f(\ell)\psi(\ell/L)\ \ll_{k}\ L(\log L)^{-C}
	\]
	uniformly for all $f$ with pretentious distance $\mathbb D(f,1;L)\ge C'\sqrt{\log\log L}$, where $C'$ depends on $C,k$. In particular the bound holds for $f(n)=\lambda(n)\chi(n)$ when $\chi$ is non-pretentious. References: Granville--Soundararajan (Pretentious multiplicative functions) and IK, §13; Harper (short intervals), with smoothing uniformity.
\end{lemma}

\begin{lemma}[Log-free exceptional-set count]\label{lem:logfree-density}
	Fix $C_1\ge 1$. For $Q\le L^{1/2}(\log L)^{-100}$, the set
	\[
		\mathcal E_{\le Q}(L;C_1):=\{\chi\ (\bmod\ q): q\le Q,\ \mathbb D(\lambda\chi,1;L)\le C_1\}
	\]
	has cardinality $\#\mathcal E_{\le Q}(L;C_1)\ll Q(\log (QL))^{-C_2}$ for some $C_2=C_2(C_1)>0$. This is a standard log-free zero-density consequence in pretentious form; see Montgomery--Vaughan, Ch. 12; Gallagher; IK, Thm. 12.2 and related log-free variants.
\end{lemma}

\begin{lemma}[Siegel-zero handling]\label{lem:siegel}
	If a single exceptional real character $\chi_0\ (\bmod\ q_0)$ exists, then for any $A>0$,
	\[
		\sum_{\ell\asymp L} b_\ell\,\lambda(\ell)\chi_0(\ell)\psi(\ell/L)\ \ll\ L\exp(-c\sqrt{\log L})
	\]
	uniformly for $b_\ell\ll \tau_k(\ell)$, with an absolute $c>0$. References: Davenport, Ch. 13; IK, §11 (Deuring--Heilbronn phenomenon).
\end{lemma}

\subsection{Admissible parameter tuple and verification}

We fix explicit values valid for large $N$:

\[
	\varepsilon=10^{-3},\qquad \eta=10^{-4},\qquad \kappa=10^{-3},\qquad \vartheta=\kappa/8=1.25\times 10^{-4}.
\]

Then $Q=N^{1/2-\varepsilon}$ and for Type~II we have $L\ge N^{\eta}$, hence $Q\le L^{1/2}(\log L)^{-100}$ for large $N$, so Lemma~\ref{lem:logfree-density} applies. In Part C, $P=X^{\vartheta}$ satisfies $\vartheta<1/6-\kappa$, and
\[
	\delta\ =\ \frac1{1000}\min\{\kappa,\tfrac12-3\vartheta\}\ \ge\ \frac{1}{1000}\min\{10^{-3},\tfrac12-3\cdot 1.25\times 10^{-4}\}\ \ge\ 5\times 10^{-7}.
\]

Choose the log-power parameters $A\ge 10$ and $B=B(A,k,\eta)$ large (from Lemma~\ref{thm:BVP2M}). With these choices all inequalities in Parts B--D (large-sieve losses, amplifier division by $|\mathcal P|^2$, dyadic counts $\ll (\log N)^C$) are satisfied simultaneously, and the net savings sum to give \eqref{eq:A1}.

\subsection{Deterministic balanced signs for the amplifier}

\begin{lemma}[Balanced signs]\label{lem:balanced-signs}
	Let $\mathcal P=\{p\in[P,2P]: p\text{ prime}\}$. There exists a deterministic choice of signs $\{\varepsilon_p\}_{p\in\mathcal P}\subset\{\pm 1\}$ with $\sum_{p\in\mathcal P}\varepsilon_p=0$. Moreover, for every integer $\Delta$,
	\[
		\Big|\sum_{p\in\mathcal P}\varepsilon_p\varepsilon_{p+\Delta}\Big|\ \le\ \#\{p\in\mathcal P: p+\Delta\in\mathcal P\}\ \le\ |\mathcal P|\cdot \mathbf 1_{|\Delta|\le 2P}.
	\]
	Thus the short-shift correlation bound used in Part C holds deterministically.
\end{lemma}

\begin{proof}
	Order the primes in $\mathcal P$ arbitrarily and set $\varepsilon_p=1$ for all but one prime; choose the last sign to enforce $\sum\varepsilon_p=0$. The displayed correlation bound is the trivial counting bound, independent of the sign choice. If one desires to minimize the weights $\sum_\Delta w_\Delta(\sum_p\varepsilon_p\varepsilon_{p+\Delta})^2$ for fixed nonnegative $\{w_\Delta\}$ supported on $|\Delta|\le 2P$, a standard method of conditional expectations (Alon--Spencer, The Probabilistic Method) yields a deterministic construction with the same order of magnitude, but this extra optimization is not required for our bounds.
\end{proof}

\bigskip

\section*{References (standard sources)}
H. Iwaniec and E. Kowalski, Analytic Number Theory, AMS Colloquium Publications, Vol. 53.
H. Montgomery and R. Vaughan, Multiplicative Number Theory I. Classical Theory, Cambridge Univ. Press.
H. Davenport, Multiplicative Number Theory, 3rd ed., Springer.
J.-M. Deshouillers and H. Iwaniec, Kloosterman sums and Fourier coefficients of cusp forms, Ann. Inst. Fourier (1982).
A. Granville and K. Soundararajan, Pretentious multiplicative functions and analytic number theory (various papers/notes).
A. Harper, Bounds for multiplicative functions in short intervals.
N. Alon and J. Spencer, The Probabilistic Method (for conditional expectations derandomization).

\bibliographystyle{plain}  % or abbrv, alpha, etc.
\bibliography{references}
\end{document}

