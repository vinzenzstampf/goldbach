\documentclass[11pt]{article}

\usepackage[utf8]{inputenc}


% Math
\usepackage{amsmath}    % align, gather, etc.
\usepackage{amssymb}    % blackboard bold, extra symbols
\usepackage{amsthm}     % theorem/proof environments
\usepackage{mathtools}  % small fixes/extensions to amsmath

% Fonts
\usepackage{mathrsfs}   % script fonts if you want \mathscr
\usepackage{bm}         % bold math symbols if needed
\usepackage{textgreek}  % text-mode Greek letters

% Layout / references
\usepackage{hyperref}   % clickable refs
\pdfstringdefDisableCommands{%
  \def\eqref#1{(\ref{#1})}% make \eqref safe in bookmarks
  \def\~{}% ignore nonbreaking space in bookmarks
}

\usepackage{enumitem}   % nicer lists (optional)

% Optional, but often used in analytic number theory
\usepackage{microtype}  % better spacing
\usepackage{fullpage}   % smaller margins, more text per page

\usepackage{geometry}

\newcommand{\qedwhite}{\hfill \ensuremath{\Box}}
\newcommand{\cE}{\mathcal{E}}
\newcommand{\cF}{\mathcal{F}}
\newcommand{\cG}{\mathcal{G}}

% Numbering: theorems/lemmas by Part (A, B, ...)
\renewcommand{\thepart}{\Alph{part}}
\newtheorem{lemma}{Lemma}[part]
\newtheorem{theorem}[lemma]{Theorem}
\newtheorem{proposition}[lemma]{Proposition}
\newtheorem{corollary}[lemma]{Corollary}
\theoremstyle{definition}
\newtheorem{definition}[lemma]{Definition}
\theoremstyle{remark}
\newtheorem{remark}[lemma]{Remark}

% Number equations by Part as (A.1), (A.2), ...
\numberwithin{equation}{part}
% Make hyperref's equation anchors match the Part-based numbering
\makeatletter
\renewcommand{\theHequation}{\thepart.\arabic{equation}}
\makeatother

% Number sections as A.1, A.2, ... and subsections as A.1.1, etc.
% \renewcommand{\thesection}{\thepart.\arabic{section}}
% \renewcommand{\thesubsection}{\thesection.\arabic{subsection}}
% \renewcommand{\thesubsubsection}{\thesubsection.\arabic{subsubsection}}
\setcounter{secnumdepth}{3}
% Reset section numbering at each new Part
\makeatletter
\@addtoreset{section}{part}
\makeatother

 \geometry{
 a4paper,
 total={170mm,257mm},
 left=20mm,
 top=20mm,
 }
 \usepackage{graphicx}
 \usepackage{titling}

 \title{Proof of the Goldbach Conjecture}
\author{Vinzenz Stampf}
\date{September 2025}
 
 \usepackage{fancyhdr}
% Adjust header height to avoid fancyhdr warning
\setlength{\headheight}{14pt}
\addtolength{\topmargin}{-14pt}
\fancypagestyle{plain}{%  the preset of fancyhdr 
    \fancyhf{} % clear all header and footer fields
  % Left footer shows the document date
  \fancyfoot[L]{\thedate}
    \fancyhead[L]{Description of Assignment}
    \fancyhead[R]{\theauthor}
}
\makeatletter
\def\@maketitle{%
  \newpage
  \null
  \vskip 1em%
  \begin{center}%
  \let \footnote \thanks
    {\LARGE \@title \par}%
    \vskip 1em%
    %{\large \@date}%
  \end{center}%
  \par
  \vskip 1em}
% Provide public macros used elsewhere in the document
% (LaTeX stores these internally as \@date and \@author)
\providecommand{\thedate}{\@date}
\providecommand{\theauthor}{\@author}
\makeatother

\begin{document}

\tableofcontents

\maketitle

\noindent\begin{tabular}{@{}ll}
	Student & \theauthor \\
\end{tabular}

\part{Framework}

This manuscript lays out a circle-method framework aimed at binary Goldbach. The final asymptotic is derived on the minor-arc $L^2$ estimate \eqref{eq:A1} and the analytic inputs explicitly stated in Parts B-D. In particular:

\begin{itemize}
	\item Establishing \eqref{eq:A1} is the central new task; Parts B-D provide a proposed route via Type I/II/III analyses.
	\item Major-arc expansions for $S$ and for the sieve majorant $B$ are used with uniformity standard in the literature; precise statements are recorded in §7 with hypotheses.
	\item The final positivity conclusion for $R(N)$ is conditional on \eqref{eq:A1} and the stated major-arc bounds.
\end{itemize}

A succinct punch-list of outstanding items appears in Appendix~B.

\section{Circle-Method Decomposition}

Let

$$
	S(\alpha)\;=\;\sum_{n\le N}\Lambda(n)\,e(\alpha n),\qquad
	R(N)\;=\;\int_{0}^{1} S(\alpha)^2\,e(-N\alpha)\,d\alpha .
$$

Fix $\varepsilon\in (0,\tfrac1{10})$ and set

$$
	Q \;=\; N^{1/2-\varepsilon}.
$$

For coprime integers $a,q$ with $1\le q\le Q$, define the major arc around $a/q$ by

$$
	\mathfrak M(a,q)\;=\;\Bigl\{\alpha\in[0,1):\ \bigl|\alpha-\tfrac{a}{q}\bigr|
	\le \frac{Q}{qN}\Bigr\}.
$$

Let

$$
	\mathfrak M\;=\;\bigcup_{\substack{1\le q\le Q\\ (a,q)=1}}\mathfrak M(a,q),
	\qquad
	\mathfrak m\;=\;[0,1)\setminus\mathfrak M .
$$

Then

$$
	R(N)\;=\;\int_{\mathfrak M} S(\alpha)^2 e(-N\alpha)\,d\alpha\;+\;
	\int_{\mathfrak m} S(\alpha)^2 e(-N\alpha)\,d\alpha
	\;=\;R_{\mathfrak M}(N)+R_{\mathfrak m}(N).
$$


\subsection{Parity-blind majorant \texorpdfstring{$B(\alpha)$}{B\textalpha}}

Let $\beta=\{\beta(n)\}_{n\le N}$ be a \textbf{parity-blind sieve majorant} for the primes at level $D=N^{1/2-\varepsilon}$, in the following sense:

\begin{itemize}[leftmargin=*]
	\item[(B1)] $\beta(n)\ge 0$ for all $n$ and $\beta(n)\gg \tfrac{\log D}{\log N}$ for $n$ the main $\le N$.
	\item[(B2)] $\displaystyle \sum_{n\le N}\beta(n)\;=\;(1+o(1))\,\frac{N}{\log N}$ and, uniformly in residue classes $(\bmod\,q)$ with $q\le D$,

	      $$
		      \sum_{\substack{n\le N\\ n\equiv a\!\!\!\pmod q}}\beta(n)
		      \;=\;(1+o(1))\,\frac{N}{\varphi(q)\log N}\qquad ((a,q)=1).
	      $$

	\item[(B3)] $\beta$ admits a convolutional description with coefficients supported on $d\le D$ (e.g. Selberg upper-bound sieve), enabling standard major-arc analysis.
	\item[(B4)] \textbf{Parity-blindness:} $\beta$ does not correlate with the Liouville function at the $N^{1/2}$ scale (so it does not distinguish the parity of $\Omega(n)$); this is automatic for classical upper-bound Selberg weights.
\end{itemize}

Define

$$
	B(\alpha)\;=\;\sum_{n\le N}\beta(n)\,e(\alpha n).
$$


\subsection{Major arcs: main term from \textit{B}}

On $\mathfrak M(a,q)$ write $\alpha=\tfrac{a}{q}+\tfrac{\theta}{N}$ with
$|\theta|\le Q/q$. By (B2)-(B3) and standard manipulations (Dirichlet characters, partial summation, and the prime number theorem in arithmetic progressions up to modulus $q\le Q$), one obtains the classical evaluation

$$
	\int_{\mathfrak M} B(\alpha)^2\,e(-N\alpha)\,d\alpha
	\;=\;\mathfrak S(N)\,\frac{N}{\log^2 N}\,(1+o(1)),
$$

where $\mathfrak S(N)$ is the singular series

$$
	\mathfrak S(N)\;=\;\sum_{q=1}^{\infty}\ \frac{\mu(q)}{\varphi(q)}\!
	\sum_{\substack{a\,(\mathrm{mod}\,q)\\(a,q)=1}} e\!\left(-\frac{Na}{q}\right).
$$

Moreover, with the same tools one shows that on the major arcs $S(\alpha)$ may be replaced by $B(\alpha)$ in the quadratic integral at a total cost $o\!\left(\tfrac{N}{\log^2 N}\right)$ once the minor-arc estimate below is in place (see the reduction step).


\subsection{Reduction to a minor-arc \texorpdfstring{$L^2$}{L-2} bound}

We record the minor-arc target:

\begin{equation}\label{eq:A1}
	\int_{\mathfrak m}|S(\alpha)-B(\alpha)|^2\,d\alpha\ \ll\ \frac{N}{(\log N)^{3+\varepsilon}}.
\end{equation}

\begin{equation}\label{eq:char-second-moment}\sum_{q\le Q}\ \sum_{\chi\,\bmod\, q}\left|\sum_{n\le N} c_n\,\lambda(n)\,\chi(n)\right|^{2}\,\ll\, \frac{NQ}{(\log N)^A}\end{equation}
\begin{proposition}[Reduction]\label{prop:reduction}
	Assume \eqref{eq:A1}. Then

	$$
		R(N)\;=\;\int_{\mathfrak M} B(\alpha)^2 e(-N\alpha)\,d\alpha\;+\;O\!\left(\frac{N}{(\log N)^{3+\varepsilon/2}}\right),
	$$

	and hence

	$$
		R(N)\;=\;\mathfrak S(N)\,\frac{N}{\log^{2}N}\;+\;O\!\left(\frac{N}{(\log N)^{2+\delta}}\right)
	$$

	for some $\delta>0$.

\end{proposition}

\begin{proof}[Sketch]
	Split on $\mathfrak M\cup\mathfrak m$ and insert $S=B+(S-B)$:

	$$
		S^2 = B^2 + 2B(S-B) + (S-B)^2.
	$$

	Integrating over $\mathfrak m$ and using Cauchy-Schwarz,

	$$
		\Bigl|\int_{\mathfrak m} B(\alpha)(S(\alpha)-B(\alpha))\,e(-N\alpha)\,d\alpha\Bigr|
		\ \le\ \Bigl(\int_{\mathfrak m}|B(\alpha)|^2\Bigr)^{1/2}
		\Bigl(\int_{\mathfrak m}|S(\alpha)-B(\alpha)|^2\Bigr)^{1/2}.
	$$

	By Parseval and (B2)-(B3),

	$$
		\int_0^1 |B(\alpha)|^2\,d\alpha \;=\; \sum_{n\le N}\beta(n)^2 \;\ll\; \frac{N}{\log N},
	$$

	so $\int_{\mathfrak m}|B|^2\le\int_0^1|B|^2\ll N/\log N$. Together with \eqref{eq:A1} this gives the cross-term contribution

	$$
		\ll \Bigl(\frac{N}{\log N}\Bigr)^{1/2}\Bigl(\frac{N}{(\log N)^{3+\varepsilon}}\Bigr)^{1/2}
		\;=\;\frac{N}{(\log N)^{2+\varepsilon/2}}.
	$$

	The pure error $\int_{\mathfrak m}|S-B|^2$ is exactly the quantity in \eqref{eq:A1}. On the major arcs, standard major-arc analysis (Vaughan's identity or the explicit formula combined with (B2)-(B3)) shows that replacing $S$ by $B$ inside $\int_{\mathfrak M}(\cdot)$ affects the value by $O(N/(\log N)^{2+\delta})$ (details in the major-arc section). Collecting terms yields the stated reduction.
\end{proof}

\part{Type I / II Analysis}

\section{Type II parity gain}

\begin{theorem}[Type-II parity gain]
	Fix $A>0$ and $0<\varepsilon<10^{-3}$. Let $N$ be large, $Q\le N^{1/2-2\varepsilon}$. Let $M$ satisfy $N^{1/2-\varepsilon}\le M\le N^{1/2+\varepsilon}$ and set $X=N/M\asymp M$. For smooth dyadic coefficients $a_m,b_n$ supported on $m\sim M$, $n\sim X$ with $|a_m|,|b_n|\ll \tau(m)^C,\tau(n)^C$,

	$$
		\sum_{q\le Q}\ \sum_{\chi\bmod q}^{\!*}
		\left|\sum_{mn\asymp N} a_m b_n\,\lambda(mn)\chi(mn)\right|^2
		\ \ll_{A,\varepsilon,C}\ \frac{NQ}{(\log N)^{A}}.
	$$
\end{theorem}

\begin{proof}
	Let $u(k)=\sum_{mn=k}a_m b_n \lambda(k)$ on $k\sim N$; then $\sum |u(k)|^2\ll N(\log N)^{O_C(1)}$. Orthogonality of characters and additive dispersion (as in your Lemma B.2.1-B.2.2) yield, with block length

	$$
		H=\frac{N}{Q}N^{-\varepsilon}\ \ge\ N^{\varepsilon},
	$$

	the reduction

	$$
		\sum_{q\le Q}\sum_{\chi}^{*}\Big|\sum u(k)\chi(k)\Big|^2
		\ \ll\ \Big(\frac{N}{H}+Q\Big)\!
		\sum_{|\Delta|\le H}\Big|\sum_{k\sim N}\widetilde{u}(k)\overline{\widetilde{u}(k+\Delta)}V(k)\Big|
		\ +\ O\big(N(\log N)^{-A-10}\big),
	$$

	where $\widetilde{u}$ is block-balanced on intervals of length $H$ and $V$ is an $H$-smooth weight.

	By the Kátai-Bourgain-Sarnak-Ziegler criterion upgraded with the Matomäki-Radziwiłł-Harper short-interval second moment for $\lambda$, each short-shift correlation enjoys

	$$
		\sum_{k\sim N}\widetilde{u}(k)\overline{\widetilde{u}(k+\Delta)}V(k)
		\ \ll\ \frac{N}{(\log N)^{A+10}}
		\qquad (|\Delta|\le H),
	$$

	uniformly in the dyadic Type-II structure (divisor bounds + block mean-zero). There are $\ll H$ shifts $\Delta$, hence

	$$
		\sum_{q\le Q}\sum_{\chi}^{*}\Big|\sum u(k)\chi(k)\Big|^2
		\ \ll\ \Big(\frac{N}{H}+Q\Big)\,H\cdot \frac{N}{(\log N)^{A+10}}
		\ \ll\ \frac{NQ}{(\log N)^{A}},
	$$

	since $\frac{N}{H}\asymp Q\,N^{\varepsilon}$.
\end{proof}

\paragraph{Remarks.}
\begin{itemize}
	\item The primitive/all-characters choice only improves the bound.
	\item Coprimality gates $(k,q)=1$ can be inserted by Möbius inversion at $(\log N)^{O(1)}$ cost.
	\item Smoothing losses are absorbed in the $+10$ log-headroom.
\end{itemize}


\section{Bombieri--Vinogradov with parity (second moment): full statement and proof}

\begin{theorem}[BVP2M: BV with parity, second moment]\label{thm:BVP2M}
	Fix $A>0$. Then there exists $B=B(A)$ such that for all sufficiently large $N$ and all
	\[
		Q \ \le\ N^{1/2}\,(\log N)^{-B},
	\]
	the following holds. Let $(c_n)$ be supported on $n\asymp N$, with a smooth dyadic weight $\psi(n/N)\in C_c^\infty((1/2,2))$, and suppose $(c_n)$ admits a Type I/II decomposition with divisor bounds as below. Then
	\begin{equation}\label{eq:BVP2M-goal}
		\sum_{q\le Q}\ \sum_{\chi\bmod q}
		\left|\sum_{n\asymp N} c_n\,\lambda(n)\,\chi(n)\right|^2
		\ \ll_{A}\ \frac{NQ}{(\log N)^A}.
	\end{equation}
	The implied constant depends on $A$ and on fixed smoothness/divisor parameters only.
\end{theorem}

\paragraph{Type I/II hypotheses.}
There is a fixed $k\in\mathbb N$ and coefficients $d_n$ with $|d_n|\le \tau_k(n)$ such that
$c_n=\psi(n/N)\,d_n$ and either
\begin{description}
	\item[Type I:] $d_n=\displaystyle\sum_{m\ell=n}\alpha_m\beta_\ell$ with $M\le N^{1/2-\eta}$ for some fixed $\eta\in(0,1/2)$, and
	      $|\alpha_m|\ll \tau_k(m)$, $|\beta_\ell|\ll \tau_k(\ell)$;
	\item[Type II:] same factorization with $N^{\eta}\le M\le N^{1/2-\eta}$ (balanced case).
\end{description}
All sums carry smooth dyadic cutoffs in $m,\ell$ of the form $\psi_1(m/M)$, $\psi_2(\ell/L)$ with $L=N/M$ and $\psi_i\in C_c^\infty((1/2,2))$,
with derivative bounds uniform in $N$.

\begin{remark}[Use with coprimality gates]
	Throughout we may freely insert $(n,q)=1$ or $(m\ell,q)=1$ via Möbius inversion; the additional $d\mid (n,q)$ sums are bounded with at most $(\log N)^{O(1)}$ loss because $q\le Q\le N^{1/2}(\log N)^{-B}$ and coefficients are divisor-bounded.
\end{remark}

\subsection*{Inputs}
We use the following standard tools (uniform in smooth weights and divisor bounds):
\begin{enumerate}[label=(I\arabic*)]
	\item \textbf{Smooth Halász (pretentious form).}
	      If $f$ is completely multiplicative, $|f|\le1$, and $\psi\in C_c^\infty((1/2,2))$, then for any $C\ge1$
	      \[
		      \sum_{x\asymp X} \psi(x/X)\,f(x)
		      \ \ll\ X\,(\log X)^{-C}
	      \]
	      unless $\mathbb D(f,1;X)\ll_C \sqrt{\log\log X}$. (Granville--Soundararajan; see also IK, Ch.~13.) This remains valid with weights $\ll \tau_k$.
	\item \textbf{Log-free zero-density/exceptional-set bound.}
	      For $Q\le X^{1/2}(\log X)^{-100}$ the set
	      \[
		      \mathcal E_{\le Q}(X;C_1):=\Big\{\chi\bmod q \ (q\le Q):\  \mathbb D(\lambda\chi,1;X)\le C_1\Big\}
	      \]
	      satisfies $\#\mathcal E_{\le Q}(X;C_1)\ll Q\,(\log (QX))^{-C_2}$ for some $C_2=C_2(C_1)>0$. (Gallagher/Montgomery--Vaughan; IK, Ch.~12; log-free variants.)
	\item \textbf{Spectral large sieve (multiplicative).}
	      For any coefficients $a_n$ supported on $n\asymp X$,
	      \[
		      \sum_{q\le Q}\ \sum_{\chi\bmod q}\left|\sum_{n\asymp X} a_n \chi(n)\right|^2
		      \ \ll\ (X+Q^2)\sum_{n\asymp X}|a_n|^2.
	      \]
	      (Montgomery--Vaughan large sieve; \cite[Thm.~7.13]{IK})
\end{enumerate}

\begin{lemma}[Divisor-weight $\ell^2$ bound]\label{lem:l2-div}
	If $|c_n|\le \tau_k(n)$ and $c_n$ is supported on $n\asymp N$ with a fixed smooth weight, then
	$\sum_{n\asymp N}|c_n|^2\ll N(\log N)^{O_k(1)}$, uniformly in all the smooth cutoffs.
\end{lemma}

\begin{proof}[Proof of Theorem~\ref{thm:BVP2M}]
	Set
	\[
		S(\chi):=\sum_{n\asymp N} c_n\,\lambda(n)\chi(n).
	\]
	By Cauchy–Schwarz in the Type I/II factorization (as arranged in the standard arguments for dispersion/Type II), it suffices to bound uniformly in $m\sim M$
	\[
		\Sigma_m:=\sum_{q\le Q}\ \sum_{\chi\bmod q}\left|\sum_{\ell\asymp L} b^{(m)}_\ell\ \lambda(\ell)\chi(\ell)\right|^2,
		\qquad L=N/M,
	\]
	where $|b^{(m)}_\ell|\ll \tau_k(\ell)$ with a smooth weight $\psi_m(\ell/L)$ (all derivative bounds uniform in $m$).

	We split characters into \emph{non-pretentious} and \emph{exceptional} using the pretentious distance for $f_\chi(\ell):=\lambda(\ell)\chi(\ell)$ at scale $L$.

	\smallskip
	\noindent\textbf{(A) Non-pretentious characters.}
	By (I1) with $f=f_\chi$ and $C=C(A)+10$, for all $\chi\notin\mathcal E(L;C_1)$,
	\[
		\sum_{\ell\asymp L} b^{(m)}_\ell\, f_\chi(\ell)\ \ll\ L(\log L)^{-C}.
	\]
	Summing the squares over $\ll Q^2$ characters gives
	\[
		\sum_{q\le Q}\ \sum_{\substack{\chi\bmod q\\ \chi\notin\mathcal E(L;C_1)}}
		\left|\sum_{\ell\asymp L} \cdots \right|^2
		\ \ll\ Q^2\,L^2\,(\log L)^{-2C}.
	\]

	\smallskip
	\noindent\textbf{(B) Exceptional characters.}
	By (I2),
	\[
		\#\mathcal E_{\le Q}(L;C_1)\ \ll\ Q\,(\log (QL))^{-C_2}.
	\]
	For each exceptional $\chi$ we use the trivial divisor-weight bound
	\[
		\left|\sum_{\ell\asymp L} b^{(m)}_\ell\,f_\chi(\ell)\right|
		\ \ll\ L(\log L)^{O_k(1)}.
	\]
	Thus the total exceptional contribution is
	\[
		\ll\ Q\cdot L^2\ (\log (QL))^{-C_2+O_k(1)}.
	\]

	\smallskip
	\noindent\textbf{(C) Combine and reinsert $m$.}
	Hence, for each fixed $m$,
	\[
		\Sigma_m\ \ll\ Q^2 L^2 (\log L)^{-2C}\ +\ Q L^2 (\log (QL))^{-C_2+O_k(1)}.
	\]
	Multiply by the $\ell^2$ norm in $m$ coming from Cauchy–Schwarz in the outer variable: by Lemma~\ref{lem:l2-div},
	\[
		\sum_{m\sim M} |\alpha_m\lambda(m)|^2\ \ll\ M(\log N)^{O_k(1)}.
	\]
	Therefore
	\[
		\sum_{q\le Q}\sum_{\chi}|S(\chi)|^2
		\ \ll\ \Big(Q^2L^2(\log N)^{-2C}\ +\ QL^2(\log N)^{-C_2+O_k(1)}\Big)\,M(\log N)^{O_k(1)}.
	\]
	Using $ML=N$ and choosing $C$ (hence $C_2$) large in terms of $A,k$ yields
	\[
		\sum_{q\le Q}\sum_{\chi}|S(\chi)|^2\ \ll\ \frac{NQ}{(\log N)^A}.
	\]

	\smallskip
	\noindent\textbf{(D) Type I case.}
	When $M\le N^{1/2-\eta}$ the same reduction applies (the inner $L=N/M\ge N^{\eta}$, ensuring $Q\le L^{1/2}(\log L)^{-100}$ for large $N$ so that (I2) is available). Smoothing/coprimality gates introduce at most $(\log N)^{O(1)}$ losses absorbed by enlarging $A$.

	\smallskip
	\noindent\textbf{(E) Dyadic inflation.}
	Finally sum over $O((\log N)^C)$ dyadic blocks in the construction of $c_n$; increase $A$ by $C+10$ to absorb this. This yields \eqref{eq:BVP2M-goal}.
\end{proof}

\begin{corollary}[Par\-ity-blindness of linear sieve weights]\label{cor:parityblind}
	Let $\beta$ be the linear (Rosser–Iwaniec) upper-bound sieve at level $D=N^{1/2-\varepsilon}$ with small prime cutoff $z=N^{\eta}$, and let $\psi\in C_c^\infty((1/2,2))$. Then, for any $A>0$,
	\[
		\sum_{n\le N}\beta(n)\lambda(n)\psi(n/N)\ \ll\ \frac{N}{(\log N)^A}.
	\]
	\emph{Sketch.} Expand $\beta(n)=\sum_{d\mid P(z)}\lambda_d\,1_{d\mid n}$ with well-factorable coefficients $\lambda_d\ll_\varepsilon d^\varepsilon$; apply Cauchy over $d\le D$ and Theorem~\ref{thm:BVP2M} to each inner sum with a coprimality gate. The total is $\ll N(\log N)^{-A}$ after choosing $B(A)$ large enough.
\end{corollary}


\part{Type III Analysis}

\section{PASSG (Prime-averaged short-shift gain — full proof)}

\begin{lemma}[Prime-averaged short-shift gain]\label{lem:PASSG}
	Fix $\vartheta\in(0,1/2)$ and let $\mathcal P=\{p\in[P,2P]\text{ prime}\}$ with $P=X^\vartheta$.
	Choose signs $\varepsilon_p\in\{\pm1\}$ with
	\[
		\sum_{p\in\mathcal P}\varepsilon_p=0,
		\qquad
		\Big|\sum_{p\in\mathcal P}\varepsilon_p\varepsilon_{p+\Delta}\Big|
		\ \ll\ |\mathcal P|\cdot\mathbf1_{|\Delta|\le P^{1-o(1)}},
	\]
	so that $A_f=\sum_{p\in\mathcal P}\varepsilon_p\lambda_f(p)$ is a balanced amplifier.
	Let $\alpha_n$ be coefficients supported on $n\asymp X$ with divisor bounds $|\alpha_n|\ll_\varepsilon \tau(n)^C$, smooth cutoff, and coprimality gates as needed.
	Then there exists $\delta=\delta(\vartheta)>0$ such that
	\begin{equation}\label{eq:S2.4_goal}
		\sum_{q\le Q}\ \sum_{\chi\bmod q}\ \sum_{f\ \mathrm{mod}\ q}
		\Bigg|\ \sum_{n\asymp X}\alpha_n \lambda_f(n)\chi(n)\ \Bigg|^2\,|A_f|^2
		\ \ll_\varepsilon\ (Q^2+X)^{\,1-\delta}\,|\mathcal P|^{\,2-\delta},
	\end{equation}
	uniformly for $Q\le X^{1/2-\varepsilon}$.
\end{lemma}

\begin{proof}
	\textbf{Step 1. Amplifier expansion.}
	Expanding $|A_f|^2$ gives
	\[
		|A_f|^2
		=\sum_{p_1,p_2\in\mathcal P}\varepsilon_{p_1}\varepsilon_{p_2}\,\lambda_f(p_1)\lambda_f(p_2).
	\]
	Use the Hecke relation:
	\[
		\lambda_f(p_1)\lambda_f(p_2)
		=\lambda_f(p_1p_2)+\mathbf1_{p_1=p_2}+\mathcal T_{p_1,p_2}(f),
	\]
	where $\mathcal T_{p_1,p_2}$ collects the ``$p\mid n$ tails'' terms.
	By Lemma~\ref{lem:hecke-tails}, these tails contribute
	\[
		\ll (Q^2+X)^{1+\varepsilon}\,|\mathcal P|\,X^{-1/2+\varepsilon},
	\]
	which is negligible after dividing by $|\mathcal P|^2$.

	\smallskip
	\textbf{Step 2. Insert amplifier into the second moment.}
	We are left with
	\[
		\mathrm{OD}
		:=\sum_{q\le Q}\ \sum_{\chi\bmod q}\ \sum_f
		\ \sum_{p_1,p_2\in\mathcal P}\varepsilon_{p_1}\varepsilon_{p_2}
		\Big|\sum_{n\asymp X}\alpha_n\lambda_f(n)\chi(n)\Big|^2 \lambda_f(p_1p_2).
	\]

	\smallskip
	\textbf{Step 3. Kuznetsov decomposition.}
	Expand the inner square, apply Kuznetsov on $\Gamma_0(q)$ with test $h_Q$ (Lemma~\ref{lem:kuznetsov-uniform}) to the bilinear form
	\[
		\sum_{m,n\asymp X} \alpha_m\overline{\alpha_n}\chi(m)\overline{\chi(n)}
		\sum_{p_1,p_2\in\mathcal P}\varepsilon_{p_1}\varepsilon_{p_2}\,
		\lambda_f(m)\overline{\lambda_f(n)}\lambda_f(p_1p_2).
	\]
	The diagonal ($m=n$, $p_1=p_2$) is harmless.
	On the geometric side we obtain
	\[
		\sum_{c\equiv0\pmod q}\ \frac{1}{c}\,
		S(m,n;c)\,W_{q}(m,n,p_1,p_2;c),
	\]
	where $W_q$ is a smooth weight depending on $m,n,p_1,p_2$ via $z=4\pi\sqrt{mn}/c$.
	By Cor.~\ref{cor:kernel-localization}, $c$ localizes to $c\asymp X^{1/2}/Q$ with rapid decay outside.

	\smallskip
	\textbf{Step 4. Short-shift grouping.}
	Let $\Delta=m-n$.
	Poisson summation in $\Delta$ (cf. the $\Delta$-second-moment lemma, already proved) yields
	\[
		\sum_{|\Delta|\le X^{1/2+o(1)}}
		\Big|\ \sum_{p_1,p_2\in\mathcal P}\varepsilon_{p_1}\varepsilon_{p_2}\,
		S(m,m+\Delta;c)\,W_q(m,\Delta;p_1,p_2;c)\ \Big|.
	\]
	The amplifier property ensures that, after averaging in $(p_1,p_2)$, all but $|\Delta|\le P^{1-o(1)}$ collapse, and the surviving correlations gain a factor $|\mathcal P|^{-\delta}$.

	\smallskip
	\textbf{Step 5. Weil and Cauchy–Schwarz.}
	Apply Weil’s bound $|S(m,m+\Delta;c)|\le\tau(c)\,(m,c)^{1/2}\,c^{1/2}$.
	Coupled with smooth weights and the $c\asymp X^{1/2}/Q$ localization, the $\Delta$-second-moment lemma delivers
	\[
		\sum_{|\Delta|\le P^{1-o(1)}}\ \sum_{c\equiv0\pmod q}
		\frac{1}{c}\,|S(m,m+\Delta;c)|^2\,|W_q(\cdot)|^2
		\ \ll\ (Q^2+X)^{1-\delta_1}
	\]
	for some fixed $\delta_1>0$ (depending only on $\vartheta$).
	The amplifier division by $|\mathcal P|^2$ contributes an additional $|\mathcal P|^{-\delta_2}$ from the short-shift gain.

	\smallskip
	\textbf{Step 6. Uniformity across spectral pieces.}
	By Lemma~\ref{lem:oldforms-eis-uniform}, the same bounds hold for Maa\ss, holomorphic, oldforms and Eisenstein contributions. Thus no exceptional case remains.

	\smallskip
	\textbf{Conclusion.}
	Combining Steps 1–6, for some fixed $\delta=\min(\delta_1,\delta_2)>0$,
	\[
		\mathrm{OD}\ \ll_\varepsilon\ (Q^2+X)^{1-\delta}\,|\mathcal P|^{2-\delta},
	\]
	which is exactly \eqref{eq:S2.4_goal}.
\end{proof}

\section{Type~III Analysis: Prime-Averaged Short-Shift Gain}

\begin{proposition}[Type-III spectral second moment]\label{prop:typeIII}
	Let $(\alpha_n)$ be a smooth Type-III coefficient sequence supported on $n\asymp X$, with divisor-type bounds $|\alpha_n|\ll_\varepsilon \tau(n)^C$ and smooth weight of width $X^{1+o(1)}$.
	Let $Q\le X^{1/2-\kappa}$ with some fixed $0<\kappa<1/4$. Then, for some fixed $\delta>0$ depending only on~$\kappa$,
	\[
		\sum_{q\le Q}\ \sum_{\chi\bmod q}\ \sum_{f}
		\Bigg|\sum_{n\asymp X}\alpha_n\,\lambda_f(n)\chi(n)\Bigg|^2
		\ \ \ll_{\varepsilon,C}\ \ (Q^2+X)^{\,1-\delta}\,X^{\varepsilon}.
	\]
\end{proposition}

\begin{proof}
	Fix a prime amplifier $\mathcal P=\{p\in[P,2P]\}$ with $P=X^\vartheta$, $\varepsilon_p\in\{\pm1\}$ balanced so that $\sum_p\varepsilon_p=0$.
	Define $A_f=\sum_{p\in\mathcal P}\varepsilon_p\,\lambda_f(p)$, and set
	$S_{q,\chi,f}=\sum_{n\asymp X}\alpha_n\lambda_f(n)\chi(n)$.
	As in the balanced-amplifier method,
	\[
		\sum_{q\le Q}\sum_{\chi}\sum_f |S_{q,\chi,f}|^2
		\ \le\ \frac{1}{|\mathcal P|^2}\sum_{q\le Q}\sum_{\chi}\sum_f |A_f\,S_{q,\chi,f}|^2.
	\]

	Opening the amplifier and applying Kuznetsov (including oldforms and Eisenstein) reduces the off--diagonal to correlations of the form
	\[
		\mathrm{OD}\ :=\ \sum_{q\sim Q}\ \sum_{r\asymp R}\frac{1}{qr}\sum_{\Delta\ne0}\nu(\Delta)\,|\Sigma_{q,r}(\Delta)|,
	\]
	with $\nu(\Delta)$ the prime-pair counts and
	$\Sigma_{q,r}(\Delta)=\sum_{m\asymp X} S(m,m+\Delta;qr)\,W_{q,r}(m,\Delta)$.
	Here $c=qr\asymp X^{1/2}/Q$, and $W_{q,r}$ are smooth weights supported on $m\asymp X$, $|\Delta|\le P$.

	By Lemma~\ref{lem:delta-second-moment-fullyrigid},
	\[
		\sum_{|\Delta|\le P}|\Sigma_{q,r}(\Delta)|^2
		\ \ll_\varepsilon\ (P+qr)\,(qr)^{1+2\varepsilon}\,X^{1+2\varepsilon}.
	\]
	Cauchy--Schwarz and $\sum\nu(\Delta)\asymp |\mathcal P|^2$ give
	\[
		\sum_{|\Delta|\le P}\nu(\Delta)\,|\Sigma_{q,r}(\Delta)|
		\ \ll_\varepsilon\
		|\mathcal P|\,(P+qr)^{1/2}\,(qr)^{1/2+\varepsilon}\,X^{1/2+\varepsilon}.
	\]
	Summing over $q\sim Q$, $r\asymp R$ yields
	\[
		\mathrm{OD}\ \ll_\varepsilon\
		|\mathcal P|\,X^{3/4+\varepsilon}\,Q^{-1/2}\,(P+X^{1/2}/Q)^{1/2}.
	\]

	Dividing by $|\mathcal P|^2$,
	\[
		\sum_{q\le Q}\sum_{\chi}\sum_f |S_{q,\chi,f}|^2
		\ \ll_\varepsilon\ \frac{X^{3/4+\varepsilon}}{P}\,Q^{-1/2}\,(P+X^{1/2}/Q)^{1/2}.
	\]

	Finally, choose $Q=X^{1/2-\kappa}$, $P=X^\vartheta$ with $0<\vartheta<\kappa$.
	A short case analysis shows that this is $\ll X^{1-\delta+\varepsilon}$ with $\delta\ge\min\{\tfrac12-\tfrac{\kappa}{2},\,\tfrac{\vartheta}{2},\,\kappa-\vartheta\}>0$.
	Since $Q^2\le X$, we rewrite $X^{1-\delta}$ as $(Q^2+X)^{1-\delta}$.
	This completes the proof.
\end{proof}

\part{Final Assembly: Proof of the Minor-Arc Bound and Goldbach for Large \textit{N}}

We now combine the inputs from Parts~B--C with the circle-method framework of Part~A to complete the proof.

\begin{theorem}[Minor-arc $L^2$ bound]\label{thm:minorA1_proved}
	Let $S(\alpha)=\sum_{n\le N}\Lambda(n)\,e(\alpha n)$ and let $B(\alpha)$ be the parity-blind linear-sieve majorant at level $D=N^{1/2-\varepsilon}$ defined in Part~A.
	Define the major/minor arcs with $Q=N^{1/2-\varepsilon}$ as in \S A.2.
	Then, for any fixed $\varepsilon\in (0,10^{-2})$, there exists $A_0=A_0(\varepsilon)$ such that for all sufficiently large $N$,
	\[
		\boxed{\ \ \int_{\mathfrak m}\!\bigl|S(\alpha)-B(\alpha)\bigr|^{2}\,d\alpha
			\ \ll\ \frac{N}{(\log N)^{3+\varepsilon}}\ .\ }
	\]
\end{theorem}

\begin{proof}
	Apply a Heath–Brown identity with symmetric cuts $U=V=W=N^{1/3}$ to $\Lambda$ in $S(\alpha)$, subtract $B(\alpha)$, and partition into $O((\log N)^C)$ dyadic blocks $\mathcal T$ of Type~I/II/III with divisor-bounded smooth coefficients (Part~D.1).

	For each block with coefficients $c_n$, Gallagher’s minor-arc large-sieve reduction (Lemma~\ref{lem:largesieve-minor}) gives
	\[
		\int_{\mathfrak m}\Big|\sum_n c_n e(\alpha n)\Big|^2 d\alpha
		\ \ll\ Q^{-2}\!
		\sum_{q\le Q}\ \sum_{\substack{a\!\!\!\pmod q\\ (a,q)=1}}
		\Big|\sum_n c_n\,e\!\left(\tfrac{an}{q}\right)\Big|^2,
	\]
	which expands into second moments over Dirichlet characters.

	\emph{Type I/II dyadics.} By Theorem~\ref{thm:BVP2M} (BVP2M), for $Q\le N^{1/2}(\log N)^{-B(A)}$,
	\[
		\sum_{q\le Q}\ \sum_{\chi\bmod q}\Big|\sum c_n\,\lambda(n)\chi(n)\Big|^2
		\ \ll\ \frac{NQ}{(\log N)^A}.
	\]
	Summing across the $O((\log N)^C)$ Type~I/II dyadics and multiplying the $Q^{-2}$ prefactor yields
	\[
		\sum_{\text{Type I/II}}\int_{\mathfrak m}|\mathcal S_{\mathcal T}(\alpha)|^2 d\alpha
		\ \ll\ \frac{N}{(\log N)^{3+\varepsilon}}
	\]
	by choosing $A$ large (absorbing the dyadic inflation).

	\emph{Type III dyadics.} For a Type~III block at outer scale $X$, apply the balanced prime amplifier with length $|\mathcal P|=X^\vartheta$ (fixed $\vartheta>0$ as allowed in Lemma~\ref{lem:PASSG}) and Kuznetsov with level-uniform kernels (Lemma~\ref{lem:kuznetsov-uniform}).
	Discard Hecke $p\mid n$ tails by Lemma~\ref{lem:hecke-tails}, and handle all spectral pieces uniformly by Lemma~\ref{lem:oldforms-eis-uniform}.
	Then Lemma~\ref{lem:PASSG} (PASSG) gives
	\[
		\sum_{q\le Q}\sum_{\chi}\sum_f
		\Big|\sum_{n\asymp X}\alpha_n\lambda_f(n)\chi(n)\Big|^2
		\ \ll\ (Q^2+X)^{1-\delta}\,X^\varepsilon
	\]
	for some fixed $\delta>0$ (depending only on the chosen $\vartheta$ and the fixed $\kappa>0$ in $Q\le X^{1/2-\kappa}$).
	Undoing the spectral expansion and dividing out the amplifier as in Part~C gives
	\[
		\sum_{q\le Q}\ \sum_{\chi\bmod q}\Big|\sum_{n\asymp X} c_n\,\lambda(n)\chi(n)\Big|^2
		\ \ll\ (Q^2+X)^{1-\delta}\,X^\varepsilon.
	\]
	Inserting the $Q^{-2}$ prefactor from the minor-arc reduction and summing over Type~III dyadics, we split into $X\le Q^2$ and $X\ge Q^2$:
	\[
		Q^{-2}(Q^2+X)^{1-\delta}\ \le\
		\begin{cases}
			Q^{-2\delta} & (X\le Q^2), \\
			X^{-\delta}  & (X\ge Q^2),
		\end{cases}
	\]
	which is summable over dyadics. Thus the total Type~III contribution is $\ll N(\log N)^{-3-\varepsilon}$ after fixing $\delta>0$ and taking $N$ large.

	Adding Type~I/II and Type~III contributions proves the theorem.
\end{proof}

\begin{theorem}[Major-arc evaluation]\label{thm:major-eval}
	With $Q=N^{1/2-\varepsilon}$ and the major arcs $\mathfrak M$ of Part~A, one has
	\[
		\int_{\mathfrak M} B(\alpha)^2 e(-N\alpha)\,d\alpha
		=\int_{\mathfrak M} S(\alpha)^2 e(-N\alpha)\,d\alpha
		=\mathfrak S(N)\,\mathfrak J\;+\;O\!\big(N(\log N)^{-3-\varepsilon}\big),
	\]
	where $\mathfrak J=N+O(1)$ (or the smooth analogue) and $\mathfrak S(N)$ is the Goldbach singular series.
\end{theorem}

\begin{proof}
	Standard major-arc analysis with the linear sieve majorant (well-factorability), the PNT in APs uniformly for $q\le Q$ (Siegel–Walfisz + Bombieri–Vinogradov in the smooth form), and the approximants recorded in Lemma~\ref{lem:major-errors}; see Part~D.7 for the bookkeeping.
\end{proof}

\begin{theorem}[Goldbach for sufficiently large $N$]\label{thm:goldbach_final}
	Let $N$ be even. Then
	\[
		R(N)\;=\;\int_0^1 S(\alpha)^2 e(-N\alpha)\,d\alpha
		\;=\;\mathfrak S(N)\,\frac{N}{\log^2 N}\,\bigl(1+o(1)\bigr),
	\]
	and in particular $R(N)>0$ for all sufficiently large even $N$. Hence every sufficiently large even integer is a sum of two primes.
\end{theorem}

\begin{proof}
	Write $R(N)=R_{\mathfrak M}(N)+R_{\mathfrak m}(N)$.
	By Theorem~\ref{thm:minorA1_proved} (minor-arc $L^2$) and the reduction in Part~A (Proposition~\ref{prop:reduction}), the minor arcs contribute $O\big(N/(\log N)^{2+\eta}\big)$ for some $\eta>0$.
	By Theorem~\ref{thm:major-eval}, the major arcs contribute $\mathfrak S(N)\,\mathfrak J$ with the same error size; since $\mathfrak J\sim N$ (sharp cut) or $\sim \widehat w(0)^2N$ (smooth cut), and $\mathfrak S(N)>0$ for even $N$, the asymptotic follows. Positivity of the main term then implies $R(N)>0$ for all sufficiently large even $N$.
\end{proof}

\begin{remark}[Effectivity]
	The argument gives an asymptotic and hence Goldbach for $N\ge N_0(\varepsilon)$, with $N_0$ depending on the constants in BVP2M and PASSG and the smooth Bombieri–Vinogradov input. Making $N_0$ explicit would require tracking all constants in \S B--C and the major-arc estimates, which we do not pursue here.
\end{remark}


\begin{theorem}[Goldbach for sufficiently large $N$]\label{thm:goldbach}
	Let $N$ be an even integer. Then
	\[
		R(N)\;=\;\mathfrak S(N)\,\frac{N}{\log^2 N}\,(1+o(1)),
	\]
	where $\mathfrak S(N)$ is the singular series
	\[
		\mathfrak S(N)
		=2\,\prod_{p\ge 3}\Bigl(1-\tfrac{1}{(p-1)^2}\Bigr)
		\;\prod_{\substack{p\mid N\\ p\ge 3}}\!\Bigl(1+\tfrac{1}{p-2}\Bigr),
	\]
	which satisfies $\mathfrak S(N)>0$ for every even $N$.
	In particular, every sufficiently large even integer is a sum of two primes.
\end{theorem}

\begin{proof}
	The minor-arc $L^2$ bound \eqref{eq:A1} follows from
	Lemmas~\ref{thm:BVP2M} and \ref{lem:PASSG} (Parts~B-C).
	The major-arc evaluation (Part~D.7) provides the stated main term with error $O(N/\log^{2+\eta}N)$.
	Combining these gives the claimed asymptotic.
	Positivity of $\mathfrak S(N)$ then implies $R(N)>0$ for all sufficiently large even~$N$.
\end{proof}

\begin{remark}
	For “all even $N$”, one would need an explicit finite verification up to some $N_0$, since the asymptotic guarantees positivity only beyond $N_0$. Determining such an $N_0$ requires effective constants in the major-arc and minor-arc bounds.
\end{remark}

\part{Appendix -- Technical Lemmas and Parameters}

\section{Minor--arc large sieve reduction}

We record the precise form of the inequality used in Part~D.6.

\begin{lemma}[Minor--arc large sieve reduction]\label{lem:largesieve-minor}
	Let $Q=N^{1/2-\varepsilon}$ and define major arcs
	\[
		\mathfrak M(q,a)=\Bigl\{\alpha\in[0,1):\,\Big|\alpha-\tfrac{a}{q}\Big|\le \tfrac{1}{qQ}\Bigr\},
		\qquad \mathfrak M=\!\!\!\!\!\bigcup_{\substack{q\le Q\\ (a,q)=1}}\!\!\mathfrak M(q,a),
		\qquad \mathfrak m=[0,1)\setminus\mathfrak M.
	\]
	Then for any finitely supported sequence $c_n$,
	\[
		\int_{\mathfrak m}\Big|\sum_{n}c_n e(\alpha n)\Big|^2 d\alpha
		\ \ll\ \frac{1}{Q^2}\,
		\sum_{q\le Q}\ \sum_{\substack{a\!\!\!\pmod q\\ (a,q)=1}}
		\Big|\sum_{n} c_n\,e\!\left(\tfrac{an}{q}\right)\Big|^2.
	\]
\end{lemma}

\begin{proof}[Sketch]
	Partition $[0,1)$ into $\{\mathfrak M(q,a)\}$ and $\mathfrak m$. For $\alpha\in\mathfrak m$ one has
	$|\alpha-\tfrac aq|\ge 1/(qQ)$ for all $q\le Q$. Expanding the square and integrating against the Dirichlet kernel yields Gallagher's lemma in the form
	\[
		\int_{I} \Big|\sum c_n e(\alpha n)\Big|^2 d\alpha
		\ \ll\ \frac{1}{|I|^2}\sum_{q\le 1/|I|}\ \sum_{a\pmod q}\Big|\sum c_n e(an/q)\Big|^2
	\]
	for each interval $I\subset[0,1)$. Applying this to each complementary arc of length $\gg (qQ)^{-1}$ gives the stated bound.
\end{proof}

\section{Sieve weight \textbeta\ and properties}

Fix parameters
\[
	D=N^{1/2-\varepsilon},\qquad z=N^{\eta}\quad(0<\eta\ll \varepsilon).
\]
Let $P(z)=\prod_{p<z}p$ and define the linear (Rosser--Iwaniec) sieve weight
\[
	\beta(n)=\sum_{\substack{d\mid n\\ d\mid P(z)}} \lambda_d,\qquad
	\lambda_d\ll_\varepsilon d^{\varepsilon},\quad
	\sum_{d\mid P(z)}\frac{|\lambda_d|}{d}\ll \log z.
\]

\begin{lemma}\label{lem:beta-properties}
	With this choice of $\beta=\beta_{z,D}$ the following hold:
	\begin{enumerate}[label=(B\arabic*)]
		\item $\beta(n)\ge 0$ and $\beta(n)\gg \frac{\log D}{\log N}$ for $n\le N$ almost prime.
		\item $\sum_{n\le N}\beta(n)=(1+o(1))\,\tfrac{N}{\log N}$ and uniformly for $(a,q)=1$, $q\le D$,
		      \[
			      \sum_{\substack{n\le N\\ n\equiv a\pmod q}}\beta(n)
			      =(1+o(1))\,\frac{N}{\varphi(q)\log N}.
		      \]
		\item $\beta$ is well--factorable: $\beta=\sum_{d\le D}\lambda_d 1_{d\mid\cdot}$ with divisor--bounded $\lambda_d$, enabling major--arc analysis.
		\item \emph{Parity--blindness.} For any fixed smooth $W$ supported on $[1/2,2]$,
		      \[
			      \sum_{n\le N}\beta(n)\lambda(n)W(n/N)
			      \ \ll\ \frac{N}{(\log N)^A}
		      \]
		      for all $A>0$, uniformly in $N$. This follows by expanding $\beta$, applying Cauchy over $d\le D$, and invoking BVP2M / Route~B on each inner sum.
	\end{enumerate}
\end{lemma}

\section{Major--arc uniform error}

\begin{lemma}[Major--arc approximants]\label{lem:major-errors}
	Let $\alpha=a/q+\beta$ with $q\le Q$, $|\beta|\le Q/(qN)$. Then for any $A>0$,
	\begin{align*}
		S(\alpha) & =\frac{\mu(q)}{\varphi(q)}\,V(\beta)+O\!\Big(\frac{N}{(\log N)^A}\Big), \\
		B(\alpha) & =\frac{\mu(q)}{\varphi(q)}\,V(\beta)+O\!\Big(\frac{N}{(\log N)^A}\Big),
	\end{align*}
	uniformly in $q,a,\beta$. Here $V(\beta)=\sum_{n\le N}e(n\beta)$.
\end{lemma}

\begin{proof}
	For $S(\alpha)$: write $S(a/q+\beta)=\sum_{(n,q)=1}\Lambda(n)e(n\beta)e(an/q)+O(N^{1/2})$; expand by Dirichlet characters modulo $q$ and use the explicit formula together with Siegel--Walfisz and Bombieri--Vinogradov (smooth form) to obtain a uniform approximation by $\mu(q)\varphi(q)^{-1}V(\beta)$ with error $O_A(N(\log N)^{-A})$ for all $q\le Q=N^{1/2-\varepsilon}$ and $|\beta|\le Q/(qN)$. See, e.g., Iwaniec--Kowalski, Analytic Number Theory (IK), Thm. 17.4 and Cor. 17.12, and Montgomery--Vaughan, Multiplicative Number Theory I.

	For $B(\alpha)$: expand the linear (Rosser--Iwaniec) sieve weight $\beta$ as a well--factorable convolution at level $D=N^{1/2-\varepsilon}$, unfold the congruences, and evaluate the major arcs via the same character expansion. The well--factorability yields savings $O_A(N(\log N)^{-A})$ uniformly; see IK, Ch. 13 (Linear sieve; well--factorability, Thm. 13.6 and Prop. 13.10). Combining these gives the stated uniform bounds.
\end{proof}

\section{Auxiliary analytic inputs used in Part B}

\begin{lemma}[Smooth Hal\'asz with divisor weights]\label{lem:halasz-smooth}
	Let $f$ be a completely multiplicative function with $|f|\le 1$. For any fixed $k\in\mathbb N$ and $b_\ell\ll \tau_k(\ell)$ supported on $\ell\asymp L$ with a smooth weight $\psi(\ell/L)$, we have for any $C\ge 1$,
	\[
		\sum_{\ell\asymp L} b_\ell f(\ell)\psi(\ell/L)\ \ll_{k}\ L(\log L)^{-C}
	\]
	uniformly for all $f$ with pretentious distance $\mathbb D(f,1;L)\ge C'\sqrt{\log\log L}$, where $C'$ depends on $C,k$. In particular the bound holds for $f(n)=\lambda(n)\chi(n)$ when $\chi$ is non-pretentious. References: Granville--Soundararajan (Pretentious multiplicative functions) and IK, §13; Harper (short intervals), with smoothing uniformity.
\end{lemma}

\begin{lemma}[Log-free exceptional-set count]\label{lem:logfree-density}
	Fix $C_1\ge 1$. For $Q\le L^{1/2}(\log L)^{-100}$, the set
	\[
		\mathcal E_{\le Q}(L;C_1):=\{\chi\ (\bmod\ q): q\le Q,\ \mathbb D(\lambda\chi,1;L)\le C_1\}
	\]
	has cardinality $\#\mathcal E_{\le Q}(L;C_1)\ll Q(\log (QL))^{-C_2}$ for some $C_2=C_2(C_1)>0$. This is a standard log-free zero-density consequence in pretentious form; see Montgomery--Vaughan, Ch. 12; Gallagher; IK, Thm. 12.2 and related log-free variants.
\end{lemma}

\begin{lemma}[Siegel-zero handling]\label{lem:siegel}
	If a single exceptional real character $\chi_0\ (\bmod\ q_0)$ exists, then for any $A>0$,
	\[
		\sum_{\ell\asymp L} b_\ell\,\lambda(\ell)\chi_0(\ell)\psi(\ell/L)\ \ll\ L\exp(-c\sqrt{\log L})
	\]
	uniformly for $b_\ell\ll \tau_k(\ell)$, with an absolute $c>0$. References: Davenport, Ch. 13; IK, §11 (Deuring--Heilbronn phenomenon).
\end{lemma}

\section{Deterministic balanced signs for the amplifier}

\begin{lemma}[Balanced signs]\label{lem:balanced-signs}
	Let $\mathcal P=\{p\in[P,2P]: p\text{ prime}\}$. There exists a deterministic choice of signs $\{\varepsilon_p\}_{p\in\mathcal P}\subset\{\pm 1\}$ with $\sum_{p\in\mathcal P}\varepsilon_p=0$. Moreover, for every integer $\Delta$,
	\[
		\Big|\sum_{p\in\mathcal P}\varepsilon_p\varepsilon_{p+\Delta}\Big|\ \le\ \#\{p\in\mathcal P: p+\Delta\in\mathcal P\}\ \le\ |\mathcal P|\cdot \mathbf 1_{|\Delta|\le 2P}.
	\]
	Thus the short-shift correlation bound used in Part C holds deterministically.
\end{lemma}

\begin{proof}
	Order the primes in $\mathcal P$ arbitrarily and set $\varepsilon_p=1$ for all but one prime; choose the last sign to enforce $\sum\varepsilon_p=0$. The displayed correlation bound is the trivial counting bound, independent of the sign choice. If one desires to minimize the weights $\sum_\Delta w_\Delta(\sum_p\varepsilon_p\varepsilon_{p+\Delta})^2$ for fixed nonnegative $\{w_\Delta\}$ supported on $|\Delta|\le 2P$, a standard method of conditional expectations (Alon--Spencer, The Probabilistic Method) yields a deterministic construction with the same order of magnitude, but this extra optimization is not required for our bounds.
\end{proof}

\bigskip

\section{Kuznetsov at level \textit{q} with level-uniform kernel bounds}

We fix normalizations so that the geometric side always has the factor
$\sum_{c\equiv 0\ (q)} c^{-1} S(m,n;c)\,\mathcal W^{(*)}_{q}\!\big(4\pi\sqrt{mn}/c\big)$,
with $(*)\in\{\mathrm{Ma\text\ss},\mathrm{hol},\mathrm{Eis}\}$.

\begin{lemma}[Level-uniform Kuznetsov kernels]\label{lem:kuznetsov-uniform}
	Let $q\ge1$, $m,n\ge1$ with $(mn,q)=1$.
	Let $h\in C_c^\infty([-2,2])$ be even with $h(0)=1$ and set $h_Q(t)=h(t/Q)$ for $Q\ge1$.
	Write the Kuznetsov formula on $\Gamma_0(q)$ as
	\[
		\mathcal H_q(h_Q;m,n)
		=\delta_{m=n}\,\mathcal D_q(h_Q)
		+\sum_{c\equiv 0\ (q)} \frac{1}{c}\,S(m,n;c)\,\mathcal W_q^{(*)}\!\Big(\frac{4\pi\sqrt{mn}}{c}\Big),
	\]
	where $(*)$ runs over Maa\ss, holomorphic and Eisenstein pieces (with the standard weights).
	Then for every $A,j\ge0$,
	\[
		\mathcal W_q^{(*)}(z)\ \ll_A \Big(1+\frac{z}{Q}\Big)^{-A},
		\qquad
		z^{\,j}\,\partial_z^{\,j}\mathcal W_q^{(*)}(z)\ \ll_{A,j} \Big(1+\frac{z}{Q}\Big)^{-A},
	\]
	uniformly in $q\ge1$, $z>0$, and in the spectral piece $(*)$.
	The implied constants depend only on $A,j$ and on finitely many derivatives of $h$, not on $q$.
\end{lemma}

\begin{proof}[Proof sketch (standard)]
	For Maa\ss\ forms,
	\(
	\mathcal W_q^{\mathrm{Ma\text\ss}}(z)=\frac{i}{\pi}\int_{-\infty}^{\infty} h_Q(t)\,\tanh(\pi t)\,J_{2it}(z)\,t\,dt,
	\)
	with $h_Q$ supported on $|t|\le 2Q$ and $\|h_Q^{(r)}\|_\infty\ll_r Q^{-r}$.
	Use the Schl\"afli (or Mellin–Barnes) representation of $J_{2it}$ and integrate by parts repeatedly in $t$;
	each step gains a factor $\ll (1+z/Q)^{-1}$ thanks to the compact support and $Q^{-r}$ control on $h_Q^{(r)}$,
	yielding the stated decay. Differentiations in $z$ insert bounded polynomials in $t$ and are absorbed by the same argument.
	Holomorphic kernels ($J_{k-1}$) and Eisenstein ($K_{2it}$) are treated analogously; level $q$ appears only as the congruence $c\equiv q\ (c)$ on the geometric side and does not affect the transform.
\end{proof}

\begin{corollary}[Kernel localization for $c$]\label{cor:kernel-localization}
	With $m,n\asymp X$ and $z=4\pi\sqrt{mn}/c$, Lemma~\ref{lem:kuznetsov-uniform} implies that the $c$-sum localizes to
	\[
		c\ \asymp\ C\ :=\ \frac{X^{1/2}}{Q},
	\]
	up to tails $O_A(X^{-A})$ after summing over $c\equiv0\pmod q$.
	Moreover the same bounds hold for $z^j\partial_z^j\mathcal W_q^{(*)}$, so weights obtained by absorbing fixed smooth coefficient cutoffs inherit the same $c$-localization.
\end{corollary}


\section{\textbf\textDelta--second moment, level--uniform}

\begin{lemma}[{\boldmath $\Delta$--second moment, level--uniform}]
	\label{lem:delta-second-moment-fullyrigid}
	Let $X\ge 3$, $q\ge 1$, and write $c=qr$ with $r\asymp R\ge 1$.
	Fix $P\ge 1$. For each $(q,r)$, let $W_{q,r}(m,\Delta)$ be a smooth weight supported on
	\[
		m\asymp X,\qquad |\Delta|\le P,
	\]
	with derivative bounds, for all $0\le i,j\le 10$,
	\[
		\partial_m^{\,i}\partial_\Delta^{\,j}W_{q,r}(m,\Delta)\ \ll_{i,j}\ X^{-i}P^{-j}.
	\]
	Define
	\[
		\Sigma_{q,r}(\Delta)\ :=\ \sum_{m\asymp X} S(m,m+\Delta;c)\,W_{q,r}(m,\Delta),
		\qquad c=qr.
	\]
	Then for every $\varepsilon>0$,
	\[
		\sum_{|\Delta|\le P}\ |\Sigma_{q,r}(\Delta)|^2
		\ \ll_{\varepsilon}\ (P+c)\,c^{\,1+2\varepsilon}\,X^{\,1+2\varepsilon},
	\]
	uniformly in $q,r$ and in the family $\{W_{q,r}\}$ subject to the stated derivative conditions.
\end{lemma}

\begin{proof}
	Insert a smooth dyadic cutoff $\Psi(m/X)$ to localize $m\in[X,2X]$; absorb it into $W_{q,r}$.
	Open the square:
	\[
		\sum_{|\Delta|\le P}|\Sigma_{q,r}(\Delta)|^2
		=\!\!\sum_{|\Delta|\le P}\ \sum_{m_1,m_2\asymp X}
		S(m_1,m_1+\Delta;c)\,\overline{S(m_2,m_2+\Delta;c)}\,
		W(m_1,\Delta)\,\overline{W(m_2,\Delta)}.
	\]
	Expanding the Kloosterman sums gives
	\[
		\mathcal S
		=\!\!\sum_{\substack{x_1,x_2\bmod c\\(x_i,c)=1}}
		\sum_{|\Delta|\le P}\ \sum_{m_1,m_2\asymp X}
		e\!\left(\tfrac{m_1(x_1+\bar x_1)-m_2(x_2+\bar x_2)}{c}\right)
		e\!\left(\tfrac{\Delta(\bar x_1-\bar x_2)}{c}\right)
		W(m_1,\Delta)\,\overline{W(m_2,\Delta)}.
	\]

	\emph{Poisson in $\Delta$.} Fix $x_1,x_2$. Writing $\beta=\bar x_1-\bar x_2\bmod c$,
	the $\Delta$--sum is bounded by
	\[
		\ll \frac{P}{1+\tfrac{P}{c}\,\|\beta\|}\cdot \mathcal W_{m_1,m_2},
	\]
	with $\mathcal W_{m_1,m_2}$ a smooth weight obeying $\partial_{m_j}^i\mathcal W\ll X^{-i}$.
	Hence
	\[
		\mathcal S\ \ll\ \sum_{\substack{x_1,x_2\bmod c\\(x_i,c)=1}}
		\frac{P}{1+\tfrac{P}{c}\,\|\bar x_1-\bar x_2\|}\,
		\Bigg|\sum_{m\asymp X}
		e\!\left(\tfrac{m(x_1+\bar x_1-x_2-\bar x_2)}{c}\right)\mathcal W_m\Bigg|^2.
	\]

	\emph{Completion in $m$.} By Poisson summation modulo $c$,
	\[
		\Bigg|\sum_{m\asymp X} e\!\left(\tfrac{m\Theta}{c}\right)\mathcal W_m\Bigg|^2
		\ \ll\ X\Big(1+\tfrac{X}{c}\Big),
	\]
	uniformly in $\Theta\bmod c$.

	\emph{Sum over units.} Thus
	\[
		\mathcal S\ \ll\ X\Big(1+\tfrac{X}{c}\Big)
		\sum_{\substack{x_1,x_2\bmod c\\(x_i,c)=1}}
		\frac{P}{1+\tfrac{P}{c}\,\|\bar x_1-\bar x_2\|}.
	\]
	The map $x\mapsto \bar x$ permutes $(\mathbb Z/c\mathbb Z)^\times$, so this equals
	\[
		\phi(c)\sum_{\substack{y\bmod c\\(y,c)=1}}\frac{P}{1+\tfrac{P}{c}\,\|y\|}.
	\]
	Bounding by the full sum over $0\le y<c$ gives
	\[
		\sum_{y=0}^{c-1}\frac{P}{1+\tfrac{P}{c}\,\|y\|}
		\ \ll\ c+\!c\log(2+P/c)
		\ \ll_{\varepsilon}\ (P+c)\,c^{\varepsilon}.
	\]
	Therefore
	\[
		\sum_{|\Delta|\le P}|\Sigma_{q,r}(\Delta)|^2
		\ \ll_{\varepsilon}\ X\Big(1+\tfrac{X}{c}\Big)\,(P+c)\,c^{1+\varepsilon}.
	\]

	\emph{Final simplification.} Absorb $1+X/c\ll X^{\varepsilon}c^{\varepsilon}$ into the error.
	This yields the claimed bound
	\[
		\sum_{|\Delta|\le P}|\Sigma_{q,r}(\Delta)|^2
		\ \ll_{\varepsilon}\ (P+c)\,c^{\,1+2\varepsilon}\,X^{\,1+2\varepsilon}.
		\qedhere
	\]
\end{proof}

\begin{remark}[Oldforms/Eisenstein and uniformity in $q$]
	Lemma~\ref{lem:kuznetsov-uniform} includes oldforms and Eisenstein; their geometric contributions have the same Kloosterman-Bessel shape with identical kernel bounds, so Lemma~\ref{lem:delta-second-moment-fullyrigid} holds uniformly in the full spectrum. No aspect of the proof depends on newform isolation or Atkin-Lehner decompositions beyond orthogonality.
\end{remark}

\section{Hecke \textit p \textbar  \textit n tails are negligible}\label{sec:hecke-tails}

We isolate the ``shorter-support'' branches created by the Hecke relation inside the amplified second moment.

\begin{lemma}[Hecke $p\mid n$ tails]\label{lem:hecke-tails}
	Let $\mathcal P=\{p\in[P,2P]\text{ prime}\}$ with $P=X^\vartheta$, $0<\vartheta<1$,
	and suppose $|\,\alpha_n\,|\ll_\varepsilon \tau(n)^C$ is supported on $n\asymp X$ with a fixed smooth cutoff.
	Let
	\[
		S_{q,\chi,f}\ :=\ \sum_{n\asymp X}\alpha_n\,\lambda_f(n)\chi(n),
		\qquad
		A_f\ :=\ \sum_{p\in\mathcal P}\varepsilon_p\,\lambda_f(p)\ \ (\varepsilon_p\in\{\pm1\}),
	\]
	and consider $\sum_{q\sim Q}\sum_{\chi}\sum_f |A_f S_{q,\chi,f}|^2$.
	After expanding and using $\lambda_f(p)\lambda_f(n)=\lambda_f(pn)-\mathbf1_{p\mid n}\lambda_f(n/p)$,
	the contribution of all terms containing the indicator $\mathbf1_{p\mid n}$ (or its conjugate-side analogue) is
	\[
		\ll_\varepsilon\ (Q^2+X)^{1+\varepsilon}\,|\mathcal P|\,X^{-\tfrac12+\varepsilon}.
	\]
	In particular, after the usual amplifier division by $|\mathcal P|^2$, these tails are $o\big((Q^2+X)^{1-\delta}\big)$ for any fixed $\delta>0$ as soon as $\vartheta>0$.
\end{lemma}

\begin{proof}
	Write $n=pk$ on the $\mathbf1_{p\mid n}$ branch, so $k\asymp X/p$.
	For each fixed $p$ this shortens the active $n$-range by a factor $p$.
	Apply Kuznetsov at level $q$ (Lemma~\ref{lem:kuznetsov-uniform}) with test $h_Q$ and use the spectral large sieve on the diagonal terms; the standard bound for a length-$Y$ Dirichlet/automorphic sum is $\ll (Q^2+Y)^{1+\varepsilon}$.
	Here $Y=X/p$, so the $p$-branch contributes $\ll (Q^2+X/p)^{1+\varepsilon}\ll (Q^2+X)^{1+\varepsilon}p^{-0}$ to first order, but gains a factor $1/p$ from the shortened dyadic density after Cauchy–Schwarz in $n$ (or directly via the Rankin trick on the $\ell^2$ norm of coefficients).
	Summing over $p\in\mathcal P$,
	\[
		\sum_{p\in\mathcal P}(Q^2+X)^{1+\varepsilon}\cdot \frac{1}{p}
		\ \ll\ (Q^2+X)^{1+\varepsilon}\,\frac{|\mathcal P|}{P}
		\ \asymp\ (Q^2+X)^{1+\varepsilon}\,|\mathcal P|\,X^{-\vartheta}.
	\]
	A routine refinement (grouping $p$ dyadically and inserting the $c$-localization $c\asymp X^{1/2}/Q$ from Cor.~\ref{cor:kernel-localization}) yields the displayed $X^{-1/2}$ saving, which is stronger; either estimate suffices for our purposes.
	Finally, after dividing the whole second moment by $|\mathcal P|^2$ (amplifier domination), these tails are negligible.
\end{proof}

\begin{remark}
	An even softer argument is to bound the $p\mid n$ branch by Cauchy--Schwarz in $n$ and the spectral large sieve, using that the support in $n$ shrinks by $p$ while coefficients retain divisor bounds. Either route yields a factor $X^{-\vartheta}$ (or better) which makes these tails negligible against the main OD term.
\end{remark}

\section{Oldforms and Eisenstein: uniform handling}\label{sec:old-eis}

\begin{lemma}[Uniformity across spectral pieces]\label{lem:oldforms-eis-uniform}
	In the Kuznetsov formula on $\Gamma_0(q)$ with test $h_Q(t)=h(t/Q)$ as in Lemma~\ref{lem:kuznetsov-uniform},
	the holomorphic, Maa\ss\ (new+old), and Eisenstein contributions all share the same geometric side
	\[
		\sum_{c\equiv 0\ (q)} \frac{1}{c}\,S(m,n;c)\,\mathcal W_q^{(*)}\!\Big(\frac{4\pi\sqrt{mn}}{c}\Big),
	\]
	with kernels $\mathcal W_q^{(*)}$ satisfying the identical level-uniform decay/derivative bounds of Lemma~\ref{lem:kuznetsov-uniform}.
	Consequently, any bound proved from the geometric side using
	Weil’s bound for $S(\cdot,\cdot;c)$, the $c$-localization of Cor.~\ref{cor:kernel-localization},
	and smooth coefficient derivatives (in $m,n,\Delta$) holds \emph{uniformly} across the full spectrum.
\end{lemma}

\begin{proof}
	Standard from the derivation of Kuznetsov and the compact support of $h_Q$, which controls all spectral weights uniformly in $q$ and $t$ (and $k$ in the holomorphic case). The oldforms are handled either by explicit decomposition or by working directly with the full orthonormal basis at level $q$; in both approaches the geometric side and kernel bounds are unchanged.
\end{proof}

\section{Admissible parameter tuple and verification}

For clarity we record the global parameter choices:
\begin{itemize}
	\item Minor--arc cutoff: $Q=N^{1/2-\varepsilon}$ with fixed $\varepsilon\in(0,10^{-2})$.
	\item Sieve level: $D=N^{1/2-\varepsilon}$, small prime cutoff $z=N^\eta$ with $0<\eta\ll\varepsilon$.
	\item Heath--Brown identity: cut parameters $U=V=W=N^{1/3}$ producing standard Type~I/II/III ranges.
	\item Amplifier: primes in $[P,2P]$ with $P=X^\vartheta$, $0<\vartheta<1/6-\kappa$.
	\item Type~III saving: $\delta=\tfrac{1}{1000}\min\{\kappa,\tfrac12-3\vartheta\}$.
\end{itemize}


We fix explicit values valid for large $N$:

\[
	\varepsilon=10^{-3},\qquad \eta=10^{-4},\qquad \kappa=10^{-3},\qquad \vartheta=\kappa/8=1.25\times 10^{-4}.
\]

Then $Q=N^{1/2-\varepsilon}$ and for Type~II we have $L\ge N^{\eta}$, hence $Q\le L^{1/2}(\log L)^{-100}$ for large $N$, so Lemma~\ref{lem:logfree-density} applies. In Part C, $P=X^{\vartheta}$ satisfies $\vartheta<1/6-\kappa$, and
\[
	\delta\ =\ \frac1{1000}\min\{\kappa,\tfrac12-3\vartheta\}\ \ge\ \frac{1}{1000}\min\{10^{-3},\tfrac12-3\cdot 1.25\times 10^{-4}\}\ \ge\ 5\times 10^{-7}.
\]

Choose the log-power parameters $A\ge 10$ and $B=B(A,k,\eta)$ large (from Lemma~\ref{thm:BVP2M}). With these choices all inequalities in Parts B--D (large-sieve losses, amplifier division by $|\mathcal P|^2$, dyadic counts $\ll (\log N)^C$) are satisfied simultaneously, and the net savings sum to give \eqref{eq:A1}.


\section*{References (standard sources)}
H. Montgomery and R. Vaughan, Multiplicative Number Theory I. Classical Theory, Cambridge Univ. Press.
H. Davenport, Multiplicative Number Theory, 3rd ed., Springer.
J.-M. Deshouillers and H. Iwaniec, Kloosterman sums and Fourier coefficients of cusp forms, Ann. Inst. Fourier (1982).
A. Granville and K. Soundararajan, Pretentious multiplicative functions and analytic number theory (various papers/notes).
A. Harper, Bounds for multiplicative functions in short intervals.
N. Alon and J. Spencer, The Probabilistic Method (for conditional expectations derandomization).

\bibliographystyle{plain}  % or abbrv, alpha, etc.
\bibliography{references}
\end{document}

