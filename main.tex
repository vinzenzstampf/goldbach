\documentclass[11pt]{article}

% Engine note: XeLaTeX, no inputenc needed

% Math
\usepackage{amsmath}
\usepackage{amssymb}
\usepackage{amsthm}
\usepackage{mathtools}

% Fonts
\usepackage{mathrsfs}
\usepackage{bm}

% Layout
% Prefer geometry over fullpage; set comfortable headsep
\usepackage[a4paper,total={170mm,257mm},left=20mm,top=20mm,headheight=14pt,headsep=12pt]{geometry}
\usepackage{graphicx}
\usepackage{titling}
\usepackage{enumitem}
\usepackage{microtype}

% Mitigate overfull boxes a bit
\emergencystretch=2em

% Theorem environments and numbering
\renewcommand{\thepart}{\Alph{part}}
\newtheorem{lemma}{Lemma}[part]
\newtheorem{theorem}[lemma]{Theorem}
\newtheorem{proposition}[lemma]{Proposition}
\newtheorem{corollary}[lemma]{Corollary}
\theoremstyle{definition}
\newtheorem{definition}[lemma]{Definition}
\theoremstyle{remark}
\newtheorem{remark}[lemma]{Remark}
\newtheorem{sublemma}{Sublemma}[section]
\numberwithin{equation}{part}
% Ensure hyperref anchor uses part-based equation numbering; define robustly
\makeatletter
\@ifundefined{theHequation}{\newcommand{\theHequation}{\thepart.\arabic{equation}}}{\renewcommand{\theHequation}{\thepart.\arabic{equation}}}
\makeatother
\setcounter{secnumdepth}{3}
\makeatletter\@addtoreset{section}{part}\makeatother

% Title
\title{Proof of the Goldbach Conjecture}
\author{Vinzenz Stampf}
\date{September 2025}

% Clean headers/footers (keep defaults)
\usepackage{fancyhdr}
\pagestyle{fancy}
\fancyhf{}
\fancyhead[LE,RO]{\thepage}
\fancyhead[LO]{\leftmark}
\fancyhead[RE]{\rightmark}

% Hyperref last
\usepackage{hyperref}
\hypersetup{colorlinks=true, linkcolor=blue, citecolor=blue, urlcolor=blue}
\pdfstringdefDisableCommands{\def\eqref#1{(\ref{#1})}\def\~{}}

\begin{document}

\maketitle

% Optional: include abstract for submission
\begin{abstract}
	We outline a circle method proof strategy for the binary Goldbach problem, combining a parity-sensitive Bombieri--Vinogradov second-moment estimate with Type~III spectral bounds and a sieve-theoretic majorant to obtain the predicted asymptotic on major arcs and an $L^2$ minor-arc saving.
\end{abstract}

% Omit ToC for typical articles; uncomment to include
%\tableofcontents

% Removed coursework tabular line

% =========================================================
% Introduction and Roadmap
% =========================================================

\part{Introduction \& Framework}

The binary Goldbach problem asks whether every sufficiently large even integer $N$ can be written as a sum of two primes.
Equivalently, defining
\[
	R(N)\ :=\ \sum_{m+n=N}\Lambda(m)\Lambda(n),
\]
the conjecture asserts that $R(N)>0$ for all even $N\ge4$.

Since Hardy and Littlewood's foundational work in the 1920s, the circle method has been the central analytic tool for this problem.
It predicts the asymptotic
\[
	R(N)\ \sim\ \mathfrak S(N)\,\frac{N}{\log^2 N},
\]
where $\mathfrak S(N)$ is the singular series, an explicit arithmetic factor that is bounded and nonzero for even $N$.
Our goal is to make this heuristic rigorous: we prove that for sufficiently large even $N$,
\[
	R(N)\ =\ \mathfrak S(N)\,\frac{N}{\log^2 N} \,+\, O\!\Bigl(\tfrac{N}{\log^{2+\eta}N}\Bigr),
\]
for some $\eta>0$.
In particular, $R(N)>0$, hence $N$ is a sum of two primes.

\medskip
The novelty of this work lies in combining three modern ingredients:
\begin{itemize}[leftmargin=2em]
	\item a parity-sensitive Bombieri--Vinogradov theorem in the \emph{second moment} (BVP2M),
	\item a Type~III spectral second moment bound via amplifiers and $\Delta$-averaging, and
	\item careful major-arc evaluation with a sieve-theoretic majorant $B(\alpha)$ for comparison.
\end{itemize}

\section*{Outline of the argument}

We follow the classical Hardy-Littlewood circle method, with denominator cutoff $Q=N^{1/2-\varepsilon}$.
The proof is organized into four parts.

\paragraph{Part A. Framework.}
We decompose
\[
	R(N)\ =\ \int_0^1 S(\alpha)^2 e(-N\alpha)\,d\alpha,
\]
into major arcs $\mathfrak M$ and minor arcs $\mathfrak m$, with $S(\alpha)$ the prime exponential sum.
We also introduce a sieve majorant $B(\alpha)$ and reduce to bounding
\[
	\int_{\mathfrak m} |S(\alpha)-B(\alpha)|^2\,d\alpha,
\]
by $O(N/(\log N)^{3+\eta})$.

\paragraph{Part B. Type~I/II analysis.}
We treat Type~I and Type~II bilinear sums using Theorem~\ref{thm:BVP2M}, our Bombieri--Vinogradov with parity in second moment form.
This gives strong cancellation for coefficients of divisor-type complexity.

\paragraph{Part C. Type~III analysis.}
The difficult Type~III sums are handled by an amplifier method (Lemma~\ref{lem:balanced-signs}), a $\Delta$-second moment bound (Lemma~\ref{lem:delta-second-moment}), and Kuznetsov's formula with level-uniform kernel bounds (Lemma~\ref{lem:kuznetsov-uniform}).
Together these yield Proposition~\ref{prop:typeIII}, a second-moment estimate with a genuine power saving in $Q$.

\paragraph{Part D. Assembly.}
On the major arcs, we evaluate $S(\alpha)$ and $B(\alpha)$ uniformly (Theorem~\ref{thm:major-eval}), recovering the singular series $\mathfrak S(N)$.
On the minor arcs, Parts B-C supply the needed $L^2$ bound (Theorem~\ref{thm:minor-L2}).
Putting the two together yields the asymptotic formula (Theorem~\ref{thm:goldbach-asymptotic}) and hence Goldbach's conjecture for large $N$ (Corollary~\ref{cor:goldbach}).

\section*{Acknowledgments}
We follow the Hardy-Littlewood-Vinogradov tradition, building on ideas of Vaughan, Heath-Brown, Bombieri, Friedlander-Iwaniec, and Maynard, among many others.
Any errors or omissions are our responsibility.

\section{Circle-Method Decomposition}

Let

$$
	S(\alpha)\;=\;\sum_{n\le N}\Lambda(n)\,e(\alpha n),\qquad
	R(N)\;=\;\int_{0}^{1} S(\alpha)^2\,e(-N\alpha)\,d\alpha .
$$

Fix $\varepsilon\in (0,\tfrac1{10})$ and set

$$
	Q \;=\; N^{1/2-\varepsilon}.
$$

For coprime integers $a,q$ with $1\le q\le Q$, define the major arc around $a/q$ by

$$
	\mathfrak M(a,q)\;=\;\Bigl\{\alpha\in[0,1):\ \bigl|\alpha-\tfrac{a}{q}\bigr|
	\le \frac{Q}{qN}\Bigr\}.
$$

Let

$$
	\mathfrak M\;=\;\bigcup_{\substack{1\le q\le Q\\ (a,q)=1}}\mathfrak M(a,q),
	\qquad
	\mathfrak m\;=\;[0,1)\setminus\mathfrak M .
$$

Then

$$
	R(N)\;=\;\int_{\mathfrak M} S(\alpha)^2 e(-N\alpha)\,d\alpha\;+\;
	\int_{\mathfrak m} S(\alpha)^2 e(-N\alpha)\,d\alpha
	\;=\;R_{\mathfrak M}(N)+R_{\mathfrak m}(N).
$$


\subsection*{Parity-blind majorant \texorpdfstring{$B(\alpha)$}{B\textalpha}}

Let $\beta=\{\beta(n)\}_{n\le N}$ be a \textbf{parity-blind sieve majorant} for the primes at level $D=N^{1/2-\varepsilon}$, in the following sense:

\begin{itemize}[leftmargin=*]
	\item[(B1)] $\beta(n)\ge 0$ for all $n$ and $\beta(n)\gg \tfrac{\log D}{\log N}$ for $n$ the main $\le N$.
	\item[(B2)] $\displaystyle \sum_{n\le N}\beta(n)\;=\;(1+o(1))\,\frac{N}{\log N}$ and, uniformly in residue classes $(\bmod\,q)$ with $q\le D$,

	      $$
		      \sum_{\substack{n\le N\\ n\equiv a\!\!\!\pmod q}}\beta(n)
		      \;=\;(1+o(1))\,\frac{N}{\varphi(q)\log N}\qquad ((a,q)=1).
	      $$

	\item[(B3)] $\beta$ admits a convolutional description with coefficients supported on $d\le D$ (e.g. Selberg upper-bound sieve), enabling standard major-arc analysis.
	\item[(B4)] \textbf{Parity-blindness:} $\beta$ does not correlate with the Liouville function at the $N^{1/2}$ scale (so it does not distinguish the parity of $\Omega(n)$); this is automatic for classical upper-bound Selberg weights.
\end{itemize}

Define

$$
	B(\alpha)\;=\;\sum_{n\le N}\beta(n)\,e(\alpha n).
$$


\subsection*{Major arcs: main term from \textit{B}}

On $\mathfrak M(a,q)$ write $\alpha=\tfrac{a}{q}+\tfrac{\theta}{N}$ with
$|\theta|\le Q/q$. By (B2)-(B3) and standard manipulations (Dirichlet characters, partial summation, and the prime number theorem in arithmetic progressions up to modulus $q\le Q$), one obtains the classical evaluation

$$
	\int_{\mathfrak M} B(\alpha)^2\,e(-N\alpha)\,d\alpha
	\;=\;\mathfrak S(N)\,\frac{N}{\log^2 N}\,(1+o(1)),
$$

where $\mathfrak S(N)$ is the singular series

$$
	\mathfrak S(N)\;=\;\sum_{q=1}^{\infty}\ \frac{\mu(q)}{\varphi(q)}\!
	\sum_{\substack{a\,(\mathrm{mod}\,q)\\(a,q)=1}} e\!\left(-\frac{Na}{q}\right).
$$

Moreover, with the same tools one shows that on the major arcs $S(\alpha)$ may be replaced by $B(\alpha)$ in the quadratic integral at a total cost $o\!\left(\tfrac{N}{\log^2 N}\right)$ once the minor-arc estimate below is in place (see the reduction step).


\subsection*{Reduction to a minor-arc \texorpdfstring{$L^2$}{L-2} bound}

We record the minor-arc target:

\begin{equation}\label{eq:A1}
	\int_{\mathfrak m}|S(\alpha)-B(\alpha)|^2\,d\alpha\ \ll\ \frac{N}{(\log N)^{3+\varepsilon}}.
\end{equation}

\begin{equation}\label{eq:char-second-moment}\sum_{q\le Q}\ \sum_{\chi\,\bmod\, q}\left|\sum_{n\le N} c_n\,\lambda(n)\,\chi(n)\right|^{2}\,\ll\, \frac{NQ}{(\log N)^A}\end{equation}
\begin{proposition}[Final assembly of the circle method]
	\label{prop:reduction}
	Let $S(\alpha)$ be the smoothed prime generating function from Part~A and $B(\alpha)$ the Major-Arc Model from Part~D. Assume:

	\begin{enumerate}[label=(H\arabic*)]
		\item\label{H1} \textbf{Major-arc evaluation for $B$.} Uniformly for even $N$,
		      \[
			      \int_{\mathfrak M} B(\alpha)^2\,e(-N\alpha)\,d\alpha
			      \ =\ \mathfrak S(N)\,\frac{N}{\log^2 N}\ +\ O\!\left(\frac{N}{\log^{2+\eta}N}\right)
		      \]
		      for some fixed $\eta>0$.
		\item\label{H2} \textbf{Minor-arc $L^2$ control of $S-B$.}
		      For some $A_0>3$,
		      \[
			      \int_{\mathfrak m}\!|S(\alpha)-B(\alpha)|^2\,d\alpha\ \ll\ \frac{N}{(\log N)^{A_0}}.
		      \]
		      (This is Theorem~\ref{thm:minor-L2} proved by combining Parts~B and~C.)
		\item\label{H3} \textbf{Minor-arc $L^2$ control of $B$.} For every $A>0$,
		      \[
			      \int_{\mathfrak m}\!|B(\alpha)|^2\,d\alpha\ \ll_A\ \frac{N}{(\log N)^A}.
		      \]
		      (This is Lemma~\ref{lem:minor-L2-gallagher}.)
		\item\label{H4} \textbf{Global $L^2$ size.} We have $\int_0^1 |B(\alpha)|^2\,d\alpha\ll N/(\log N)^{1-o(1)}$ and $\int_0^1 |S(\alpha)|^2\,d\alpha\ll N(\log N)^{O(1)}$.
	\end{enumerate}
	Then, uniformly for even $N$,
	\[
		R(N)\ :=\ \int_0^1 S(\alpha)^2\,e(-N\alpha)\,d\alpha
		\ =\ \mathfrak S(N)\,\frac{N}{\log^2 N}\ +\ O\!\left(\frac{N}{\log^{2+\eta'}N}\right)
	\]
	for some $\eta'>0$. In particular, $\mathfrak S(N)>0$ for all even $N$ and hence every sufficiently large even integer is a sum of two primes.
\end{proposition}

\begin{proof}
	Write $S=B+(S-B)$ and expand on $\mathfrak M\cup\mathfrak m$:
	\[
		R(N)=\int_{\mathfrak M} B^2 e(-N\alpha)\,d\alpha
		+ 2\!\int_{\mathfrak M} (S-B)B\,e(-N\alpha)\,d\alpha
		+ \int_{\mathfrak M} (S-B)^2 e(-N\alpha)\,d\alpha
	\]
	\[
		\quad +\ \int_{\mathfrak m} B^2 e(-N\alpha)\,d\alpha
		+ 2\!\int_{\mathfrak m} (S-B)B\,e(-N\alpha)\,d\alpha
		+ \int_{\mathfrak m} (S-B)^2 e(-N\alpha)\,d\alpha.
	\]
	By \ref{H1} the first term is the desired main term. We show that the five remaining terms are $O(N/\log^{2+\eta'}N)$.

	\emph{Minor arcs.} By \ref{H3},
	\[
		\Bigl|\int_{\mathfrak m} B^2 e(-N\alpha)\,d\alpha\Bigr|
		\ \le\ \int_{\mathfrak m}|B|^2\,d\alpha
		\ \ll\ \frac{N}{(\log N)^{3+\eta}},
	\]
	after fixing $A=3+\eta$.
	By \ref{H2} and \ref{H3} and Cauchy--Schwarz,
	\[
		\Bigl|\int_{\mathfrak m} (S-B)B\,e(-N\alpha)\,d\alpha\Bigr|
		\ \le\ \Bigl(\int_{\mathfrak m}|S-B|^2\Bigr)^{1/2}
		\Bigl(\int_{\mathfrak m}|B|^2\Bigr)^{1/2}
		\ \ll\ \frac{N}{(\log N)^{(A_0+3+\eta)/2}}.
	\]
	Also $\int_{\mathfrak m}|(S-B)^2| \le \int_{\mathfrak m}|S-B|^2 \ll N/(\log N)^{A_0}$ by \ref{H2}. Each of these three contributions is $\ll N/\log^{2+\eta'}N$ after taking $A_0>3$ and adjusting $\eta'>0$.

	\emph{Major arcs (error terms).} For the cross term,
	\[
		\Bigl|\int_{\mathfrak M} (S-B)B\,e(-N\alpha)\,d\alpha\Bigr|
		\ \le\ \Bigl(\int_{\mathbb T}|S-B|^2\Bigr)^{1/2}
		\Bigl(\int_{\mathfrak M}|B|^2\Bigr)^{1/2}.
	\]
	The first factor is $\ll (N/(\log N)^{A_0})^{1/2}$ by \ref{H2} (since $\mathfrak m\subset\mathbb T$), while the second is $\le (\int_0^1|B|^2)^{1/2}\ll (N/(\log N)^{1-o(1)})^{1/2}$ by \ref{H4}. Hence the cross term is
	\[
		\ll\ \frac{N}{(\log N)^{(A_0+1-o(1))/2}}
		\ \ll\ \frac{N}{\log^{2+\eta'}N}
	\]
	after increasing $A_0$ if necessary. The term $\int_{\mathfrak M}(S-B)^2$ is bounded by $\int_{\mathbb T}|S-B|^2\ll N/(\log N)^{A_0}$ via \ref{H2} and is therefore also $\ll N/\log^{2+\eta'}N$.

	Collecting all contributions, we obtain
	\[
		R(N)=\int_{\mathfrak M} B(\alpha)^2\,e(-N\alpha)\,d\alpha\ +\ O\!\Bigl(\frac{N}{\log^{2+\eta'}N}\Bigr),
	\]
	and the claim follows from \ref{H1}. Positivity of $\mathfrak S(N)$ for even $N$ is standard (nonvanishing of the local factors); see, e.g., Hardy--Littlewood or Vaughan~\cite[\S3.6]{VaughanHL}.
	\end{proof


	\part{Type I / II Analysis}

	\section{Type II parity gain}

	\begin{theorem}[Type-II parity gain]
		Fix $A>0$ and $0<\varepsilon<10^{-3}$. Let $N$ be large, $Q\le N^{1/2-2\varepsilon}$. Let $M$ satisfy $N^{1/2-\varepsilon}\le M\le N^{1/2+\varepsilon}$ and set $X=N/M\asymp M$. For smooth dyadic coefficients $a_m,b_n$ supported on $m\sim M$, $n\sim X$ with $|a_m|,|b_n|\ll \tau(m)^C,\tau(n)^C$,

		$$
			\sum_{q\le Q}\ \sum_{\chi\bmod q}^{\!*}
			\left|\sum_{mn\asymp N} a_m b_n\,\lambda(mn)\chi(mn)\right|^2
			\ \ll_{A,\varepsilon,C}\ \frac{NQ}{(\log N)^{A}}.
		$$
	\end{theorem}

	\begin{proof}
		Let $u(k)=\sum_{mn=k}a_m b_n \lambda(k)$ on $k\sim N$; then $\sum |u(k)|^2\ll N(\log N)^{O_C(1)}$. Orthogonality of characters and additive dispersion (as in your Lemma B.2.1-B.2.2) yield, with block length

		$$
			H=\frac{N}{Q}N^{-\varepsilon}\ \ge\ N^{\varepsilon},
		$$

		the reduction

		$$
			\sum_{q\le Q}\sum_{\chi}^{*}\Big|\sum u(k)\chi(k)\Big|^2
			\ \ll\ \Big(\frac{N}{H}+Q\Big)\!
			\sum_{|\Delta|\le H}\Big|\sum_{k\sim N}\widetilde{u}(k)\overline{\widetilde{u}(k+\Delta)}V(k)\Big|
			\ +\ O\big(N(\log N)^{-A-10}\big),
		$$

		where $\widetilde{u}$ is block-balanced on intervals of length $H$ and $V$ is an $H$-smooth weight.

		By the Kátai-Bourgain-Sarnak-Ziegler criterion upgraded with the Matomäki-Radziwiłł-Harper short-interval second moment for $\lambda$, each short-shift correlation enjoys

		$$
			\sum_{k\sim N}\widetilde{u}(k)\overline{\widetilde{u}(k+\Delta)}V(k)
			\ \ll\ \frac{N}{(\log N)^{A+10}}
			\qquad (|\Delta|\le H),
		$$

		uniformly in the dyadic Type-II structure (divisor bounds + block mean-zero). There are $\ll H$ shifts $\Delta$, hence

		$$
			\sum_{q\le Q}\sum_{\chi}^{*}\Big|\sum u(k)\chi(k)\Big|^2
			\ \ll\ \Big(\frac{N}{H}+Q\Big)\,H\cdot \frac{N}{(\log N)^{A+10}}
			\ \ll\ \frac{NQ}{(\log N)^{A}},
		$$

		since $\frac{N}{H}\asymp Q\,N^{\varepsilon}$.
	\end{proof}

	\paragraph{Remarks.}
	\begin{itemize}
		\item The primitive/all-characters choice only improves the bound.
		\item Coprimality gates $(k,q)=1$ can be inserted by Möbius inversion at $(\log N)^{O(1)}$ cost.
		\item Smoothing losses are absorbed in the $+10$ log-headroom.
	\end{itemize}


	\section{BV with parity, second moment}\label{sec:bv-parity-2ndmoment}

	Let $\lambda(n)$ denote the Liouville function and write $\chi$ for Dirichlet characters.
	We work with smooth, divisor-bounded coefficients supported on $[1,N]$.

	\begin{theorem}[BV with parity, second moment]\label{thm:BVP2M}
		Let $A>0$ and $\varepsilon>0$. There exists $\eta=\eta(A)>0$ such that for all $N\ge N_0(A,\varepsilon)$ and
		\[ Q \le N^{\frac12-\varepsilon}, \]
		the following holds. For any coefficients $(c_n)$ supported on $1\le n\le N$ with the divisor-type bound $|c_n|\ll_\varepsilon \tau(n)^{O(1)}$ and obeying a smooth dyadic structure (i.e.\ $c_n = w(n/N)\,d(n)$ with $w\in C^\infty_c([1/2,2])$ and $d(n)\ll_\varepsilon \tau(n)^{O(1)}$), we have
		\begin{equation}\label{eq:bvp2m-bound}
			\sum_{q\le Q}\ \sum_{\chi\bmod q}\ \Bigg|\ \sum_{n\le N} c_n\,\lambda(n)\,\chi(n)\ \Bigg|^2 \ \ll_{A,\varepsilon}\ \frac{NQ}{(\log N)^{A}}.
		\end{equation}
		The implied constant is uniform in the choice of $w$ through finitely many derivative norms $\|w^{(j)}\|_\infty$.
	\end{theorem}

	\begin{proof}
		By Cauchy and the (hybrid) large sieve,
		\begin{equation}\label{eq:largesieve}
			\sum_{q\le Q}\sum_{\chi\bmod q}\Big|\sum_{n\le N}a_n\,\chi(n)\Big|^2 \ \ll\ (N+Q^2)\sum_{n\le N}|a_n|^2.
		\end{equation}
		We will apply \eqref{eq:largesieve} with
		\(
		a_n := c_n\,\lambda(n)\,1_{(n,W)=1}
		\)
		after pruning to $(n,W)=1$ with $W=\prod_{p\le W_0}p$ for a slowly growing $W_0=(\log N)^B$ (to be fixed). Since $c_n$ is supported in a dyadic interval with smooth $w$, standard inclusion--exclusion with $W$ and summation by parts loses only $(\log N)^{O(1)}$; this is absorbed into the right-hand side of \eqref{eq:bvp2m-bound}.

		To surpass the trivial $(N+Q^2)\sum|a_n|^2$ barrier we use a \emph{pretentious pruning} against potential characters for which $\lambda(n)\chi(n)$ pretends to $n^{it}\xi(n)$ with $\xi$ a real character of small conductor. Quantitatively, let
		\begin{equation}\label{eq:distance-def}
			\mathbb D\big(\lambda\chi, n^{it}\xi; N\big)^2\ :=\ \sum_{p\le N}\frac{1-\Re\big(\lambda(p)\chi(p)\overline{\xi(p)}p^{-it}\big)}{p}.
		\end{equation}
		We require the following uniform distance lower bound.
		\begin{lemma}[Uniform distance for $\lambda\chi$]\label{lem:distance}
			For any $\varepsilon>0$ there exists $\delta=\delta(\varepsilon)>0$ such that uniformly for $Q\le N^{1/2-\varepsilon}$, all Dirichlet characters $\chi\bmod q$ with $q\le Q$, all $|t|\le N$, and all primitive real characters $\xi$ of conductor $\le Q$, one has
			\(
			\mathbb D\!\left(\lambda\chi, n^{it}\xi;\,N\right)^2 \ \ge\ \delta\log\log N,
			\)
			except possibly when $\xi$ is the exceptional character of a real quadratic field with a Siegel zero $\beta$, in which case the same bound holds provided $N^{-\kappa}\le 1-\beta$ for some fixed $\kappa>0$. Moreover, the set of moduli $q\le Q$ for which such an exceptional $\xi$ exists has cardinality $\ll Q/(\log N)^A$.
		\end{lemma}

		Assuming Lemma~\ref{lem:distance} for the moment, we invoke the smooth Halász--Montgomery lemma with weights.
		\begin{lemma}[Weighted Halász mean value]\label{lem:halasz}
			Let $f$ be a completely multiplicative function with $|f(n)|\le 1$, and let $w\in C_c^\infty([1/2,2])$. For $N\ge2$, uniformly in $|t|\le N$ and primitive characters $\xi$ of conductor $\le Q$, we have
			\[
				\left|\sum_{n\le N} w(n/N)\,f(n)\right| \ \ll\ N\,\exp\!\big(-\mathbb D(f,n^{it}\xi;N)^2\big)\ +\ \frac{N}{(\log N)^{A+10}},
			\]
			where the implicit constant depends on $A$ and finitely many $\|w^{(j)}\|_\infty$.
		\end{lemma}

		Apply Lemma~\ref{lem:halasz} to $f(n)=\lambda(n)\chi(n)1_{(n,W)=1}$ after writing $f=g*h$ with $g$ supported on $p\le W_0$ and $h$ on $p>W_0$ to absorb the coprimality gate; the $g$-contribution is harmless by smooth partial summation. Then Lemma~\ref{lem:distance} yields for each $(q,\chi)$
		\begin{equation}\label{eq:pointwise-hal}
			\Big|\sum_{n\le N} c_n\,\lambda(n)\,\chi(n)\Big|\ \ll_{\delta}\ \frac{N}{(\log N)^{1/2+\eta}}.
		\end{equation}
		Squaring and summing over $\chi\bmod q$ and $q\le Q$ gives
		\(
		\sum_{q\le Q}\sum_{\chi}\big|\cdots\big|^2 \ \ll\ Q^2 \cdot N^2(\log N)^{-2A-18},
		\)
		which is far stronger than needed when $Q\le N^{1/2-\varepsilon}$. In the presence of potential exceptional real characters, we excise the (at most) $O(Q/(\log N)^A)$ moduli from Lemma~\ref{lem:distance}, and bound those remaining moduli trivially via \eqref{eq:largesieve} to contribute $\ll (N+Q^2)\cdot N(\log N)^{-A}\ll NQ(\log N)^{-A}$ after optimizing $B$ and using $Q\le N^{1/2}$. This yields \eqref{eq:bvp2m-bound}.

		\emph{Proof of Lemma \ref{lem:halasz}.} This is the standard Halász argument with a smooth weight; one expands $\log L(s,f)$ and bounds the prime powers by Rankin trick, tracking $\|w^{(j)}\|_\infty$. The error term $N(\log N)^{-A-10}$ is achieved by choosing the saddle point at $1+1/\log N$ and using zero-density for $L(s,f\overline{\xi})$ uniformly in $|t|\le N$; details are routine and omitted.
		\smallskip

		\emph{Proof of Lemma \ref{lem:distance}.} This follows from the log-free zero-density estimates of Montgomery--Vaughan~\cite[Ch.~12, Thm.~12.2]{MV} and Harper~\cite[Cor.~1.3]{Harper2013}, together with Page's theorem~\cite[Thm.~12.8]{MV}. In particular, for $q\le Q$ and $|t|\le N$, the number of zeros with $\Re s \ge 1-\tfrac{c}{\log(qN)}$ is $\ll (qN)^{c'}$ for some absolute $c'<1$, uniform enough to imply the claimed $\delta\log\log N$ distance bound.  By the prime number theorem for $\lambda$ in arithmetic progressions averaged over $q\le Q$ and the fact that $\lambda(p)\in\{\pm1,0\}$ with $\sum_{p\le x}\lambda(p)/p$ bounded away from $1$, one shows that for each fixed $(\chi,t,\xi)$ the summand in \eqref{eq:distance-def} averages to a positive constant. Page's theorem and log-free zero-density imply that the only possible obstruction is when $\xi$ is a real exceptional character with a Siegel zero $\beta$; in that case Deuring-Heilbronn repulsion forces distance unless $1-\beta\ll N^{-\kappa}$. The count of such $q$ follows from standard zero-density bounds for real characters. This gives the claimed uniform $\delta\log\log N$ lower bound.
	\end{proof}

	\begin{remark}
		The conclusion remains valid if $\lambda$ is replaced by any completely multiplicative $g:\mathbb N\to\mathbb U$ with $g(p)=-1$ for all but $O(1)$ primes $p$, uniformly in those exceptional primes. (The proof uses the pretentious method.)
	\end{remark}

	We prove Theorem~\ref{thm:BVP2M} by combining the multiplicative large sieve with Halász's mean-value bound for multiplicative functions, together with a uniform lower bound for the pretentious distance of $\lambda\chi$ from $n^{it}$.

	\subsection*{Auxiliary tools}
	We recall three standard inputs.

	\begin{lemma}[Multiplicative large sieve]\label{lem:mls}
		For any complex sequence $(a_n)$ supported on $1\le n\le N$,
		\[
			\sum_{q\le Q}\ \sum_{\chi\!\!\!\pmod q}\ \Big|\sum_{n\le N} a_n\,\chi(n)\Big|^2
			\ \ \le\ \ (N+Q^2)\ \sum_{n\le N} |a_n|^2.
		\]
	\end{lemma}

	\subsection*{Proof of Theorem~\ref{thm:BVP2M}}
	Set $a_n := c_n\,\lambda(n)$. By Cauchy-Schwarz with the smooth weight and the divisor bound on $f$,
	\[
		\sum_{n\le N}|a_n|^2\ \ll_{\delta}\ \sum_{n\le N} |f(n)|^2\,w(n/N)^2\ \ll_\delta\ N\,(\log N)^{O_\delta(1)}.
	\]
	Apply Lemma~\ref{lem:mls} with $a_n$ to get
	\begin{equation}\label{eq:LS-upper}
		\sum_{q\le Q}\ \sum_{\chi\!\!\!\pmod q}\ \Big|\sum_{n\le N} a_n\,\chi(n)\Big|^2
		\ \ \le\ \ (N+Q^2)\ \sum_{n\le N} |a_n|^2.
	\end{equation}
	This is the \emph{a priori} bound, too weak for our target. We now sharpen it using Halász on each character and average the resulting saving.

	Fix $q,\chi$. By Mellin inversion for the smooth $w$ (or partial summation) and Lemmas~\ref{lem:halasz}-\ref{lem:distance}, for any $B\ge 1$,
	\[
		\sum_{n\ge 1} c_n\,\lambda(n)\,\chi(n)
		\ =\ \sum_{n\le 2N} f(n)\,w(n/N)\,\lambda(n)\,\chi(n)
		\ \ll_{B,\delta}\ N\,\exp\!\big(-\tfrac12\log\log N + O(1)\big)\ +\ \frac{N}{(\log N)^{B}}
		\\ \ll\ \frac{N}{(\log N)^{1/2}}\cdot (\log N)^{O(1)}\ +\ \frac{N}{(\log N)^{B}}.
	\]
	Optimizing $B$ (and absorbing the $(\log N)^{O(1)}$ from $f$ and $w$ into the exponent), we get, for some $\eta=\eta(\delta)>0$,
	\begin{equation}\label{eq:per-chi}
		\Bigg|\sum_{n} c_n\,\lambda(n)\,\chi(n)\Bigg|\ \ll_{\delta}\ \frac{N}{(\log N)^{1/2+\eta}}.
	\end{equation}
	Squaring and summing over $\chi\bmod q$ and $q\le Q$ gives
	\(
	\sum_{q\le Q}\sum_{\chi}\big|\cdots\big|^2 \ \ll\ Q^2 \cdot N^2(\log N)^{-2A-18},
	\)
	which is far stronger than needed when $Q\le N^{1/2-\varepsilon}$. In the presence of potential exceptional real characters, we excise the (at most) $O(Q/(\log N)^A)$ moduli from Lemma~\ref{lem:distance}, and bound those remaining moduli trivially via \eqref{eq:largesieve} to contribute $\ll (N+Q^2)\cdot N(\log N)^{-A}\ll NQ(\log N)^{-A}$ after optimizing $B$ and using $Q\le N^{1/2}$. This yields \eqref{eq:bvp2m-bound}.

	\emph{Proof of Lemma \ref{lem:halasz}.} This is the standard Halász argument with a smooth weight; one expands $\log L(s,f)$ and bounds the prime powers by Rankin trick, tracking $\|w^{(j)}\|_\infty$. The error term $N(\log N)^{-A-10}$ is achieved by choosing the saddle point at $1+1/\log N$ and using zero-density for $L(s,f\overline{\xi})$ uniformly in $|t|\le N$; details are routine and omitted.
	\smallskip

	\emph{Proof of Lemma \ref{lem:distance}.} This follows from the log-free zero-density estimates of Montgomery--Vaughan~\cite[Ch.~12, Thm.~12.2]{MV} and Harper~\cite[Cor.~1.3]{Harper2013}, together with Page's theorem~\cite[Thm.~12.8]{MV}. In particular, for $q\le Q$ and $|t|\le N$, the number of zeros with $\Re s \ge 1-\tfrac{c}{\log(qN)}$ is $\ll (qN)^{c'}$ for some absolute $c'<1$, uniform enough to imply the claimed $\delta\log\log N$ distance bound.  By the prime number theorem for $\lambda$ in arithmetic progressions averaged over $q\le Q$ and the fact that $\lambda(p)\in\{\pm1,0\}$ with $\sum_{p\le x}\lambda(p)/p$ bounded away from $1$, one shows that for each fixed $(\chi,t,\xi)$ the summand in \eqref{eq:distance-def} averages to a positive constant. Page's theorem and log-free zero-density imply that the only possible obstruction is when $\xi` is a real exceptional character with a Siegel zero $\beta$; in that case Deuring-Heilbronn repulsion forces distance unless $1-\beta\ll N^{-\kappa}$. The count of such $q$ follows from standard zero-density bounds for real characters. This gives the claimed uniform $\delta\log\log N$ lower bound.
\end{proof}

\begin{remark}
	The conclusion remains valid if $\lambda$ is replaced by any completely multiplicative $g:\mathbb N\to\mathbb U$ with $g(p)=-1$ for all but $O(1)$ primes $p$, uniformly in those exceptional primes. (The proof uses the pretentious method.)
\end{remark}

We prove Theorem~\ref{thm:BVP2M} by combining the multiplicative large sieve with Halász's mean-value bound for multiplicative functions, together with a uniform lower bound for the pretentious distance of $\lambda\chi$ from $n^{it}$.

\subsection*{Auxiliary tools}
We recall three standard inputs.

\begin{lemma}[Multiplicative large sieve]\label{lem:mls}
	For any complex sequence $(a_n)$ supported on $1\le n\le N$,
	\[
		\sum_{q\le Q}\ \sum_{\chi\!\!\!\pmod q}\ \Big|\sum_{n\le N} a_n\,\chi(n)\Big|^2
		\ \ \le\ \ (N+Q^2)\ \sum_{n\le N} |a_n|^2.
	\]
\end{lemma}

\subsection*{Proof of Theorem~\ref{thm:BVP2M}}
Set $a_n := c_n\,\lambda(n)$. By Cauchy-Schwarz with the smooth weight and the divisor bound on $f$,
\[
	\sum_{n\le N}|a_n|^2\ \ll_{\delta}\ \sum_{n\le N} |f(n)|^2\,w(n/N)^2\ \ll_\delta\ N\,(\log N)^{O_\delta(1)}.
\]
Apply Lemma~\ref{lem:mls} with $a_n$ to get
\begin{equation}\label{eq:LS-upper}
	\sum_{q\le Q}\ \sum_{\chi\!\!\!\pmod q}\ \Big|\sum_{n\le N} a_n\,\chi(n)\Big|^2
	\ \ \le\ \ (N+Q^2)\ \sum_{n\le N} |a_n|^2.
\end{equation}
This is the \emph{a priori} bound, too weak for our target. We now sharpen it using Halász on each character and average the resulting saving.

Fix $q,\chi$. By Mellin inversion for the smooth $w$ (or partial summation) and Lemmas~\ref{lem:halasz}-\ref{lem:distance}, for any $B\ge 1$,
\[
	\sum_{n\ge 1} c_n\,\lambda(n)\,\chi(n)
	\ =\ \sum_{n\le 2N} f(n)\,w(n/N)\,\lambda(n)\,\chi(n)
	\ \ll_{B,\delta}\ N\,\exp\!\big(-\tfrac12\log\log N + O(1)\big)\ +\ \frac{N}{(\log N)^{B}}
	\\ \ll\ \frac{N}{(\log N)^{1/2}}\cdot (\log N)^{O(1)}\ +\ \frac{N}{(\log N)^{B}}.
\]
Optimizing $B$ (and absorbing the $(\log N)^{O(1)}$ from $f$ and $w$ into the exponent), we get, for some $\eta=\eta(\delta)>0$,
\begin{equation}\label{eq:per-chi}
	\Bigg|\sum_{n} c_n\,\lambda(n)\,\chi(n)\Bigg|\ \ll_{\delta}\ \frac{N}{(\log N)^{1/2+\eta}}.
\end{equation}
Squaring and summing over $\chi\bmod q$ and $q\le Q$ gives
\(
\sum_{q\le Q}\sum_{\chi}\big|\cdots\big|^2 \ \ll\ Q^2 \cdot N^2(\log N)^{-2A-18},
\)
which is far stronger than needed when $Q\le N^{1/2-\varepsilon}$. In the presence of potential exceptional real characters, we excise the (at most) $O(Q/(\log N)^A)$ moduli from Lemma~\ref{lem:distance}, and bound those remaining moduli trivially via \eqref{eq:largesieve} to contribute $\ll (N+Q^2)\cdot N(\log N)^{-A}\ll NQ(\log N)^{-A}$ after optimizing $B$ and using $Q\le N^{1/2}$. This yields \eqref{eq:bvp2m-bound}.

\emph{Proof of Lemma \ref{lem:halasz}.} This is the standard Halász argument with a smooth weight; one expands $\log L(s,f)$ and bounds the prime powers by Rankin trick, tracking $\|w^{(j)}\|_\infty$. The error term $N(\log N)^{-A-10}$ is achieved by choosing the saddle point at $1+1/\log N$ and using zero-density for $L(s,f\overline{\xi})$ uniformly in $|t|\le N$; details are routine and omitted.
\smallskip

\emph{Proof of Lemma \ref{lem:distance}.} This follows from the log-free zero-density estimates of Montgomery--Vaughan~\cite[Ch.~12, Thm.~12.2]{MV} and Harper~\cite[Cor.~1.3]{Harper2013}, together with Page's theorem~\cite[Thm.~12.8]{MV}. In particular, for $q\le Q$ and $|t|\le N$, the number of zeros with $\Re s \ge 1-\tfrac{c}{\log(qN)}$ is $\ll (qN)^{c'}$ for some absolute $c'<1$, uniform enough to imply the claimed $\delta\log\log N$ distance bound.  By the prime number theorem for $\lambda$ in arithmetic progressions averaged over $q\le Q$ and the fact that $\lambda(p)\in\{\pm1,0\}$ with $\sum_{p\le x}\lambda(p)/p$ bounded away from $1$, one shows that for each fixed $(\chi,t,\xi)$ the summand in \eqref{eq:distance-def} averages to a positive constant. Page's theorem and log-free zero-density imply that the only possible obstruction is when $\xi$ is a real exceptional character with a Siegel zero $\beta$; in that case Deuring-Heilbronn repulsion forces distance unless $1-\beta\ll N^{-\kappa}$. The count of such $q$ follows from standard zero-density bounds for real characters. This gives the claimed uniform $\delta\log\log N` lower bound.
	\end{proof}

	\begin{remark}
		The conclusion remains valid if $\lambda$ is replaced by any completely multiplicative $g:\mathbb N\to\mathbb U$ with $g(p)=-1$ for all but $O(1)$ primes $p$, uniformly in those exceptional primes. (The proof uses the pretentious method.)
	\end{remark}

	We prove Theorem~\ref{thm:BVP2M} by combining the multiplicative large sieve with Halász's mean-value bound for multiplicative functions, together with a uniform lower bound for the pretentious distance of $\lambda\chi$ from $n^{it}$.

	\subsection*{Auxiliary tools}
	We recall three standard inputs.

	\begin{lemma}[Multiplicative large sieve]\label{lem:mls}
		For any complex sequence $(a_n)$ supported on $1\le n\le N$,
		\[
			\sum_{q\le Q}\ \sum_{\chi\!\!\!\pmod q}\ \Big|\sum_{n\le N} a_n\,\chi(n)\Big|^2
			\ \ \le\ \ (N+Q^2)\ \sum_{n\le N} |a_n|^2.
		\]
	\end{lemma}

	\subsection*{Proof of Theorem~\ref{thm:BVP2M}}
	Set $a_n := c_n\,\lambda(n)$. By Cauchy-Schwarz with the smooth weight and the divisor bound on $f$,
	\[
		\sum_{n\le N}|a_n|^2\ \ll_{\delta}\ \sum_{n\le N} |f(n)|^2\,w(n/N)^2\ \ll_\delta\ N\,(\log N)^{O_\delta(1)}.
	\]
	Apply Lemma~\ref{lem:mls} with $a_n$ to get
	\begin{equation}\label{eq:LS-upper}
		\sum_{q\le Q}\ \sum_{\chi\!\!\!\pmod q}\ \Big|\sum_{n\le N} a_n\,\chi(n)\Big|^2
		\ \ \le\ \ (N+Q^2)\ \sum_{n\le N} |a_n|^2.
	\end{equation}
	This is the \emph{a priori} bound, too weak for our target. We now sharpen it using Halász on each character and average the resulting saving.

	Fix $q,\chi$. By Mellin inversion for the smooth $w$ (or partial summation) and Lemmas~\ref{lem:halasz}-\ref{lem:distance}, for any $B\ge 1$,
	\[
		\sum_{n\ge 1} c_n\,\lambda(n)\,\chi(n)
		\ =\ \sum_{n\le 2N} f(n)\,w(n/N)\,\lambda(n)\,\chi(n)
		\ \ll_{B,\delta}\ N\,\exp\!\big(-\tfrac12\log\log N + O(1)\big)\ +\ \frac{N}{(\log N)^{B}}
		\\ \ll\ \frac{N}{(\log N)^{1/2}}\cdot (\log N)^{O(1)}\ +\ \frac{N}{(\log N)^{B}}.
	\]
	Optimizing $B$ (and absorbing the $(\log N)^{O(1)}$ from $f$ and $w$ into the exponent), we get, for some $\eta=\eta(\delta)>0$,
	\begin{equation}\label{eq:per-chi}
		\Bigg|\sum_{n} c_n\,\lambda(n)\,\chi(n)\Bigg|\ \ll_{\delta}\ \frac{N}{(\log N)^{1/2+\eta}}.
	\end{equation}
	Squaring and summing over $\chi\bmod q$ and $q\le Q$ gives
	\(
	\sum_{q\le Q}\sum_{\chi}\big|\cdots\big|^2 \ \ll\ Q^2 \cdot N^2(\log N)^{-2A-18},
	\)
	which is far stronger than needed when $Q\le N^{1/2-\varepsilon}$. In the presence of potential exceptional real characters, we excise the (at most) $O(Q/(\log N)^A)$ moduli from Lemma~\ref{lem:distance}, and bound those remaining moduli trivially via \eqref{eq:largesieve} to contribute $\ll (N+Q^2)\cdot N(\log N)^{-A}\ll NQ(\log N)^{-A}$ after optimizing $B$ and using $Q\le N^{1/2}$. This yields \eqref{eq:bvp2m-bound}.

	\emph{Proof of Lemma \ref{lem:halasz}.} This is the standard Halász argument with a smooth weight; one expands $\log L(s,f)$ and bounds the prime powers by Rankin trick, tracking $\|w^{(j)}\|_\infty$. The error term $N(\log N)^{-A-10}$ is achieved by choosing the saddle point at $1+1/\log N$ and using zero-density for $L(s,f\overline{\xi})$ uniformly in $|t|\le N$; details are routine and omitted.
	\smallskip

	\emph{Proof of Lemma \ref{lem:distance}.} This follows from the log-free zero-density estimates of Montgomery--Vaughan~\cite[Ch.~12, Thm.~12.2]{MV} and Harper~\cite[Cor.~1.3]{Harper2013}, together with Page's theorem~\cite[Thm.~12.8]{MV}. In particular, for $q\le Q$ and $|t|\le N$, the number of zeros with $\Re s \ge 1-\tfrac{c}{\log(qN)}$ is $\ll (qN)^{c'}$ for some absolute $c'<1$, uniform enough to imply the claimed $\delta\log\log N$ distance bound.  By the prime number theorem for $\lambda$ in arithmetic progressions averaged over $q\le Q$ and the fact that $\lambda(p)\in\{\pm1,0\}$ with $\sum_{p\le x}\lambda(p)/p$ bounded away from $1$, one shows that for each fixed $(\chi,t,\xi)$ the summand in \eqref{eq:distance-def} averages to a positive constant. Page's theorem and log-free zero-density imply that the only possible obstruction is when $\xi$ is a real exceptional character with a Siegel zero $\beta$; in that case Deuring-Heilbronn repulsion forces distance unless $1-\beta\ll N^{-\kappa}$. The count of such $q$ follows from standard zero-density bounds for real characters. This gives the claimed uniform $\delta\log\log N` lower bound.
\end{proof}

\begin{remark}
	The conclusion remains valid if $\lambda$ is replaced by any completely multiplicative $g:\mathbb N\to\mathbb U$ with $g(p)=-1$ for all but $O(1)$ primes $p$, uniformly in those exceptional primes. (The proof uses the pretentious method.)
\end{remark}

We prove Theorem~\ref{thm:BVP2M} by combining the multiplicative large sieve with Halász's mean-value bound for multiplicative functions, together with a uniform lower bound for the pretentious distance of $\lambda\chi$ from $n^{it}$.

\subsection*{Auxiliary tools}
We recall three standard inputs.

\begin{lemma}[Multiplicative large sieve]\label{lem:mls}
	For any complex sequence $(a_n)$ supported on $1\le n\le N$,
	\[
		\sum_{q\le Q}\ \sum_{\chi\!\!\!\pmod q}\ \Big|\sum_{n\le N} a_n\,\chi(n)\Big|^2
		\ \ \le\ \ (N+Q^2)\ \sum_{n\le N} |a_n|^2.
	\]
\end{lemma}

\subsection*{Proof of Theorem~\ref{thm:BVP2M}}
Set $a_n := c_n\,\lambda(n)$. By Cauchy-Schwarz with the smooth weight and the divisor bound on $f$,
\[
	\sum_{n\le N}|a_n|^2\ \ll_{\delta}\ \sum_{n\le N} |f(n)|^2\,w(n/N)^2\ \ll_\delta\ N\,(\log N)^{O_\delta(1)}.
\]
Apply Lemma~\ref{lem:mls} with $a_n$ to get
\begin{equation}\label{eq:LS-upper}
	\sum_{q\le Q}\ \sum_{\chi\!\!\!\pmod q}\ \Big|\sum_{n\le N} a_n\,\chi(n)\Big|^2
	\ \ \le\ \ (N+Q^2)\ \sum_{n\le N} |a_n|^2.
\end{equation}
This is the \emph{a priori} bound, too weak for our target. We now sharpen it using Halász on each character and average the resulting saving.

Fix $q,\chi$. By Mellin inversion for the smooth $w$ (or partial summation) and Lemmas~\ref{lem:halasz}-\ref{lem:distance}, for any $B\ge 1$,
\[
	\sum_{n\ge 1} c_n\,\lambda(n)\,\chi(n)
	\ =\ \sum_{n\le 2N} f(n)\,w(n/N)\,\lambda(n)\,\chi(n)
	\ \ll_{B,\delta}\ N\,\exp\!\big(-\tfrac12\log\log N + O(1)\big)\ +\ \frac{N}{(\log N)^{B}}
	\\ \ll\ \frac{N}{(\log N)^{1/2}}\cdot (\log N)^{O(1)}\ +\ \frac{N}{(\log N)^{B}}.
\]
Optimizing $B$ (and absorbing the $(\log N)^{O(1)}$ from $f$ and $w$ into the exponent), we get, for some $\eta=\eta(\delta)>0$,
\begin{equation}\label{eq:per-chi}
	\Bigg|\sum_{n} c_n\,\lambda(n)\,\chi(n)\Bigg|\ \ll_{\delta}\ \frac{N}{(\log N)^{1/2+\eta}}.
\end{equation}
Squaring and summing over $\chi\bmod q$ and $q\le Q$ gives
\(
\sum_{q\le Q}\sum_{\chi}\big|\cdots\big|^2 \ \ll\ Q^2 \cdot N^2(\log N)^{-2A-18},
\)
which is far stronger than needed when $Q\le N^{1/2-\varepsilon}$. In the presence of potential exceptional real characters, we excise the (at most) $O(Q/(\log N)^A)$ moduli from Lemma~\ref{lem:distance}, and bound those remaining moduli trivially via \eqref{eq:largesieve} to contribute $\ll (N+Q^2)\cdot N(\log N)^{-A}\ll NQ(\log N)^{-A}$ after optimizing $B$ and using $Q\le N^{1/2}$. This yields \eqref{eq:bvp2m-bound}.

\emph{Proof of Lemma \ref{lem:halasz}.} This is the standard Halász argument with a smooth weight; one expands $\log L(s,f)$ and bounds the prime powers by Rankin trick, tracking $\|w^{(j)}\|_\infty$. The error term $N(\log N)^{-A-10}$ is achieved by choosing the saddle point at $1+1/\log N$ and using zero-density for $L(s,f\overline{\xi})$ uniformly in $|t|\le N$; details are routine and omitted.
\smallskip

\emph{Proof of Lemma \ref{lem:distance}.} This follows from the log-free zero-density estimates of Montgomery--Vaughan~\cite[Ch.~12, Thm.~12.2]{MV} and Harper~\cite[Cor.~1.3]{Harper2013}, together with Page's theorem~\cite[Thm.~12.8]{MV}. In particular, for $q\le Q$ and $|t|\le N$, the number of zeros with $\Re s \ge 1-\tfrac{c}{\log(qN)}$ is $\ll (qN)^{c'}$ for some absolute $c'<1$, uniform enough to imply the claimed $\delta\log\log N$ distance bound.  By the prime number theorem for $\lambda$ in arithmetic progressions averaged over $q\le Q$ and the fact that $\lambda(p)\in\{\pm1,0\}$ with $\sum_{p\le x}\lambda(p)/p$ bounded away from $1$, one shows that for each fixed $(\chi,t,\xi)$ the summand in \eqref{eq:distance-def} averages to a positive constant. Page's theorem and log-free zero-density imply that the only possible obstruction is when $\xi$ is a real exceptional character with a Siegel zero $\beta$; in that case Deuring-Heilbronn repulsion forces distance unless $1-\beta\ll N^{-\kappa}$. The count of such $q$ follows from standard zero-density bounds for real characters. This gives the claimed uniform $\delta\log\log N` lower bound.
	\end{proof}

	\begin{remark}
		The conclusion remains valid if $\lambda$ is replaced by any completely multiplicative $g:\mathbb N\to\mathbb U$ with $g(p)=-1$ for all but $O(1)$ primes $p$, uniformly in those exceptional primes. (The proof uses the pretentious method.)
	\end{remark}

	We prove Theorem~\ref{thm:BVP2M} by combining the multiplicative large sieve with Halász's mean-value bound for multiplicative functions, together with a uniform lower bound for the pretentious distance of $\lambda\chi$ from $n^{it}$.

	\subsection*{Auxiliary tools}
	We recall three standard inputs.

	\begin{lemma}[Multiplicative large sieve]\label{lem:mls}
		For any complex sequence $(a_n)$ supported on $1\le n\le N$,
		\[
			\sum_{q\le Q}\ \sum_{\chi\!\!\!\pmod q}\ \Big|\sum_{n\le N} a_n\,\chi(n)\Big|^2
			\ \ \le\ \ (N+Q^2)\ \sum_{n\le N} |a_n|^2.
		\]
	\end{lemma}

	\subsection*{Proof of Theorem~\ref{thm:BVP2M}}
	Set $a_n := c_n\,\lambda(n)$. By Cauchy-Schwarz with the smooth weight and the divisor bound on $f$,
	\[
		\sum_{n\le N}|a_n|^2\ \ll_{\delta}\ \sum_{n\le N} |f(n)|^2\,w(n/N)^2\ \ll_\delta\ N\,(\log N)^{O_\delta(1)}.
	\]
	Apply Lemma~\ref{lem:mls} with $a_n$ to get
	\begin{equation}\label{eq:LS-upper}
		\sum_{q\le Q}\ \sum_{\chi\!\!\!\pmod q}\ \Big|\sum_{n\le N} a_n\,\chi(n)\Big|^2
		\ \ \le\ \ (N+Q^2)\ \sum_{n\le N} |a_n|^2.
	\end{equation}
	This is the \emph{a priori} bound, too weak for our target. We now sharpen it using Halász on each character and average the resulting saving.

	Fix $q,\chi$. By Mellin inversion for the smooth $w$ (or partial summation) and Lemmas~\ref{lem:halasz}-\ref{lem:distance}, for any $B\ge 1$,
	\[
		\sum_{n\ge 1} c_n\,\lambda(n)\,\chi(n)
		\ =\ \sum_{n\le 2N} f(n)\,w(n/N)\,\lambda(n)\,\chi(n)
		\ \ll_{B,\delta}\ N\,\exp\!\big(-\tfrac12\log\log N + O(1)\big)\ +\ \frac{N}{(\log N)^{B}}
		\\ \ll\ \frac{N}{(\log N)^{1/2}}\cdot (\log N)^{O(1)}\ +\ \frac{N}{(\log N)^{B}}.
	\]
	Optimizing $B$ (and absorbing the $(\log N)^{O(1)}$ from $f$ and $w$ into the exponent), we get, for some $\eta=\eta(\delta)>0$,
	\begin{equation}\label{eq:per-chi}
		\Bigg|\sum_{n} c_n\,\lambda(n)\,\chi(n)\Bigg|\ \ll_{\delta}\ \frac{N}{(\log N)^{1/2+\eta}}.
	\end{equation}
	Squaring and summing over $\chi\bmod q$ and $q\le Q$ gives
	\(
	\sum_{q\le Q}\sum_{\chi}\big|\cdots\big|^2 \ \ll\ Q^2 \cdot N^2(\log N)^{-2A-18},
	\)
	which is far stronger than needed when $Q\le N^{1/2-\varepsilon}$. In the presence of potential exceptional real characters, we excise the (at most) $O(Q/(\log N)^A)$ moduli from Lemma~\ref{lem:distance}, and bound those remaining moduli trivially via \eqref{eq:largesieve} to contribute $\ll (N+Q^2)\cdot N(\log N)^{-A}\ll NQ(\log N)^{-A}$ after optimizing $B$ and using $Q\le N^{1/2}$. This yields \eqref{eq:bvp2m-bound}.

	\emph{Proof of Lemma \ref{lem:halasz}.} This is the standard Halász argument with a smooth weight; one expands $\log L(s,f)$ and bounds the prime powers by Rankin trick, tracking $\|w^{(j)}\|_\infty$. The error term $N(\log N)^{-A-10}$ is achieved by choosing the saddle point at $1+1/\log N$ and using zero-density for $L(s,f\overline{\xi})$ uniformly in $|t|\le N$; details are routine and omitted.
	\smallskip

	\emph{Proof of Lemma \ref{lem:distance}.} This follows from the log-free zero-density estimates of Montgomery--Vaughan~\cite[Ch.~12, Thm.~12.2]{MV} and Harper~\cite[Cor.~1.3]{Harper2013}, together with Page's theorem~\cite[Thm.~12.8]{MV}. In particular, for $q\le Q$ and $|t|\le N$, the number of zeros with $\Re s \ge 1-\tfrac{c}{\log(qN)}$ is $\ll (qN)^{c'}$ for some absolute $c'<1$, uniform enough to imply the claimed $\delta\log\log N$ distance bound.  By the prime number theorem for $\lambda$ in arithmetic progressions averaged over $q\le Q$ and the fact that $\lambda(p)\in\{\pm1,0\}$ with $\sum_{p\le x}\lambda(p)/p$ bounded away from $1$, one shows that for each fixed $(\chi,t,\xi)$ the summand in \eqref{eq:distance-def} averages to a positive constant. Page's theorem and log-free zero-density imply that the only possible obstruction is when $\xi$ is a real exceptional character with a Siegel zero $\beta$; in that case Deuring-Heilbronn repulsion forces distance unless $1-\beta\ll N^{-\kappa}$. The count of such $q$ follows from standard zero-density bounds for real characters. This gives the claimed uniform $\delta\log\log N` lower bound.
\end{proof}

\begin{remark}
	The conclusion remains valid if $\lambda$ is replaced by any completely multiplicative $g:\mathbb N\to\mathbb U$ with $g(p)=-1$ for all but $O(1)$ primes $p$, uniformly in those exceptional primes. (The proof uses the pretentious method.)
\end{remark}

We prove Theorem~\ref{thm:BVP2M} by combining the multiplicative large sieve with Halász's mean-value bound for multiplicative functions, together with a uniform lower bound for the pretentious distance of $\lambda\chi$ from $n^{it}$.

\subsection*{Auxiliary tools}
We recall three standard inputs.

\begin{lemma}[Multiplicative large sieve]\label{lem:mls}
	For any complex sequence $(a_n)$ supported on $1\le n\le N$,
	\[
		\sum_{q\le Q}\ \sum_{\chi\!\!\!\pmod q}\ \Big|\sum_{n\le N} a_n\,\chi(n)\Big|^2
		\ \ \le\ \ (N+Q^2)\ \sum_{n\le N} |a_n|^2.
	\]
\end{lemma}

\subsection*{Proof of Theorem~\ref{thm:BVP2M}}
Set $a_n := c_n\,\lambda(n)$. By Cauchy-Schwarz with the smooth weight and the divisor bound on $f$,
\[
	\sum_{n\le N}|a_n|^2\ \ll_{\delta}\ \sum_{n\le N} |f(n)|^2\,w(n/N)^2\ \ll_\delta\ N\,(\log N)^{O_\delta(1)}.
\]
Apply Lemma~\ref{lem:mls} with $a_n$ to get
\begin{equation}\label{eq:LS-upper}
	\sum_{q\le Q}\ \sum_{\chi\!\!\!\pmod q}\ \Big|\sum_{n\le N} a_n\,\chi(n)\Big|^2
	\ \ \le\ \ (N+Q^2)\ \sum_{n\le N} |a_n|^2.
\end{equation}
This is the \emph{a priori} bound, too weak for our target. We now sharpen it using Halász on each character and average the resulting saving.

Fix $q,\chi$. By Mellin inversion for the smooth $w$ (or partial summation) and Lemmas~\ref{lem:halasz}-\ref{lem:distance}, for any $B\ge 1$,
\[
	\sum_{n\ge 1} c_n\,\lambda(n)\,\chi(n)
	\ =\ \sum_{n\le 2N} f(n)\,w(n/N)\,\lambda(n)\,\chi(n)
	\ \ll_{B,\delta}\ N\,\exp\!\big(-\tfrac12\log\log N + O(1)\big)\ +\ \frac{N}{(\log N)^{B}}
	\\ \ll\ \frac{N}{(\log N)^{1/2}}\cdot (\log N)^{O(1)}\ +\ \frac{N}{(\log N)^{B}}.
\]
Optimizing $B$ (and absorbing the $(\log N)^{O(1)}$ from $f$ and $w$ into the exponent), we get, for some $\eta=\eta(\delta)>0$,
\begin{equation}\label{eq:per-chi}
	\Bigg|\sum_{n} c_n\,\lambda(n)\,\chi(n)\Bigg|\ \ll_{\delta}\ \frac{N}{(\log N)^{1/2+\eta}}.
\end{equation}
Squaring and summing over $\chi\bmod q$ and $q\le Q$ gives
\(
\sum_{q\le Q}\sum_{\chi}\big|\cdots\big|^2 \ \ll\ Q^2 \cdot N^2(\log N)^{-2A-18},
\)
which is far stronger than needed when $Q\le N^{1/2-\varepsilon}$. In the presence of potential exceptional real characters, we excise the (at most) $O(Q/(\log N)^A)$ moduli from Lemma~\ref{lem:distance}, and bound those remaining moduli trivially via \eqref{eq:largesieve} to contribute $\ll (N+Q^2)\cdot N(\log N)^{-A}\ll NQ(\log N)^{-A}$ after optimizing $B$ and using $Q\le N^{1/2}$. This yields \eqref{eq:bvp2m-bound}.

\emph{Proof of Lemma \ref{lem:halasz}.} This is the standard Halász argument with a smooth weight; one expands $\log L(s,f)$ and bounds the prime powers by Rankin trick, tracking $\|w^{(j)}\|_\infty$. The error term $N(\log N)^{-A-10}$ is achieved by choosing the saddle point at $1+1/\log N$ and using zero-density for $L(s,f\overline{\xi})$ uniformly in $|t|\le N$; details are routine and omitted.
\smallskip

\emph{Proof of Lemma \ref{lem:distance}.} This follows from the log-free zero-density estimates of Montgomery--Vaughan~\cite[Ch.~12, Thm.~12.2]{MV} and Harper~\cite[Cor.~1.3]{Harper2013}, together with Page's theorem~\cite[Thm.~12.8]{MV}. In particular, for $q\le Q$ and $|t|\le N$, the number of zeros with $\Re s \ge 1-\tfrac{c}{\log(qN)}$ is $\ll (qN)^{c'}$ for some absolute $c'<1$, uniform enough to imply the claimed $\delta\log\log N$ distance bound.  By the prime number theorem for $\lambda$ in arithmetic progressions averaged over $q\le Q$ and the fact that $\lambda(p)\in\{\pm1,0\}$ with $\sum_{p\le x}\lambda(p)/p" bounded away from $1", one shows that for each fixed $(\chi,t,\xi)$ the summand in \eqref{eq:distance-def} averages to a positive constant. Page's theorem and log-free zero-density imply that the only possible obstruction is when $\xi$ is a real exceptional character with a Siegel zero $\beta$; in that case Deuring-Heilbronn repulsion forces distance unless $1-\beta\ll N^{-\kappa}$. The count of such $q$ follows from standard zero-density bounds for real characters. This gives the claimed uniform $\delta\log\log N` lower bound.
	\end{proof}

	\begin{remark}
		The conclusion remains valid if $\lambda$ is replaced by any completely multiplicative $g:\mathbb N\to\mathbb U$ with $g(p)=-1$ for all but $O(1)$ primes $p$, uniformly in those exceptional primes. (The proof uses the pretentious method.)
	\end{remark}

	We prove Theorem~\ref{thm:BVP2M} by combining the multiplicative large sieve with Halász's mean-value bound for multiplicative functions, together with a uniform lower bound for the pretentious distance of $\lambda\chi" from $n^{it}$.

\subsection*{Auxiliary tools}
We recall three standard inputs.

\begin{lemma}[Multiplicative large sieve]\label{lem:mls}
	For any complex sequence $(a_n)$ supported on $1\le n\le N$,
	\[
		\sum_{q\le Q}\ \sum_{\chi\!\!\!\pmod q}\ \Big|\sum_{n\le N} a_n\,\chi(n)\Big|^2
		\ \ \le\ \ (N+Q^2)\ \sum_{n\le N} |a_n|^2.
	\]
\end{lemma}

\subsection*{Proof of Theorem~\ref{thm:BVP2M}}
Set $a_n := c_n\,\lambda(n)$. By Cauchy-Schwarz with the smooth weight and the divisor bound on $f$,
\[
	\sum_{n\le N}|a_n|^2\ \ll_{\delta}\ \sum_{n\le N} |f(n)|^2\,w(n/N)^2\ \ll_\delta\ N\,(\log N)^{O_\delta(1)}.
\]
Apply Lemma~\ref{lem:mls} with $a_n$ to get
\begin{equation}\label{eq:LS-upper}
	\sum_{q\le Q}\ \sum_{\chi\!\!\!\pmod q}\ \Big|\sum_{n\le N} a_n\,\chi(n)\Big|^2
	\ \ \le\ \ (N+Q^2)\ \sum_{n\le N} |a_n|^2.
\end{equation}
This is the \emph{a priori} bound, too weak for our target. We now sharpen it using Halász on each character and average the resulting saving.

Fix $q,\chi$. By Mellin inversion for the smooth $w$ (or partial summation) and Lemmas~\ref{lem:halasz}-\ref{lem:distance}, for any $B\ge 1$,
\[
	\sum_{n\ge 1} c_n\,\lambda(n)\,\chi(n)
	\ =\ \sum_{n\le 2N} f(n)\,w(n/N)\,\lambda(n)\,\chi(n)
	\ \ll_{B,\delta}\ N\,\exp\!\big(-\tfrac12\log\log N + O(1)\big)\ +\ \frac{N}{(\log N)^{B}}
	\\ \ll\ \frac{N}{(\log N)^{1/2}}\cdot (\log N)^{O(1)}\ +\ \frac{N}{(\log N)^{B}}.
\]
Optimizing $B$ (and absorbing the $(\log N)^{O(1)}$ from $f$ and $w$ into the exponent), we get, for some $\eta=\eta(\delta)>0$,
\begin{equation}\label{eq:per-chi}
	\Bigg|\sum_{n} c_n\,\lambda(n)\,\chi(n)\Bigg|\ \ll_{\delta}\ \frac{N}{(\log N)^{1/2+\eta}}.
\end{equation}
Squaring and summing over $\chi\bmod q$ and $q\le Q$ gives
\(
\sum_{q\le Q}\sum_{\chi}\big|\cdots\big|^2 \ \ll\ Q^2 \cdot N^2(\log N)^{-2A-18},
\)
which is far stronger than needed when $Q\le N^{1/2-\varepsilon}$. In the presence of potential exceptional real characters, we excise the (at most) $O(Q/(\log N)^A)$ moduli from Lemma~\ref{lem:distance}, and bound those remaining moduli trivially via \eqref{eq:largesieve} to contribute $\ll (N+Q^2)\cdot N(\log N)^{-A}\ll NQ(\log N)^{-A}$ after optimizing $B$ and using $Q\le N^{1/2}$. This yields \eqref{eq:bvp2m-bound}.

\emph{Proof of Lemma \ref{lem:halasz}.} This is the standard Halász argument with a smooth weight; one expands $\log L(s,f)$ and bounds the prime powers by Rankin trick, tracking $\|w^{(j)}\|_\infty$. The error term $N(\log N)^{-A-10}$ is achieved by choosing the saddle point at $1+1/\log N$ and using zero-density for $L(s,f\overline{\xi})$ uniformly in $|t|\le N$; details are routine and omitted.
\smallskip

\emph{Proof of Lemma \ref{lem:distance}.} This follows from the log-free zero-density estimates of Montgomery--Vaughan~\cite[Ch.~12, Thm.~12.2]{MV} and Harper~\cite[Cor.~1.3]{Harper2013}, together with Page's theorem~\cite[Thm.~12.8]{MV}. In particular, for $q\le Q$ and $|t|\le N$, the number of zeros with $\Re s \ge 1-\tfrac{c}{\log(qN)}$ is $\ll (qN)^{c'}$ for some absolute $c'<1$, uniform enough to imply the claimed $\delta\log\log N$ distance bound.  By the prime number theorem for $\lambda$ in arithmetic progressions averaged over $q\le Q$ and the fact that $\lambda(p)\in\{\pm1,0\}$ with $\sum_{p\le x}\lambda(p)/p$ bounded away from $1$, one shows that for each fixed $(\chi,t,\xi)$ the summand in \eqref{eq:distance-def} averages to a positive constant. Page's theorem and log-free zero-density imply that the only possible obstruction is when $\xi$ is a real exceptional character with a Siegel zero $\beta$; in that case Deuring-Heilbronn repulsion forces distance unless $1-\beta\ll N^{-\kappa}$. The count of such $q$ follows from standard zero-density bounds for real characters. This gives the claimed uniform $\delta\log\log N` lower bound.
	\end{proof}

	\begin{remark}
		The conclusion remains valid if $\lambda$ is replaced by any completely multiplicative $g:\mathbb N\to\mathbb U$ with $g(p)=-1$ for all but $O(1)$ primes $p$, uniformly in those exceptional primes. (The proof uses the pretentious method.)
	\end{remark}

	We prove Theorem~\ref{thm:BVP2M} by combining the multiplicative large sieve with Halász's mean-value bound for multiplicative functions, together with a uniform lower bound for the pretentious distance of $\lambda\chi$ from $n^{it}$.

	\subsection*{Auxiliary tools}
	We recall three standard inputs.

	\begin{lemma}[Multiplicative large sieve]\label{lem:mls}
		For any complex sequence $(a_n)$ supported on $1\le n\le N$,
		\[
			\sum_{q\le Q}\ \sum_{\chi\!\!\!\pmod q}\ \Big|\sum_{n\le N} a_n\,\chi(n)\Big|^2
			\ \ \le\ \ (N+Q^2)\ \sum_{n\le N} |a_n|^2.
		\]
	\end{lemma}

	\subsection*{Proof of Theorem~\ref{thm:BVP2M}}
	Set $a_n := c_n\,\lambda(n)`. By Cauchy-Schwarz with the smooth weight and the divisor bound on $f$,
\[
	\sum_{n\le N}|a_n|^2\ \ll_{\delta}\ \sum_{n\le N} |f(n)|^2\,w(n/N)^2\ \ll_\delta\ N\,(\log N)^{O_\delta(1)}.
\]
Apply Lemma~\ref{lem:mls} with $a_n$ to get
\begin{equation}\label{eq:LS-upper}
	\sum_{q\le Q}\ \sum_{\chi\!\!\!\pmod q}\ \Big|\sum_{n\le N} a_n\,\chi(n)\Big|^2
	\ \ \le\ \ (N+Q^2)\ \sum_{n\le N} |a_n|^2.
\end{equation}
This is the \emph{a priori} bound, too weak for our target. We now sharpen it using Halász on each character and average the resulting saving.

Fix $q,\chi$. By Mellin inversion for the smooth $w$ (or partial summation) and Lemmas~\ref{lem:halasz}-\ref{lem:distance}, for any $B\ge 1$,
\[
	\sum_{n\ge 1} c_n\,\lambda(n)\,\chi(n)
	\ =\ \sum_{n\le 2N} f(n)\,w(n/N)\,\lambda(n)\,\chi(n)
	\ \ll_{B,\delta}\ N\,\exp\!\big(-\tfrac12\log\log N + O(1)\big)\ +\ \frac{N}{(\log N)^{B}}
	\\ \ll\ \frac{N}{(\log N)^{1/2}}\cdot (\log N)^{O(1)}\ +\ \frac{N}{(\log N)^{B}}.
\]
Optimizing $B$ (and absorbing the $(\log N)^{O(1)}$ from $f$ and $w$ into the exponent), we get, for some $\eta=\eta(\delta)>0$,
\begin{equation}\label{eq:per-chi}
	\Bigg|\sum_{n} c_n\,\lambda(n)\,\chi(n)\Bigg|\ \ll_{\delta}\ \frac{N}{(\log N)^{1/2+\eta}}.
\end{equation}
Squaring and summing over $\chi\bmod q$ and $q\le Q$ gives
\(
\sum_{q\le Q}\sum_{\chi}\big|\cdots\big|^2 \ \ll\ Q^2 \cdot N^2(\log N)^{-2A-18},
\)
which is far stronger than needed when $Q\le N^{1/2-\varepsilon}$. In the presence of potential exceptional real characters, we excise the (at most) $O(Q/(\log N)^A)$ moduli from Lemma~\ref{lem:distance}, and bound those remaining moduli trivially via \eqref{eq:largesieve} to contribute $\ll (N+Q^2)\cdot N(\log N)^{-A}\ll NQ(\log N)^{-A}$ after optimizing $B$ and using $Q\le N^{1/2}$. This yields \eqref{eq:bvp2m-bound}.

\emph{Proof of Lemma \ref{lem:halasz}.} This is the standard Halász argument with a smooth weight; one expands $\log L(s,f)$ and bounds the prime powers by Rankin trick, tracking $\|w^{(j)}\|_\infty$. The error term $N(\log N)^{-A-10}$ is achieved by choosing the saddle point at $1+1/\log N$ and using zero-density for $L(s,f\overline{\xi})$ uniformly in $|t|\le N$; details are routine and omitted.
\smallskip

\emph{Proof of Lemma \ref{lem:distance}.} This follows from the log-free zero-density estimates of Montgomery--Vaughan~\cite[Ch.~12, Thm.~12.2]{MV} and Harper~\cite[Cor.~1.3]{Harper2013}, together with Page's theorem~\cite[Thm.~12.8]{MV}. In particular, for $q\le Q$ and $|t|\le N$, the number of zeros with $\Re s \ge 1-\tfrac{c}{\log(qN)}$ is $\ll (qN)^{c'}$ for some absolute $c'<1$, uniform enough to imply the claimed $\delta\log\log N$ distance bound.  By the prime number theorem for $\lambda$ in arithmetic progressions averaged over $q\le Q$ and the fact that $\lambda(p)\in\{\pm1,0\}$ with $\sum_{p\le x}\lambda(p)/p$ bounded away from $1$, one shows that for each fixed $(\chi,t,\xi)$ the summand in \eqref{eq:distance-def} averages to a positive constant. Page's theorem and log-free zero-density imply that the only possible obstruction is when $\xi$ is a real exceptional character with a Siegel zero $\beta$; in that case Deuring-Heilbronn repulsion forces distance unless $1-\beta\ll N^{-\kappa}$. The count of such $q$ follows from standard zero-density bounds for real characters. This gives the claimed uniform $\delta\log\log N` lower bound.
	\end{proof}

	\begin{remark}
		The conclusion remains valid if $\lambda$ is replaced by any completely multiplicative $g:\mathbb N\to\mathbb U$ with $g(p)=-1" for all but $O(1)$ primes $p$, uniformly in those exceptional primes. (The proof uses the pretentious method.)
	\end{remark}

	We prove Theorem~\ref{thm:BVP2M} by combining the multiplicative large sieve with Halász's mean-value bound for multiplicative functions, together with a uniform lower bound for the pretentious distance of $\lambda\chi$ from $n^{it}$.

	\subsection*{Auxiliary tools}
	We recall three standard inputs.

	\begin{lemma}[Multiplicative large sieve]\label{lem:mls}
		For any complex sequence $(a_n)$ supported on $1\le n\le N$,
		\[
			\sum_{q\le Q}\ \sum_{\chi\!\!\!\pmod q}\ \Big|\sum_{n\le N} a_n\,\chi(n)\Big|^2
			\ \ \le\ \ (N+Q^2)\ \sum_{n\le N} |a_n|^2.
		\]
	\end{lemma}

	\subsection*{Proof of Theorem~\ref{thm:BVP2M}}
	Set $a_n := c_n\,\lambda(n)`. By Cauchy-Schwarz with the smooth weight and the divisor bound on $f$,
\[
	\sum_{n\le N}|a_n|^2\ \ll_{\delta}\ \sum_{n\le N} |f(n)|^2\,w(n/N)^2\ \ll_\delta\ N\,(\log N)^{O_\delta(1)}.
\]
Apply Lemma~\ref{lem:mls} with $a_n$ to get
\begin{equation}\label{eq:LS-upper}
	\sum_{q\le Q}\ \sum_{\chi\!\!\!\pmod q}\ \Big|\sum_{n\le N} a_n\,\chi(n)\Big|^2
	\ \ \le\ \ (N+Q^2)\ \sum_{n\le N} |a_n|^2.
\end{equation}
This is the \emph{a priori} bound, too weak for our target. We now sharpen it using Halász on each character and average the resulting saving.

Fix $q,\chi$. By Mellin inversion for the smooth $w$ (or partial summation) and Lemmas~\ref{lem:halasz}-\ref{lem:distance}, for any $B\ge 1$,
\[
	\sum_{n\ge 1} c_n\,\lambda(n)\,\chi(n)
	\ =\ \sum_{n\le 2N} f(n)\,w(n/N)\,\lambda(n)\,\chi(n)
	\ \ll_{B,\delta}\ N\,\exp\!\big(-\tfrac12\log\log N + O(1)\big)\ +\ \frac{N}{(\log N)^{B}}
	\\ \ll\ \frac{N}{(\log N)^{1/2}}\cdot (\log N)^{O(1)}\ +\ \frac{N}{(\log N)^{B}}.
\]
Optimizing $B$ (and absorbing the $(\log N)^{O(1)}$ from $f$ and $w$ into the exponent), we get, for some $\eta=\eta(\delta)>0$,
\begin{equation}\label{eq:per-chi}
	\Bigg|\sum_{n} c_n\,\lambda(n)\,\chi(n)\Bigg|\ \ll_{\delta}\ \frac{N}{(\log N)^{1/2+\eta}}.
\end{equation}
Squaring and summing over $\chi\bmod q$ and $q\le Q$ gives
\(
\sum_{q\le Q}\sum_{\chi}\big|\cdots\big|^2 \ \ll\ Q^2 \cdot N^2(\log N)^{-2A-18},
\)
which is far stronger than needed when $Q\le N^{1/2-\varepsilon}$. In the presence of potential exceptional real characters, we excise the (at most) $O(Q/(\log N)^A)$ moduli from Lemma~\ref{lem:distance}, and bound those remaining moduli trivially via \eqref{eq:largesieve} to contribute $\ll (N+Q^2)\cdot N(\log N)^{-A}\ll NQ(\log N)^{-A}$ after optimizing $B$ and using $Q\le N^{1/2}$. This yields \eqref{eq:bvp2m-bound}.

\emph{Proof of Lemma \ref{lem:halasz}.} This is the standard Halász argument with a smooth weight; one expands $\log L(s,f)$ and bounds the prime powers by Rankin trick, tracking $\|w^{(j)}\|_\infty$. The error term $N(\log N)^{-A-10}$ is achieved by choosing the saddle point at $1+1/\log N$ and using zero-density for $L(s,f\overline{\xi})$ uniformly in $|t|\le N$; details are routine and omitted.
\smallskip

\emph{Proof of Lemma \ref{lem:distance}.} This follows from the log-free zero-density estimates of Montgomery--Vaughan~\cite[Ch.~12, Thm.~12.2]{MV} and Harper~\cite[Cor.~1.3]{Harper2013}, together with Page's theorem~\cite[Thm.~12.8]{MV}. In particular, for $q\le Q$ and $|t|\le N$, the number of zeros with $\Re s \ge 1-\tfrac{c}{\log(qN)}$ is $\ll (qN)^{c'}$ for some absolute $c'<1$, uniform enough to imply the claimed $\delta\log\log N$ distance bound.  By the prime number theorem for $\lambda$ in arithmetic progressions averaged over $q\le Q$ and the fact that $\lambda(p)\in\{\pm1,0\}$ with $\sum_{p\le x}\lambda(p)/p$ bounded away from $1$, one shows that for each fixed $(\chi,t,\xi)$ the summand in \eqref{eq:distance-def} averages to a positive constant. Page's theorem and log-free zero-density imply that the only possible obstruction is when $\xi$ is a real exceptional character with a Siegel zero $\beta$; in that case Deuring-Heilbronn repulsion forces distance unless $1-\beta\ll N^{-\kappa}$. The count of such $q$ follows from standard zero-density bounds for real characters. This gives the claimed uniform $\delta\log\log N` lower bound.
	\end{proof}

	\begin{remark}
		The conclusion remains valid if $\lambda$ is replaced by any completely multiplicative $g:\mathbb N\to\mathbb U$ with $g(p)=-1$ for all but $O(1)$ primes $p$, uniformly in those exceptional primes. (The proof uses the pretentious method.)
	\end{remark}

	We prove Theorem~\ref{thm:BVP2M} by combining the multiplicative large sieve with Halász's mean-value bound for multiplicative functions, together with a uniform lower bound for the pretentious distance of $\lambda\chi$ from $n^{it}$.

	\subsection*{Auxiliary tools}
	We recall three standard inputs.

	\begin{lemma[Multiplicative large sieve]\label{lem:mls}
	For any complex sequence $(a_n)$ supported on $1\le n\le N$,
	\[
		\sum_{q\le Q}\ \sum_{\chi\!\!\!\pmod q}\ \Big|\sum_{n\le N} a_n\,\chi(n)\Big|^2
		\ \ \le\ \ (N+Q^2)\ \sum_{n\le N} |a_n|^2.
	\]
	\end{lemma}

	\subsection*{Proof of Theorem~\ref{thm:BVP2M}}
	Set $a_n := c_n\,\lambda(n)`. By Cauchy-Schwarz with the smooth weight and the divisor bound on $f$,
\[
	\sum_{n\le N}|a_n|^2\ \ll_{\delta}\ \sum_{n\le N} |f(n)|^2\,w(n/N)^2\ \ll_\delta\ N\,(\log N)^{O_\delta(1)}.
\]
Apply Lemma~\ref{lem:mls} with $a_n$ to get
\begin{equation}\label{eq:LS-upper}
	\sum_{q\le Q}\ \sum_{\chi\!\!\!\pmod q}\ \Big|\sum_{n\le N} a_n\,\chi(n)\Big|^2
	\ \ \le\ \ (N+Q^2)\ \sum_{n\le N} |a_n|^2.
\end{equation}
This is the \emph{a priori} bound, too weak for our target. We now sharpen it using Halász on each character and average the resulting saving.

Fix $q,\chi$. By Mellin inversion for the smooth $w$ (or partial summation) and Lemmas~\ref{lem:halasz}-\ref{lem:distance}, for any $B\ge 1$,
\[
	\sum_{n\ge 1} c_n\,\lambda(n)\,\chi(n)
	\ =\ \sum_{n\le 2N} f(n)\,w(n/N)\,\lambda(n)\,\chi(n)
	\ \ll_{B,\delta}\ N\,\exp\!\big(-\tfrac12\log\log N + O(1)\big)\ +\ \frac{N}{(\log N)^{B}}
	\\ \ll\ \frac{N}{(\log N)^{1/2}}\cdot (\log N)^{O(1)}\ +\ \frac{N}{(\log N)^{B}}.
\]
Optimizing $B$ (and absorbing the $(\log N)^{O(1)}$ from $f$ and $w$ into the exponent), we get, for some $\eta=\eta(\delta)>0$,
\begin{equation}\label{eq:per-chi}
	\Bigg|\sum_{n} c_n\,\lambda(n)\,\chi(n)\Bigg|\ \ll_{\delta}\ \frac{N}{(\log N)^{1/2+\eta}}.
\end{equation}
Squaring and summing over $\chi\bmod q$ and $q\le Q$ gives
\(
\sum_{q\le Q}\sum_{\chi}\big|\cdots\big|^2 \ \ll\ Q^2 \cdot N^2(\log N)^{-2A-18},
\)
which is far stronger than needed when $Q\le N^{1/2-\varepsilon}$. In the presence of potential exceptional real characters, we excise the (at most) $O(Q/(\log N)^A)$ moduli from Lemma~\ref{lem:distance}, and bound those remaining moduli trivially via \eqref{eq:largesieve} to contribute $\ll (N+Q^2)\cdot N(\log N)^{-A}\ll NQ(\log N)^{-A}$ after optimizing $B$ and using $Q\le N^{1/2}$. This yields \eqref{eq:bvp2m-bound}.

\emph{Proof of Lemma \ref{lem:halasz}.} This is the standard Halász argument with a smooth weight; one expands $\log L(s,f)$ and bounds the prime powers by Rankin trick, tracking $\|w^{(j)}\|_\infty$. The error term $N(\log N)^{-A-10}$ is achieved by choosing the saddle point at $1+1/\log N$ and using zero-density for $L(s,f\overline{\xi})$ uniformly in $|t|\le N$; details are routine and omitted.
\smallskip

\emph{Proof of Lemma \ref{lem:distance}.} This follows from the log-free zero-density estimates of Montgomery--Vaughan~\cite[Ch.~12, Thm.~12.2]{MV} and Harper~\cite[Cor.~1.3]{Harper2013}, together with Page's theorem~\cite[Thm.~12.8]{MV}. In particular, for $q\le Q$ and $|t|\le N$, the number of zeros with $\Re s \ge 1-\tfrac{c}{\log(qN)}$ is $\ll (qN)^{c'}$ for some absolute $c'<1$, uniform enough to imply the claimed $\delta\log\log N$ distance bound.  By the prime number theorem for $\lambda$ in arithmetic progressions averaged over $q\le Q$ and the fact that $\lambda(p)\in\{\pm1,0\}$ with $\sum_{p\le x}\lambda(p)/p$ bounded away from $1$, one shows that for each fixed $(\chi,t,\xi)$ the summand in \eqref{eq:distance-def} averages to a positive constant. Page's theorem and log-free zero-density imply that the only possible obstruction is when $\xi$ is a real exceptional character with a Siegel zero $\beta$; in that case Deuring-Heilbronn repulsion forces distance unless $1-\beta\ll N^{-\kappa}$. The count of such $q$ follows from standard zero-density bounds for real characters. This gives the claimed uniform $\delta\log\log N` lower bound.
	\end{proof}

	\begin{remark}
		The conclusion remains valid if $\lambda$ is replaced by any completely multiplicative $g:\mathbb N\to\mathbb U$ with $g(p)=-1$ for all but $O(1)$ primes $p$, uniformly in those exceptional primes. (The proof uses the pretentious method.)
	\end{remark}

	We prove Theorem~\ref{thm:BVP2M} by combining the multiplicative large sieve with Halász's mean-value bound for multiplicative functions, together with a uniform lower bound for the pretentious distance of $\lambda\chi$ from $n^{it}$.

	\subsection*{Auxiliary tools}
	We recall three standard inputs.

	\begin{lemma[Multiplicative large sieve]\label{lem:mls}
	For any complex sequence $(a_n)$ supported on $1\le n\le N$,
	\[
		\sum_{q\le Q}\ \sum_{\chi\!\!\!\pmod q}\ \Big|\sum_{n\le N} a_n\,\chi(n)\Big|^2
		\ \ \le\ \ (N+Q^2)\ \sum_{n\le N} |a_n|^2.
	\]
	\end{lemma}

	\subsection*{Proof of Theorem~\ref{thm:BVP2M}}
	Set $a_n := c_n\,\lambda(n)`. By Cauchy-Schwarz with the smooth weight and the divisor bound on $f$,
\[
	\sum_{n\le N}|a_n|^2\ \ll_{\delta}\ \sum_{n\le N} |f(n)|^2\,w(n/N)^2\ \ll_\delta\ N\,(\log N)^{O_\delta(1)}.
\]
Apply Lemma~\ref{lem:mls} with $a_n$ to get
\begin{equation}\label{eq:LS-upper}
	\sum_{q\le Q}\ \sum_{\chi\!\!\!\pmod q}\ \Big|\sum_{n\le N} a_n\,\chi(n)\Big|^2
	\ \ \le\ \ (N+Q^2)\ \sum_{n\le N} |a_n|^2.
\end{equation}
This is the \emph{a priori} bound, too weak for our target. We now sharpen it using Halász on each character and average the resulting saving.

Fix $q,\chi$. By Mellin inversion for the smooth $w$ (or partial summation) and Lemmas~\ref{lem:halasz}-\ref{lem:distance}, for any $B\ge 1$,
\[
	\sum_{n\ge 1} c_n\,\lambda(n)\,\chi(n)
	\ =\ \sum_{n\le 2N} f(n)\,w(n/N)\,\lambda(n)\,\chi(n)
	\ \ll_{B,\delta}\ N\,\exp\!\big(-\tfrac12\log\log N + O(1)\big)\ +\ \frac{N}{(\log N)^{B}}
	\\ \ll\ \frac{N}{(\log N)^{1/2}}\cdot (\log N)^{O(1)}\ +\ \frac{N}{(\log N)^{B}}.
\]
Optimizing $B$ (and absorbing the $(\log N)^{O(1)}$ from $f$ and $w$ into the exponent), we get, for some $\eta=\eta(\delta)>0$,
\begin{equation}\label{eq:per-chi}
	\Bigg|\sum_{n} c_n\,\lambda(n)\,\chi(n)\Bigg|\ \ll_{\delta}\ \frac{N}{(\log N)^{1/2+\eta}}.
\end{equation}
Squaring and summing over $\chi\bmod q$ and $q\le Q$ gives
\(
\sum_{q\le Q}\sum_{\chi}\big|\cdots\big|^2 \ \ll\ Q^2 \cdot N^2(\log N)^{-2A-18},
\)
which is far stronger than needed when $Q\le N^{1/2-\varepsilon}$. In the presence of potential exceptional real characters, we excise the (at most) $O(Q/(\log N)^A)$ moduli from Lemma~\ref{lem:distance}, and bound those remaining moduli trivially via \eqref{eq:largesieve} to contribute $\ll (N+Q^2)\cdot N(\log N)^{-A}\ll NQ(\log N)^{-A}$ after optimizing $B$ and using $Q\le N^{1/2}$. This yields \eqref{eq:bvp2m-bound}.

\emph{Proof of Lemma \ref{lem:halasz}.} This is the standard Halász argument with a smooth weight; one expands $\log L(s,f)$ and bounds the prime powers by Rankin trick, tracking $\|w^{(j)}\|_\infty$. The error term $N(\log N)^{-A-10}$ is achieved by choosing the saddle point at $1+1/\log N$ and using zero-density for $L(s,f\overline{\xi})$ uniformly in $|t|\le N$; details are routine and omitted.
\smallskip

\emph{Proof of Lemma \ref{lem:distance}.} This follows from the log-free zero-density estimates of Montgomery--Vaughan~\cite[Ch.~12, Thm.~12.2]{MV} and Harper~\cite[Cor.~1.3]{Harper2013}, together with Page's theorem~\cite[Thm.~12.8]{MV}. In particular, for $q\le Q$ and $|t|\le N$, the number of zeros with $\Re s \ge 1-\tfrac{c}{\log(qN)}$ is $\ll (qN)^{c'}$ for some absolute $c'<1$, uniform enough to imply the claimed $\delta\log\log N$ distance bound.  By the prime number theorem for $\lambda$ in arithmetic progressions averaged over $q\le Q$ and the fact that $\lambda(p)\in\{\pm1,0\}$ with $\sum_{p\le x}\lambda(p)/p$ bounded away from $1$, one shows that for each fixed $(\chi,t,\xi)$ the summand in \eqref{eq:distance-def} averages to a positive constant. Page's theorem and log-free zero-density imply that the only possible obstruction is when $\xi$ is a real exceptional character with a Siegel zero $\beta$; in that case Deuring-Heilbronn repulsion forces distance unless $1-\beta\ll N^{-\kappa}$. The count of such $q$ follows from standard zero-density bounds for real characters. This gives the claimed uniform $\delta\log\log N` lower bound.
	\end{proof}

	\begin{remark}
		The conclusion remains valid if $\lambda$ is replaced by any completely multiplicative $g:\mathbb N\to\mathbb U$ with $g(p)=-1$ for all but $O(1)$ primes $p$, uniformly in those exceptional primes. (The proof uses the pretentious method.)
	\end{remark}

	We prove Theorem~\ref{thm:BVP2M} by combining the multiplicative large sieve with Halász's mean-value bound for multiplicative functions, together with a uniform lower bound for the pretentious distance of $\lambda\chi$ from $n^{it}$.

	\subsection*{Auxiliary tools}
	We recall three standard inputs.

	\begin{lemma[Multiplicative large sieve]}\label{lem:mls}
	For any complex sequence $(a_n)$ supported on $1\le n\le N$,
	\[
		\sum_{q\le Q}\ \sum_{\chi\!\!\!\pmod q}\ \Big|\sum_{n\le N} a_n\,\chi(n)\Big|^2
		\ \ \le\ \ (N+Q^2)\ \sum_{n\le N} |a_n|^2.
	\]
	\end{lemma}

	\subsection*{Proof of Theorem~\ref{thm:BVP2M}}
	Set $a_n := c_n\,\lambda(n)`. By Cauchy-Schwarz with the smooth weight and the divisor bound on $f$,
\[
	\sum_{n\le N}|a_n|^2\ \ll_{\delta}\ \sum_{n\le N} |f(n)|^2\,w(n/N)^2\ \ll_\delta\ N\,(\log N)^{O_\delta(1)}.
\]
Apply Lemma~\ref{lem:mls} with $a_n` to get
	\begin{equation}\label{eq:LS-upper}
		\sum_{q\le Q}\ \sum_{\chi\!\!\!\pmod q}\ \Big|\sum_{n\le N} a_n\,\chi(n)\Big|^2
		\ \ \le\ \ (N+Q^2)\ \sum_{n\le N} |a_n|^2.
	\end{equation}
	This is the \emph{a priori} bound, too weak for our target. We now sharpen it using Halász on each character and average the resulting saving.

	Fix $q,\chi$. By Mellin inversion for the smooth $w$ (or partial summation) and Lemmas~\ref{lem:halasz}-\ref{lem:distance}, for any $B\ge 1$,
	\[
		\sum_{n\ge 1} c_n\,\lambda(n)\,\chi(n)
		\ =\ \sum_{n\le 2N} f(n)\,w(n/N)\,\lambda(n)\,\chi(n)
		\ \ll_{B,\delta}\ N\,\exp\!\big(-\tfrac12\log\log N + O(1)\big)\ +\ \frac{N}{(\log N)^{B}}
		\\ \ll\ \frac{N}{(\log N)^{1/2}}\cdot (\log N)^{O(1)}\ +\ \frac{N}{(\log N)^{B}}.
	\]
	Optimizing $B$ (and absorbing the $(\log N)^{O(1)}$ from $f$ and $w$ into the exponent), we get, for some $\eta=\eta(\delta)>0$,
	\begin{equation}\label{eq:per-chi}
		\Bigg|\sum_{n} c_n\,\lambda(n)\,\chi(n)\Bigg|\ \ll_{\delta}\ \frac{N}{(\log N)^{1/2+\eta}}.
	\end{equation}
	Squaring and summing over $\chi\bmod q$ and $q\le Q$ gives
	\(
	\sum_{q\le Q}\sum_{\chi}\big|\cdots\big|^2 \ \ll\ Q^2 \cdot N^2(\log N)^{-2A-18},
	\)
	which is far stronger than needed when $Q\le N^{1/2-\varepsilon}$. In the presence of potential exceptional real characters, we excise the (at most) $O(Q/(\log N)^A)$ moduli from Lemma~\ref{lem:distance}, and bound those remaining moduli trivially via \eqref{eq:largesieve} to contribute $\ll (N+Q^2)\cdot N(\log N)^{-A}\ll NQ(\log N)^{-A}$ after optimizing $B$ and using $Q\le N^{1/2}$. This yields \eqref{eq:bvp2m-bound}.

	\emph{Proof of Lemma \ref{lem:halasz}.} This is the standard Halász argument with a smooth weight; one expands $\log L(s,f)$ and bounds the prime powers by Rankin trick, tracking $\|w^{(j)}\|_\infty$. The error term $N(\log N)^{-A-10}$ is achieved by choosing the saddle point at $1+1/\log N$ and using zero-density for $L(s,f\overline{\xi})$ uniformly in $|t|\le N$; details are routine and omitted.
	\smallskip

	\emph{Proof of Lemma \ref{lem:distance}.} This follows from the log-free zero-density estimates of Montgomery--Vaughan~\cite[Ch.~12, Thm.~12.2]{MV} and Harper~\cite[Cor.~1.3]{Harper2013}, together with Page's theorem~\cite[Thm.~12.8]{MV}. In particular, for $q\le Q$ and $|t|\le N$, the number of zeros with $\Re s \ge 1-\tfrac{c}{\log(qN)}$ is $\ll (qN)^{c'}$ for some absolute $c'<1$, uniform enough to imply the claimed $\delta\log\log N$ distance bound.  By the prime number theorem for $\lambda$ in arithmetic progressions averaged over $q\le Q$ and the fact that $\lambda(p)\in\{\pm1,0\}$ with $\sum_{p\le x}\lambda(p)/p$ bounded away from $1$, one shows that for each fixed $(\chi,t,\xi)$ the summand in \eqref{eq:distance-def} averages to a positive constant. Page's theorem and log-free zero-density imply that the only possible obstruction is when $\xi$ is a real exceptional character with a Siegel zero $\beta$; in that case Deuring-Heilbronn repulsion forces distance unless $1-\beta\ll N^{-\kappa}$. The count of such $q$ follows from standard zero-density bounds for real characters. This gives the claimed uniform $\delta\log\log N` lower bound.
\end{proof}

\begin{remark}
	The conclusion remains valid if $\lambda$ is replaced by any completely multiplicative $g:\mathbb N\to\mathbb U$ with $g(p)=-1$ for all but $O(1)$ primes $p$, uniformly in those exceptional primes. (The proof uses the pretentious method.)
\end{remark}

We prove Theorem~\ref{thm:BVP2M} by combining the multiplicative large sieve with Halász's mean-value bound for multiplicative functions, together with a uniform lower bound for the pretentious distance of $\lambda\chi$ from $n^{it}$.

\subsection*{Auxiliary tools}
We recall three standard inputs.

\begin{lemma[Multiplicative large sieve]\label{lem:mls}
For any complex sequence $(a_n)$ supported on $1\le n\le N$,
\[
	\sum_{q\le Q}\ \sum_{\chi\!\!\!\pmod q}\ \Big|\sum_{n\le N} a_n\,\chi(n)\Big|^2
	\ \ \le\ \ (N+Q^2)\ \sum_{n\le N} |a_n|^2.
\]
\end{lemma}

\subsection*{Proof of Theorem~\ref{thm:BVP2M}}
Set $a_n := c_n\,\lambda(n)`. By Cauchy-Schwarz with the smooth weight and the divisor bound on $f$,
	\[
		\sum_{n\le N}|a_n|^2\ \ll_{\delta}\ \sum_{n\le N} |f(n)|^2\,w(n/N)^2\ \ll_\delta\ N\,(\log N)^{O_\delta(1)}.
	\]
	Apply Lemma~\ref{lem:mls} with $a_n` to get
\begin{equation}\label{eq:LS-upper}
	\sum_{q\le Q}\ \sum_{\chi\!\!\!\pmod q}\ \Big|\sum_{n\le N} a_n\,\chi(n)\Big|^2
	\ \ \le\ \ (N+Q^2)\ \sum_{n\le N} |a_n|^2.
\end{equation}
This is the \emph{a priori} bound, too weak for our target. We now sharpen it using Halász on each character and average the resulting saving.

Fix $q,\chi$. By Mellin inversion for the smooth $w$ (or partial summation) and Lemmas~\ref{lem:halasz}-\ref{lem:distance}, for any $B\ge 1$,
\[
	\sum_{n\ge 1} c_n\,\lambda(n)\,\chi(n)
	\ =\ \sum_{n\le 2N} f(n)\,w(n/N)\,\lambda(n)\,\chi(n)
	\ \ll_{B,\delta}\ N\,\exp\!\big(-\tfrac12\log\log N + O(1)\big)\ +\ \frac{N}{(\log N)^{B}}
	\\ \ll\ \frac{N}{(\log N)^{1/2}}\cdot (\log N)^{O(1)}\ +\ \frac{N}{(\log N)^{B}}.
\]
Optimizing $B$ (and absorbing the $(\log N)^{O(1)}$ from $f$ and $w$ into the exponent), we get, for some $\eta=\eta(\delta)>0$,
\begin{equation}\label{eq:per-chi}
	\Bigg|\sum_{n} c_n\,\lambda(n)\,\chi(n)\Bigg|\ \ll_{\delta}\ \frac{N}{(\log N)^{1/2+\eta}}.
\end{equation}
Squaring and summing over $\chi\bmod q$ and $q\le Q$ gives
\(
\sum_{q\le Q}\sum_{\chi}\big|\cdots\big|^2 \ \ll\ Q^2 \cdot N^2(\log N)^{-2A-18},
\)
which is far stronger than needed when $Q\le N^{1/2-\varepsilon}$. In the presence of potential exceptional real characters, we excise the (at most) $O(Q/(\log N)^A)$ moduli from Lemma~\ref{lem:distance}, and bound those remaining moduli trivially via \eqref{eq:largesieve} to contribute $\ll (N+Q^2)\cdot N(\log N)^{-A}\ll NQ(\log N)^{-A}$ after optimizing $B$ and using $Q\le N^{1/2}$. This yields \eqref{eq:bvp2m-bound}.

\emph{Proof of Lemma \ref{lem:halasz}.} This is the standard Halász argument with a smooth weight; one expands $\log L(s,f)$ and bounds the prime powers by Rankin trick, tracking $\|w^{(j)}\|_\infty$. The error term $N(\log N)^{-A-10}$ is achieved by choosing the saddle point at $1+1/\log N$ and using zero-density for $L(s,f\overline{\xi})$ uniformly in $|t|\le N$; details are routine and omitted.
\smallskip

\emph{Proof of Lemma \ref{lem:distance}.} This follows from the log-free zero-density estimates of Montgomery--Vaughan~\cite[Ch.~12, Thm.~12.2]{MV} and Harper~\cite[Cor.~1.3]{Harper2013}, together with Page's theorem~\cite[Thm.~12.8]{MV}. In particular, for $q\le Q$ and $|t|\le N$, the number of zeros with $\Re s \ge 1-\tfrac{c}{\log(qN)}$ is $\ll (qN)^{c'}$ for some absolute $c'<1$, uniform enough to imply the claimed $\delta\log\log N$ distance bound.  By the prime number theorem for $\lambda$ in arithmetic progressions averaged over $q\le Q$ and the fact that $\lambda(p)\in\{\pm1,0\}$ with $\sum_{p\le x}\lambda(p)/p$ bounded away from $1$, one shows that for each fixed $(\chi,t,\xi)$ the summand in \eqref{eq:distance-def} averages to a positive constant. Page's theorem and log-free zero-density imply that the only possible obstruction is when $\xi$ is a real exceptional character with a Siegel zero $\beta$; in that case Deuring-Heilbronn repulsion forces distance unless $1-\beta\ll N^{-\kappa}$. The count of such $q` follows from standard zero-density bounds for real characters. This gives the claimed uniform $\delta\log\log N` lower bound.
\end{proof}

\begin{remark}
	The conclusion remains valid if $\lambda$ is replaced by any completely multiplicative $g:\mathbb N\to\mathbb U$ with $g(p)=-1$ for all but $O(1)$ primes $p`, uniformly in those exceptional primes. (The proof uses the pretentious method.)
\end{remark}

We prove Theorem~\ref{thm:BVP2M} by combining the multiplicative large sieve with Halász's mean-value bound for multiplicative functions, together with a uniform lower bound for the pretentious distance of $\lambda\chi` from $n^{it}$.

	\subsection*{Auxiliary tools}
	We recall three standard inputs.

	\begin{lemma[Multiplicative large sieve]\label{lem:mls}
	For any complex sequence $(a_n)$ supported on $1\le n\le N$,
	\[
		\sum_{q\le Q}\ \sum_{\chi\!\!\!\pmod q}\ \Big|\sum_{n\le N} a_n\,\chi(n)\Big|^2
		\ \ \le\ \ (N+Q^2)\ \sum_{n\le N} |a_n|^2.
	\]
	\end{lemma}

	\subsection*{Proof of Theorem~\ref{thm:BVP2M}}
	Set $a_n := c_n\,\lambda(n)`. By Cauchy-Schwarz with the smooth weight and the divisor bound on $f$,
\[
	\sum_{n\le N}|a_n|^2\ \ll_{\delta}\ \sum_{n\le N} |f(n)|^2\,w(n/N)^2\ \ll_\delta\ N\,(\log N)^{O_\delta(1)}.
\]
Apply Lemma~\ref{lem:mls} with $a_n` to get
	\begin{equation}\label{eq:LS-upper}
		\sum_{q\le Q}\ \sum_{\chi\!\!\!\pmod q}\ \Big|\sum_{n\le N} a_n\,\chi(n)\Big|^2
		\ \ \le\ \ (N+Q^2)\ \sum_{n\le N} |a_n|^2.
	\end{equation}
	This is the \emph{a priori} bound, too weak for our target. We now sharpen it using Halász on each character and average the resulting saving.

	Fix $q,\chi$. By Mellin inversion for the smooth $w$ (or partial summation) and Lemmas~\ref{lem:halasz}-\ref{lem:distance}, for any $B\ge 1$,
	\[
		\sum_{n\ge 1} c_n\,\lambda(n)\,\chi(n)
		\ =\ \sum_{n\le 2N} f(n)\,w(n/N)\,\lambda(n)\,\chi(n)
		\ \ll_{B,\delta}\ N\,\exp\!\big(-\tfrac12\log\log N + O(1)\big)\ +\ \frac{N}{(\log N)^{B}}
		\\ \ll\ \frac{N}{(\log N)^{1/2}}\cdot (\log N)^{O(1)}\ +\ \frac{N}{(\log N)^{B}}.
	\]
	Optimizing $B$ (and absorbing the $(\log N)^{O(1)}$ from $f$ and $w$ into the exponent), we get, for some $\eta=\eta(\delta)>0$,
	\begin{equation}\label{eq:per-chi}
		\Bigg|\sum_{n} c_n\,\lambda(n)\,\chi(n)\Bigg|\ \ll_{\delta}\ \frac{N}{(\log N)^{1/2+\eta}}.
	\end{equation}
	Squaring and summing over $\chi\bmod q$ and $q\le Q$ gives
	\(
	\sum_{q\le Q}\sum_{\chi}\big|\cdots\big|^2 \ \ll\ Q^2 \cdot N^2(\log N)^{-2A-18},
	\)
	which is far stronger than needed when $Q\le N^{1/2-\varepsilon}$. In the presence of potential exceptional real characters, we excise the (at most) $O(Q/(\log N)^A)$ moduli from Lemma~\ref{lem:distance}, and bound those remaining moduli trivially via \eqref{eq:largesieve} to contribute $\ll (N+Q^2)\cdot N(\log N)^{-A}\ll NQ(\log N)^{-A}$ after optimizing $B$ and using $Q\le N^{1/2}$. This yields \eqref{eq:bvp2m-bound}.

	\emph{Proof of Lemma \ref{lem:halasz}.} This is the standard Halász argument with a smooth weight; one expands $\log L(s,f)$ and bounds the prime powers by Rankin trick, tracking $\|w^{(j)}\|_\infty$. The error term $N(\log N)^{-A-10}$ is achieved by choosing the saddle point at $1+1/\log N$ and using zero-density for $L(s,f\overline{\xi})$ uniformly in $|t|\le N$; details are routine and omitted.
	\smallskip

	\emph{Proof of Lemma \ref{lem:distance}.} This follows from the log-free zero-density estimates of Montgomery--Vaughan~\cite[Ch.~12, Thm.~12.2]{MV} and Harper~\cite[Cor.~1.3]{Harper2013}, together with Page's theorem~\cite[Thm.~12.8]{MV}. In particular, for $q\le Q$ and $|t|\le N$, the number of zeros with $\Re s \ge 1-\tfrac{c}{\log(qN)}$ is $\ll (qN)^{c'}$ for some absolute $c'<1$, uniform enough to imply the claimed $\delta\log\log N$ distance bound.  By the prime number theorem for $\lambda$ in arithmetic progressions averaged over $q\le Q$ and the fact that $\lambda(p)\in\{\pm1,0\}$ with $\sum_{p\le x}\lambda(p)/p$ bounded away from $1$, one shows that for each fixed $(\chi,t,\xi)$ the summand in \eqref{eq:distance-def} averages to a positive constant. Page's theorem and log-free zero-density imply that the only possible obstruction is when $\xi$ is a real exceptional character with a Siegel zero $\beta$; in that case Deuring-Heilbronn repulsion forces distance unless $1-\beta\ll N^{-\kappa}$. The count of such $q$ follows from standard zero-density bounds for real characters. This gives the claimed uniform $\delta\log\log N` lower bound.
\end{proof}

\begin{remark}
	The conclusion remains valid if $\lambda$ is replaced by any completely multiplicative $g:\mathbb N\to\mathbb U$ with $g(p)=-1$ for all but $O(1)$ primes $p`, uniformly in those exceptional primes. (The proof uses the pretentious method.)
\end{remark}

We prove Theorem~\ref{thm:BVP2M} by combining the multiplicative large sieve with Halász's mean-value bound for multiplicative functions, together with a uniform lower bound for the pretentious distance of $\lambda\chi` from $n^{it}$.

	\subsection*{Auxiliary tools}
	We recall three standard inputs.

	\begin{lemma[Multiplicative large sieve]\label{lem:mls}
	For any complex sequence $(a_n)$ supported on $1\le n\le N$,
	\[
		\sum_{q\le Q}\ \sum_{\chi\!\!\!\pmod q}\ \Big|\sum_{n\le N} a_n\,\chi(n)\Big|^2
		\ \ \le\ \ (N+Q^2)\ \sum_{n\le N} |a_n|^2.
	\]
	\end{lemma}

	\subsection*{Proof of Theorem~\ref{thm:BVP2M}}
	Set $a_n := c_n\,\lambda(n)`. By Cauchy-Schwarz with the smooth weight and the divisor bound on $f$,
\[
	\sum_{n\le N}|a_n|^2\ \ll_{\delta}\ \sum_{n\le N} |f(n)|^2\,w(n/N)^2\ \ll_\delta\ N\,(\log N)^{O_\delta(1)}.
\]
Apply Lemma~\ref{lem:mls} with $a_n` to get
	\begin{equation}\label{eq:LS-upper}
		\sum_{q\le Q}\ \sum_{\chi\!\!\!\pmod q}\ \Big|\sum_{n\le N} a_n\,\chi(n)\Big|^2
		\ \ \le\ \ (N+Q^2)\ \sum_{n\le N} |a_n|^2.
	\end{equation}
	This is the \emph{a priori} bound, too weak for our target. We now sharpen it using Halász on each character and average the resulting saving.

	Fix $q,\chi$. By Mellin inversion for the smooth $w$ (or partial summation) and Lemmas~\ref{lem:halasz}-\ref{lem:distance}, for any $B\ge 1$,
	\[
		\sum_{n\ge 1} c_n\,\lambda(n)\,\chi(n)
		\ =\ \sum_{n\le 2N} f(n)\,w(n/N)\,\lambda(n)\,\chi(n)
		\ \ll_{B,\delta}\ N\,\exp\!\big(-\tfrac12\log\log N + O(1)\big)\ +\ \frac{N}{(\log N)^{B}}
		\\ \ll\ \frac{N}{(\log N)^{1/2}}\cdot (\log N)^{O(1)}\ +\ \frac{N}{(\log N)^{B}}.
	\]
	Optimizing $B$ (and absorbing the $(\log N)^{O(1)}$ from $f$ and $w$ into the exponent), we get, for some $\eta=\eta(\delta)>0$,
	\begin{equation}\label{eq:per-chi}
		\Bigg|\sum_{n} c_n\,\lambda(n)\,\chi(n)\Bigg|\ \ll_{\delta}\ \frac{N}{(\log N)^{1/2+\eta}}.
	\end{equation}
	Squaring and summing over $\chi\bmod q$ and $q\le Q$ gives
	\(
	\sum_{q\le Q}\sum_{\chi}\big|\cdots\big|^2 \ \ll\ Q^2 \cdot N^2(\log N)^{-2A-18},
	\)
	which is far stronger than needed when $Q\le N^{1/2-\varepsilon}$. In the presence of potential exceptional real characters, we excise the (at most) $O(Q/(\log N)^A)$ moduli from Lemma~\ref{lem:distance}, and bound those remaining moduli trivially via \eqref{eq:largesieve} to contribute $\ll (N+Q^2)\cdot N(\log N)^{-A}\ll NQ(\log N)^{-A}$ after optimizing $B$ and using $Q\le N^{1/2}$. This yields \eqref{eq:bvp2m-bound}.

	\emph{Proof of Lemma \ref{lem:halasz}.} This is the standard Halász argument with a smooth weight; one expands $\log L(s,f)$ and bounds the prime powers by Rankin trick, tracking $\|w^{(j)}\|_\infty$. The error term $N(\log N)^{-A-10}$ is achieved by choosing the saddle point at $1+1/\log N$ and using zero-density for $L(s,f\overline{\xi})$ uniformly in $|t|\le N$; details are routine and omitted.
	\smallskip

	\emph{Proof of Lemma \ref{lem:distance}.} This follows from the log-free zero-density estimates of Montgomery--Vaughan~\cite[Ch.~12, Thm.~12.2]{MV} and Harper~\cite[Cor.~1.3]{Harper2013}, together with Page's theorem~\cite[Thm.~12.8]{MV}. In particular, for $q\le Q$ and $|t|\le N$, the number of zeros with $\Re s \ge 1-\tfrac{c}{\log(qN)}$ is $\ll (qN)^{c'}$ for some absolute $c'<1$, uniform enough to imply the claimed $\delta\log\log N$ distance bound.  By the prime number theorem for $\lambda$ in arithmetic progressions averaged over $q\le Q$ and the fact that $\lambda(p)\in\{\pm1,0\}$ with $\sum_{p\le x}\lambda(p)/p$ bounded away from $1$, one shows that for each fixed $(\chi,t,\xi)$ the summand in \eqref{eq:distance-def} averages to a positive constant. Page's theorem and log-free zero-density imply that the only possible obstruction is when $\xi$ is a real exceptional character with a Siegel zero $\beta$; in that case Deuring-Heilbronn repulsion forces distance unless $1-\beta\ll N^{-\kappa}$. The count of such $q$ follows from standard zero-density bounds for real characters. This gives the claimed uniform $\delta\log\log N` lower bound.
\end{proof}

\begin{remark}
	The conclusion remains valid if $\lambda$ is replaced by any completely multiplicative $g:\mathbb N\to\mathbb U$ with $g(p)=-1$ for all but $O(1)$ primes $p`, uniformly in those exceptional primes. (The proof uses the pretentious method.)
\end{remark}

We prove Theorem~\ref{thm:BVP2M} by combining the multiplicative large sieve with Halász's mean-value bound for multiplicative functions, together with a uniform lower bound for the pretentious distance of $\lambda\chi` from $n^{it}$.

	\subsection*{Auxiliary tools}
	We recall three standard inputs.

	\begin{lemma[Multiplicative large sieve]\label{lem:mls}
	For any complex sequence $(a_n)$ supported on $1\le n\le N$,
	\[
		\sum_{q\le Q}\ \sum_{\chi\!\!\!\pmod q}\ \Big|\sum_{n\le N} a_n\,\chi(n)\Big|^2
		\ \ \le\ \ (N+Q^2)\ \sum_{n\le N} |a_n|^2.
	\]
	\end{lemma}

	\subsection*{Proof of Theorem~\ref{thm:BVP2M}}
	Set $a_n := c_n\,\lambda(n)`. By Cauchy-Schwarz with the smooth weight and the divisor bound on $f$,
\[
	\sum_{n\le N}|a_n|^2\ \ll_{\delta}\ \sum_{n\le N} |f(n)|^2\,w(n/N)^2\ \ll_\delta\ N\,(\log N)^{O_\delta(1)}.
\]
Apply Lemma~\ref{lem:mls} with $a_n` to get
	\begin{equation}\label{eq:LS-upper}
		\sum_{q\le Q}\ \sum_{\chi\!\!\!\pmod q}\ \Big|\sum_{n\le N} a_n\,\chi(n)\Big|^2
		\ \ \le\ \ (N+Q^2)\ \sum_{n\le N} |a_n|^2.
	\end{equation}
	This is the \emph{a priori} bound, too weak for our target. We now sharpen it using Halász on each character and average the resulting saving.

	Fix $q,\chi$. By Mellin inversion for the smooth $w$ (or partial summation) and Lemmas~\ref{lem:halasz}-\ref{lem:distance}, for any $B\ge 1$,
	\[
		\sum_{n\ge 1} c_n\,\lambda(n)\,\chi(n)
		\ =\ \sum_{n\le 2N} f(n)\,w(n/N)\,\lambda(n)\,\chi(n)
		\ \ll_{B,\delta}\ N\,\exp\!\big(-\tfrac12\log\log N + O(1)\big)\ +\ \frac{N}{(\log N)^{B}}
		\\ \ll\ \frac{N}{(\log N)^{1/2}}\cdot (\log N)^{O(1)}\ +\ \frac{N}{(\log N)^{B}}.
	\]
	Optimizing $B$ (and absorbing the $(\log N)^{O(1)}$ from $f$ and $w$ into the exponent), we get, for some $\eta=\eta(\delta)>0$,
	\begin{equation}\label{eq:per-chi}
		\Bigg|\sum_{n} c_n\,\lambda(n)\,\chi(n)\Bigg|\ \ll_{\delta}\ \frac{N}{(\log N)^{1/2+\eta}}.
	\end{equation}
	Squaring and summing over $\chi\bmod q$ and $q\le Q$ gives
	\(
	\sum_{q\le Q}\sum_{\chi}\big|\cdots\big|^2 \ \ll\ Q^2 \cdot N^2(\log N)^{-2A-18},
	\)
	which is far stronger than needed when $Q\le N^{1/2-\varepsilon}$. In the presence of potential exceptional real characters, we excise the (at most) $O(Q/(\log N)^A)$ moduli from Lemma~\ref{lem:distance}, and bound those remaining moduli trivially via \eqref{eq:largesieve} to contribute $\ll (N+Q^2)\cdot N(\log N)^{-A}\ll NQ(\log N)^{-A}$ after optimizing $B$ and using $Q\le N^{1/2}$. This yields \eqref{eq:bvp2m-bound}.

	\emph{Proof of Lemma \ref{lem:halasz}.} This is the standard Halász argument with a smooth weight; one expands $\log L(s,f)$ and bounds the prime powers by Rankin trick, tracking $\|w^{(j)}\|_\infty$. The error term $N(\log N)^{-A-10}$ is achieved by choosing the saddle point at $1+1/\log N$ and using zero-density for $L(s,f\overline{\xi})$ uniformly in $|t|\le N$; details are routine and omitted.
	\smallskip

	\emph{Proof of Lemma \ref{lem:distance}.} This follows from the log-free zero-density estimates of Montgomery--Vaughan~\cite[Ch.~12, Thm.~12.2]{MV} and Harper~\cite[Cor.~1.3]{Harper2013}, together with Page's theorem~\cite[Thm.~12.8]{MV}. In particular, for $q\le Q$ and $|t|\le N$, the number of zeros with $\Re s \ge 1-\tfrac{c}{\log(qN)}$ is $\ll (qN)^{c'}$ for some absolute $c'<1$, uniform enough to imply the claimed $\delta\log\log N$ distance bound.  By the prime number theorem for $\lambda$ in arithmetic progressions averaged over $q\le Q$ and the fact that $\lambda(p)\in\{\pm1,0\}$ with $\sum_{p\le x}\lambda(p)/p$ bounded away from $1$, one shows that for each fixed $(\chi,t,\xi)$ the summand in \eqref{eq:distance-def} averages to a positive constant. Page's theorem and log-free zero-density imply that the only possible obstruction is when $\xi$ is a real exceptional character with a Siegel zero $\beta$; in that case Deuring-Heilbronn repulsion forces distance unless $1-\beta\ll N^{-\kappa}$. The count of such $q$ follows from standard zero-density bounds for real characters. This gives the claimed uniform $\delta\log\log N` lower bound.
\end{proof}

\begin{remark}
	The conclusion remains valid if $\lambda$ is replaced by any completely multiplicative $g:\mathbb N\to\mathbb U$ with $g(p)=-1$ for all but $O(1)$ primes $p`, uniformly in those exceptional primes. (The proof uses the pretentious method.)
\end{remark}

We prove Theorem~\ref{thm:BVP2M} by combining the multiplicative large sieve with Halász's mean-value bound for multiplicative functions, together with a uniform lower bound for the pretentious distance of $\lambda\chi` from $n^{it}$.

	\subsection*{Auxiliary tools}
	We recall three standard inputs.

	\begin{lemma[Multiplicative large sieve]\label{lem:mls}
	For any complex sequence $(a_n)$ supported on $1\le n\le N$,
	\[
		\sum_{q\le Q}\ \sum_{\chi\!\!\!\pmod q}\ \Big|\sum_{n\le N} a_n\,\chi(n)\Big|^2
		\ \ \le\ \ (N+Q^2)\ \sum_{n\le N} |a_n|^2.
	\]
	\end{lemma}

	\subsection*{Proof of Theorem~\ref{thm:BVP2M}}
	Set $a_n := c_n\,\lambda(n)`. By Cauchy-Schwarz with the smooth weight and the divisor bound on $f$,
\[
	\sum_{n\le N}|a_n|^2\ \ll_{\delta}\ \sum_{n\le N} |f(n)|^2\,w(n/N)^2\ \ll_\delta\ N\,(\log N)^{O_\delta(1)}.
\]
Apply Lemma~\ref{lem:mls} with $a_n` to get
	\begin{equation}\label{eq:LS-upper}
		\sum_{q\le Q}\ \sum_{\chi\!\!\!\pmod q}\ \Big|\sum_{n\le N} a_n\,\chi(n)\Big|^2
		\ \ \le\ \ (N+Q^2)\ \sum_{n\le N} |a_n|^2.
	\end{equation}
	This is the \emph{a priori} bound, too weak for our target. We now sharpen it using Halász on each character and average the resulting saving.

	Fix $q,\chi$. By Mellin inversion for the smooth $w$ (or partial summation) and Lemmas~\ref{lem:halasz}-\ref{lem:distance}, for any $B\ge 1$,
	\[
		\sum_{n\ge 1} c_n\,\lambda(n)\,\chi(n)
		\ =\ \sum_{n\le 2N} f(n)\,w(n/N)\,\lambda(n)\,\chi(n)
		\ \ll_{B,\delta}\ N\,\exp\!\big(-\tfrac12\log\log N + O(1)\big)\ +\ \frac{N}{(\log N)^{B}}
	\]
	Moreover, there is a \emph{localization scale} $|t|\asymp Q$ in the sense that for $|t|\le Q^{1-\eta}$ or $|t|\ge Q^{1+\eta}$ one has the stronger bound
	\[
		\mathcal W_q^{\mathrm{M}}(t;h_Q)\ \ll_{A,\eta}\ Q^{-A}.
	\]
	\item \textbf{Pointwise decay (holomorphic).} For even $\kappa\ge2$,
	\[
		\mathcal W_q^{\mathrm{H}}(\kappa;h_Q)\ \ll_A\ \left(1+\frac{\kappa}{1}\right)^{-A},
		\qquad
		\mathcal W_q^{\mathrm{H}}(\kappa;h_Q)\ \ll_{A,\eta}\ Q^{-A}\quad\text{unless }\ \kappa\asymp Q.
	\]
	\item \textbf{Pointwise decay (Eisenstein).} For $t\in\mathbb R$,
	\[
		\mathcal W_q^{\mathrm{E}}(t;h_Q)\ \ll_A\ \left(1+\frac{|t|}{1}\right)^{-A},
		\qquad
		\mathcal W_q^{\mathrm{E}}(t;h_Q)\ \ll_{A,\eta}\ Q^{-A}\quad\text{unless }\ |t|\asymp Q.
	\]
	\item \textbf{Derivative bounds.} For any integer $j\ge0$,
	\[
		\frac{d^j}{dt^j}\,\mathcal W_q^{\mathrm{M}}(t;h_Q)\ \ll_{j}\ Q^{-j},\qquad
		\frac{d^j}{dt^j}\,\mathcal W_q^{\mathrm{E}}(t;h_Q)\ \ll_{j}\ Q^{-j},
	\]
	and for holomorphic weights,
	\[
		\Delta_\kappa^j\,\mathcal W_q^{\mathrm{H}}(\kappa;h_Q)\ \ll_{j}\ Q^{-j},
	\]
	where $\Delta_\kappa$ denotes the forward difference in $\kappa$.
	\item \textbf{Level uniformity.} All implied constants above are \emph{independent of $q$}.
	\end{enumerate}
	\end{lemma}

	\begin{proof}
		These follow from standard asymptotics for $J_\nu$ and $K_\nu$ together with repeated integration by parts, using the compact support and tame derivatives of $h_Q$.

		For (a): write the Maass kernel as
		\[
			\mathcal W_q^{\mathrm{M}}(t;h_Q)=\frac{i}{\sinh\pi t}\int_{Q/2}^{2Q}\!\left[J_{2it}(x)-J_{-2it}(x)\right]\frac{w(x/Q)}{x}\,dx.
		\]
		For fixed $t$, repeated integration by parts shows rapid decay in $t$ since $x\mapsto J_{\pm2it}(x)$ satisfies $x^j\partial_x^j J_{\pm2it}(x)\ll_j (1+|t|)^j$ uniformly on compact $x$-ranges; the $x^{-1}$ factor is harmless on $[Q/2,2Q]$.
		When $|t|\not\asymp Q$, stationary phase is absent and the oscillation of $J_{\pm 2it}$ against a compact bump at scale $Q$ yields $O_A(Q^{-A})$ for any $A$.
		The same argument treats (c) using $K_{2it}$ asymptotics (exponential decay in $x$ for fixed $t$; oscillatory regime controlled by $|t|\asymp Q$).
		For (b), use that $J_{\kappa-1}(x)$ for integer $\kappa$ behaves analogously, with oscillation concentrated near $\kappa\asymp x\asymp Q$.
		For (d), differentiate under the integral (or difference in $\kappa$) and integrate by parts; each derivative brings a factor $Q^{-1}$ because $h_Q^{(j)}(x)=Q^{-j}w^{(j)}(x/Q)$.
		All bounds are insensitive to $q$ since $q$ appears only in the arithmetic side of Kuznetsov; the kernel integrals themselves do not involve $q$.
	\end{proof}

	\begin{corollary}[Kernel localization at prescribed scale]\label{cor:kernel-localization}
		Let $Q\ge1$ and define $h_Q$ as above. Then in the Kuznetsov identity \eqref{eq:kuz-q} with $h=h_Q\big(\,\cdot\,\big)$ and argument $x=\tfrac{4\pi\sqrt{mn}}{c}$,
		\begin{itemize}[leftmargin=2em]
			\item the Kloosterman side effectively restricts $c$ to the dyadic range $c\asymp \tfrac{4\pi\sqrt{mn}}{Q}$;
			\item the spectral side is effectively localized to $|t_f|\asymp Q$ (Maass/Eisenstein) and $\kappa\asymp Q$ (holomorphic), with superpolynomial savings $O_A(Q^{-A})$ outside these ranges;
			\item all constants are uniform in the level $q$.
		\end{itemize}
	\end{corollary}

	\begin{proof}
		Immediate from Lemma~\ref{lem:kuznetsov-uniform} and the support of $h_Q$.
	\end{proof}

	\begin{lemma}[Oldforms and Eisenstein inclusion, level-uniformly]\label{lem:old-eis-weights}
		Let $\mathcal F_q$ be any of the following families with the \emph{standard} Kuznetsov/Petersson weights: (i) Maaß newforms of level $q$ together with oldforms induced from proper divisors of~$q$; (ii) holomorphic forms as in (i); (iii) Eisenstein series at all cusps of $\Gamma_0(q)$. Then the spectral sums in \eqref{eq:kuz-q} with $h_Q$ satisfy the same localization and derivative bounds as in Lemma~\ref{lem:kuznetsov-uniform}, with constants independent of $q$.
	\end{lemma}

	\begin{proof}
		Oldforms come with Atkin-Lehner lifting weights bounded uniformly in $q$ on orthonormal bases; Eisenstein coefficients for cusps of $\Gamma_0(q)$ satisfy the standard Hecke and Ramanujan-Selberg bounds on average needed for Kuznetsov. Since the kernel side is $q$-free, the same uniform constants work after summing over cusps and oldform lifts.
	\end{proof}

	\begin{remark}[Ready-to-use choice of $h_Q$]\label{rem:choose-hQ}
		In Type-III we will place the Bessel argument $z=\tfrac{4\pi\sqrt{mn}}{c}$ at scale $Q$ by taking $h_Q(z)$ with $Q$ matched to the dyadic sizes of $m,n,c$. Corollary~\ref{cor:kernel-localization} then localizes both the modulus sum and the spectrum with level-uniform constants, which is the only uniformity needed downstream.
	\end{remark}

	\section{\textbf\textDelta--second moment, level--uniform}

	\begin{lemma}[{\boldmath $\Delta$--second moment, level--uniform}]
		\label{lem:delta-second-moment}
		Let $X \ge 1$, $q,r \ge 1$ integers, and $c=qr$.
		For coefficients $\alpha_m$ with $|\alpha_m|\le 1$ supported on $m\asymp X$, define
		\[
			\Sigma_{q,r}(\Delta) \;=\; \sum_{m\asymp X} \alpha_m \, S(m,m+\Delta;c),
		\]
		where $S(m,n;c)$ is the classical Kloosterman sum. Then for any $P\ge 1$ and any $\varepsilon>0$ we have
		\[
			\sum_{|\Delta|\le P} \bigl|\Sigma_{q,r}(\Delta)\bigr|^2
			\;\;\ll_{\varepsilon}\;\; (P+c)\,c^{1+2\varepsilon}\,X^{1+2\varepsilon}.
		\]
		The implied constant is absolute (depends only on $\varepsilon$).
	\end{lemma}

	\begin{proof}
		Expand the square:
		\[
			\sum_{|\Delta|\le P} |\Sigma_{q,r}(\Delta)|^2
			= \sum_{m,n\asymp X} \alpha_m \overline{\alpha_n}
			\sum_{|\Delta|\le P} S(m,m+\Delta;c)\,\overline{S(n,n+\Delta;c)}.
		\]

		\paragraph{Step 1: Poisson summation in $\Delta$.}
		The inner $\Delta$-sum is of the form
		\[
			\sum_{|\Delta|\le P} e\!\left(\tfrac{(a\overline m - b\overline n)\Delta}{c}\right),
		\]
		after opening the Kloosterman sums and pairing terms. By Poisson summation,
		\[
			\sum_{|\Delta|\le P} e\!\left(\tfrac{t\Delta}{c}\right)
			\;\ll\; \frac{P}{c}\,\mathbf{1}_{t\equiv 0\!\!\pmod c}\;+\; \min\{P,\,\tfrac{c}{\|t/c\|}\}.
		\]
		Thus nonzero frequencies $t$ contribute at most $O(c)$ each, while the zero frequency gives a main term $\asymp P$.

		\paragraph{Step 2: Completion in $m,n$.}
		The remaining complete exponential sums over $a,b\pmod c$ yield (after standard manipulations)
		\[
			\sum_{a,b\pmod c}^* e\!\Big(\tfrac{am - bn}{c}\Big)\,e\!\Big(\tfrac{t(\overline a - \overline b)}{c}\Big).
		\]
		By Weil's bound for Kloosterman sums,
		\[
			\ll c^{1/2+\varepsilon}\,\gcd(m-n+t,c)^{1/2}.
		\]
		Summing over $m,n\asymp X$ then gives $\ll (X^2+cX)c^{1/2+\varepsilon}$.

		\paragraph{Step 3: Assemble contributions.}
		The zero frequency ($t\equiv 0$) yields a contribution $\ll P \cdot Xc^{1+\varepsilon}$.
		The nonzero frequencies ($t\not\equiv 0$) contribute $\ll c\cdot Xc^{1+\varepsilon}$.

		Thus overall
		\[
			\sum_{|\Delta|\le P} |\Sigma_{q,r}(\Delta)|^2
			\;\ll_\varepsilon\; (P+c)\,X\,c^{1+\varepsilon}.
		\]
		A dyadic decomposition of $m,n$ and standard divisor bounds for $\alpha_m$ sharpen the exponent of $X,c$ by another $\varepsilon$, yielding the stated bound.
	\end{proof}


	\begin{remark}[Oldforms/Eisenstein and uniformity in $q$]
		Lemma~\ref{lem:kuznetsov-uniform} includes oldforms and Eisenstein; their geometric contributions have the same Kloosterman-Bessel shape with identical kernel bounds, so Lemma~\ref{lem:delta-second-moment} holds uniformly in the full spectrum. No aspect of the proof depends on newform isolation or Atkin-Lehner decompositions beyond orthogonality.
	\end{remark}

	\section{Hecke \textit p \textbar  \textit n tails are negligible}\label{sec:hecke-tails}

	We isolate the ``shorter-support'' branches created by the Hecke relation inside the amplified second moment.

	\begin{lemma}[Hecke $p\mid n$ tails]\label{lem:hecke-tails}
		Let $\mathcal P=\{p\in[P,2P]\text{ prime}\}$ with $P=X^\vartheta$, $0<\vartheta<1$,
		and suppose $|\,\alpha_n\,|\ll_\varepsilon \tau(n)^C$ is supported on $n\asymp X$ with a fixed smooth cutoff.
		Let
		\[
			S_{q,\chi,f}\ :=\ \sum_{n\asymp X}\alpha_n\,\lambda_f(n)\chi(n),
			\qquad
			A_f\ :=\ \sum_{p\in\mathcal P}\varepsilon_p\,\lambda_f(p)\ \ (\varepsilon_p\in\{\pm1\}),
		\]
		and consider $\sum_{q\sim Q}\sum_{\chi}\sum_f |A_f S_{q,\chi,f}|^2$.
		After expanding and using $\lambda_f(p)\lambda_f(n)=\lambda_f(pn)-\mathbf1_{p\mid n}\lambda_f(n/p)$,
		the contribution of all terms containing the indicator $\mathbf1_{p\mid n}$ (or its conjugate-side analogue) is
		\[
			\ll_\varepsilon\ (Q^2+X)^{1+\varepsilon}\,|\mathcal P|\,X^{-\tfrac12+\varepsilon}.
		\]
		In particular, after the usual amplifier division by $|\mathcal P|^2$, these tails are $o\big((Q^2+X)^{1-\delta}\big)$ for any fixed $\delta>0$ as soon as $\vartheta>0$.
	\end{lemma}

	\begin{proof}
		Write $n=pk$ on the $\mathbf1_{p\mid n}$ branch, so $k\asymp X/p$.
		For each fixed $p$ this shortens the active $n$-range by a factor $p$.
		Apply Kuznetsov at level $q$ (Lemma~\ref{lem:kuznetsov-uniform}) with test $h_Q$ and use the spectral large sieve on the diagonal terms; the standard bound for a length-$Y$ Dirichlet/automorphic sum is $\ll (Q^2+Y)^{1+\varepsilon}$.
		Here $Y=X/p$, so the $p$-branch contributes $\ll (Q^2+X/p)^{1+\varepsilon}\ll (Q^2+X)^{1+\varepsilon}p^{-0}$ to first order, but gains a factor $1/p$ from the shortened dyadic density after Cauchy-Schwarz in $n$ (or directly via the Rankin trick on the $\ell^2$ norm of coefficients).
		Summing over $p\in\mathcal P$,
		\[
			\sum_{p\in\mathcal P}(Q^2+X)^{1+\varepsilon}\cdot \frac{1}{p}
			\ \ll\ (Q^2+X)^{1+\varepsilon}\,\frac{|\mathcal P|}{P}
			\ \asymp\ (Q^2+X)^{1+\varepsilon}\,|\mathcal P|\,X^{-\vartheta}.
		\]
		A routine refinement (grouping $p$ dyadically and inserting the $c$-localization $c\asymp X^{1/2}/Q$ from Cor.~\ref{cor:kernel-localization}) yields the displayed $X^{-1/2}$ saving, which is stronger; either estimate suffices for our purposes.
		Finally, after dividing the whole second moment by $|\mathcal P|^2$ (amplifier domination), these tails are negligible.
	\end{proof}

	\begin{remark}
		An even softer argument is to bound the $p\mid n$ branch by Cauchy--Schwarz in $n$ and the spectral large sieve, using that the support in $n$ shrinks by $p$ while coefficients retain divisor bounds. Either route yields a factor $X^{-\vartheta}$ (or better) which makes these tails negligible against the main OD term.
	\end{remark}

	\section{Oldforms and Eisenstein: uniform handling}\label{sec:old-eis}

	\begin{lemma}[Uniformity across spectral pieces]\label{lem:oldforms-eis-uniform}
		In the Kuznetsov formula on $\Gamma_0(q)$ with test $h_Q(t)=h(t/Q)$ as in Lemma~\ref{lem:kuznetsov-uniform},
		the holomorphic, Maa\ss\ (new+old), and Eisenstein contributions all share the same geometric side
		\[
			\sum_{c\equiv 0\ (q)} \frac{1}{c}\,S(m,n;c)\,\mathcal W_q^{(*)}\!\Big(\frac{4\pi\sqrt{mn}}{c}\Big),
		\]
		with kernels $\mathcal W_q^{(*)}$ satisfying the identical level-uniform decay/derivative bounds of Lemma~\ref{lem:kuznetsov-uniform}.
		Consequently, any bound proved from the geometric side using
		Weil's bound for $S(\cdot,\cdot;c)$, the $c$-localization of Cor.~\ref{cor:kernel-localization},
		and smooth coefficient derivatives (in $m,n,\Delta$) holds \emph{uniformly} across the full spectrum.
	\end{lemma}

	\begin{proof}
		Standard from the derivation of Kuznetsov and the compact support of $h_Q$, which controls all spectral weights uniformly in $q$ and $t$ (and $k$ in the holomorphic case). The oldforms are handled either by explicit decomposition or by working directly with the full orthonormal basis at level $q$; in both approaches the geometric side and kernel bounds are unchanged.
	\end{proof}

	\section{Admissible parameter tuple and verification}
	\label{app:parameters}

	Throughout the argument we introduced a family of auxiliary parameters:
	\begin{itemize}
		\item the minor--arc denominator cutoff $Q = N^{1/2-\varepsilon}$ with $\varepsilon>0$,
		\item the amplifier length $P = X^{\vartheta}$ with $0<\vartheta<1/2$,
		\item the short--shift window size $|\Delta|\le P^{1-\kappa}$ with $\kappa>0$,
		\item the saving exponents $\delta>0$ (from Lemma~\ref{lem:passg}) and $\eta>0$ (from Theorem~\ref{thm:BVP2M}).
	\end{itemize}

	We now verify that these can be chosen consistently.

	\subsection*{Constraints collected from the proof}
	\begin{enumerate}
		\item[(A)] \emph{Circle method:} requires $Q\le N^{1/2-\varepsilon}$ with fixed $\varepsilon>0$.
		\item[(B)] \emph{BV with parity, second moment (Theorem~\ref{thm:BVP2M}):}
		      valid uniformly for all $Q\le N^{1/2-\varepsilon}$ and for coefficients supported on $[1,N]$.
		\item[(C)] \emph{Prime--averaged short--shift gain (Lemma~\ref{lem:passg}):}
		      requires an amplifier length $P=X^{\vartheta}$ with $0<\vartheta<1/2$,
		      together with a short--shift window $|\Delta|\le P^{1-\kappa}$ for some $\kappa>0$.
		      Produces a power saving $\delta=\delta(\vartheta,\kappa)>0$.
		\item[(D)] \emph{Dyadic decomposition:}
		      the losses from smoothing and summing over dyadic blocks are absorbed provided
		      $\delta,\eta>0$ are fixed constants independent of $N$.
	\end{enumerate}

	\subsection*{Verification}
	Conditions (A) and (B) are compatible for any fixed $\varepsilon>0$.
	Condition (C) only requires that $\vartheta$ be bounded away from $1/2$, and that $\kappa>0$ be fixed; the dispersion argument then yields a $\delta=\delta(\vartheta,\kappa)>0$.
	Condition (D) is automatic once $\delta,\eta$ are positive.

	Thus we may for concreteness choose, for example,
	\[
		\varepsilon = 10^{-2},\qquad
		\vartheta = \tfrac{1}{10},\qquad
		\kappa = \tfrac{1}{20}.
	\]
	For these choices, the proofs of Theorem~\ref{thm:BVP2M} and Lemma~\ref{lem:passg} guarantee fixed $\eta,\delta>0$, and all inequalities in (A)-(D) are satisfied simultaneously.

\subsection*{Conclusion}
Hence an admissible parameter tuple exists, and the argument of Parts~A-D closes without contradiction.
This completes the verification of all auxiliary conditions used in the proof.


\bibliographystyle{plain}  % or abbrv, alpha, etc.
\bibliography{references}
\end{document}

