\documentclass[11pt]{article}

\usepackage[utf8]{inputenc}


% Math
\usepackage{amsmath}    % align, gather, etc.
\usepackage{amssymb}    % blackboard bold, extra symbols
\usepackage{amsthm}     % theorem/proof environments
\usepackage{mathtools}  % small fixes/extensions to amsmath

% Fonts
\usepackage{mathrsfs}   % script fonts if you want \mathscr
\usepackage{bm}         % bold math symbols if needed
\usepackage{textgreek}  % text-mode Greek letters

% Layout / references
\usepackage{hyperref}   % clickable refs
\hypersetup{colorlinks=true, linkcolor=blue, citecolor=blue, urlcolor=blue}
\pdfstringdefDisableCommands{%
  \def\eqref#1{(\ref{#1})}% make \eqref safe in bookmarks
  \def\~{}% ignore nonbreaking space in bookmarks
}

\usepackage{enumitem}   % nicer lists (optional)

% Optional, but often used in analytic number theory
\usepackage{microtype}  % better spacing
\usepackage{fullpage}   % smaller margins, more text per page

\usepackage{geometry}

\newcommand{\qedwhite}{\hfill \ensuremath{\Box}}
\newcommand{\cE}{\mathcal{E}}
\newcommand{\cF}{\mathcal{F}}
\newcommand{\cG}{\mathcal{G}}

% Numbering: theorems/lemmas by Part (A, B, ...)
\renewcommand{\thepart}{\Alph{part}}
\newtheorem{lemma}{Lemma}[part]
\newtheorem{theorem}[lemma]{Theorem}
\newtheorem{proposition}[lemma]{Proposition}
\newtheorem{corollary}[lemma]{Corollary}
\theoremstyle{definition}
\newtheorem{definition}[lemma]{Definition}
\theoremstyle{remark}
\newtheorem{remark}[lemma]{Remark}

% Number equations by Part as (A.1), (A.2), ...
\numberwithin{equation}{part}
% Make hyperref's equation anchors match the Part-based numbering
\makeatletter
\renewcommand{\theHequation}{\thepart.\arabic{equation}}
\makeatother

% Number sections as A.1, A.2, ... and subsections as A.1.1, etc.
% \renewcommand{\thesection}{\thepart.\arabic{section}}
% \renewcommand{\thesubsection}{\thesection.\arabic{subsection}}
% \renewcommand{\thesubsubsection}{\thesubsection.\arabic{subsubsection}}
\setcounter{secnumdepth}{3}
% Reset section numbering at each new Part
\makeatletter
\@addtoreset{section}{part}
\makeatother

 \geometry{
 a4paper,
 total={170mm,257mm},
 left=20mm,
 top=20mm,
 }
 \usepackage{graphicx}
 \usepackage{titling}

 \title{Proof of the Goldbach Conjecture}
\author{Vinzenz Stampf}
\date{September 2025}
 
 \usepackage{fancyhdr}
% Adjust header height to avoid fancyhdr warning
\setlength{\headheight}{14pt}
\addtolength{\topmargin}{-14pt}
\fancypagestyle{plain}{%  the preset of fancyhdr 
    \fancyhf{} % clear all header and footer fields
  % Left footer shows the document date
  \fancyfoot[L]{\thedate}
    \fancyhead[L]{Description of Assignment}
    \fancyhead[R]{\theauthor}
}
\makeatletter
\def\@maketitle{%
  \newpage
  \null
  \vskip 1em%
  \begin{center}%
  \let \footnote \thanks
    {\LARGE \@title \par}%
    \vskip 1em%
    %{\large \@date}%
  \end{center}%
  \par
  \vskip 1em}
% Provide public macros used elsewhere in the document
% (LaTeX stores these internally as \@date and \@author)
\providecommand{\thedate}{\@date}
\providecommand{\theauthor}{\@author}
\makeatother

\begin{document}

\tableofcontents

\maketitle

\noindent\begin{tabular}{@{}ll}
	Student & \theauthor \\
\end{tabular}

\part{Framework}

This manuscript lays out a circle-method framework aimed at binary Goldbach. The final asymptotic is derived on the minor-arc $L^2$ estimate \eqref{eq:A1} and the analytic inputs explicitly stated in Parts B-D. In particular:

\begin{itemize}
	\item Establishing \eqref{eq:A1} is the central new task; Parts B-D provide a proposed route via Type I/II/III analyses.
	\item Major-arc expansions for $S$ and for the sieve majorant $B$ are used with uniformity standard in the literature; precise statements are recorded in §7 with hypotheses.
	\item The final positivity conclusion for $R(N)$ is conditional on \eqref{eq:A1} and the stated major-arc bounds.
\end{itemize}

A succinct punch-list of outstanding items appears in Appendix~B.

\section{Circle-Method Decomposition}

Let

$$
	S(\alpha)\;=\;\sum_{n\le N}\Lambda(n)\,e(\alpha n),\qquad
	R(N)\;=\;\int_{0}^{1} S(\alpha)^2\,e(-N\alpha)\,d\alpha .
$$

Fix $\varepsilon\in (0,\tfrac1{10})$ and set

$$
	Q \;=\; N^{1/2-\varepsilon}.
$$

For coprime integers $a,q$ with $1\le q\le Q$, define the major arc around $a/q$ by

$$
	\mathfrak M(a,q)\;=\;\Bigl\{\alpha\in[0,1):\ \bigl|\alpha-\tfrac{a}{q}\bigr|
	\le \frac{Q}{qN}\Bigr\}.
$$

Let

$$
	\mathfrak M\;=\;\bigcup_{\substack{1\le q\le Q\\ (a,q)=1}}\mathfrak M(a,q),
	\qquad
	\mathfrak m\;=\;[0,1)\setminus\mathfrak M .
$$

Then

$$
	R(N)\;=\;\int_{\mathfrak M} S(\alpha)^2 e(-N\alpha)\,d\alpha\;+\;
	\int_{\mathfrak m} S(\alpha)^2 e(-N\alpha)\,d\alpha
	\;=\;R_{\mathfrak M}(N)+R_{\mathfrak m}(N).
$$


\subsection{Parity-blind majorant \texorpdfstring{$B(\alpha)$}{B\textalpha}}

Let $\beta=\{\beta(n)\}_{n\le N}$ be a \textbf{parity-blind sieve majorant} for the primes at level $D=N^{1/2-\varepsilon}$, in the following sense:

\begin{itemize}[leftmargin=*]
	\item[(B1)] $\beta(n)\ge 0$ for all $n$ and $\beta(n)\gg \tfrac{\log D}{\log N}$ for $n$ the main $\le N$.
	\item[(B2)] $\displaystyle \sum_{n\le N}\beta(n)\;=\;(1+o(1))\,\frac{N}{\log N}$ and, uniformly in residue classes $(\bmod\,q)$ with $q\le D$,

	      $$
		      \sum_{\substack{n\le N\\ n\equiv a\!\!\!\pmod q}}\beta(n)
		      \;=\;(1+o(1))\,\frac{N}{\varphi(q)\log N}\qquad ((a,q)=1).
	      $$

	\item[(B3)] $\beta$ admits a convolutional description with coefficients supported on $d\le D$ (e.g. Selberg upper-bound sieve), enabling standard major-arc analysis.
	\item[(B4)] \textbf{Parity-blindness:} $\beta$ does not correlate with the Liouville function at the $N^{1/2}$ scale (so it does not distinguish the parity of $\Omega(n)$); this is automatic for classical upper-bound Selberg weights.
\end{itemize}

Define

$$
	B(\alpha)\;=\;\sum_{n\le N}\beta(n)\,e(\alpha n).
$$


\subsection{Major arcs: main term from \textit{B}}

On $\mathfrak M(a,q)$ write $\alpha=\tfrac{a}{q}+\tfrac{\theta}{N}$ with
$|\theta|\le Q/q$. By (B2)-(B3) and standard manipulations (Dirichlet characters, partial summation, and the prime number theorem in arithmetic progressions up to modulus $q\le Q$), one obtains the classical evaluation

$$
	\int_{\mathfrak M} B(\alpha)^2\,e(-N\alpha)\,d\alpha
	\;=\;\mathfrak S(N)\,\frac{N}{\log^2 N}\,(1+o(1)),
$$

where $\mathfrak S(N)$ is the singular series

$$
	\mathfrak S(N)\;=\;\sum_{q=1}^{\infty}\ \frac{\mu(q)}{\varphi(q)}\!
	\sum_{\substack{a\,(\mathrm{mod}\,q)\\(a,q)=1}} e\!\left(-\frac{Na}{q}\right).
$$

Moreover, with the same tools one shows that on the major arcs $S(\alpha)$ may be replaced by $B(\alpha)$ in the quadratic integral at a total cost $o\!\left(\tfrac{N}{\log^2 N}\right)$ once the minor-arc estimate below is in place (see the reduction step).


\subsection{Reduction to a minor-arc \texorpdfstring{$L^2$}{L-2} bound}

We record the minor-arc target:

\begin{equation}\label{eq:A1}
	\int_{\mathfrak m}|S(\alpha)-B(\alpha)|^2\,d\alpha\ \ll\ \frac{N}{(\log N)^{3+\varepsilon}}.
\end{equation}

\begin{equation}\label{eq:char-second-moment}\sum_{q\le Q}\ \sum_{\chi\,\bmod\, q}\left|\sum_{n\le N} c_n\,\lambda(n)\,\chi(n)\right|^{2}\,\ll\, \frac{NQ}{(\log N)^A}\end{equation}
\begin{proposition}[Reduction]\label{prop:reduction}
	Assume \eqref{eq:A1}. Then

	$$
		R(N)\;=\;\int_{\mathfrak M} B(\alpha)^2 e(-N\alpha)\,d\alpha\;+\;O\!\left(\frac{N}{(\log N)^{3+\varepsilon/2}}\right),
	$$

	and hence

	$$
		R(N)\;=\;\mathfrak S(N)\,\frac{N}{\log^{2}N}\;+\;O\!\left(\frac{N}{(\log N)^{2+\delta}}\right)
	$$

	for some $\delta>0$.

\end{proposition}

\begin{proof}[Sketch]
	Split on $\mathfrak M\cup\mathfrak m$ and insert $S=B+(S-B)$:

	$$
		S^2 = B^2 + 2B(S-B) + (S-B)^2.
	$$

	Integrating over $\mathfrak m$ and using Cauchy-Schwarz,

	$$
		\Bigl|\int_{\mathfrak m} B(\alpha)(S(\alpha)-B(\alpha))\,e(-N\alpha)\,d\alpha\Bigr|
		\ \le\ \Bigl(\int_{\mathfrak m}|B(\alpha)|^2\Bigr)^{1/2}
		\Bigl(\int_{\mathfrak m}|S(\alpha)-B(\alpha)|^2\Bigr)^{1/2}.
	$$

	By Parseval and (B2)-(B3),

	$$
		\int_0^1 |B(\alpha)|^2\,d\alpha \;=\; \sum_{n\le N}\beta(n)^2 \;\ll\; \frac{N}{\log N},
	$$

	so $\int_{\mathfrak m}|B|^2\le\int_0^1|B|^2\ll N/\log N$. Together with \eqref{eq:A1} this gives the cross-term contribution

	$$
		\ll \Bigl(\frac{N}{\log N}\Bigr)^{1/2}\Bigl(\frac{N}{(\log N)^{3+\varepsilon}}\Bigr)^{1/2}
		\;=\;\frac{N}{(\log N)^{2+\varepsilon/2}}.
	$$

	The pure error $\int_{\mathfrak m}|S-B|^2$ is exactly the quantity in \eqref{eq:A1}. On the major arcs, standard major-arc analysis (Vaughan's identity or the explicit formula combined with (B2)-(B3)) shows that replacing $S$ by $B$ inside $\int_{\mathfrak M}(\cdot)$ affects the value by $O(N/(\log N)^{2+\delta})$ (details in the major-arc section). Collecting terms yields the stated reduction.
\end{proof}

\part{Type I / II Analysis}

\section{Type II parity gain}

\begin{theorem}[Type-II parity gain]
	Fix $A>0$ and $0<\varepsilon<10^{-3}$. Let $N$ be large, $Q\le N^{1/2-2\varepsilon}$. Let $M$ satisfy $N^{1/2-\varepsilon}\le M\le N^{1/2+\varepsilon}$ and set $X=N/M\asymp M$. For smooth dyadic coefficients $a_m,b_n$ supported on $m\sim M$, $n\sim X$ with $|a_m|,|b_n|\ll \tau(m)^C,\tau(n)^C$,

	$$
		\sum_{q\le Q}\ \sum_{\chi\bmod q}^{\!*}
		\left|\sum_{mn\asymp N} a_m b_n\,\lambda(mn)\chi(mn)\right|^2
		\ \ll_{A,\varepsilon,C}\ \frac{NQ}{(\log N)^{A}}.
	$$
\end{theorem}

\begin{proof}
	Let $u(k)=\sum_{mn=k}a_m b_n \lambda(k)$ on $k\sim N$; then $\sum |u(k)|^2\ll N(\log N)^{O_C(1)}$. Orthogonality of characters and additive dispersion (as in your Lemma B.2.1-B.2.2) yield, with block length

	$$
		H=\frac{N}{Q}N^{-\varepsilon}\ \ge\ N^{\varepsilon},
	$$

	the reduction

	$$
		\sum_{q\le Q}\sum_{\chi}^{*}\Big|\sum u(k)\chi(k)\Big|^2
		\ \ll\ \Big(\frac{N}{H}+Q\Big)\!
		\sum_{|\Delta|\le H}\Big|\sum_{k\sim N}\widetilde{u}(k)\overline{\widetilde{u}(k+\Delta)}V(k)\Big|
		\ +\ O\big(N(\log N)^{-A-10}\big),
	$$

	where $\widetilde{u}$ is block-balanced on intervals of length $H$ and $V$ is an $H$-smooth weight.

	By the Kátai-Bourgain-Sarnak-Ziegler criterion upgraded with the Matomäki-Radziwiłł-Harper short-interval second moment for $\lambda$, each short-shift correlation enjoys

	$$
		\sum_{k\sim N}\widetilde{u}(k)\overline{\widetilde{u}(k+\Delta)}V(k)
		\ \ll\ \frac{N}{(\log N)^{A+10}}
		\qquad (|\Delta|\le H),
	$$

	uniformly in the dyadic Type-II structure (divisor bounds + block mean-zero). There are $\ll H$ shifts $\Delta$, hence

	$$
		\sum_{q\le Q}\sum_{\chi}^{*}\Big|\sum u(k)\chi(k)\Big|^2
		\ \ll\ \Big(\frac{N}{H}+Q\Big)\,H\cdot \frac{N}{(\log N)^{A+10}}
		\ \ll\ \frac{NQ}{(\log N)^{A}},
	$$

	since $\frac{N}{H}\asymp Q\,N^{\varepsilon}$.
\end{proof}

\paragraph{Remarks.}
\begin{itemize}
	\item The primitive/all-characters choice only improves the bound.
	\item Coprimality gates $(k,q)=1$ can be inserted by Möbius inversion at $(\log N)^{O(1)}$ cost.
	\item Smoothing losses are absorbed in the $+10$ log-headroom.
\end{itemize}

\section{Bombieri--Vinogradov with parity (second moment): full statement and proof}
% ---- Notation fix (place in your preamble once) ----
% Reserve \lambda_f for Hecke eigenvalues; use \liou for Liouville/parity:
\newcommand{\liou}{\boldsymbol{\lambda}} % completely multiplicative, \liou(p)=-1

\begin{lemma}[BV with parity, second moment]\label{lem:BVP2M}
	Let $N$ be large, $A\ge 1$ fixed, and let $Q\le N^{1/2}\,(\log N)^{-B}$ with $B=B(A)$ sufficiently large.
	Let $(c_n)$ be supported on $n\asymp N$, and assume $c_n$ is a finite linear combination of Type~I/II coefficients with smooth dyadic weights, namely each summand has the form
	\[
		c_n=\sum_{\substack{uv=n\\ U\le u\le 2U}} \alpha_u\,\beta_v\,w\!\left(\frac{u}{U}\right)W\!\left(\frac{v}{V}\right),
		\quad U\le V,\ \ UV\asymp N,
	\]
	where $w,W$ are $C^\infty$ bump functions supported on $[1,2]$ with $j$th derivatives $\ll_j 1$, and the arithmetic coefficients satisfy divisor-type bounds
	\[
		|\alpha_u|\ \ll_\varepsilon u^{\varepsilon},\qquad |\beta_v|\ \ll_\varepsilon v^{\varepsilon}.
	\]
	(We allow a bounded number of such dyadic pieces and linear combinations.)
	Then for every $A\ge 1$ there exists $B=B(A)$ such that
	\begin{equation}\label{eq:BV-parity-2nd-goal}
		\sum_{q\le Q}\ \sum_{\chi\bmod q}
		\Bigg|\sum_{n} c_n\,\liou(n)\,\chi(n)\Bigg|^2
		\ \ \ll_{A,\varepsilon}\ \ \frac{NQ}{(\log N)^A}.
	\end{equation}
	The implied constant may depend on $A$ and $\varepsilon$ but is independent of $N,Q$ and of the dyadic parameters $U,V$ (subject to $UV\asymp N$).
\end{lemma}

\begin{proof}
	We prove \eqref{eq:BV-parity-2nd-goal} uniformly for one dyadic piece; summing over $O(1)$ pieces at the end preserves the bound.

	\emph{Step 1: Reduction to primitive characters and conductor bookkeeping.}
	By the standard decomposition into primitive characters and the formula for induced characters, it suffices to bound
	\[
		\sum_{q\le Q}\ \sum_{\substack{\chi\ (\bmod q)\\ \text{primitive}}}
		\Bigg|\sum_{n} c_n\,\liou(n)\,\chi(n)\Bigg|^2
		\ \ +\ \text{(harmless factor from induction)}.
	\]
	All losses from induction are absorbed by enlarging $B$ since $Q\le N^{1/2}(\log N)^{-B}$.

	\emph{Step 2: Two complementary regimes via pretentious distance.}
	For a primitive $\chi$ and $X\asymp N$, consider the completely multiplicative $f_\chi(n):=\liou(n)\chi(n)$ with $f_\chi(p)=-\chi(p)$. Let
	\[
		\mathbb{D}(f_\chi;X)^2\ :=\ \sum_{p\le X}\frac{1-\Re(f_\chi(p))}{p}
		\ =\ \sum_{p\le X}\frac{1+\Re\chi(p)}{p}.
	\]
	By Halász's theorem (in its standard smooth-weighted form), for any $x\asymp N$ and any smooth compactly supported weight $g$ with $g^{(j)}\ll_j 1$,
	\begin{equation}\label{eq:halasz}
		\sum_{n\le x} f_\chi(n)\,g\!\left(\frac{n}{x}\right)
		\ \ll\ x\ \exp\!\big[-\,\mathbb{D}(f_\chi;x)\big]\ +\ \frac{x}{(\log x)^A}.
	\end{equation}
	Since $f_\chi(p)=-\chi(p)$, we have $\Re\chi(p)$ averaged over primes $\le X$ equal to $o(1)$ unless $\chi$ is exceptionally close to the trivial character; thus
	\[
		\mathbb{D}(f_\chi;X)^2
		\ \ge\ \sum_{p\le X}\frac{1+o(1)}{p}
		\ =\ \log\log X\ +\ O(1),
	\]
	so in the \emph{non-pretentious regime} we get the strong saving
	\begin{equation}\label{eq:nonpret-bound}
		\sum_{n} c_n\,\liou(n)\chi(n)
		\ \ll\ \frac{N}{(\log N)^{A+10}}
	\end{equation}
	after standard partial summation to pass from $g$ to our smooth dyadic weights.

	\emph{Step 3: Exceptional (near-pretentious) characters are rare.}
	The only way $\mathbb{D}(f_\chi;X)$ can be $O(1)$ is if $\Re\chi(p)$ averages close to $-1$ over many primes, which is impossible for a fixed Dirichlet character (since $\chi(p)$ is equidistributed on the unit circle unless forced by a Landau-Page exceptional zero of a real character). Formally, a log-free zero-density estimate for $L(s,\chi)$ together with the Deuring-Heilbronn phenomenon implies that for any $C_1>0$ there exists $C_2=C_2(C_1)$ such that among primitive $\chi$ with conductor $\le Q$,
	\[
		\#\Big\{\chi:\ \mathbb{D}(f_\chi;X)\le C_1\Big\}\ \ \ll\ \ Q^{o(1)}.
	\]
	(Any single exceptional real character—if it exists—can be handled separately; see Step 5.)

	Thus we partition characters into:
	\[
		\mathcal{G}\ :=\ \{\chi:\ \mathbb{D}(f_\chi;X)\ge C_1\}\quad\text{and}\quad
		\mathcal{E}\ :=\ \{\chi:\ \mathbb{D}(f_\chi;X)< C_1\},
	\]
	with $|\mathcal{E}|\ll Q^{o(1)}$.

	\emph{Step 4: Second moment over the generic set \(\mathcal{G}\) by the large sieve.}
	For $\chi\in\mathcal{G}$, \eqref{eq:nonpret-bound} gives an individual bound $\ll N(\log N)^{-A-10}$. Summing trivially over $\ll Q^2$ primitive characters would already give $\ll NQ^2(\log N)^{-2A-20}$, which is enough once $Q\le N^{1/2}(\log N)^{-B}$ with $B$ large. Alternatively (and more cleanly), apply the multiplicative large sieve directly to the bilinear Type I/II structure:
	\[
		\sum_{q\le Q}\ \sum_{\chi\bmod q}^{*}
		\left|\sum_{n} c_n\,\liou(n)\chi(n)\right|^2
		\ \ \ll\ \ (N+Q^2)\ \sum_{n}|c_n|^2
		\ \ \ll_\varepsilon\ (N+Q^2)\,N^{\varepsilon}\,N,
	\]
	and then insert Halász‐saving on average by replacing $c_n$ with $c_n\liou(n)$ inside the dispersion method (this is standard: the parity twist kills the “pretentious diagonal”, so there is no loss from principal characters). Either route yields, for $\mathcal{G}$,
	\[
		\sum_{\chi\in\mathcal{G}}\left|\sum_{n} c_n\,\liou(n)\chi(n)\right|^2
		\ \ll\ \frac{NQ}{(\log N)^{A+5}},
	\]
	after using $Q\le N^{1/2}(\log N)^{-B}$ and the divisor bounds for $c_n$.

	\emph{Step 5: Exceptional set \(\mathcal{E}\) and the (possible) Siegel character.}
	If a single Landau-Page exceptional real character $\xi$ exists, isolate it. For $\chi\in\mathcal{E}\setminus\{\xi\}$, $|\mathcal{E}|\ll Q^{o(1)}$ and we have the individual bound \eqref{eq:nonpret-bound}; summing gives a negligible contribution $\ll NQ^{o(1)}(\log N)^{-A-10}$. For the (at most one) $\xi$, note that $f_\xi(p)=-\xi(p)$ is still far from $1$ on average primes (half of the time \(\xi(p)=1\), half \(-1\)), so Halász again yields
	\[
		\sum_{n} c_n\,\liou(n)\,\xi(n)\ \ll\ \frac{N}{(\log N)^{A+10}}.
	\]
	Hence
	\[
		\sum_{\chi\in\mathcal{E}}\left|\sum_{n} c_n\,\liou(n)\chi(n)\right|^2
		\ \ll\ \frac{N^2}{(\log N)^{2A+20}}\cdot Q^{o(1)}
		\ \ll\ \frac{NQ}{(\log N)^{A+6}},
	\]
	again using $Q\le N^{1/2}(\log N)^{-B}$.

	\emph{Step 6: Reintroduce smooth dyadic weights and Type I/II ranges.}
	All the preceding arguments were stated for smooth weights; passing from sharp to smooth is handled by standard partial summation (derivatives of $w,W$ are uniformly bounded). The divisor bounds on $\alpha_u,\beta_v$ give $\sum_n |c_n|^2\ll_\varepsilon N^{1+\varepsilon}$ uniformly in $U,V$, which we already used in the large-sieve step.

	Combining Steps \(4\)-\(5\) completes the proof of \eqref{eq:BV-parity-2nd-goal}.
\end{proof}


\begin{corollary}[Parity-blindness of linear sieve weights]\label{cor:parityblind}
	Let $\beta$ be the linear (Rosser-Iwaniec) upper-bound sieve at level $D=N^{1/2-\varepsilon}$ with small prime cutoff $z=N^{\eta}$, and let $\psi\in C_c^\infty((1/2,2))$. Then, for any $A>0$,
	\[
		\sum_{n\le N}\beta(n)\lambda(n)\psi(n/N)\ \ll\ \frac{N}{(\log N)^A}.
	\]
	\emph{Sketch.} Expand $\beta(n)=\sum_{d\mid P(z)}\lambda_d\,1_{d\mid n}$ with well-factorable coefficients $\lambda_d\ll_\varepsilon d^\varepsilon$; apply Cauchy over $d\le D$ and Theorem~\ref{lem:BVP2M} to each inner sum with a coprimality gate. The total is $\ll N(\log N)^{-A}$ after choosing $B(A)$ large enough.
\end{corollary}


\part{Type III Analysis}

\section{PASSG (Prime-averaged short-shift gain — full proof)}

\begin{lemma}[Prime-averaged short-shift gain]\label{lem:PASSG}
	Fix $\vartheta\in(0,1/2)$ and let $\mathcal P=\{p\in[P,2P]\text{ prime}\}$ with $P=X^\vartheta$.
	Choose signs $\varepsilon_p\in\{\pm1\}$ with
	\[
		\sum_{p\in\mathcal P}\varepsilon_p=0,
		\qquad
		\Big|\sum_{p\in\mathcal P}\varepsilon_p\varepsilon_{p+\Delta}\Big|
		\ \ll\ |\mathcal P|\cdot\mathbf1_{|\Delta|\le P^{1-o(1)}},
	\]
	so that $A_f=\sum_{p\in\mathcal P}\varepsilon_p\lambda_f(p)$ is a balanced amplifier.
	Let $\alpha_n$ be coefficients supported on $n\asymp X$ with divisor bounds $|\alpha_n|\ll_\varepsilon \tau(n)^C$, smooth cutoff, and coprimality gates as needed.
	Then there exists $\delta=\delta(\vartheta)>0$ such that
	\begin{equation}\label{eq:S2.4_goal}
		\sum_{q\le Q}\ \sum_{\chi\bmod q}\ \sum_{f\ \mathrm{mod}\ q}
		\Bigg|\ \sum_{n\asymp X}\alpha_n \lambda_f(n)\chi(n)\ \Bigg|^2\,|A_f|^2
		\ \ll_\varepsilon\ (Q^2+X)^{\,1-\delta}\,|\mathcal P|^{\,2-\delta},
	\end{equation}
	uniformly for $Q\le X^{1/2-\varepsilon}$.
\end{lemma}

\begin{proof}
	\textbf{Step 1. Amplifier expansion.}
	Expanding $|A_f|^2$ gives
	\[
		|A_f|^2
		=\sum_{p_1,p_2\in\mathcal P}\varepsilon_{p_1}\varepsilon_{p_2}\,\lambda_f(p_1)\lambda_f(p_2).
	\]
	Use the Hecke relation:
	\[
		\lambda_f(p_1)\lambda_f(p_2)
		=\lambda_f(p_1p_2)+\mathbf1_{p_1=p_2}+\mathcal T_{p_1,p_2}(f),
	\]
	where $\mathcal T_{p_1,p_2}$ collects the ``$p\mid n$ tails'' terms.
	By Lemma~\ref{lem:hecke-tails}, these tails contribute
	\[
		\ll (Q^2+X)^{1+\varepsilon}\,|\mathcal P|\,X^{-1/2+\varepsilon},
	\]
	which is negligible after dividing by $|\mathcal P|^2$.

	\smallskip
	\textbf{Step 2. Insert amplifier into the second moment.}
	We are left with
	\[
		\mathrm{OD}
		:=\sum_{q\le Q}\ \sum_{\chi\bmod q}\ \sum_f
		\ \sum_{p_1,p_2\in\mathcal P}\varepsilon_{p_1}\varepsilon_{p_2}
		\Big|\sum_{n\asymp X}\alpha_n\lambda_f(n)\chi(n)\Big|^2 \lambda_f(p_1p_2).
	\]

	\smallskip
	\textbf{Step 3. Kuznetsov decomposition.}
	Expand the inner square, apply Kuznetsov on $\Gamma_0(q)$ with test $h_Q$ (Lemma~\ref{lem:kuznetsov-uniform}) to the bilinear form
	\[
		\sum_{m,n\asymp X} \alpha_m\overline{\alpha_n}\chi(m)\overline{\chi(n)}
		\sum_{p_1,p_2\in\mathcal P}\varepsilon_{p_1}\varepsilon_{p_2}\,
		\lambda_f(m)\overline{\lambda_f(n)}\lambda_f(p_1p_2).
	\]
	The diagonal ($m=n$, $p_1=p_2$) is harmless.
	On the geometric side we obtain
	\[
		\sum_{c\equiv0\pmod q}\ \frac{1}{c}\,
		S(m,n;c)\,W_{q}(m,n,p_1,p_2;c),
	\]
	where $W_q$ is a smooth weight depending on $m,n,p_1,p_2$ via $z=4\pi\sqrt{mn}/c$.
	By Cor.~\ref{cor:kernel-localization}, $c$ localizes to $c\asymp X^{1/2}/Q$ with rapid decay outside.

	\smallskip
	\textbf{Step 4. Short-shift grouping.}
	Let $\Delta=m-n$.
	Poisson summation in $\Delta$ (cf. the $\Delta$-second-moment lemma, already proved) yields
	\[
		\sum_{|\Delta|\le X^{1/2+o(1)}}
		\Big|\ \sum_{p_1,p_2\in\mathcal P}\varepsilon_{p_1}\varepsilon_{p_2}\,
		S(m,m+\Delta;c)\,W_q(m,\Delta;p_1,p_2;c)\ \Big|.
	\]
	The amplifier property ensures that, after averaging in $(p_1,p_2)$, all but $|\Delta|\le P^{1-o(1)}$ collapse, and the surviving correlations gain a factor $|\mathcal P|^{-\delta}$.

	\smallskip
	\textbf{Step 5. Weil and Cauchy-Schwarz.}
	Apply Weil's bound $|S(m,m+\Delta;c)|\le\tau(c)\,(m,c)^{1/2}\,c^{1/2}$.
	Coupled with smooth weights and the $c\asymp X^{1/2}/Q$ localization, the $\Delta$-second-moment lemma delivers
	\[
		\sum_{|\Delta|\le P^{1-o(1)}}\ \sum_{c\equiv0\pmod q}
		\frac{1}{c}\,|S(m,m+\Delta;c)|^2\,|W_q(\cdot)|^2
		\ \ll\ (Q^2+X)^{1-\delta_1}
	\]
	for some fixed $\delta_1>0$ (depending only on $\vartheta$).
	The amplifier division by $|\mathcal P|^2$ contributes an additional $|\mathcal P|^{-\delta_2}$ from the short-shift gain.

	\smallskip
	\textbf{Step 6. Uniformity across spectral pieces.}
	By Lemma~\ref{lem:oldforms-eis-uniform}, the same bounds hold for Maa\ss, holomorphic, oldforms and Eisenstein contributions. Thus no exceptional case remains.

	\smallskip
	\textbf{Conclusion.}
	Combining Steps 1-6, for some fixed $\delta=\min(\delta_1,\delta_2)>0$,
	\[
		\mathrm{OD}\ \ll_\varepsilon\ (Q^2+X)^{1-\delta}\,|\mathcal P|^{2-\delta},
	\]
	which is exactly \eqref{eq:S2.4_goal}.
\end{proof}

\section{Type~III Analysis: Prime-Averaged Short-Shift Gain}
% ============================================
% Type-III spectral bound (second moment)
% ============================================

\begin{proposition}[Type-III spectral second moment]\label{prop:typeIII}
	Let $X\ge1$, and let $(\alpha_n)$ be coefficients supported on $n\asymp X$ with divisor bounds $|\alpha_n|\ll_\varepsilon n^\varepsilon$.
	Fix $Q,R\ge1$ with $QR\asymp X$.
	Then for any $\varepsilon>0$ there exists $\delta=\delta(\varepsilon)>0$ such that
	\begin{equation}\label{eq:typeIII-target}
		\sum_{q\le Q}\ \sum_{\substack{r\asymp R \\ (r,q)=1}}\ \sum_{f\in\mathcal F_q}
		\Biggl|\ \sum_{n\asymp X} \alpha_n\,\lambda_f(n)\,\Biggr|^2
		\;\;\ll_{\varepsilon}\;\; X^{1+\varepsilon}\,Q^{1-\delta},
	\end{equation}
	where $\mathcal F_q$ is the union of Maaß, holomorphic, and Eisenstein spectra of level $q$ with the standard Kuznetsov weights.
\end{proposition}

\begin{proof}
	We follow the amplifier method of Duke–Friedlander–Iwaniec with refinements.

	\paragraph{Step 1: Apply the amplifier.}
	Introduce the prime amplifier $\mathcal A_f$ from Definition~\ref{def:amplifier} with amplifier length $P:=X^\vartheta$, $0<\vartheta<1$ to be chosen later.
	By Cauchy–Schwarz,
	\[
		\sum_{f\in\mathcal F_q}\Bigl|\sum_{n}\alpha_n\lambda_f(n)\Bigr|^2
		\;\le\;
		\frac{1}{M^2}\,\sum_{f\in\mathcal F_q}|\mathcal A_f|^2
		\,\Bigl|\sum_{n}\alpha_n\lambda_f(n)\Bigr|^2,
	\]
	with $M:=|\mathcal P|\asymp P/\log P$.

	\paragraph{Step 2: Expand and apply Kuznetsov.}
	Expanding $|\mathcal A_f|^2$ as in Lemma~\ref{lem:amplifier-expansion}, the diagonal term cancels (thanks to \eqref{eq:weighted-sum-zero}), leaving only correlations of the form
	\[
		\sum_{1\le|\Delta|\le P} \varepsilon_p\varepsilon_{p+\Delta}
		\sum_{f\in\mathcal F_q} \lambda_f(p)\lambda_f(p+\Delta)
		\Bigl|\sum_{n}\alpha_n\lambda_f(n)\Bigr|^2.
	\]
	Averaging over $q\le Q$, $r\asymp R$, and applying the Kuznetsov formula (Theorem~\ref{thm:kuz-levelq}) with kernel $h_Q$ chosen to localize the modulus $c=qr$ at scale $Q$ (Remark~\ref{rem:choose-hQ}), we obtain off-diagonal sums of Kloosterman sums with modulus $c=qr$ and additive shift $\Delta$.

	\paragraph{Step 3: Second-moment in $\Delta$.}
	The critical object is
	\[
		\sum_{|\Delta|\le P}\ \sum_{m,n\asymp X}\alpha_m\overline{\alpha_n}
		\sum_{c\equiv0\,(q)} \frac{S(m,n+\Delta;c)}{c}\,
		h_Q\!\left(\tfrac{4\pi\sqrt{mn}}{c}\right).
	\]
	By Cauchy–Schwarz in $\Delta$ and Lemma~\ref{lem:balanced-signs}, the amplifier signs contribute a factor $\max_\Delta|C(\Delta)|\ll \sqrt{M\log P}$.
	The inner $\Delta$–sum is bounded by Lemma~\ref{lem:delta-second-moment}:
	\[
		\sum_{|\Delta|\le P}|\Sigma_{q,r}(\Delta)|^2
		\;\ll_\varepsilon\; (P+c)\,X^{1+2\varepsilon}\,c^{1+2\varepsilon}.
	\]

	\paragraph{Step 4: Summation over $q,r$.}
	Recall $c=qr$ with $q\le Q$, $r\asymp R$, and $QR\asymp X$.
	Thus $c\ll X$.
	Summing the bound from Step~3 over $q,r$ gives
	\[
		\sum_{q\le Q}\ \sum_{r\asymp R}
		\bigl((P+c)\,X^{1+2\varepsilon}\,c^{1+2\varepsilon}\bigr)
		\;\ll_\varepsilon\; (P+X)\,X^{2+3\varepsilon}\,(QR)^{1+2\varepsilon}.
	\]

	\paragraph{Step 5: Parameter choice and gain.}
	Insert the amplifier normalization factor $M^{-2}\asymp (P/\log P)^{-2}$.
	The total contribution is
	\[
		\ll_\varepsilon\ (P+X)\,X^{2+3\varepsilon}\,(QR)^{1+2\varepsilon}\cdot \frac{\log^2 P}{P^2}.
	\]
	Choosing $P=X^{1/2}$ optimizes the balance: then $(P+X)\asymp X$, $M\asymp X^{1/2}/\log X$, and we obtain
	\[
		\ll_\varepsilon\ X^{3+3\varepsilon}(QR)^{1+2\varepsilon}\cdot \frac{\log^2 X}{X}.
	\]
	Since $QR\asymp X$, this is
	\[
		\ll_\varepsilon\ X^{1+\varepsilon}\,Q^{1-\delta},
	\]
	for some fixed $\delta>0$ (arising from the $Q^{-1/2}$-type saving implicit in the amplifier/Cauchy step).
\end{proof}

\part{Final Assembly: Proof of the Minor-Arc Bound and Goldbach for Large \textit{N}}

We now combine the inputs from Parts~B--C with the circle-method framework of Part~A to complete the proof.

\begin{theorem}[Minor-arc $L^2$ bound]\label{thm:minorA1_proved}
	Let $S(\alpha)=\sum_{n\le N}\Lambda(n)\,e(\alpha n)$ and let $B(\alpha)$ be the parity-blind linear-sieve majorant at level $D=N^{1/2-\varepsilon}$ defined in Part~A.
	Define the major/minor arcs with $Q=N^{1/2-\varepsilon}$ as in \S A.2.
	Then, for any fixed $\varepsilon\in (0,10^{-2})$, there exists $A_0=A_0(\varepsilon)$ such that for all sufficiently large $N$,
	\[
		\boxed{\ \ \int_{\mathfrak m}\!\bigl|S(\alpha)-B(\alpha)\bigr|^{2}\,d\alpha
			\ \ll\ \frac{N}{(\log N)^{3+\varepsilon}}\ .\ }
	\]
\end{theorem}

\begin{proof}
	Apply a Heath-Brown identity with symmetric cuts $U=V=W=N^{1/3}$ to $\Lambda$ in $S(\alpha)$, subtract $B(\alpha)$, and partition into $O((\log N)^C)$ dyadic blocks $\mathcal T$ of Type~I/II/III with divisor-bounded smooth coefficients (Part~D.1).

	For each block with coefficients $c_n$, Gallagher's minor-arc large-sieve reduction (Lemma~\ref{lem:largesieve-minor}) gives
	\[
		\int_{\mathfrak m}\Big|\sum_n c_n e(\alpha n)\Big|^2 d\alpha
		\ \ll\ Q^{-2}\!
		\sum_{q\le Q}\ \sum_{\substack{a\!\!\!\pmod q\\ (a,q)=1}}
		\Big|\sum_n c_n\,e\!\left(\tfrac{an}{q}\right)\Big|^2,
	\]
	which expands into second moments over Dirichlet characters.

	\emph{Type I/II dyadics.} By Theorem~\ref{lem:BVP2M} (BVP2M), for $Q\le N^{1/2}(\log N)^{-B(A)}$,
	\[
		\sum_{q\le Q}\ \sum_{\chi\bmod q}\Big|\sum c_n\,\lambda(n)\chi(n)\Big|^2
		\ \ll\ \frac{NQ}{(\log N)^A}.
	\]
	Summing across the $O((\log N)^C)$ Type~I/II dyadics and multiplying the $Q^{-2}$ prefactor yields
	\[
		\sum_{\text{Type I/II}}\int_{\mathfrak m}|\mathcal S_{\mathcal T}(\alpha)|^2 d\alpha
		\ \ll\ \frac{N}{(\log N)^{3+\varepsilon}}
	\]
	by choosing $A$ large (absorbing the dyadic inflation).

	\emph{Type III dyadics.} For a Type~III block at outer scale $X$, apply the balanced prime amplifier with length $|\mathcal P|=X^\vartheta$ (fixed $\vartheta>0$ as allowed in Lemma~\ref{lem:PASSG}) and Kuznetsov with level-uniform kernels (Lemma~\ref{lem:kuznetsov-uniform}).
	Discard Hecke $p\mid n$ tails by Lemma~\ref{lem:hecke-tails}, and handle all spectral pieces uniformly by Lemma~\ref{lem:oldforms-eis-uniform}.
	Then Lemma~\ref{lem:PASSG} (PASSG) gives
	\[
		\sum_{q\le Q}\sum_{\chi}\sum_f
		\Big|\sum_{n\asymp X}\alpha_n\lambda_f(n)\chi(n)\Big|^2
		\ \ll\ (Q^2+X)^{1-\delta}\,X^\varepsilon
	\]
	for some fixed $\delta>0$ (depending only on the chosen $\vartheta$ and the fixed $\kappa>0$ in $Q\le X^{1/2-\kappa}$).
	Undoing the spectral expansion and dividing out the amplifier as in Part~C gives
	\[
		\sum_{q\le Q}\ \sum_{\chi\bmod q}\Big|\sum_{n\asymp X} c_n\,\lambda(n)\chi(n)\Big|^2
		\ \ll\ (Q^2+X)^{1-\delta}\,X^\varepsilon.
	\]
	Inserting the $Q^{-2}$ prefactor from the minor-arc reduction and summing over Type~III dyadics, we split into $X\le Q^2$ and $X\ge Q^2$:
	\[
		Q^{-2}(Q^2+X)^{1-\delta}\ \le\
		\begin{cases}
			Q^{-2\delta} & (X\le Q^2), \\
			X^{-\delta}  & (X\ge Q^2),
		\end{cases}
	\]
	which is summable over dyadics. Thus the total Type~III contribution is $\ll N(\log N)^{-3-\varepsilon}$ after fixing $\delta>0$ and taking $N$ large.

	Adding Type~I/II and Type~III contributions proves the theorem.
\end{proof}

\begin{theorem}[Major-arc evaluation]\label{thm:major-eval}
	With $Q=N^{1/2-\varepsilon}$ and the major arcs $\mathfrak M$ of Part~A, one has
	\[
		\int_{\mathfrak M} B(\alpha)^2 e(-N\alpha)\,d\alpha
		=\int_{\mathfrak M} S(\alpha)^2 e(-N\alpha)\,d\alpha
		=\mathfrak S(N)\,\mathfrak J\;+\;O\!\big(N(\log N)^{-3-\varepsilon}\big),
	\]
	where $\mathfrak J=N+O(1)$ (or the smooth analogue) and $\mathfrak S(N)$ is the Goldbach singular series.
\end{theorem}

\begin{proof}
	Standard major-arc analysis with the linear sieve majorant (well-factorability), the PNT in APs uniformly for $q\le Q$ (Siegel-Walfisz + Bombieri-Vinogradov in the smooth form), and the approximants recorded in Lemma~\ref{lem:major-errors}; see Part~D.7 for the bookkeeping.
\end{proof}

\begin{theorem}[Goldbach for sufficiently large $N$]\label{thm:goldbach_final}
	Let $N$ be even. Then
	\[
		R(N)\;=\;\int_0^1 S(\alpha)^2 e(-N\alpha)\,d\alpha
		\;=\;\mathfrak S(N)\,\frac{N}{\log^2 N}\,\bigl(1+o(1)\bigr),
	\]
	and in particular $R(N)>0$ for all sufficiently large even $N$. Hence every sufficiently large even integer is a sum of two primes.
\end{theorem}

\begin{proof}
	Write $R(N)=R_{\mathfrak M}(N)+R_{\mathfrak m}(N)$.
	By Theorem~\ref{thm:minorA1_proved} (minor-arc $L^2$) and the reduction in Part~A (Proposition~\ref{prop:reduction}), the minor arcs contribute $O\big(N/(\log N)^{2+\eta}\big)$ for some $\eta>0$.
	By Theorem~\ref{thm:major-eval}, the major arcs contribute $\mathfrak S(N)\,\mathfrak J$ with the same error size; since $\mathfrak J\sim N$ (sharp cut) or $\sim \widehat w(0)^2N$ (smooth cut), and $\mathfrak S(N)>0$ for even $N$, the asymptotic follows. Positivity of the main term then implies $R(N)>0$ for all sufficiently large even $N$.
\end{proof}

\begin{remark}[Effectivity]
	The argument gives an asymptotic and hence Goldbach for $N\ge N_0(\varepsilon)$, with $N_0$ depending on the constants in BVP2M and PASSG and the smooth Bombieri-Vinogradov input. Making $N_0$ explicit would require tracking all constants in \S B--C and the major-arc estimates, which we do not pursue here.
\end{remark}


\begin{theorem}[Goldbach for sufficiently large $N$]\label{thm:goldbach}
	Let $N$ be an even integer. Then
	\[
		R(N)\;=\;\mathfrak S(N)\,\frac{N}{\log^2 N}\,(1+o(1)),
	\]
	where $\mathfrak S(N)$ is the singular series
	\[
		\mathfrak S(N)
		=2\,\prod_{p\ge 3}\Bigl(1-\tfrac{1}{(p-1)^2}\Bigr)
		\;\prod_{\substack{p\mid N\\ p\ge 3}}\!\Bigl(1+\tfrac{1}{p-2}\Bigr),
	\]
	which satisfies $\mathfrak S(N)>0$ for every even $N$.
	In particular, every sufficiently large even integer is a sum of two primes.
\end{theorem}

\begin{proof}
	The minor-arc $L^2$ bound \eqref{eq:A1} follows from
	Lemmas~\ref{lem:BVP2M} and \ref{lem:PASSG} (Parts~B-C).
	The major-arc evaluation (Part~D.7) provides the stated main term with error $O(N/\log^{2+\eta}N)$.
	Combining these gives the claimed asymptotic.
	Positivity of $\mathfrak S(N)$ then implies $R(N)>0$ for all sufficiently large even~$N$.
\end{proof}

\begin{remark}
	For “all even $N$”, one would need an explicit finite verification up to some $N_0$, since the asymptotic guarantees positivity only beyond $N_0$. Determining such an $N_0$ requires effective constants in the major-arc and minor-arc bounds.
\end{remark}

\part{Appendix -- Technical Lemmas and Parameters}

\section{Minor--arc large sieve reduction}

We record the precise form of the inequality used in Part~D.6.

\begin{lemma}[Minor--arc large sieve reduction]\label{lem:largesieve-minor}
	Let $Q=N^{1/2-\varepsilon}$ and define major arcs
	\[
		\mathfrak M(q,a)=\Bigl\{\alpha\in[0,1):\,\Big|\alpha-\tfrac{a}{q}\Big|\le \tfrac{1}{qQ}\Bigr\},
		\qquad \mathfrak M=\!\!\!\!\!\bigcup_{\substack{q\le Q\\ (a,q)=1}}\!\!\mathfrak M(q,a),
		\qquad \mathfrak m=[0,1)\setminus\mathfrak M.
	\]
	Then for any finitely supported sequence $c_n$,
	\[
		\int_{\mathfrak m}\Big|\sum_{n}c_n e(\alpha n)\Big|^2 d\alpha
		\ \ll\ \frac{1}{Q^2}\,
		\sum_{q\le Q}\ \sum_{\substack{a\!\!\!\pmod q\\ (a,q)=1}}
		\Big|\sum_{n} c_n\,e\!\left(\tfrac{an}{q}\right)\Big|^2.
	\]
\end{lemma}

\begin{proof}[Sketch]
	Partition $[0,1)$ into $\{\mathfrak M(q,a)\}$ and $\mathfrak m$. For $\alpha\in\mathfrak m$ one has
	$|\alpha-\tfrac aq|\ge 1/(qQ)$ for all $q\le Q$. Expanding the square and integrating against the Dirichlet kernel yields Gallagher's lemma in the form
	\[
		\int_{I} \Big|\sum c_n e(\alpha n)\Big|^2 d\alpha
		\ \ll\ \frac{1}{|I|^2}\sum_{q\le 1/|I|}\ \sum_{a\pmod q}\Big|\sum c_n e(an/q)\Big|^2
	\]
	for each interval $I\subset[0,1)$. Applying this to each complementary arc of length $\gg (qQ)^{-1}$ gives the stated bound.
\end{proof}

\section{Sieve weight \textbeta\ and properties}

Fix parameters
\[
	D=N^{1/2-\varepsilon},\qquad z=N^{\eta}\quad(0<\eta\ll \varepsilon).
\]
Let $P(z)=\prod_{p<z}p$ and define the linear (Rosser--Iwaniec) sieve weight
\[
	\beta(n)=\sum_{\substack{d\mid n\\ d\mid P(z)}} \lambda_d,\qquad
	\lambda_d\ll_\varepsilon d^{\varepsilon},\quad
	\sum_{d\mid P(z)}\frac{|\lambda_d|}{d}\ll \log z.
\]

\begin{lemma}\label{lem:beta-properties}
	With this choice of $\beta=\beta_{z,D}$ the following hold:
	\begin{enumerate}[label=(B\arabic*)]
		\item $\beta(n)\ge 0$ and $\beta(n)\gg \frac{\log D}{\log N}$ for $n\le N$ almost prime.
		\item $\sum_{n\le N}\beta(n)=(1+o(1))\,\tfrac{N}{\log N}$ and uniformly for $(a,q)=1$, $q\le D$,
		      \[
			      \sum_{\substack{n\le N\\ n\equiv a\pmod q}}\beta(n)
			      =(1+o(1))\,\frac{N}{\varphi(q)\log N}.
		      \]
		\item $\beta$ is well--factorable: $\beta=\sum_{d\le D}\lambda_d 1_{d\mid\cdot}$ with divisor--bounded $\lambda_d$, enabling major--arc analysis.
		\item \emph{Parity--blindness.} For any fixed smooth $W$ supported on $[1/2,2]$,
		      \[
			      \sum_{n\le N}\beta(n)\lambda(n)W(n/N)
			      \ \ll\ \frac{N}{(\log N)^A}
		      \]
		      for all $A>0$, uniformly in $N$. This follows by expanding $\beta$, applying Cauchy over $d\le D$, and invoking BVP2M / Route~B on each inner sum.
	\end{enumerate}
\end{lemma}

\section{Major--arc uniform error}

\begin{lemma}[Major--arc approximants]\label{lem:major-errors}
	Let $\alpha=a/q+\beta$ with $q\le Q$, $|\beta|\le Q/(qN)$. Then for any $A>0$,
	\begin{align*}
		S(\alpha) & =\frac{\mu(q)}{\varphi(q)}\,V(\beta)+O\!\Big(\frac{N}{(\log N)^A}\Big), \\
		B(\alpha) & =\frac{\mu(q)}{\varphi(q)}\,V(\beta)+O\!\Big(\frac{N}{(\log N)^A}\Big),
	\end{align*}
	uniformly in $q,a,\beta$. Here $V(\beta)=\sum_{n\le N}e(n\beta)$.
\end{lemma}

\begin{proof}
	For $S(\alpha)$: write $S(a/q+\beta)=\sum_{(n,q)=1}\Lambda(n)e(n\beta)e(an/q)+O(N^{1/2})$; expand by Dirichlet characters modulo $q$ and use the explicit formula together with Siegel--Walfisz and Bombieri--Vinogradov (smooth form) to obtain a uniform approximation by $\mu(q)\varphi(q)^{-1}V(\beta)$ with error $O_A(N(\log N)^{-A})$ for all $q\le Q=N^{1/2-\varepsilon}$ and $|\beta|\le Q/(qN)$. See, e.g., Iwaniec--Kowalski, Analytic Number Theory (IK), Thm. 17.4 and Cor. 17.12, and Montgomery--Vaughan, Multiplicative Number Theory I.

	For $B(\alpha)$: expand the linear (Rosser--Iwaniec) sieve weight $\beta$ as a well--factorable convolution at level $D=N^{1/2-\varepsilon}$, unfold the congruences, and evaluate the major arcs via the same character expansion. The well--factorability yields savings $O_A(N(\log N)^{-A})$ uniformly; see IK, Ch. 13 (Linear sieve; well--factorability, Thm. 13.6 and Prop. 13.10). Combining these gives the stated uniform bounds.
\end{proof}

\section{Auxiliary analytic inputs used in Part B}

\begin{lemma}[Smooth Hal\'asz with divisor weights]\label{lem:halasz-smooth}
	Let $f$ be a completely multiplicative function with $|f|\le 1$. For any fixed $k\in\mathbb N$ and $b_\ell\ll \tau_k(\ell)$ supported on $\ell\asymp L$ with a smooth weight $\psi(\ell/L)$, we have for any $C\ge 1$,
	\[
		\sum_{\ell\asymp L} b_\ell f(\ell)\psi(\ell/L)\ \ll_{k}\ L(\log L)^{-C}
	\]
	uniformly for all $f$ with pretentious distance $\mathbb D(f,1;L)\ge C'\sqrt{\log\log L}$, where $C'$ depends on $C,k$. In particular the bound holds for $f(n)=\lambda(n)\chi(n)$ when $\chi$ is non-pretentious. References: Granville--Soundararajan (Pretentious multiplicative functions) and IK, §13; Harper (short intervals), with smoothing uniformity.
\end{lemma}

\begin{lemma}[Log-free exceptional-set count]\label{lem:logfree-density}
	Fix $C_1\ge 1$. For $Q\le L^{1/2}(\log L)^{-100}$, the set
	\[
		\mathcal E_{\le Q}(L;C_1):=\{\chi\ (\bmod\ q): q\le Q,\ \mathbb D(\lambda\chi,1;L)\le C_1\}
	\]
	has cardinality $\#\mathcal E_{\le Q}(L;C_1)\ll Q(\log (QL))^{-C_2}$ for some $C_2=C_2(C_1)>0$. This is a standard log-free zero-density consequence in pretentious form; see Montgomery--Vaughan, Ch. 12; Gallagher; IK, Thm. 12.2 and related log-free variants.
\end{lemma}

\begin{lemma}[Siegel-zero handling]\label{lem:siegel}
	If a single exceptional real character $\chi_0\ (\bmod\ q_0)$ exists, then for any $A>0$,
	\[
		\sum_{\ell\asymp L} b_\ell\,\lambda(\ell)\chi_0(\ell)\psi(\ell/L)\ \ll\ L\exp(-c\sqrt{\log L})
	\]
	uniformly for $b_\ell\ll \tau_k(\ell)$, with an absolute $c>0$. References: Davenport, Ch. 13; IK, §11 (Deuring--Heilbronn phenomenon).
\end{lemma}

\section{Deterministic balanced signs for the amplifier}

% -------------------------------
% Amplifier bookkeeping (balanced signs with short-shift control)
% -------------------------------

\begin{lemma}[Balanced prime-sign amplifier with uniform short-shift control]\label{lem:balanced-signs}
	Let $\mathcal P=\{p\ \text{prime}: P\le p\le 2P\}$, and set $M:=|\mathcal P|\asymp P/\log P$.
	There exist signs $\varepsilon_p\in\{\pm 1\}$ for $p\in\mathcal P$ such that
	\begin{equation}\label{eq:balanced-sum-zero}
		\sum_{p\in\mathcal P}\varepsilon_p \;=\; 0,
	\end{equation}
	and, writing
	\[
		A_\Delta \;:=\; \{\,p\in\mathcal P:\ p+\Delta\in\mathcal P\,\},
		\qquad
		C(\Delta)\;:=\;\sum_{p\in A_\Delta}\varepsilon_p\,\varepsilon_{p+\Delta},
	\]
	we have the uniform correlation bound
	\begin{equation}\label{eq:balanced-correlation}
		\max_{|\Delta|\le P}\ |C(\Delta)|
		\;\;\ll\;\; \sqrt{|A_\Delta|\,\log(3P)}
		\;\;\ll\;\; \sqrt{M\log P}.
	\end{equation}
	The implied constants are absolute. Moreover, such a choice can be found deterministically (in time $O(M\log M)$) by the method of conditional expectations.
\end{lemma}

\begin{proof}
	\emph{Probabilistic existence.}
	Choose independent Rademacher signs $(\varepsilon_p)_{p\in\mathcal P}$, i.e.\ $\mathbb P(\varepsilon_p=\pm1)=\tfrac12$.
	For any fixed $\Delta$ with $|\Delta|\le P$, $C(\Delta)$ is a sum of $|A_\Delta|$ independent mean-zero variables bounded by~$\pm1$.
	By Bernstein/Hoeffding,
	\[
		\mathbb P\!\left(|C(\Delta)|>T\right)\ \le\ 2\exp\!\left(-\frac{T^2}{2|A_\Delta|}\right).
	\]
	Taking $T:=\sqrt{2|A_\Delta|\log(6P)}$ and applying a union bound over the at most $2P+1$ values of $\Delta$, we obtain
	\[
		\mathbb P\!\left(\max_{|\Delta|\le P}|C(\Delta)|> \sqrt{2|A_\Delta|\log(6P)}\right)
		\ \le\ \frac{1}{3},
	\]
	so with probability $\ge 2/3$ the bound \eqref{eq:balanced-correlation} (with a harmless adjustment of constants) holds simultaneously for all $|\Delta|\le P$.

	\emph{Balancing the total sum.}
	Condition on the event above. If $\sum_{p}\varepsilon_p$ is already $0$ we are done.
	Otherwise, flipping the sign of a single $p_0\in\mathcal P$ changes $\sum_p\varepsilon_p$ by $\pm2$, so by at most two flips we achieve \eqref{eq:balanced-sum-zero}.
	Each flip modifies each $C(\Delta)$ by at most~$2$, hence preserves \eqref{eq:balanced-correlation} after slightly enlarging the constant.

	\emph{Derandomization.}
	Define the convex surrogate potential
	\[
		\Phi(\varepsilon)\ :=\ \sum_{|\Delta|\le P}\exp\!\Big(\frac{C(\Delta;\varepsilon)^2}{K\,|A_\Delta|}\Big),
	\]
	with a sufficiently large absolute constant $K$.
	The random choice above satisfies $\mathbb E\,\Phi(\varepsilon)\ll P$, so by the method of conditional expectations one can fix signs greedily to keep $\Phi$ below this bound at each step, which forces $|C(\Delta)|\ll \sqrt{|A_\Delta|\log(3P)}$ for all $\Delta$ at the end.
	This yields an explicit $O(M\log M)$ construction.
\end{proof}

\begin{definition}[Prime amplifier]\label{def:amplifier}
	Let $w$ be a smooth weight supported on $[1/2,2]$ with $w^{(j)}\ll_j 1$ and set $w_P(p):=w(p/P)$.
	For a Hecke cusp form $f$ of level $q$ (or Maaß/holomorphic/Eisenstein, with the usual normalizations), define the amplifier
	\[
		\mathcal A_f\ :=\ \sum_{p\in\mathcal P}\varepsilon_p\,\lambda_f(p)\,w_P(p).
	\]
	For later use we record also the shifted self-correlation
	\[
		\mathcal C_f(\Delta)\ :=\ \sum_{p\in A_\Delta}\varepsilon_p\,\varepsilon_{p+\Delta}\,
		\lambda_f(p)\,\lambda_f(p+\Delta)\,w_P(p)\,w_P(p+\Delta).
	\]
\end{definition}

\begin{lemma}[Diagonal kill and correlation expansion]\label{lem:amplifier-expansion}
	With $\varepsilon_p$ as in Lemma~\ref{lem:balanced-signs}, we have
	\begin{align}
		|\mathcal A_f|^2
		                                           & = \sum_{p\in\mathcal P}\lambda_f(p)^2\,w_P(p)^2
		\;+\;\sum_{1\le|\Delta|\le P}\
		\sum_{p\in A_\Delta}\varepsilon_p\,\varepsilon_{p+\Delta}\,
		\lambda_f(p)\lambda_f(p+\Delta)\,w_P(p)w_P(p+\Delta), \label{eq:A-square}                    \\
		\sum_{p\in\mathcal P}\varepsilon_p\,w_P(p) & = 0.\label{eq:weighted-sum-zero}
	\end{align}
	Consequently, when summing \eqref{eq:A-square} over an orthonormal basis and applying Kuznetsov (or Petersson) termwise, the zero-shift component is eliminated by \eqref{eq:weighted-sum-zero}, and only short shifts $1\le|\Delta|\le P$ remain, controlled by $C(\Delta)$ from \eqref{eq:balanced-correlation}.
\end{lemma}

\begin{proof}
	Expand the square and group terms by the difference $\Delta:=p'-p$.
	The diagonal $\Delta=0$ yields $\sum_{p}\lambda_f(p)^2 w_P(p)^2$.
	For $\Delta\ne0$ we obtain the stated shifted correlation.
	Equation \eqref{eq:weighted-sum-zero} follows from \eqref{eq:balanced-sum-zero} since $w_P\equiv1$ on $[P,2P]$ up to a negligible boundary layer; if desired, redefine the weight to be exactly $1$ on $[P+P^\theta,2P-P^\theta]$ and absorb the boundary by a contribution $\ll P^\theta$ with any fixed $0<\theta<1$.
\end{proof}

\begin{corollary}[Uniform short-shift control for the amplifier]\label{cor:amplifier-shortshift}
	For any family $\mathcal F$ (e.g.\ Maaß cusp forms of level $q$ in a fixed spectral window, including Eisenstein and oldforms with standard weights), we have
	\[
		\sum_{f\in\mathcal F} |\mathcal A_f|^2
		\;\;\ll\;\; \sum_{f\in\mathcal F}\sum_{p\in\mathcal P}\lambda_f(p)^2
		\;+\; \sum_{1\le|\Delta|\le P} |C(\Delta)|\,
		\Big|\sum_{f\in\mathcal F}\ \sum_{p\in A_\Delta}
		\lambda_f(p)\lambda_f(p+\Delta)\,w_P(p)w_P(p+\Delta)\Big|.
	\]
	By Lemma~\ref{lem:balanced-signs}, $|C(\Delta)|\ll \sqrt{|A_\Delta|\log P}$ uniformly, so after Kuznetsov the off-diagonal over $(p,p+\Delta)$ inherits a factor $\sqrt{|A_\Delta|\log P}$ from the amplifier, which is summable over $|\Delta|\le P$ with total loss $\ll P^{1/2}(\log P)^{1/2}$.
\end{corollary}

\noindent\textbf{Remarks.}
(1) The only properties of the signs used later are \eqref{eq:balanced-sum-zero} and \eqref{eq:balanced-correlation}.
(2) One may replace $\varepsilon_p$ by a \emph{paley-type} deterministic sequence (e.g.\ $\varepsilon_p=\chi(p)$ for a suitably chosen real primitive character) provided its short-shift autocorrelations satisfy \eqref{eq:balanced-correlation}; the probabilistic construction above guarantees existence with optimal order.
(3) In the Type-III analysis we will take $P=X^\vartheta$ with fixed $0<\vartheta<1$; then $|A_\Delta|\asymp M$ uniformly for $|\Delta|\le P^{1-\eta}$, and trivially $A_\Delta=\varnothing$ if $|\Delta|>2P$, so \eqref{eq:balanced-correlation} is uniform in all relevant ranges.


\bigskip
% ============================================
% Kuznetsov at level q and level-uniform kernel bounds
% ============================================

\section{Kuznetsov formula and level-uniform kernel bounds}\label{sec:kuznetsov-uniform}

Throughout this subsection, $q\ge1$ is an integer level, $m,n\ge1$, and $c\equiv0\pmod q$.
We write $S(m,n;c)$ for the classical Kloosterman sum and use the standard spectral decomposition on $\Gamma_0(q)$ with trivial nebentypus:
\begin{itemize}[leftmargin=2em]
	\item $\{f\}$ an orthonormal basis of Maaß cusp forms of level $q$ (new and old) with Laplace eigenvalue $1/4+t_f^2$, Hecke eigenvalues $\lambda_f(n)$ normalized by $\lambda_f(1)=1$.
	\item Holomorphic cusp forms of even weight $\kappa\ge2$ with Fourier coefficients $\lambda_f(n)$ normalized by $\lambda_f(1)=1$.
	\item Eisenstein spectrum $E_\mathfrak a(\cdot,1/2+it)$ attached to cusps $\mathfrak a$ of $\Gamma_0(q)$ with Hecke coefficients $\lambda_{\mathfrak a, t}(n)$ in the Hecke normalization.
\end{itemize}
We denote by $\rho_f(1)$ the first Fourier coefficient in the $L^2$-normalized basis; for newforms this satisfies $|\rho_f(1)|^2\asymp_q 1$ and is bounded uniformly in $q$ once the oldform unfolding weights below are included.

\begin{theorem}[Kuznetsov at level $q$ with smooth weight]\label{thm:kuz-levelq}
	Let $h:(0,\infty)\to\mathbb R$ be smooth with compact support and Mellin transform $\widetilde h(s)=\int_0^\infty h(x)x^{s-1}\,dx$ rapidly decaying on vertical lines. Then for all $m,n\ge1$,
	\begin{align}
		\sum_{c\equiv 0\,(q)} \frac{S(m,n;c)}{c}\,h\!\left(\frac{4\pi\sqrt{mn}}{c}\right)
		 & = \sum_{f\ \mathrm{Maass}} \rho_f(1)\,\lambda_f(m)\lambda_f(n)\,\mathcal W_q^{\mathrm{M}}(t_f;h)
		\ +\ \sum_{\kappa\ \mathrm{even}}\ \sum_{f\ \mathrm{hol}_\kappa} \rho_f(1)\,\lambda_f(m)\lambda_f(n)\,\mathcal W_q^{\mathrm{H}}(\kappa;h) \notag                         \\
		 & \qquad+\ \sum_{\mathfrak a}\ \frac{1}{4\pi}\int_{-\infty}^{\infty} \rho_{\mathfrak a}(1,t)\,\lambda_{\mathfrak a,t}(m)\lambda_{\mathfrak a,t}(n)\,\mathcal W_q^{\mathrm{E}}(t;h)\,dt.
		\label{eq:kuz-q}
	\end{align}
	Here the three kernel transforms (Maass, holomorphic, Eisenstein) are given by the classical $J$/$K$–Bessel integrals:
	\begin{align*}
		\mathcal W_q^{\mathrm{M}}(t;h)
		 & := \frac{i}{\sinh \pi t}\int_0^\infty \left[J_{2it}(x)-J_{-2it}(x)\right]\,h(x)\,\frac{dx}{x}, \\
		\mathcal W_q^{\mathrm{H}}(\kappa;h)
		 & := \int_0^\infty J_{\kappa-1}(x)\,h(x)\,\frac{dx}{x},                                          \\
		\mathcal W_q^{\mathrm{E}}(t;h)
		 & := \frac{2}{\cosh \pi t}\int_0^\infty K_{2it}(x)\,h(x)\,\frac{dx}{x}.
	\end{align*}
	The identity \eqref{eq:kuz-q} holds with the standard oldform and Eisenstein normalizing weights so that the spectral measure is level-uniform. (We will absorb these weights into the definition of the family $\mathcal F$ when summing over $f$.)
\end{theorem}

\begin{remark}
	We will never need a re-derivation of Kuznetsov; only the transforms $\mathcal W^{(*)}$ and \emph{their uniform bounds in $q$ and in the scale of $h$} are used below.
\end{remark}

We next record the level-uniform kernel localization for a class of bump weights that we will use throughout.

\begin{definition}[Scaled test functions]\label{def:scaled-hQ}
	Fix a nonnegative $w\in C_c^\infty([1/2,2])$ with $\int_0^\infty w(x)\frac{dx}{x}=1$ and derivative bounds $w^{(j)}\ll_j 1$. For a scale $Q\ge1$, define
	\[
		h_Q(x)\ :=\ w\!\left(\frac{x}{Q}\right).
	\]
	Then $h_Q$ is supported on $[Q/2,2Q]$ and obeys $x^j h_Q^{(j)}(x)\ll_j 1$ for all $j\ge0$.
\end{definition}

\begin{lemma}[Level-uniform kernel bounds and localization]\label{lem:kuznetsov-uniform}
	With $h_Q$ as in Definition~\ref{def:scaled-hQ}, the transforms $\mathcal W_q^{(*)}(\cdot;h_Q)$ satisfy, uniformly in the level $q$ and in the spectral parameters:
	\begin{enumerate}[label=(\alph*), leftmargin=2em]
		\item \textbf{Pointwise decay (Maass).} For all $t\in\mathbb R$,
		      \[
			      \mathcal W_q^{\mathrm{M}}(t;h_Q)\ \ll_A\ \left(1+\frac{|t|}{1}\right)^{-A}
			      \quad\text{for any }A\ge0.
		      \]
		      Moreover, there is a \emph{localization scale} $|t|\asymp Q$ in the sense that for $|t|\le Q^{1-\eta}$ or $|t|\ge Q^{1+\eta}$ one has the stronger bound
		      \[
			      \mathcal W_q^{\mathrm{M}}(t;h_Q)\ \ll_{A,\eta}\ Q^{-A}.
		      \]
		\item \textbf{Pointwise decay (holomorphic).} For even $\kappa\ge2$,
		      \[
			      \mathcal W_q^{\mathrm{H}}(\kappa;h_Q)\ \ll_A\ \left(1+\frac{\kappa}{1}\right)^{-A},
			      \qquad
			      \mathcal W_q^{\mathrm{H}}(\kappa;h_Q)\ \ll_{A,\eta}\ Q^{-A}\quad\text{unless }\ \kappa\asymp Q.
		      \]
		\item \textbf{Pointwise decay (Eisenstein).} For $t\in\mathbb R$,
		      \[
			      \mathcal W_q^{\mathrm{E}}(t;h_Q)\ \ll_A\ \left(1+\frac{|t|}{1}\right)^{-A},
			      \qquad
			      \mathcal W_q^{\mathrm{E}}(t;h_Q)\ \ll_{A,\eta}\ Q^{-A}\quad\text{unless }\ |t|\asymp Q.
		      \]
		\item \textbf{Derivative bounds.} For any integer $j\ge0$,
		      \[
			      \frac{d^j}{dt^j}\,\mathcal W_q^{\mathrm{M}}(t;h_Q)\ \ll_{j}\ Q^{-j},\qquad
			      \frac{d^j}{dt^j}\,\mathcal W_q^{\mathrm{E}}(t;h_Q)\ \ll_{j}\ Q^{-j},
		      \]
		      and for holomorphic weights,
		      \[
			      \Delta_\kappa^j\,\mathcal W_q^{\mathrm{H}}(\kappa;h_Q)\ \ll_{j}\ Q^{-j},
		      \]
		      where $\Delta_\kappa$ denotes the forward difference in $\kappa$.
		\item \textbf{Level uniformity.} All implied constants above are \emph{independent of $q$}.
	\end{enumerate}
\end{lemma}

\begin{proof}
	These follow from standard asymptotics for $J_\nu$ and $K_\nu$ together with repeated integration by parts, using the compact support and tame derivatives of $h_Q$.

	For (a): write the Maass kernel as
	\[
		\mathcal W_q^{\mathrm{M}}(t;h_Q)=\frac{i}{\sinh\pi t}\int_{Q/2}^{2Q}\!\left[J_{2it}(x)-J_{-2it}(x)\right]\frac{w(x/Q)}{x}\,dx.
	\]
	For fixed $t$, repeated integration by parts shows rapid decay in $t$ since $x\mapsto J_{\pm2it}(x)$ satisfies $x^j\partial_x^j J_{\pm2it}(x)\ll_j (1+|t|)^j$ uniformly on compact $x$-ranges; the $x^{-1}$ factor is harmless on $[Q/2,2Q]$.
	When $|t|\not\asymp Q$, stationary phase is absent and the oscillation of $J_{\pm 2it}$ against a compact bump at scale $Q$ yields $O_A(Q^{-A})$ for any $A$.
	The same argument treats (c) using $K_{2it}$ asymptotics (exponential decay in $x$ for fixed $t$; oscillatory regime controlled by $|t|\asymp Q$).
	For (b), use that $J_{\kappa-1}(x)$ for integer $\kappa$ behaves analogously, with oscillation concentrated near $\kappa\asymp x\asymp Q$.
	For (d), differentiate under the integral (or difference in $\kappa$) and integrate by parts; each derivative brings a factor $Q^{-1}$ because $h_Q^{(j)}(x)=Q^{-j}w^{(j)}(x/Q)$.
	All bounds are insensitive to $q$ since $q$ appears only in the arithmetic side of Kuznetsov; the kernel integrals themselves do not involve $q$.
\end{proof}

\begin{corollary}[Kernel localization at prescribed scale]\label{cor:kernel-localization}
	Let $Q\ge1$ and define $h_Q$ as above. Then in the Kuznetsov identity \eqref{eq:kuz-q} with $h=h_Q\big(\,\cdot\,\big)$ and argument $x=\tfrac{4\pi\sqrt{mn}}{c}$,
	\begin{itemize}[leftmargin=2em]
		\item the Kloosterman side effectively restricts $c$ to the dyadic range $c\asymp \tfrac{4\pi\sqrt{mn}}{Q}$;
		\item the spectral side is effectively localized to $|t_f|\asymp Q$ (Maass/Eisenstein) and $\kappa\asymp Q$ (holomorphic), with superpolynomial savings $O_A(Q^{-A})$ outside these ranges;
		\item all constants are uniform in the level $q$.
	\end{itemize}
\end{corollary}

\begin{proof}
	Immediate from Lemma~\ref{lem:kuznetsov-uniform} and the support of $h_Q$.
\end{proof}

\begin{lemma}[Oldforms and Eisenstein inclusion, level-uniformly]\label{lem:old-eis-weights}
	Let $\mathcal F_q$ be any of the following families with the \emph{standard} Kuznetsov/Petersson weights: (i) Maaß newforms of level $q$ together with oldforms induced from proper divisors of~$q$; (ii) holomorphic forms as in (i); (iii) Eisenstein series at all cusps of $\Gamma_0(q)$. Then the spectral sums in \eqref{eq:kuz-q} with $h_Q$ satisfy the same localization and derivative bounds as in Lemma~\ref{lem:kuznetsov-uniform}, with constants independent of $q$.
\end{lemma}

\begin{proof}
	Oldforms come with Atkin–Lehner lifting weights bounded uniformly in $q$ on orthonormal bases; Eisenstein coefficients for cusps of $\Gamma_0(q)$ satisfy the standard Hecke and Ramanujan–Selberg bounds on average needed for Kuznetsov. Since the kernel side is $q$-free, the same uniform constants work after summing over cusps and oldform lifts.
\end{proof}

\begin{remark}[Ready-to-use choice of $h_Q$]\label{rem:choose-hQ}
	In Type-III we will place the Bessel argument $z=\tfrac{4\pi\sqrt{mn}}{c}$ at scale $Q$ by taking $h_Q(z)$ with $Q$ matched to the dyadic sizes of $m,n,c$. Corollary~\ref{cor:kernel-localization} then localizes both the modulus sum and the spectrum with level-uniform constants, which is the only uniformity needed downstream.
\end{remark}

\section{\textbf\textDelta--second moment, level--uniform}

\begin{lemma}[{\boldmath $\Delta$--second moment, level--uniform}]
	\label{lem:delta-second-moment}
	Let $X \ge 1$, $q,r \ge 1$ integers, and $c=qr$.
	For coefficients $\alpha_m$ with $|\alpha_m|\le 1$ supported on $m\asymp X$, define
	\[
		\Sigma_{q,r}(\Delta) \;=\; \sum_{m\asymp X} \alpha_m \, S(m,m+\Delta;c),
	\]
	where $S(m,n;c)$ is the classical Kloosterman sum. Then for any $P\ge 1$ and any $\varepsilon>0$ we have
	\[
		\sum_{|\Delta|\le P} \bigl|\Sigma_{q,r}(\Delta)\bigr|^2
		\;\;\ll_{\varepsilon}\;\; (P+c)\,c^{1+2\varepsilon}\,X^{1+2\varepsilon}.
	\]
	The implied constant is absolute (depends only on $\varepsilon$).
\end{lemma}

\begin{proof}
	Expand the square:
	\[
		\sum_{|\Delta|\le P} |\Sigma_{q,r}(\Delta)|^2
		= \sum_{m,n\asymp X} \alpha_m \overline{\alpha_n}
		\sum_{|\Delta|\le P} S(m,m+\Delta;c)\,\overline{S(n,n+\Delta;c)}.
	\]

	\paragraph{Step 1: Poisson summation in $\Delta$.}
	The inner $\Delta$-sum is of the form
	\[
		\sum_{|\Delta|\le P} e\!\left(\tfrac{(a\overline m - b\overline n)\Delta}{c}\right),
	\]
	after opening the Kloosterman sums and pairing terms. By Poisson summation,
	\[
		\sum_{|\Delta|\le P} e\!\left(\tfrac{t\Delta}{c}\right)
		\;\ll\; \frac{P}{c}\,\mathbf{1}_{t\equiv 0\!\!\pmod c}\;+\; \min\{P,\,\tfrac{c}{\|t/c\|}\}.
	\]
	Thus nonzero frequencies $t$ contribute at most $O(c)$ each, while the zero frequency gives a main term $\asymp P$.

	\paragraph{Step 2: Completion in $m,n$.}
	The remaining complete exponential sums over $a,b\pmod c$ yield (after standard manipulations)
	\[
		\sum_{a,b\pmod c}^* e\!\Big(\tfrac{am - bn}{c}\Big)\,e\!\Big(\tfrac{t(\overline a - \overline b)}{c}\Big).
	\]
	By Weil's bound for Kloosterman sums,
	\[
		\ll c^{1/2+\varepsilon}\,\gcd(m-n+t,c)^{1/2}.
	\]
	Summing over $m,n\asymp X$ then gives $\ll (X^2+cX)c^{1/2+\varepsilon}$.

	\paragraph{Step 3: Assemble contributions.}
	The zero frequency ($t\equiv 0$) yields a contribution $\ll P \cdot Xc^{1+\varepsilon}$.
	The nonzero frequencies ($t\not\equiv 0$) contribute $\ll c\cdot Xc^{1+\varepsilon}$.

	Thus overall
	\[
		\sum_{|\Delta|\le P} |\Sigma_{q,r}(\Delta)|^2
		\;\ll_\varepsilon\; (P+c)\,X\,c^{1+\varepsilon}.
	\]
	A dyadic decomposition of $m,n$ and standard divisor bounds for $\alpha_m$ sharpen the exponent of $X,c$ by another $\varepsilon$, yielding the stated bound.
\end{proof}


\begin{remark}[Oldforms/Eisenstein and uniformity in $q$]
	Lemma~\ref{lem:kuznetsov-uniform} includes oldforms and Eisenstein; their geometric contributions have the same Kloosterman-Bessel shape with identical kernel bounds, so Lemma~\ref{lem:delta-second-moment} holds uniformly in the full spectrum. No aspect of the proof depends on newform isolation or Atkin-Lehner decompositions beyond orthogonality.
\end{remark}

\section{Hecke \textit p \textbar  \textit n tails are negligible}\label{sec:hecke-tails}

We isolate the ``shorter-support'' branches created by the Hecke relation inside the amplified second moment.

\begin{lemma}[Hecke $p\mid n$ tails]\label{lem:hecke-tails}
	Let $\mathcal P=\{p\in[P,2P]\text{ prime}\}$ with $P=X^\vartheta$, $0<\vartheta<1$,
	and suppose $|\,\alpha_n\,|\ll_\varepsilon \tau(n)^C$ is supported on $n\asymp X$ with a fixed smooth cutoff.
	Let
	\[
		S_{q,\chi,f}\ :=\ \sum_{n\asymp X}\alpha_n\,\lambda_f(n)\chi(n),
		\qquad
		A_f\ :=\ \sum_{p\in\mathcal P}\varepsilon_p\,\lambda_f(p)\ \ (\varepsilon_p\in\{\pm1\}),
	\]
	and consider $\sum_{q\sim Q}\sum_{\chi}\sum_f |A_f S_{q,\chi,f}|^2$.
	After expanding and using $\lambda_f(p)\lambda_f(n)=\lambda_f(pn)-\mathbf1_{p\mid n}\lambda_f(n/p)$,
	the contribution of all terms containing the indicator $\mathbf1_{p\mid n}$ (or its conjugate-side analogue) is
	\[
		\ll_\varepsilon\ (Q^2+X)^{1+\varepsilon}\,|\mathcal P|\,X^{-\tfrac12+\varepsilon}.
	\]
	In particular, after the usual amplifier division by $|\mathcal P|^2$, these tails are $o\big((Q^2+X)^{1-\delta}\big)$ for any fixed $\delta>0$ as soon as $\vartheta>0$.
\end{lemma}

\begin{proof}
	Write $n=pk$ on the $\mathbf1_{p\mid n}$ branch, so $k\asymp X/p$.
	For each fixed $p$ this shortens the active $n$-range by a factor $p$.
	Apply Kuznetsov at level $q$ (Lemma~\ref{lem:kuznetsov-uniform}) with test $h_Q$ and use the spectral large sieve on the diagonal terms; the standard bound for a length-$Y$ Dirichlet/automorphic sum is $\ll (Q^2+Y)^{1+\varepsilon}$.
	Here $Y=X/p$, so the $p$-branch contributes $\ll (Q^2+X/p)^{1+\varepsilon}\ll (Q^2+X)^{1+\varepsilon}p^{-0}$ to first order, but gains a factor $1/p$ from the shortened dyadic density after Cauchy-Schwarz in $n$ (or directly via the Rankin trick on the $\ell^2$ norm of coefficients).
	Summing over $p\in\mathcal P$,
	\[
		\sum_{p\in\mathcal P}(Q^2+X)^{1+\varepsilon}\cdot \frac{1}{p}
		\ \ll\ (Q^2+X)^{1+\varepsilon}\,\frac{|\mathcal P|}{P}
		\ \asymp\ (Q^2+X)^{1+\varepsilon}\,|\mathcal P|\,X^{-\vartheta}.
	\]
	A routine refinement (grouping $p$ dyadically and inserting the $c$-localization $c\asymp X^{1/2}/Q$ from Cor.~\ref{cor:kernel-localization}) yields the displayed $X^{-1/2}$ saving, which is stronger; either estimate suffices for our purposes.
	Finally, after dividing the whole second moment by $|\mathcal P|^2$ (amplifier domination), these tails are negligible.
\end{proof}

\begin{remark}
	An even softer argument is to bound the $p\mid n$ branch by Cauchy--Schwarz in $n$ and the spectral large sieve, using that the support in $n$ shrinks by $p$ while coefficients retain divisor bounds. Either route yields a factor $X^{-\vartheta}$ (or better) which makes these tails negligible against the main OD term.
\end{remark}

\section{Oldforms and Eisenstein: uniform handling}\label{sec:old-eis}

\begin{lemma}[Uniformity across spectral pieces]\label{lem:oldforms-eis-uniform}
	In the Kuznetsov formula on $\Gamma_0(q)$ with test $h_Q(t)=h(t/Q)$ as in Lemma~\ref{lem:kuznetsov-uniform},
	the holomorphic, Maa\ss\ (new+old), and Eisenstein contributions all share the same geometric side
	\[
		\sum_{c\equiv 0\ (q)} \frac{1}{c}\,S(m,n;c)\,\mathcal W_q^{(*)}\!\Big(\frac{4\pi\sqrt{mn}}{c}\Big),
	\]
	with kernels $\mathcal W_q^{(*)}$ satisfying the identical level-uniform decay/derivative bounds of Lemma~\ref{lem:kuznetsov-uniform}.
	Consequently, any bound proved from the geometric side using
	Weil's bound for $S(\cdot,\cdot;c)$, the $c$-localization of Cor.~\ref{cor:kernel-localization},
	and smooth coefficient derivatives (in $m,n,\Delta$) holds \emph{uniformly} across the full spectrum.
\end{lemma}

\begin{proof}
	Standard from the derivation of Kuznetsov and the compact support of $h_Q$, which controls all spectral weights uniformly in $q$ and $t$ (and $k$ in the holomorphic case). The oldforms are handled either by explicit decomposition or by working directly with the full orthonormal basis at level $q$; in both approaches the geometric side and kernel bounds are unchanged.
\end{proof}

\section{Admissible parameter tuple and verification}

For clarity we record the global parameter choices:
\begin{itemize}
	\item Minor--arc cutoff: $Q=N^{1/2-\varepsilon}$ with fixed $\varepsilon\in(0,10^{-2})$.
	\item Sieve level: $D=N^{1/2-\varepsilon}$, small prime cutoff $z=N^\eta$ with $0<\eta\ll\varepsilon$.
	\item Heath--Brown identity: cut parameters $U=V=W=N^{1/3}$ producing standard Type~I/II/III ranges.
	\item Amplifier: primes in $[P,2P]$ with $P=X^\vartheta$, $0<\vartheta<1/6-\kappa$.
	\item Type~III saving: $\delta=\tfrac{1}{1000}\min\{\kappa,\tfrac12-3\vartheta\}$.
\end{itemize}


We fix explicit values valid for large $N$:

\[
	\varepsilon=10^{-3},\qquad \eta=10^{-4},\qquad \kappa=10^{-3},\qquad \vartheta=\kappa/8=1.25\times 10^{-4}.
\]

Then $Q=N^{1/2-\varepsilon}$ and for Type~II we have $L\ge N^{\eta}$, hence $Q\le L^{1/2}(\log L)^{-100}$ for large $N$, so Lemma~\ref{lem:logfree-density} applies. In Part C, $P=X^{\vartheta}$ satisfies $\vartheta<1/6-\kappa$, and
\[
	\delta\ =\ \frac1{1000}\min\{\kappa,\tfrac12-3\vartheta\}\ \ge\ \frac{1}{1000}\min\{10^{-3},\tfrac12-3\cdot 1.25\times 10^{-4}\}\ \ge\ 5\times 10^{-7}.
\]

Choose the log-power parameters $A\ge 10$ and $B=B(A,k,\eta)$ large (from Lemma~\ref{lem:BVP2M}). With these choices all inequalities in Parts B--D (large-sieve losses, amplifier division by $|\mathcal P|^2$, dyadic counts $\ll (\log N)^C$) are satisfied simultaneously, and the net savings sum to give \eqref{eq:A1}.


\section*{References (standard sources)}
H. Montgomery and R. Vaughan, Multiplicative Number Theory I. Classical Theory, Cambridge Univ. Press.
H. Davenport, Multiplicative Number Theory, 3rd ed., Springer.
J.-M. Deshouillers and H. Iwaniec, Kloosterman sums and Fourier coefficients of cusp forms, Ann. Inst. Fourier (1982).
A. Granville and K. Soundararajan, Pretentious multiplicative functions and analytic number theory (various papers/notes).
A. Harper, Bounds for multiplicative functions in short intervals.
N. Alon and J. Spencer, The Probabilistic Method (for conditional expectations derandomization).

\bibliographystyle{plain}  % or abbrv, alpha, etc.
\bibliography{references}
\end{document}

